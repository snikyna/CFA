% Financial Analysis Study Notes for CFA Level I
% This document contains comprehensive study notes covering the Financial Statement Analysis framework
% and key concepts for the CFA Level I examination
% Author: CFA Level I Candidate
% Created: 2024 Study Session
% Last Updated: October 2025

\documentclass[12pt]{article}
\usepackage{amsmath}
\usepackage{geometry}
\usepackage{graphicx} % for including images and figures
\usepackage{booktabs}
\usepackage{caption}
\usepackage{titlesec}
\usepackage{float}
\usepackage{makecell}
\usepackage{tabularx}
\usepackage{enumitem}
\usepackage[utf8]{inputenc}
\usepackage{textcomp}


\geometry{margin=1in}

\title{Financial Analysis}
\author{}
\date{}

\begin{document}
\maketitle
\subsection*{Module 29.1: Financial Statement Roles}

\subsubsection*{LOS 29.a: Steps in the Financial Statement Analysis Framework}
\begin{itemize}
  \item \textbf{Step 1: State the objective and context}
    \begin{itemize}
      \item Define key questions: e.g., “Should we invest in this company’s bonds?”
      \item Decide reporting format (memo, detailed report, presentation).
      \item Consider time and resources available.
    \end{itemize}
  \item \textbf{Step 2: Gather data}
    \begin{itemize}
      \item Collect company’s financial statements (10-K, annual reports).
      \item Industry reports, macroeconomic data.
      \item Field research: interviews with management, suppliers, site visits.
    \end{itemize}
  \item \textbf{Step 3: Process the data}
    \begin{itemize}
      \item Adjust statements (e.g., leases capitalized).
      \item Compute ratios: liquidity, profitability, leverage.
      \item Prepare exhibits: graphs, common-size balance sheets.
    \end{itemize}
  \item \textbf{Step 4: Analyze and interpret the data}
    \begin{itemize}
      \item Compare with peers and historical data.
      \item Identify risk factors and growth opportunities.
    \end{itemize}
  \item \textbf{Step 5: Report conclusions or recommendations}
    \begin{itemize}
      \item Ensure compliance with CFA Code and Standards.
      \item Adapt report to audience (investors, management, regulators).
    \end{itemize}
  \item \textbf{Step 6: Update the analysis}
    \begin{itemize}
      \item Continuous monitoring of new data.
      \item Adjust recommendations as conditions change.
    \end{itemize}
\end{itemize}

\subsubsection*{LOS 29.b: Roles of Financial Statement Analysis}
\begin{itemize}
  \item Uses accounting information to support \textbf{economic decisions}.
  \item Examples of decisions:
    \begin{itemize}
      \item Buy/sell recommendations for equity or debt securities.
      \item Assigning credit ratings.
      \item Extending trade or bank credit.
    \end{itemize}
  \item Analysts evaluate:
    \begin{itemize}
      \item Past performance and financial position.
      \item Future ability to generate profits and cash flows.
      \item Risk factors impacting profitability and stability.
    \end{itemize}
\end{itemize}

\subsubsection*{LOS 29.c: Importance of Regulatory Filings and Disclosures}
\begin{itemize}
  \item \textbf{Standard-setters:}
    \begin{itemize}
      \item \textbf{FASB (U.S.\@)}: U.S.\@ GAAP.
      \item \textbf{IASB (International)}: IFRS\@.
    \end{itemize}
  \item \textbf{Regulators:}
    \begin{itemize}
      \item SEC (U.S.), FCA (UK), ESMA (EU).
      \item Members of \textbf{IOSCO} regulate $95\%$ of global markets.
    \end{itemize}
  \item \textbf{IOSCO Objectives:}
    \begin{enumerate}
      \item Protect investors.
      \item Ensure fair, efficient, transparent markets.
      \item Reduce systemic risk.
    \end{enumerate}
  \item \textbf{SEC Example Requirements:}
    \begin{itemize}
      \item Compliance with Sarbanes---Oxley Act (SOX 2002).
      \item CEO/CFO certification of financial statements.
      \item Auditor independence (cannot provide consulting services).
      \item Internal controls effectiveness statement.
    \end{itemize}
\end{itemize}

\paragraph{Financial Statement Notes (Footnotes):}
\begin{itemize}
  \item Provide basis of presentation (IFRS vs U.S. GAAP, fiscal year end).
  \item Disclose accounting methods, assumptions, estimates.
  \item Contain details on acquisitions, legal contingencies, pensions, related parties.
  \item Segment disclosures:
    \begin{itemize}
      \item Revenue (external + inter-segment).
      \item Assets, liabilities, profit/loss.
      \item CapEx, D\&A, income taxes.
    \end{itemize}
  \item Segments must account for $\geq 75\%$ of external sales.
\end{itemize}

\paragraph{Management Commentary (MD\&A):}
\begin{itemize}
  \item Nature of business, strategy, past performance.
  \item Key risks, relationships, forward-looking statements.
  \item U.S. SEC requires MD\&A to cover:
    \begin{itemize}
      \item Liquidity and capital resources.
      \item Effects of inflation.
      \item Off-balance sheet obligations.
      \item Critical accounting policies.
    \end{itemize}
\end{itemize}

\paragraph{Audit Reports:}
\begin{itemize}
  \item \textbf{Unqualified opinion (clean)}: No material errors.
  \item \textbf{Qualified opinion}: Exceptions exist.
  \item \textbf{Adverse opinion}: Misstated or misleading.
  \item \textbf{Disclaimer}: No opinion possible (scope limitation).
  \item Key Audit Matters (KAMs) / Critical Audit Matters (CAMs) disclose:
    \begin{itemize}
      \item Most significant accounting judgments.
      \item Challenging/subjective areas of audit.
    \end{itemize}
\end{itemize}

\subsubsection*{LOS 29.d: Alternative Reporting Systems and Monitoring}
\begin{itemize}
  \item \textbf{Key issue:} IFRS vs. U.S. GAAP differences can distort cross-border comparisons.
  \item Example differences:
    \begin{itemize}
      \item \textbf{IFRS:} Principle-based, allows revaluation of PP\&E.
      \item \textbf{U.S. GAAP:} Rule-based, historical cost model.
    \end{itemize}
  \item Analysts must track:
    \begin{itemize}
      \item New products and financial innovations.
      \item Emerging accounting standards.
      \item Significant changes in company disclosures.
    \end{itemize}
  \item Sources: IASB, FASB websites, CFA Institute position papers.
\end{itemize}

\subsubsection*{LOS 29.e: Additional Information Sources}
\begin{itemize}
  \item \textbf{Issuer sources:}
    \begin{itemize}
      \item Earnings calls (Q\&A with management).
      \item Ad hoc presentations, press releases.
      \item Direct communications with management / IR\@.
    \end{itemize}
  \item \textbf{Public third-party sources:}
    \begin{itemize}
      \item Industry reports, whitepapers, trade journals.
      \item Government statistics.
      \item Media and social media.
    \end{itemize}
  \item \textbf{Proprietary third-party sources:}
    \begin{itemize}
      \item Bloomberg, FactSet, Wind.
      \item Analyst/consultancy reports.
    \end{itemize}
  \item \textbf{Proprietary primary research:}
    \begin{itemize}
      \item Commissioned studies.
      \item First-hand product usage.
      \item Technical expert consultations.
    \end{itemize}
\end{itemize}

\subsubsection*{Exhibit: Comparison Table}
\begin{table}[H]
\centering
\resizebox{\textwidth}{!}{%
\footnotesize
\begin{tabular}{|l|p{5cm}|p{6cm}|}
\hline
\textbf{Source} & \textbf{Strengths} & \textbf{Limitations} \\
\hline
Financial Statements & Audited, standardized (IFRS/GAAP) & Backward-looking, limited qualitative info \\
\hline
Management Commentary & Forward-looking, strategic insights & Partially unaudited, potential bias \\
\hline
Footnotes & Detail on assumptions, methods, risks & Complex, requires expertise to interpret \\
\hline
Audit Report & Provides assurance, highlights key issues & Only “reasonable assurance,” not absolute \\
\hline
Earnings Calls / Press Releases & Timely updates, direct access to management & Not audited, selective disclosure risk \\
\hline
Third-party Reports (Bloomberg, FactSet) & Independent analysis, benchmarks & Expensive, potential conflicts of interest \\
\hline
Proprietary Research & Tailored, unique insights & Costly, time-intensive \\
\hline
\end{tabular}}
\caption{Comparison of Information Sources in Financial Analysis}
\end{table}

\subsection*{Module 30.1: Revenue Recognition}

\subsubsection*{LOS 30.a: General Principles of Revenue Recognition}
\begin{itemize}
  \item \textbf{Core principle:} Revenue is recognized when control of goods/services transfers to the customer, not necessarily when cash is received.
  \item \textbf{Accrual basis:} 
    \begin{itemize}
      \item Credit sales $\rightarrow$ Revenue recognized at sale; Accounts Receivable created.
      \item Cash received in advance $\rightarrow$ Recorded as \textit{Unearned Revenue (liability)} until goods/services delivered.
      \item Example: Magazine subscription $\rightarrow$ Cash received upfront, liability recognized, revenue recognized as issues delivered.
    \end{itemize}
  \item \textbf{Revenue is reported net of:}
    \begin{itemize}
      \item Returns
      \item Allowances
      \item Discounts
      \item Warranty provisions
    \end{itemize}
\end{itemize}

\subsubsection*{Five-Step Model under Converged IFRS/US GAAP (IFRS 15 / ASC 606)}
\begin{enumerate}
  \item Identify the \textbf{contract (s)} with a customer.
  \item Identify distinct \textbf{performance obligations}.
  \item Determine the \textbf{transaction price}.
  \item Allocate the transaction price to the performance obligations.
  \item Recognize revenue when/as performance obligations are satisfied.
\end{enumerate}

\paragraph{Definitions:}
\begin{itemize}
  \item \textbf{Contract:} Agreement with enforceable rights/obligations; collectability must be probable (definition of “probable” differs under IFRS vs US GAAP).
  \item \textbf{Performance obligation:} Promise to deliver a distinct good/service.
    \begin{itemize}
      \item Distinct if:
        \begin{enumerate}
          \item Customer can benefit independently or with other resources.
          \item Transfer promise is identifiable separately.
        \end{enumerate}
    \end{itemize}
  \item \textbf{Transaction price:} Expected amount of consideration (fixed or variable).
  \item \textbf{Revenue recognition:} Only when highly probable it won’t be reversed.
  \item \textbf{Indicators of control transfer:} Physical possession, acceptance, transfer of risks/benefits, legal title, right to payment.
\end{itemize}

\subsubsection*{Revenue Recognition in Long-Term Contracts}
\begin{itemize}
  \item Revenue recognized \textbf{over time} if:
    \begin{enumerate}
      \item Customer benefits continuously as supplier performs (e.g., maintenance contracts).
      \item Customer controls asset being created/enhanced (e.g., construction projects).
      \item Asset has no alternative use + supplier has right to payment for completed work.
    \end{enumerate}
  \item Measurement:
    \begin{itemize}
      \item \textbf{Input method:} % of completion costs incurred.
      \item \textbf{Output method:} Engineering milestones, % delivered.
    \end{itemize}
  \item Costs to secure contracts (e.g., sales commissions) must be \textbf{capitalized}.
\end{itemize}

\subsubsection*{Examples (IFRS 15 Applications)}
\paragraph{1. Long-term contract (Warehouse construction)}
\begin{itemize}
  \item Contract price = \$10m; total costs estimated = \$8m.
  \item Year 1: Costs incurred = \$4m (50\% completion) $\rightarrow$ Revenue recognized = 0.5 $\times$ \$10m = \$5m.
  \item Year 2: Costs incurred additional \$2m \rightarrow Cumulative costs = \$6m (75\% completion).  
  Revenue to date = 0.75 $\times$ \$10m = \$7.5m.  
  Revenue recognized in Year 2 = \$7.5m $-$ \$5m = \$2.5m.
  \item Equivalent to \textbf{Percentage-of-Completion Method}.
\end{itemize}

\paragraph{2. Acting as an Agent (Travel Agent)}
\begin{itemize}
  \item Ticket price = \$10,000.  
  \item Commission = \$1,000 (no credit or inventory risk).
  \item Revenue recognized = \$1,000 (net).  
  \item If treated as principal \rightarrow Revenue = \$10,000, Expense = \$9,000, GP = \$1,000.  
  \item \textbf{Gross profit margin differences:}
    \begin{itemize}
      \item As principal: \( \frac{1,000}{10,000} = 10\% \).
      \item As agent: \( \frac{1,000}{1,000} = 100\% \).
    \end{itemize}
\end{itemize}

\paragraph{3. Franchising and Licensing (Fast Food Chain)}
\begin{itemize}
  \item Revenue categories:
    \begin{enumerate}
      \item Company-owned restaurants.
      \item Franchise royalties \& fees (deferred then amortized over franchise term).
      \item Supplies to franchisees (equipment, food).
    \end{enumerate}
  \item Royalties (e.g., 2\% turnover) recognized when payable.
\end{itemize}

\paragraph{4. Service vs License (Software Supplier)}
\begin{itemize}
  \item \textbf{Case A\@: License with continuous updates.}  
  Revenue recognized over contract life.
  \item \textbf{Case B: License “as is”.}  
  Revenue recognized at outset; updates covered in separate contract.
  \item \textbf{Cloud service (SaaS).}  
  Revenue recognized over subscription life (service).
\end{itemize}

\paragraph{5. Bill-and-Hold Agreements}
\begin{itemize}
  \item Customer pays ahead of shipping; normally \rightarrow deferred revenue.
  \item Revenue may be recognized before delivery if:
    \begin{itemize}
      \item Customer requests arrangement.
      \item Goods are identified as belonging to customer.
      \item Goods are complete and ready to ship.
      \item Supplier cannot redirect goods.
    \end{itemize}
\end{itemize}

\subsubsection*{Required Disclosures (IFRS 15 / ASC 606)}
\begin{itemize}
  \item Disaggregation of revenue (by product/service category).
  \item Assets \& liabilities from contracts (balances, changes).
  \item Outstanding performance obligations + allocated transaction prices.
  \item Management judgments on timing/amount of revenue.
\end{itemize}

\subsubsection*{Exhibit: Examples of Revenue Recognition}
\begin{table}[H]
\centering
\resizebox{\textwidth}{!}{%
\footnotesize
\begin{tabular}{|l|p{5cm}|p{5cm}|}
\hline
\textbf{Scenario} & \textbf{Revenue Recognition} & \textbf{Implications for Analysis} \\
\hline
Credit Sale & Recognized at sale (A/R created) & Cash flow timing differs from revenue; analysts adjust for working capital. \\
\hline
Advance Payment (Magazine subscription) & Initially liability (unearned revenue); recognized as delivered & Liability inflates until service performed. \\
\hline
Long-term Contract & Over time using input/output methods & Smooths revenue; requires estimate reliability. \\
\hline
Agent vs Principal & Agent \rightarrow Net revenue (commission only). Principal \rightarrow Gross revenue & Gross margin ratios differ; important for comparability. \\
\hline
Franchise Fees \& Royalties & Fees deferred, amortized; royalties when payable & Analysts separate recurring vs one-time revenue streams. \\
\hline
Software License vs SaaS & License revenue upfront vs over contract term & Recognition timing significantly affects earnings profile. \\
\hline
Bill-and-Hold & Recognize if customer controls goods & May accelerate revenue; analysts should check substance. \\
\hline
\end{tabular}}
\caption{Revenue Recognition Scenarios and Implications}
\end{table}

\subsection*{Module 30.2: Expense Recognition}

\subsubsection*{LOS 30.b: General Principles of Expense Recognition}
\begin{itemize}
  \item \textbf{Definition (IASB):} Expenses = decreases in economic benefits during an accounting period in the form of:
    \begin{itemize}
      \item Outflows or depletions of assets
      \item Increases in liabilities
      \item Resulting in decreases in equity (other than distributions to owners)
    \end{itemize}
  \item \textbf{Accrual vs Cash Basis:}
    \begin{itemize}
      \item \textbf{Cash basis:} Expense when paid.
      \item \textbf{Accrual basis:} Expense when economic benefit is consumed.
    \end{itemize}
  \item \textbf{Three recognition methods:}
    \begin{enumerate}
      \item \textbf{Matching principle:} Match expense with revenue generated (e.g., COGS, warranty provisions).
      \item \textbf{Capitalization:} Record as asset \rightarrow amortized/depreciated as benefits consumed.
      \item \textbf{Expensing as incurred:} Period costs (admin, rent, utilities).
    \end{enumerate}
  \item \textbf{Conservatism vs Aggressiveness:}
    \begin{itemize}
      \item Expensing earlier = conservative.
      \item Deferring via capitalization = aggressive.
    \end{itemize}
\end{itemize}

\subsubsection*{Example: Matching Principle with Inventory}
\begin{itemize}
  \item Firm sells 100 units during the year.
  \item Beginning inventory = 20 units @ \$400 total.
  \item Purchases = 90 units (various costs). Units available = 110.
  \item Ending inventory = 10 units (8 from most recent purchase, 2 from prior).
  \item \textbf{Matching:} Remove 10 units from COGS \rightarrow report them as inventory (asset).
  \item Ensures COGS = cost of 100 units sold.
\end{itemize}

\paragraph{Note:} If exact identification is not possible \rightarrow use cost flow methods:
\begin{enumerate}
  \item FIFO (First-in, First-out)
  \item LIFO (Last-in, First-out)
  \item Weighted Average Cost
\end{enumerate}

\subsubsection*{Capitalization vs Expensing}
\begin{itemize}
  \item \textbf{Capitalization:} 
    \begin{itemize}
      \item Expected future economic benefit \rightarrow recorded as asset.
      \item Cost spread via depreciation, amortization, or depletion.
      \item Land and indefinite-life intangibles (goodwill) not amortized.
    \end{itemize}
  \item \textbf{Expensing:}
    \begin{itemize}
      \item No future benefit or highly uncertain \rightarrow expense immediately.
      \item Reduces current pretax income fully in period incurred.
    \end{itemize}
  \item \textbf{Subsequent expenditures:}
    \begin{itemize}
      \item \textbf{Extend life/increase benefits} \rightarrow capitalize.
      \item \textbf{Maintenance/repairs} \rightarrow expense.
    \end{itemize}
\end{itemize}

\subsubsection*{Example: Northwood Equipment}
\begin{itemize}
  \item Equipment cost = \$250,000 (incl.\ freight + taxes).
  \item Installation = \$10,000 \rightarrow capitalize.
  \item Training = \$7,500 \rightarrow expense (benefits employees, not asset).
  \item Repairs \& maintenance = \$35,000 \rightarrow expense.
  \item Motor rebuild = \$85,000 \rightarrow capitalize (extends life).
\end{itemize}

\subsubsection*{Example: Chair Ltd. (Impact of Capitalization vs Expensing)}
\begin{itemize}
  \item Equipment cost = £12,000, useful life = 4 years, straight-line depreciation.
  \item Annual revenue = £30,000, operating margin = 40\%, tax = 30\%.
\end{itemize}

\paragraph{Impacts:}
\begin{itemize}
  \item \textbf{Income Statement:}
    \begin{itemize}
      \item Capitalization: Expense spread (£3,000/year depreciation).
      \item Expensing: Entire £12,000 in Year 1.
      \item Result: Expensing = lower NI in Year 1, higher NI in later years. Capitalization smooths earnings.
    \end{itemize}
  \item \textbf{Balance Sheet:}
    \begin{itemize}
      \item Capitalization: Higher assets (equipment net of depreciation), higher retained earnings.
      \item Expensing: No asset recorded \rightarrow lower equity in early years.
    \end{itemize}
  \item \textbf{Cash Flow Statement:}
    \begin{itemize}
      \item Capitalization: Cash outflow \rightarrow investing activities.
      \item Expensing: Cash outflow \rightarrow operating activities.
      \item Expensing gives full tax benefit upfront, capitalization spreads it.
    \end{itemize}
  \item \textbf{Ratios:}
    \begin{itemize}
      \item Asset turnover = lower if capitalized (assets higher).
      \item Net profit margin = higher in Year 1 if capitalized.
      \item ROE = higher in Year 1 if capitalized, lower in later years.
    \end{itemize}
\end{itemize}

\subsubsection*{Capitalized Interest}
\begin{itemize}
  \item When firm builds asset for own use or resale \rightarrow interest during construction is capitalized.
  \item Treatment:
    \begin{itemize}
      \item Included in asset cost.
      \item Expensed later via depreciation (if held for use) or COGS (if held for sale).
    \end{itemize}
  \item Cash flow effect:
    \begin{itemize}
      \item Capitalized interest \rightarrow investing outflow.
      \item Expensed interest \rightarrow operating outflow (GAAP) or operating/financing (IFRS).
    \end{itemize}
  \item \textbf{Analyst adjustment:} Add capitalized interest back to interest expense for solvency ratios.
\end{itemize}

\paragraph{Example: Willock AG}
\begin{itemize}
  \item EBIT = €160m, reported interest expense = €80m.
  \item €20m capitalized, €10m depreciation from prior capitalized interest.
  \item Adjusted EBIT = €160m + €10m = €170m.  
  \item Adjusted interest = €80m + €20m = €100m.  
  \item Interest coverage = \( \frac{170}{100} = 1.7 \) (vs reported \( \frac{160}{80} = 2.0 \)).
\end{itemize}

\subsubsection*{R\&D Costs}
\begin{itemize}
  \item \textbf{IFRS:}
    \begin{itemize}
      \item Research costs \rightarrow expensed.
      \item Development costs \rightarrow capitalized if criteria met (e.g., feasibility, intent to use/sell).
    \end{itemize}
  \item \textbf{U.S. GAAP:}
    \begin{itemize}
      \item R\&D \rightarrow expensed.
      \item Software development: expensed until technological feasibility, then capitalized.
    \end{itemize}
  \item \textbf{Analyst adjustment:}  
    \begin{itemize}
      \item Expense capitalized development costs for comparability.
      \item Remove amortization of past capitalized costs.
      \item Adjust CFO downward (include costs in operations).
    \end{itemize}
\end{itemize}

\subsubsection*{Bad Debt \& Warranty Expense Recognition}
\begin{itemize}
  \item Matching principle requires recognition \textbf{at time of sale}.
  \item Estimates involved \rightarrow risk of earnings management.
  \item Analyst checks:
    \begin{itemize}
      \item Compare to peers (e.g., unusually low warranty expense).
      \item Assess whether estimate changes reflect real improvements or manipulation.
    \end{itemize}
\end{itemize}

\subsubsection*{Exhibit: Capitalization vs Expensing --- Financial Statement Effects}
\begin{table}[H]
\centering
\footnotesize
\begin{tabular}{|l|p{5cm}|p{5cm}|}
\hline
\textbf{Aspect} & \textbf{Capitalization} & \textbf{Expensing} \\
\hline
Income Statement & Spreads cost over asset life (depreciation) & Entire cost in Year 1 \\
\hline
Balance Sheet & Higher assets (PP\&E), higher equity (retained earnings) & No asset, lower equity early \\
\hline
Cash Flow Statement & Outflow under investing activities & Outflow under operating activities \\
\hline
Tax Effect & Tax benefit spread over years & Immediate tax benefit in Year 1 \\
\hline
Ratios & Lower asset turnover, smoother NI, higher margins in Year 1 & Higher turnover, volatile NI, margins lower in Year 1 \\
\hline
Earnings Profile & Smooth, less volatile & Front-loaded cost, volatile earnings \\
\hline
\end{tabular}
\caption{Comparison of Capitalization vs Expensing}
\end{table}

\subsection*{Module 30.3: Nonrecurring Items}

\subsubsection*{LOS 30.c: Financial Reporting Treatment and Analysis of Nonrecurring Items}

\subsubsection*{1. Unusual or Infrequent Items}
\begin{itemize}
  \item \textbf{Definition:} Events that are unusual in nature or infrequent in occurrence, and \textbf{material} enough to affect decisions.
  \item \textbf{Examples:}
    \begin{itemize}
      \item Gains/losses from sale of assets or business units (not part of ordinary operations).
      \item Impairments, write-offs, write-downs.
      \item Restructuring costs.
    \end{itemize}
  \item \textbf{Reporting:}
    \begin{itemize}
      \item Included in \textit{income from continuing operations}.
      \item Reported \textbf{before tax}.
    \end{itemize}
  \item \textbf{Analyst consideration:}
    \begin{itemize}
      \item Should assess whether such items are truly nonrecurring.
      \item Some firms report “one-off” charges frequently \rightarrow signals recurring issues.
      \item Adjust forecasts by excluding these from “core” earnings if justified.
    \end{itemize}
\end{itemize}

\subsubsection*{2. Discontinued Operations}
\begin{itemize}
  \item \textbf{Definition:} Component of business that is physically and operationally distinct, and management has decided to dispose of.
  \item \textbf{Phases:}
    \begin{itemize}
      \item \textbf{Measurement date:} When formal plan to dispose is announced.
      \item \textbf{Phaseout period:} Between measurement date and disposal.
    \end{itemize}
  \item \textbf{Accounting treatment:}
    \begin{itemize}
      \item Reported separately in income statement, \textbf{net of tax}, after continuing operations.
      \item Prior-period statements restated for comparability.
      \item Losses during phaseout and estimated loss on sale recognized at measurement date.
      \item Gains only recognized when disposal completed.
    \end{itemize}
  \item \textbf{Analyst treatment:}
    \begin{itemize}
      \item Exclude discontinued operations from future earnings forecasts.
      \item Consider disposal impact on firm’s future cash flows and structure.
    \end{itemize}
\end{itemize}

\subsubsection*{3. Changes in Accounting Policies, Estimates, and Errors}
\begin{itemize}
  \item \textbf{Types of accounting changes:}
    \begin{enumerate}
      \item \textbf{Accounting policy changes:} (e.g., inventory method, capitalization vs expensing).  
      \begin{itemize}
        \item Require \textbf{retrospective application} unless impractical.  
        \item Enhances comparability across periods.  
        \item Example: IFRS 15 revenue recognition \rightarrow allowed modified retrospective application (adjust cumulative balances, no restatement of prior periods).  
      \end{itemize}
      \item \textbf{Accounting estimate changes:} (e.g., useful life of asset, bad debt allowance).  
      \begin{itemize}
        \item Require \textbf{prospective application}.  
        \item Do not affect prior periods; only future results.  
        \item Do not directly affect cash flows.  
      \end{itemize}
      \item \textbf{Corrections of errors / prior-period adjustments:} (e.g., correcting from non-GAAP to GAAP method).  
      \begin{itemize}
        \item Require \textbf{retrospective restatement}.  
        \item Disclosure required (nature of error and impact).  
        \item May indicate weak internal controls.  
      \end{itemize}
    \end{enumerate}
  \item \textbf{Analyst adjustments:}  
    \begin{itemize}
      \item Scrutinize policy changes for earnings management.  
      \item Adjust comparability when firms adopt different policies.  
      \item For estimates, determine whether changes reflect genuine new information or manipulation.  
    \end{itemize}
\end{itemize}

\subsubsection*{4. Changes in Scope and Exchange Rates}
\begin{itemize}
  \item \textbf{Changes in scope:} Acquisitions, mergers, or disposals \rightarrow affect comparability of financial statements before vs after.
  \item \textbf{Exchange rates:} Affect overseas subsidiaries’ revenues, expenses, and assets when translated to reporting currency.
  \item \textbf{Disclosure:} Not explicitly required, but analysts should monitor.
\end{itemize}

\subsubsection*{Exhibit: Summary of Nonrecurring Items Treatment}
\begin{table}[H]
\centering
\resizebox{\textwidth}{!}{%
\footnotesize
\begin{tabular}{|l|p{5cm}|p{5cm}|}
\hline
\textbf{Item} & \textbf{Reporting Treatment} & \textbf{Analyst Implications} \\
\hline
Unusual / Infrequent Items & Included in continuing operations (before tax) & Adjust if not truly one-off; recurring charges reduce quality of earnings \\
\hline
Discontinued Operations & Separate line, net of tax, after continuing operations; prior periods restated & Exclude from future earnings forecasts; assess disposal impact on cash flows \\
\hline
Change in Accounting Policy & Retrospective application (unless impractical) & Improves comparability, but check for management bias \\
\hline
Change in Accounting Estimate & Prospective application & No restatement; assess impact on future earnings \\
\hline
Correction of Errors (Prior-period Adjustment) & Retrospective restatement; disclosure required & May signal weak internal controls; usually no cash flow effect \\
\hline
Change in Scope (M\&A) & Not separately disclosed & Reduces comparability; analyst should adjust historical trends \\
\hline
Exchange Rate Effects & Not separately disclosed & Affects revenues/assets of foreign subsidiaries; adjust for FX volatility \\
\hline
\end{tabular}}
\caption{Nonrecurring Items --- Reporting Treatment and Analyst Considerations}
\end{table}

\subsubsection*{Key Analytical Insights}
\begin{itemize}
  \item Nonrecurring items distort earnings comparability.  
  \item Analysts should focus on \textbf{income from continuing operations} as basis for forecasting.  
  \item Frequent “one-off” losses may reveal poor operations or aggressive accounting.  
  \item Restatements (policy or error corrections) improve comparability but highlight potential internal control issues.  
  \item Scope and FX changes \rightarrow require careful normalization in trend analysis.  
\end{itemize}

\subsection*{Module 30.4: Earnings Per Share (EPS)}

\subsubsection*{LOS 30.d: Basic and Diluted EPS --- Principles and Calculations}

\subsubsection*{1. Overview}
\begin{itemize}
  \item EPS = most widely used measure of corporate profitability for publicly traded firms.
  \item EPS is reported only for \textbf{common stock}.
  \item \textbf{Capital structure types:}
    \begin{itemize}
      \item \textbf{Simple:} Only common stock, nonconvertible debt, nonconvertible preferred.  
      $\Rightarrow$ Report only \textbf{basic EPS}.
      \item \textbf{Complex:} Contains potentially dilutive securities (options, warrants, convertible bonds, convertible preferred).  
      $\Rightarrow$ Report both \textbf{basic and diluted EPS}.
    \end{itemize}
\end{itemize}

\subsubsection*{2. Basic EPS}
\paragraph{Formula:}
\[
\text{Basic EPS} = \frac{\text{Net income} - \text{Preferred dividends}}{\text{Weighted average number of common shares outstanding}}
\]

\begin{itemize}
  \item Preferred dividends are subtracted (common shareholders’ claim).  
  \item Common dividends are \textbf{not} subtracted.  
  \item Weighted average shares = shares outstanding adjusted for:
    \begin{itemize}
      \item Issue or repurchase (time-weighted by days/months).
      \item Stock splits/dividends \rightarrow applied retroactively to beginning of year and prior periods.
    \end{itemize}
\end{itemize}

\paragraph{Example --- Weighted Average Shares (Johnson Co.):}
\begin{itemize}
  \item 10,000 shares at start.
  \item April 1: issue 4,000 shares.
  \item July 1: 10\% stock dividend (retroactive adjustment).
  \item Sept 1: repurchase 3,000 shares.
\end{itemize}
\[
\text{Weighted Average Shares} = \text{time-adjusted and dividend-adjusted count (new shares)}
\]

\paragraph{Example --- Basic EPS (Johnson Co.):}
\begin{itemize}
  \item Net income = \$10,000.
  \item Preferred dividends = \$1,000.  
  \item Weighted average shares (from above) = used in denominator.  
  \item Cash dividends to common (\$1,750) ignored in EPS\@.  
\end{itemize}

\[
\text{Basic EPS} = \frac{10,000 - 1,000}{\text{Weighted Avg. Shares}}
\]

\subsubsection*{3. Diluted EPS}
\paragraph{Definition:}
\begin{itemize}
  \item Diluted EPS considers effects of all \textbf{potentially dilutive securities}.
  \item \textbf{Dilutive security:} reduces EPS if converted (included).  
  \item \textbf{Antidilutive security:} increases EPS if converted (excluded).
\end{itemize}

\paragraph{Formula:}
\[
\text{Diluted EPS} = \frac{\text{Net income available to common (adjusted)}}{\text{Weighted average shares outstanding + shares from conversion (if dilutive)}}
\]

\paragraph{Adjustments:}
\begin{itemize}
  \item \textbf{Convertible Preferred Stock:} Add back preferred dividends to numerator if dilutive.
  \item \textbf{Convertible Debt:} Add back after-tax interest expense:
  \[
  \text{Adj. Net Income} = \text{Net Income} + \text{Interest} \times (1 - t)
  \]
  \item \textbf{Options/Warrants:} Use Treasury Stock Method:
    \begin{itemize}
      \item Assumes exercise proceeds used to buy back shares at average market price.
      \item Net increase = new shares issued $-$ shares repurchased.
    \end{itemize}
\end{itemize}

\subsubsection*{4. Worked Examples}

\paragraph{Example 1 --- Convertible Preferred Stock (ZZZ Corp.)}
\begin{itemize}
  \item Net income = \$4.35m.
  \item Shares outstanding = 2m.
  \item Preferred stock = \$5m par, 7\% dividend, convertible 1.1 shares per \$10 par.
  \item Step 1 --- Basic EPS\@:
  \[
  \text{Basic EPS} = \frac{4.35m - 0.35m}{2m} = 2.00
  \]
  \item Step 2 --- Diluted EPS\@:
    \begin{itemize}
      \item New shares = (5m / 10) $\times$ 1.1 = 550,000 shares.
      \item Add back preferred dividends (\$0.35m).  
      \item Diluted EPS\@:
      \[
      \frac{4.35m}{2.55m} = 1.71
      \]
    \end{itemize}
  \item Since diluted EPS (1.71) < basic (2.00) \rightarrow dilutive.
\end{itemize}

\paragraph{Example 2 --- Convertible Debt (YYY Corp.)}
\begin{itemize}
  \item Net income available = \$2.5m.
  \item Shares outstanding = 1m.
  \item Basic EPS = 2.50.
  \item Convertible bonds = 2,000 bonds $\times$ \$1,000 $\times$ 5\% = \$100,000 interest.
  \item Tax rate = 30\%.
  \item Step 1 --- Extra shares if converted:
  \[
  2,000 \times 120 = 240,000
  \]
  \item Step 2 --- Add back after-tax interest:
  \[
  100,000 \times (1 - 0.30) = 70,000
  \]
  \item Step 3 --- Diluted EPS\@:
  \[
  \frac{2.5m + 70,000}{1m + 240,000} = 2.07
  \]
  \item Since 2.07 < 2.50 \rightarrow dilutive.
\end{itemize}

\paragraph{Example 3 --- Stock Options/Warrants (XXX Corp.)}
\begin{itemize}
  \item Net income = \$1.2m.
  \item Shares = 500,000.
  \item Basic EPS = 2.40.
  \item Options outstanding = 100,000 @ \$15 exercise price.
  \item Average market price = \$20.
  \item Step 1 --- Shares issued if exercised = 100,000.
  \item Step 2 --- Proceeds = 100,000 $\times$ 15 = 1.5m.
  \item Step 3 --- Shares repurchased = 1.5m / 20 = 75,000.
  \item Step 4 --- Net new shares = 25,000.
  \item Step 5 --- Diluted EPS\@:
  \[
  \frac{1.2m}{500,000 + 25,000} = 2.29
  \]
  \item Options are dilutive since 2.29 < 2.40.
\end{itemize}

\subsubsection*{5. Summary Table --- Basic vs Diluted EPS}
\begin{table}[H]
\centering
\footnotesize
\begin{tabular}{|l|p{5cm}|p{6cm}|}
\hline
\textbf{Case} & \textbf{Numerator Adjustment} & \textbf{Denominator Adjustment} \\
\hline
Basic EPS & Net income $-$ preferred dividends & Weighted average shares outstanding \\
\hline
Convertible Preferred & Add back preferred dividends if dilutive & Add new shares if converted \\
\hline
Convertible Debt & Add back interest $\times$ (1 $-$ tax) if dilutive & Add new shares if converted \\
\hline
Options/Warrants & No adjustment & Treasury stock method: new shares $-$ repurchased shares \\
\hline
Antidilutive Securities & Excluded & Excluded (ignored in denominator) \\
\hline
\end{tabular}
\caption{Basic vs Diluted EPS Adjustments}
\end{table}

\subsubsection*{6. Key Analyst Considerations}
\begin{itemize}
  \item Always test each potential security separately for dilution.  
  \item Exclude antidilute securities even if they are convertible.  
  \item Stock splits/dividends \rightarrow retroactively adjust prior years to ensure comparability.  
  \item Diluted EPS provides more conservative measure of per-share profitability.  
  \item Frequent issuance of dilutive securities = red flag for shareholders (dilution of ownership).  
\end{itemize}

\subsection*{Module 30.5: Ratios and Common-Size Income Statements}

\subsubsection*{LOS 30.e: Evaluate Performance Using Common-Size Income Statements and Ratios}

\subsubsection*{1. Common-Size Income Statements}
\begin{itemize}
  \item \textbf{Definition:} Expresses each line item as a percentage of revenue.  
  \item \textbf{Purpose:}
    \begin{itemize}
      \item Eliminates firm size effect $\Rightarrow$ allows \textbf{time-series} and \textbf{cross-sectional} analysis.
      \item Facilitates comparison across peers and over time.
    \end{itemize}
  \item \textbf{Key points:}
    \begin{itemize}
      \item Reveals structural differences in costs and profitability.
      \item Highlights strategic focus (e.g., high R\&D vs low R\&D firms).
      \item Exception: Income tax is more meaningful as \textbf{percentage of pretax income} = effective tax rate.
    \end{itemize}
\end{itemize}

\subsubsection*{2. Example: North vs South Company}
\paragraph{Absolute Results (in \$):}
\begin{itemize}
  \item North: Revenue = 75,000,000; Gross Profit = 22,500,000; Operating Profit = 7,500,000.
  \item South: Revenue = 3,500,000; Gross Profit = 2,800,000; Operating Profit = 1,575,000.
  \item North larger and higher absolute profit.
\end{itemize}

\paragraph{Common-Size Results (relative \% of revenue):}
\begin{table}[H]
\centering
\footnotesize
\begin{tabular}{|l|c|c|}
\hline
\textbf{Metric} & \textbf{North (\% of Revenue)} & \textbf{South (\% of Revenue)} \\
\hline
Gross Profit Margin & 30\% & 80\% \\
Operating Profit Margin & 10\% & 45\% \\
R\&D Expense & Lower proportion & Higher proportion \\
\hline
\end{tabular}
\caption{Common-Size Income Statement Comparison: North vs South}
\end{table}

\paragraph{Insights:}
\begin{itemize}
  \item South is \textbf{more profitable relatively}, despite smaller size.
  \item High gross margin suggests \textbf{technological differentiation or pricing power}.
  \item Higher R\&D share indicates innovation-driven strategy.
\end{itemize}

\subsubsection*{3. Margin Ratios (Profitability Metrics)}
\paragraph{Formulas:}
\begin{itemize}
  \item \textbf{Gross Profit Margin:}
  \[
  \text{GPM} = \frac{\text{Gross Profit}}{\text{Revenue}} = \frac{\text{Revenue $-$ COGS}}{\text{Revenue}}
  \]
  \item \textbf{Operating Profit Margin:}
  \[
  \text{OPM} = \frac{\text{Operating Profit}}{\text{Revenue}}
  \]
  \item \textbf{Pretax Margin:}
  \[
  \text{Pretax Margin} = \frac{\text{Pretax Accounting Profit}}{\text{Revenue}}
  \]
  \item \textbf{Net Profit Margin:}
  \[
  \text{NPM} = \frac{\text{Net Income}}{\text{Revenue}}
  \]
  \item \textbf{Effective Tax Rate:}
  \[
  \text{ETR} = \frac{\text{Income Tax Expense}}{\text{Pretax Income}}
  \]
\end{itemize}

\paragraph{Interpretation:}
\begin{itemize}
  \item \textbf{Gross Profit Margin (GPM):}
    \begin{itemize}
      \item Indicates ability to cover production costs.
      \item Improved via: raising prices, reducing production costs.
      \item Higher GPM often reflects product differentiation (brand, technology, patents).
    \end{itemize}
  \item \textbf{Operating Profit Margin (OPM):}
    \begin{itemize}
      \item Accounts for operating expenses (R\&D, SG\&A).
      \item Measures efficiency of operations and cost control.
    \end{itemize}
  \item \textbf{Net Profit Margin (NPM):}
    \begin{itemize}
      \item Includes all expenses (interest, tax).
      \item Best measure of bottom-line profitability.
    \end{itemize}
  \item \textbf{Pretax Margin:}
    \begin{itemize}
      \item Useful for comparing firms across different tax jurisdictions.
    \end{itemize}
\end{itemize}

\subsubsection*{4. Example: Ratio Analysis of North vs South}
\begin{table}[H]
\centering
\footnotesize
\begin{tabular}{|l|c|c|c|}
\hline
\textbf{Ratio} & \textbf{Formula} & \textbf{North} & \textbf{South} \\
\hline
Gross Profit Margin & GP / Revenue & 30\% & 80\% \\
\hline
Operating Profit Margin & OP / Revenue & 10\% & 45\% \\
\hline
Net Profit Margin & NI / Revenue & Lower & Higher \\
\hline
R\&D as \% of Revenue & R\&D / Revenue & Low & High \\
\hline
Effective Tax Rate & Tax Expense / Pretax Income & Apply separately & Apply separately \\
\hline
\end{tabular}
\caption{Comparison of Profitability Ratios: North vs South}
\end{table}

\paragraph{Insights:}
\begin{itemize}
  \item North: Economies of scale, but low margins. Strategy: volume-driven.
  \item South: Differentiated products, high pricing power, innovation-focused.
  \item Indicates South may sustain higher profitability despite smaller size.
\end{itemize}

\subsubsection*{5. Key Analyst Considerations}
\begin{itemize}
  \item Common-size analysis reveals \textbf{underlying strategy and structure}, not visible in absolute figures.
  \item Margin ratios should be tracked \textbf{over time} (trend analysis) and \textbf{against peers} (cross-sectional).
  \item Tax effects should be separated using effective tax rate.
  \item High margins may indicate differentiation but may also suggest risk if not sustainable.
  \item Low margins may suggest commoditization, reliance on cost leadership.
\end{itemize}

\subsection*{Module 31.1: Intangible Assets and Marketable Securities}

\subsubsection*{LOS 31.a: Intangible Assets}
\begin{itemize}
  \item \textbf{Definition:} Non-monetary assets lacking physical substance.
  \item \textbf{Types:}
    \begin{itemize}
      \item \textbf{Identifiable:} Can be acquired separately (patents, trademarks, copyrights).
      \item \textbf{Unidentifiable:} Cannot be separated, often indefinite life (e.g., goodwill).
    \end{itemize}
  \item \textbf{IFRS Treatment:}
    \begin{itemize}
      \item Purchased intangibles: cost model or revaluation model (if active market exists).
      \item Internally created intangibles:
        \begin{itemize}
          \item Research costs $\Rightarrow$ expensed.
          \item Development costs $\Rightarrow$ capitalized if criteria met (e.g., feasibility, intent to use/sell).
        \end{itemize}
    \end{itemize}
  \item \textbf{U.S. GAAP Treatment:}
    \begin{itemize}
      \item Only cost model allowed.
      \item Internally created intangibles (R\&D) generally expensed (except certain legal costs).
    \end{itemize}
  \item \textbf{Subsequent Treatment:}
    \begin{itemize}
      \item Finite-lived $\Rightarrow$ amortized + impairment testing.
      \item Indefinite-lived $\Rightarrow$ no amortization, annual impairment test.
    \end{itemize}
  \item \textbf{Costs Always Expensed (IFRS \& GAAP):} start-up, training, admin, advertising, relocation, termination.
\end{itemize}

\paragraph{Example: Lowe S.A. R\&D Projects (IFRS)}
\begin{itemize}
  \item Project 1: Hydrogen fuel cells (research stage) $\Rightarrow$ costs expensed.
  \item Project 2: Catalytic converter (development stage, prototype exists, resources and market available) $\Rightarrow$ costs capitalized.
  \[
  \text{Capitalized Costs} = 120 + 60 + 30 = \text{€210} \text{ million}
  \]
  \item Admin costs $\Rightarrow$ expensed.
\end{itemize}

\subsubsection*{LOS 31.b: Goodwill}
\begin{itemize}
  \item \textbf{Definition:} Excess purchase price over fair value of net assets in an acquisition.
  \[
  \text{Goodwill} = \text{Purchase Price} - \text{Fair Value of Net Assets}
  \]
  \item \textbf{Key Points:}
    \begin{itemize}
      \item Created only in acquisitions (not internally generated).
      \item Indefinite life $\Rightarrow$ not amortized, but tested annually for impairment.
      \item Impairment recognized as loss (no cash flow impact).
    \end{itemize}
  \item \textbf{Special Case:} If purchase price $<$ fair value $\Rightarrow$ gain recognized in income statement.
  \item \textbf{Analyst Considerations:}
    \begin{itemize}
      \item Some analysts exclude goodwill from balance sheets (improves comparability).
      \item Goodwill impairments can signal poor acquisitions.
      \item Firms may allocate more cost to goodwill (not amortized) vs assets (which depreciate), inflating net income.
    \end{itemize}
\end{itemize}

\paragraph{Types of Goodwill:}
\begin{itemize}
  \item \textbf{Accounting Goodwill:} Arises from past acquisitions.
  \item \textbf{Economic Goodwill:} PV of expected future excess returns.
\end{itemize}

\subsubsection*{LOS 31.c: Financial Instruments}
\begin{itemize}
  \item \textbf{Definition:} Contracts that create both a financial asset (for one party) and a liability/equity instrument (for the other).
  \item \textbf{Examples (Assets):} investment securities, derivatives, loans, receivables.
  \item \textbf{Measurement Bases:}
    \begin{itemize}
      \item \textbf{Historical Cost:} e.g., unquoted equity investments, loans.
      \item \textbf{Amortized Cost:} held-to-maturity (GAAP), debt securities with intent to hold to maturity.
      \item \textbf{Fair Value:} trading securities, available-for-sale (GAAP), derivatives.
    \end{itemize}
\end{itemize}

\paragraph{U.S. GAAP Classification:}
\begin{table}[H]
\centering
\footnotesize
\begin{tabular}{|l|l|p{4cm}|}
\hline
\textbf{Category} & \textbf{Measurement} & \textbf{Income Statement Impact} \\
\hline
Held-to-Maturity & Amortized cost & Interest income only \\
\hline
Trading Securities & Fair value & Unrealized gains/losses + income \\
\hline
Available-for-Sale & Fair value & Realized gains/losses + income; Unrealized gains/losses $\to$ OCI \\
\hline
\end{tabular}
\caption{Financial Assets under U.S. GAAP}
\end{table}

\paragraph{Example: Triple D Bond (\$1M, 6\%, decline by \$20k)}
\begin{itemize}
  \item \textbf{Held-to-Maturity:} Report \$1,000,000, interest income \$60,000.
  \item \textbf{Trading:} Report \$980,000, interest \$60,000 + unrealized loss \$20,000.
  \item \textbf{Available-for-Sale:} Report \$980,000, interest \$60,000 in IS, \$20,000 unrealized loss in OCI.
\end{itemize}

\paragraph{IFRS Classification:}
\begin{itemize}
  \item Amortized Cost (hold to collect).
  \item Fair Value through OCI (collect + sell).
  \item Fair Value through P\&L (trading/default).
  \item Key Differences: Equity can be FVOCI under IFRS (choice at purchase), not under U.S. GAAP.
\end{itemize}

\subsubsection*{LOS 31.d: Non-Current Liabilities}
\begin{itemize}
  \item \textbf{Examples:} Bank loans, notes payable, bonds payable, some derivatives.
  \item \textbf{Measurement:}
    \begin{itemize}
      \item Usually reported at amortized cost:
     \[
\begin{aligned}
\text{Amortized Cost} &= \text{Issue Price} \\
&- \text{Principal Payments} \\
&+ \text{Amortized Discount} \\
&- \text{Amortized Premium}
\end{aligned}
\]
      \item Premium/discount amortized into interest expense.
      \item Liability approaches face value at maturity.
      \item Some liabilities (e.g., trading, derivatives, hedged) measured at fair value.
    \end{itemize}
  \item \textbf{Deferred Tax Liabilities (DTL):}
    \begin{itemize}
      \item Taxes payable in future due to timing differences between financial vs tax reporting.
      \item Created when:
        \begin{itemize}
          \item Tax deductions occur before expense recognition (e.g., accelerated tax depreciation).
          \item Revenues recognized before taxable (e.g., subsidiary earnings).
        \end{itemize}
      \item Eventually reverse when taxes are paid.
    \end{itemize}
\end{itemize}

\subsection*{Module 32.1: Cash Flow Introduction and Direct Method CFO}

\textbf{LOS 32.a: How the Cash Flow Statement Links to Income Statement and Balance Sheet}

\begin{itemize}
    \item The \textbf{Cash Flow Statement (CFS)} provides insights not visible in the Income Statement (IS) and Balance Sheet (BS):
    \begin{itemize}
        \item Cash receipts and cash payments during the period.
        \item Classification into: Operating (CFO), Investing (CFI), Financing (CFF).
        \item Quality of earnings: accrual vs.\ cash-backed profits.
    \end{itemize}
    
    \item \textbf{Uses of CFS by analysts:}
    \begin{itemize}
        \item Liquidity $\rightarrow$ ability to sustain business with operating cash.
        \item Solvency $\rightarrow$ ability to meet long-term obligations.
        \item Financial flexibility $\rightarrow$ ability to fund growth or meet surprises.
    \end{itemize}

    \item \textbf{Link to Financial Statements:}
    \begin{itemize}
        \item IS = performance between two BS dates (flow statement).
        \item CFS reconciles change in cash between beginning and end of BS period.
        \item Operating Activities $\leftrightarrow$ Current Assets \& Liabilities. \\
              Investing Activities $\leftrightarrow$ Noncurrent Assets. \\
              Financing Activities $\leftrightarrow$ Noncurrent Liabilities \& Equity.
    \end{itemize}
\end{itemize}

\bigskip
\textbf{Example: Accounts Receivable (AR)}

\[
\text{Ending AR} = \text{Beginning AR} + \text{Sales} - \text{Cash Collections}
\]

\[
\text{Cash Collections} = \text{Sales} - (\text{Ending AR} - \text{Beginning AR})
\]

\noindent\fbox{\begin{minipage}{\textwidth}
\textbf{Numerical Example:}\\
Beginning AR = €10,000, Ending AR = €15,000, Sales = €68,000  
\[
\text{Cash Collections} = 68{,}000 - (15{,}000 - 10{,}000) = 63{,}000
\]
\end{minipage}}

- An \textbf{increase in AR} $\rightarrow$ use of cash.  
- An \textbf{increase in Unearned Revenue} $\rightarrow$ source of cash.

\bigskip
\textbf{General Rules (Sources vs Uses of Cash):}
\begin{itemize}
    \item Increase in an Asset $\rightarrow$ Use of Cash ($-$).
    \item Decrease in an Asset $\rightarrow$ Source of Cash (+).
    \item Increase in a Liability $\rightarrow$ Source of Cash (+).
    \item Decrease in a Liability $\rightarrow$ Use of Cash ($-$).
\end{itemize}

\bigskip
\textbf{LOS 32.b: Direct vs. Indirect CFO Presentation}

\begin{itemize}
    \item CFO can be presented using:
    \begin{enumerate}
        \item \textbf{Direct Method:} Lists actual cash inflows/outflows.  
        \item \textbf{Indirect Method:} Adjusts net income for non-cash items and accruals.  
    \end{enumerate}
    \item CFI and CFF are presented the same under both methods.  
\end{itemize}

\bigskip
\textbf{Direct Method for CFO\@: Step-by-Step}

\begin{enumerate}
    \item Start with Revenue (top of IS).
    \item Adjust for changes in related BS accounts:
    \begin{itemize}
        \item Subtract increase in asset (use of cash).
        \item Add decrease in asset (source of cash).
        \item Add increase in liability (source of cash).
        \item Subtract decrease in liability (use of cash).
    \end{itemize}
    \item Treat expenses as negative values before adjustments.
    \item Ignore non-cash items (e.g., depreciation, unrealized gains/losses).
    \item Sum adjusted inflows/outflows = CFO\@.
\end{enumerate}

\bigskip
\textbf{Components of Direct Method CFO:}
\begin{itemize}
    \item Cash collected from customers.
    \item Cash paid to suppliers (COGS adjusted for Inventory \& AP).
    \item Cash operating expenses (e.g., wages, rent).
    \item Cash interest paid.
    \item Cash taxes paid.
\end{itemize}

\bigskip
\textbf{Example: Direct Method CFO Calculation}

\begin{table}[H]
\centering
\caption*{Income Statement (20X7)}
\begin{tabular}{|l|r|}
\hline
Sales Revenue & 500,000 \\
COGS & (300,000) \\
Depreciation & (20,000) \\
Operating Expenses & (100,000) \\
Interest Expense & (10,000) \\
Tax Expense & (20,000) \\
\hline
Net Income & 50,000 \\
\hline
\end{tabular}
\end{table}

\begin{table}[H]
\centering
\caption*{Balance Sheet Changes (20X6 $\rightarrow$ 20X7)}
\begin{tabular}{|l|r|}
\hline
Accounts Receivable & +15,000 \\
Inventory & +5,000 \\
Accounts Payable & +8,000 \\
Taxes Payable & +3,000 \\
\hline
\end{tabular}
\end{table}

\bigskip
\textbf{Step-by-Step CFO\@:}

\begin{itemize}
    \item Cash Collected from Customers:
    \[
    500{,}000 - 15{,}000 = 485{,}000
    \]
    \item Cash Paid to Suppliers (COGS adj.):
    \[
    300{,}000 + 5{,}000 - 8{,}000 = 297{,}000
    \]
    \item Cash Operating Expenses:
    \[
    100{,}000 \quad (\text{no adjustment assumed})
    \]
    \item Cash Interest Paid:
    \[
    10{,}000 \quad (\text{no adjustment assumed})
    \]
    \item Cash Taxes Paid:
    \[
    20{,}000 - 3{,}000 = 17{,}000
    \]
\end{itemize}

\[
\text{CFO} = 485{,}000 - 297{,}000 - 100{,}000 - 10{,}000 - 17{,}000 = 61{,}000
\]

\bigskip
\textbf{Interpretation:}
\begin{itemize}
    \item CFO = 61,000 (positive cash flow from operations).
    \item Quality of earnings is high if CFO $\geq$ Net Income (50,000).
    \item Indicates earnings are backed by real cash collections.
\end{itemize}

\subsection*{Module 32.2: Indirect Method CFO}

\subsubsection*{LOS 32.b: Prepare and interpret CFO using the indirect method}

\paragraph{Core idea}
\begin{itemize}
  \item Start from \textbf{Net Income (NI)} and reconcile to \textbf{Cash Flow from Operations (CFO)} by:
  \begin{enumerate}
    \item \textbf{Adding back} noncash charges and \textbf{removing} non-operating gains/losses that flowed through NI.
    \item \textbf{Adjusting for working capital} changes (operating current assets and operating current liabilities).
  \end{enumerate}
\end{itemize}

\paragraph{Bridge formula}
\[
\boxed{\;\; \text{CFO} \;=\; \text{NI} \;+\; \text{NCC} \;-\; \text{WCINV} \;\;}
\]
\begin{itemize}
  \item \textbf{NCC (Noncash Charges):} Items in NI with no current-period cash effect (e.g., depreciation). Gains reduce NI without cash classification in CFO; losses increase NI similarly. Under the indirect method: \textit{add back charges}, \textit{subtract gains}, \textit{add losses}.
  \item \textbf{WCINV (Working Capital Investment):} Net increase in \emph{noncash} operating current assets minus the net increase in \emph{operating} current liabilities.
\end{itemize}

\paragraph{What counts as operating for WCINV}
\begin{itemize}
  \item \textbf{Include (typical):} Accounts receivable, inventory, prepaid expenses, other operating CAs; Accounts payable, accrued expenses, taxes payable, unearned (deferred) revenue.
  \item \textbf{Exclude:} Cash and cash equivalents; short-term \emph{interest-bearing} debt and dividends payable (CFF); short-term investments (except trading securities which are CFO under U.S. GAAP).
\end{itemize}

\paragraph{Sign rules for working capital adjustments}
\begin{table}[H]
\centering
\resizebox{\textwidth}{!}{%
\footnotesize
\begin{tabular}{|l|c|p{7cm}|}
\hline
\textbf{Account change} & \textbf{CFO effect} & \textbf{Logic} \\
\hline
$\uparrow$ Operating current asset (e.g., AR, Inventory) & $-$ & Cash not yet received or tied up in inventory \\
$\downarrow$ Operating current asset & $+$ & Release of cash (collection or inventory run-down) \\
$\uparrow$ Operating current liability (e.g., AP, Accrued, Taxes Payable) & $+$ & Paying later preserves cash \\
$\downarrow$ Operating current liability & $-$ & Paying earlier uses cash \\
\hline
\end{tabular}}
\caption{Working capital adjustments under the indirect method}
\end{table}

\paragraph{Typical noncash charges, gains, and losses (NCC bucket)}
\begin{table}[H]
\centering
\footnotesize
\begin{tabular}{|l|c|}
\hline
\textbf{Income statement item} & \textbf{Indirect CFO adjustment} \\
\hline
Depreciation and amortization & Add back \\
Impairment losses, write-downs & Add back \\
Stock-based compensation expense & Add back \\
Bad-debt expense (allowance build) & Add back \\
Deferred tax expense (net) & Add back (if noncash) \\
Unrealized FX losses (noncash) & Add back \\
Gains on sale of PPE/investments (CFI item) & Subtract \\
Losses on sale of PPE/investments (CFI item) & Add back \\
\hline
\end{tabular}
\caption{Common noncash and non-operating items in the NI-to-CFO bridge}
\end{table}

\paragraph{Step-by-step algorithm (indirect method)}
\begin{enumerate}[label=\arabic*.]
  \item Begin with \textbf{Net Income}.
  \item \textbf{Add back} noncash charges and \textbf{remove} non-operating gains/losses: $+\,$Depreciation/Amortization, $+\,$Impairments, $-\,$Gains on asset sales, $+\,$Losses on asset sales, $+\,$Deferred tax expense, etc.
  \item Adjust for \textbf{working capital}:
  \begin{itemize}
    \item Subtract increases / add decreases in operating current \emph{assets}.
    \item Add increases / subtract decreases in operating current \emph{liabilities}.
  \end{itemize}
  \item Resulting total is \textbf{CFO}.
\end{enumerate}

\subsubsection*{Worked example (reconciling to the 32.1 direct-method numbers)}

\textbf{Given (from 32.1):}
\begin{itemize}
  \item Income Statement (20X7): NI = \$50{,}000; Depreciation = \$20{,}000; no gains/losses assumed.
  \item Balance sheet changes (20X6 $\rightarrow$ 20X7): AR = $+15{,}000$; Inventory = $+5{,}000$; AP = $+8{,}000$; Taxes Payable = $+3{,}000$.
\end{itemize}

\textbf{Indirect method CFO:}
\[
\begin{aligned}
\text{CFO} &= \text{NI} \;+\; \text{NCC} \;+\;(\Delta \text{AP}) \;+\;(\Delta \text{Taxes Payable}) \;-\;(\Delta \text{AR}) \;-\;(\Delta \text{Inventory}) \\
           &= 50{,}000 \;+\; 20{,}000 \;+\; 8{,}000 \;+\; 3{,}000 \;-\; 15{,}000 \;-\; 5{,}000 \\
           &= \boxed{61{,}000}
\end{aligned}
\]
This matches the \textbf{direct method CFO} computed in Module 32.1, as required.

\subsubsection*{Aggregate working capital formulation}
\[
\text{WCINV} \;=\; \Delta(\text{AR} + \text{Inventory} + \text{Prepaids} + \text{Other Op. CAs}) \;-\; \Delta(\text{AP} + \text{Accrued} + \text{Taxes Pay.} + \text{Unearned})
\]
\[
\text{CFO} \;=\; \text{NI} \;+\; \text{NCC} \;-\; \text{WCINV}
\]

\subsubsection*{Standards notes and exam tips}
\begin{itemize}
  \item Both IFRS and U.S. GAAP \emph{encourage} the direct format, but most issuers present \textbf{indirect} CFO. Under U.S. GAAP, if direct is shown, an indirect reconciliation is required in the notes.
  \item Remember classification options: under IFRS, interest and dividends received/paid may be classified as CFO or CFI/CFF (policy choice), while U.S. GAAP generally classifies interest paid/received and dividends received as \textbf{CFO}, and dividends paid as \textbf{CFF}.
  \item Quality of earnings: persistent gap where NI $>$ CFO can indicate aggressive accruals or working-capital build.
\end{itemize}

\subsection*{Module 32.3: Investing and Financing Cash Flows and IFRS/U.S. GAAP Differences}

\subsubsection*{Cash Flow From Investing (CFI) and From Financing (CFF)}

\paragraph{Definitions}
\begin{itemize}
  \item \textbf{CFI:} Cash inflows/outflows from acquiring or disposing of \emph{long-term assets} and certain investments.
  \item \textbf{CFF:} Cash inflows/outflows from transactions affecting \emph{capital structure} (debt and equity).
\end{itemize}

\paragraph{U.S. GAAP classification (typical)}
\begin{table}[H]
\centering
\footnotesize
\begin{tabular}{|l|p{5.3cm}|p{5.3cm}|}
\hline
\textbf{Bucket} & \textbf{Inflows} & \textbf{Outflows} \\
\hline
\textbf{CFI} &
Proceeds from sale of PP\&E, intangibles, long-term investments &
Purchase of PP\&E and intangibles; purchase of debt/equity investments (other than trading); loans made to others \\
\hline
\textbf{CFF} &
Proceeds from issuing debt (bonds/notes) or equity (shares); proceeds from new borrowings &
Repayment of principal on debt; share repurchases (treasury stock); cash dividends paid \\
\hline
\textbf{CFO (relevant contrasts)} &
Interest received; dividends received &
Interest paid; income taxes paid (all taxes under U.S. GAAP) \\
\hline
\end{tabular}
\caption{Cash flow classifications under U.S. GAAP (selected items)}
\end{table}

\paragraph{Professor's note}
\begin{itemize}
  \item Do not confuse \emph{dividends received} (CFO under U.S. GAAP) with \emph{dividends paid} (CFF).
\end{itemize}

\subsubsection*{Worked example: Computing CFI}
\textbf{Given:}
\begin{itemize}
  \item Footnote: PP\&E purchases during the year = \$25{,}000.
  \item Gross PP\&E: beginning = \$60{,}000; ending = \$69{,}000.
  \item Depreciation expense = \$7{,}000.
  \item Accumulated depreciation increased only \$3{,}000 (implies disposal removed accumulated depreciation).
  \item Loss on sale of PP\&E in the income statement (therefore proceeds $<$ carrying value).
  \item Land: beginning carrying value = \$40{,}000; ending = \$35{,}000; no land purchases disclosed.
\end{itemize}

\paragraph{1) Identify PP\&E disposals (gross and accumulated depreciation)}
\[
\begin{aligned}
\text{Disposed gross cost} &= \text{Beg. gross PP\&E} + \text{Additions} - \text{End. gross PP\&E} \\
&= 60{,}000 + 25{,}000 - 69{,}000 = 16{,}000
\end{aligned}
\]
Accumulated depreciation removed with the disposal:
\[
\text{AD removed} = \text{Beg. AD} + \text{Depreciation} - \text{End. AD} = 7{,}000 - 3{,}000 = 4{,}000
\]
\textit{(We only need the net change; exact beg/end AD levels are not required.)}

\paragraph{2) Carrying value and proceeds on PP\&E sale}
\[
\text{Carrying value disposed} = \text{Disposed gross cost} - \text{AD removed} = 16{,}000 - 4{,}000 = 12{,}000
\]
Given a \emph{loss} on disposal and that only \emph{proceeds} are cash:
\[
\text{Proceeds on PP\&E sale} = 10{,}000 \quad (\text{implied by loss of \$2{,}000})
\]

\paragraph{3) Land disposal}
\[
\text{Carrying value disposed (land)} = 40{,}000 - 35{,}000 = 5{,}000 \quad (\text{no depreciation on land})
\]
\[
\text{Proceeds on land sale} = 15{,}000
\]

\paragraph{4) Compute CFI}
\[
\begin{aligned}
\text{CFI} &= -\text{Cash paid for PP\&E additions} + \text{Proceeds on PP\&E sale} + \text{Proceeds on land sale} \\
&= -25{,}000 + 10{,}000 + 15{,}000 \\
&= \boxed{0}
\end{aligned}
\]
\textit{In this case, new asset purchases were exactly offset by disposal proceeds.}

\paragraph{One-line shortcut for carrying value of PP\&E disposed}
\[
\text{Carrying value disposed} = \text{Beg. net PP\&E} - \text{Depreciation} + \text{Additions} - \text{End. net PP\&E}
\]
\textit{Then use: Proceeds = Carrying value $+$ Gain (or $-$ Loss).}

\subsubsection*{Worked example: Computing CFF}
\textbf{Given:}
\begin{itemize}
  \item Bonds outstanding issued at par. Bonds payable: beginning = \$10{,}000; ending = \$15{,}000.
  \item Contributed capital (Common stock + APIC): beginning = \$40{,}000; ending = \$50{,}000.
  \item Retained earnings: beginning = \$30,500; ending = \$61,000. Net income = \$39,000.
  \item Dividends payable change as applicable (not shown below; include if provided).
\end{itemize}

\paragraph{1) Net principal cash flow from debt (issued at par)}
\[
\Delta \text{Bonds payable} = 15{,}000 - 10{,}000 = \boxed{+5{,}000 \text{ (CFF inflow)}}
\]

\paragraph{2) Net equity cash flow}
\[
\Delta \text{Contributed capital} = 50{,}000 - 40{,}000 = \boxed{+10{,}000 \text{ (CFF inflow)}}
\]
\textit{If contributed capital decreased, it would be an outflow (share repurchase).}

\paragraph{3) Cash dividends paid}
\[
\text{Dividends declared} = \text{Beg. RE} + \text{Net income} - \text{End. RE} = 30,500 + 39,000 - 61,000 = 8,500
\]
Adjust for change in dividends payable (DP):
\[
\text{Dividends paid} = \text{Dividends declared} + \text{Beg. DP} - \text{End. DP}
\]
\textit{Use provided DP figures; if none, assume declared = paid.} \\
Thus, \(\boxed{\text{Dividends paid} = 8,500 \text{ (CFF outflow)}}\).

\paragraph{4) Total CFF (sign convention: inflows positive)}
\[
\text{CFF} = (+5{,}000) + (+10{,}000) - (8{,}500) = \boxed{+6{,}500}
\]

\subsubsection*{Completing the cash flow statement}
\begin{itemize}
  \item Compute \textbf{CFO} (from 32.1/32.2), \textbf{CFI}, and \textbf{CFF}.
  \item \(\text{Net change in cash} = \text{CFO} + \text{CFI} + \text{CFF}\).
  \item Check: \(\text{End cash} - \text{Beg cash} = \text{Net change in cash}\).
\end{itemize}

\subsubsection*{Converting Indirect CFO to Direct CFO (LOS 32.c)}
\paragraph{Three-step method}
\begin{enumerate}[label=\arabic*.]
  \item \textbf{Aggregate} all revenues \& gains and all expenses \& losses.
  \item \textbf{Remove} all noncash items and \textbf{disaggregate} the remainder into natural cash categories.
  \item \textbf{Convert accruals to cash} by adjusting each category for related working-capital changes.
\end{enumerate}

\paragraph{Useful direct-method building blocks}
\[
\text{Cash collected from customers} = \text{Sales} - \Delta \text{Accounts receivable}
\]
\[
\text{Cash paid to suppliers} = \text{COGS} + \Delta \text{Inventory} - \Delta \text{Accounts payable}
\]
\[
\text{Cash operating expenses} \approx \text{SG\&A} - \Delta \text{Accrued expenses} - \Delta \text{Prepaids}
\]
\[
\text{Cash interest paid} = \text{Interest expense} - \Delta \text{Interest payable}
\]
\[
\text{Cash taxes paid} = \text{Tax expense} - \Delta \text{Taxes payable} - \Delta \text{Deferred taxes}
\]
Sum the adjusted cash inflows/outflows to get \(\text{CFO}_{\text{direct}}\) (it must equal indirect CFO).

\subsubsection*{IFRS vs U.S. GAAP classification differences (LOS 32.d)}
\begin{table}[H]
\centering
\footnotesize
\begin{tabular}{|l|c|c|}
\hline
\textbf{Item} & \textbf{U.S. GAAP} & \textbf{IFRS} \\
\hline
Interest received & CFO & CFO or CFI \\
Interest paid & CFO & CFO or CFF \\
Dividends received & CFO & CFO or CFI \\
Dividends paid & CFF & CFO or CFF \\
Income taxes paid & CFO (all) & CFO unless specifically attributable to CFI or CFF \\
\hline
\end{tabular}
\caption{Key classification differences: IFRS vs U.S. GAAP}
\end{table}

\paragraph{Illustration: tax on investing transaction}
Sell land for \$1{,}000{,}000; income tax on sale = \$160{,}000.
\begin{itemize}
  \item \textbf{U.S. GAAP:} CFI inflow \$1{,}000{,}000; CFO outflow \$160{,}000.
  \item \textbf{IFRS:} May present net CFI inflow \$840{,}000 if taxes are directly attributable to the investing transaction.
\end{itemize}

\subsubsection*{Exam tips and analyst notes}
\begin{itemize}
  \item When bonds are issued at par (Level I simplification), \(\Delta\) Bonds payable equals \emph{cash} from debt issuance/repayment.
  \item Premium/discount amortization affects interest expense and carrying value but not cash; focus on principal cash flows in CFF.
  \item For equity, \(\Delta\) (Common stock + APIC) approximates net share issuance (inflow) or repurchase (outflow); differences from issue price affect retained earnings in practice.
  \item Always reconcile totals to the change in cash as a validation step.
\end{itemize}

\subsection*{Module 33.1: Introduction to Financial Ratios}
\textbf{LOS 39.a:} Describe tools and techniques used in financial analysis, including their uses and limitations.

\subsubsection*{1. Overview of Analytical Tools}

\begin{itemize}
  \item Financial analysis converts accounting data into decision-useful information.
  \item Techniques help identify trends, relationships, and anomalies — but must be interpreted contextually.
  \item Core analytical methods:
        \begin{itemize}
          \item \textbf{Ratio Analysis}
          \item \textbf{Common-Size Analysis} (vertical and horizontal)
          \item \textbf{Graphical Analysis}
          \item \textbf{Regression Analysis}
        \end{itemize}
  \item These tools correspond to \textbf{Step 3} of the financial analysis framework:
        \[
        \text{“Adjust financial statements, compute ratios, and prepare exhibits.”}
        \]
\end{itemize}

\subsubsection*{2. Ratio Analysis}

\paragraph{Purpose and Usefulness:}
\begin{itemize}
  \item Expresses relationships among financial variables.
  \item Provides quick insight into performance, efficiency, liquidity, solvency, and profitability.
  \item Useful for:
        \begin{itemize}
          \item Projecting future earnings and cash flows.
          \item Assessing flexibility (ability to meet obligations under stress).
          \item Evaluating management performance.
          \item Comparing firm and industry trends.
          \item Benchmarking against peers and historical results.
        \end{itemize}
\end{itemize}

\paragraph{Analyst Objective:}
\begin{itemize}
  \item Ratios raise questions rather than provide final answers.
  \item Must be interpreted together — no single ratio suffices.
\end{itemize}

\begin{tabular}{|l|p{5.5cm}|p{5.5cm}|}
\hline
\textbf{Analytical Goal} & \textbf{Related Ratio Type} & \textbf{Interpretation Focus} \\
\hline
Profitability & Net margin, ROA, ROE & Earnings generation vs. resources used \\
\hline
Liquidity & Current, Quick, Cash ratios & Short-term solvency \\
\hline
Leverage & Debt-to-assets, Debt-to-equity & Long-term solvency and capital structure \\
\hline
Efficiency & Inventory turnover, Receivable turnover & Asset utilization effectiveness \\
\hline
Valuation & P/E, EV/EBITDA, P/B & Market perception and expectations \\
\hline
\end{tabular}

\paragraph{Limitations:}
\begin{itemize}
  \item Ratios are meaningless in isolation — require benchmarking.
  \item Accounting methods differ (e.g., IFRS vs. U.S. GAAP \rightarrow inconsistent comparability).
  \item Conglomerates complicate peer comparison (multi-industry operations).
  \item Ratios vary across industries — what’s “strong” in one sector may be “weak” in another.
  \item Definitions differ across analysts (e.g., “debt” may or may not include leases).
  \item Requires contextual analysis:
        \begin{itemize}
          \item Prior-period trends.
          \item Business cycle stage.
          \item Company strategy and expectations.
        \end{itemize}
\end{itemize}

\paragraph{Analytical Tip:}
\begin{itemize}
  \item Consistency of calculation method is crucial.
  \item Always specify formula and inputs (e.g., average vs. ending balances).
\end{itemize}

\subsubsection*{3. Common-Size Analysis}

\paragraph{Purpose:}
\begin{itemize}
  \item Standardizes financial statements for comparison across time or peers.
  \item Removes size effect — ideal for multi-period and cross-sectional analysis.
\end{itemize}

\paragraph{Types:}
\begin{itemize}
  \item \textbf{Vertical Common-Size Statements:}
        \begin{itemize}
          \item Express each item as a \% of a key total.
          \item \textbf{Balance Sheet:} Each item ÷ Total Assets.
                \[
                \text{Common-size BS ratio} = \frac{\text{Balance Sheet Item}}{\text{Total Assets}}
                \]
          \item \textbf{Income Statement:} Each item ÷ Sales.
                \[
                \text{Common-size IS ratio} = \frac{\text{Income Statement Item}}{\text{Sales}}
                \]
        \end{itemize}
  \item \textbf{Horizontal Common-Size Statements:}
        \begin{itemize}
          \item Express each item relative to a base-year value (set = 1.0).
          \item Useful for trend and growth analysis over multiple periods.
        \end{itemize}
\end{itemize}

\begin{tabular}{|l|p{5.5cm}|p{5.5cm}|}
\hline
\textbf{Type} & \textbf{Divisor / Base} & \textbf{Analytical Use} \\
\hline
Vertical BS & Total Assets & Capital structure, liquidity mix \\
\hline
Vertical IS & Sales (Revenue) & Margin structure, cost efficiency \\
\hline
Horizontal BS/IS & First-year values (index = 1.0) & Trend growth and volatility \\
\hline
\end{tabular}

\paragraph{Advantages:}
\begin{itemize}
  \item Facilitates identification of cost drivers and margin trends.
  \item Enables structural comparison (e.g., asset mix, leverage composition).
  \item Reveals operating leverage through relative expense movement.
\end{itemize}

\paragraph{Example Interpretation:}
\begin{itemize}
  \item Suppose net profit margin rises from 7\% \rightarrow 12\%.  
        \rightarrow Analyst investigates whether this stems from:
        \begin{itemize}
          \item Lower amortization (noncash \rightarrow temporary effect).  
          \item Lower interest expense (improved capital efficiency).  
          \item Permanent operational gains vs. one-time savings.
        \end{itemize}
  \item Common-size analysis identifies areas to \textbf{investigate further} — not final conclusions.
\end{itemize}

\paragraph{Analyst Caution:}
\begin{itemize}
  \item Presentation format can differ — some show latest year leftmost (as in CFA examples).
  \item Always check which year is the “base” for horizontal statements.
\end{itemize}

\subsubsection*{4. Graphical Analysis}

\paragraph{Purpose:}
\begin{itemize}
  \item Visualizes relationships and time trends for easier pattern recognition.
  \item Useful for presentations and quick diagnostics.
\end{itemize}

\paragraph{Common Formats:}
\begin{itemize}
  \item \textbf{Stacked Column (Bar) Graph:}  
        -- Shows composition (e.g., asset categories) over multiple years.  
        -- Reveals shifts in structure (e.g., rise in payables, fall in cash).
  \item \textbf{Line Graph:}  
        -- Tracks item trends (e.g., revenue, margins, leverage ratios).  
        -- Highlights anomalies (e.g., diverging growth between assets and sales).
\end{itemize}

\paragraph{Example Interpretation:}
\begin{itemize}
  \item Rising trade payables and declining cash \rightarrow possible liquidity issues.
  \item Rapid growth in receivables vs. flat sales \rightarrow possible aggressive revenue recognition.
\end{itemize}

\begin{tabular}{|l|p{5.5cm}|p{5.5cm}|}
\hline
\textbf{Graph Type} & \textbf{Best For} & \textbf{Analyst Insight} \\
\hline
Stacked Bar & Composition over time & Balance sheet structure shifts \\
\hline
Line Graph & Trend analysis & Growth, seasonality, or volatility patterns \\
\hline
Pie Chart & Cross-sectional composition & Segment contribution to total \\
\hline
\end{tabular}

\subsubsection*{5. Regression Analysis}

\paragraph{Purpose:}
\begin{itemize}
  \item Quantitative tool linking dependent (e.g., sales) and independent variables (e.g., GDP, advertising).
  \item Used for \textbf{forecasting and scenario analysis.}
\end{itemize}

\[
\text{Sales}_t = \alpha + \beta \times \text{GDP}_t + \varepsilon_t
\]

\paragraph{Analytical Application:}
\begin{itemize}
  \item Identify drivers of performance (macroeconomic sensitivity).  
  \item Forecast future metrics (e.g., revenues, margins, default probabilities).  
  \item Evaluate consistency of management forecasts.
\end{itemize}

\paragraph{Limitations:}
\begin{itemize}
  \item Correlation $\neq$ causation.  
  \item Historical relationships may not persist.  
  \item Sensitive to outliers and structural breaks (e.g., pandemic, crisis).
\end{itemize}

\subsubsection*{6. Comparative Overview of Analytical Tools}

\begin{tabular}{|l|p{5.5cm}|p{5.5cm}|}
\hline
\textbf{Tool} & \textbf{Primary Use} & \textbf{Limitation / Risk} \\
\hline
Ratio Analysis & Quick diagnostics, cross-firm comparison & Accounting differences, context dependency \\
\hline
Common-Size Analysis & Normalized structure, cost trend detection & No insight into absolute scale or cash flows \\
\hline
Graphical Analysis & Visualization of trends & Potential for oversimplification \\
\hline
Regression Analysis & Forecasting relationships & Model risk, multicollinearity, data sensitivity \\
\hline
\end{tabular}

\subsubsection*{7. Analytical Insights for Interpretation}

\begin{itemize}
  \item Always interpret ratios and common-size trends together:
        \[
        \text{E.g., Rising ROE but falling CFO/NI \rightarrow possible accrual-based boost.}
        \]
  \item Combine horizontal trends (growth) with vertical proportions (structure).
  \item Investigate anomalies:
        \begin{itemize}
          \item Decline in gross margin with stable sales \rightarrow cost inflation or product mix change.
          \item Decrease in total asset turnover \rightarrow inefficient asset use or overinvestment.
        \end{itemize}
  \item Verify if improvements are operational (sustainable) or accounting-based (temporary).
\end{itemize}

\subsubsection*{8. Key Takeaways}

\begin{itemize}
  \item \textbf{Ratio analysis} provides relationships but requires context.  
  \item \textbf{Common-size analysis} standardizes data for comparability.  
  \item \textbf{Graphical tools} reveal trends intuitively.  
  \item \textbf{Regression analysis} quantifies predictive relationships.  
  \item Always evaluate results within:
        \begin{itemize}
          \item Historical trend.
          \item Peer and industry context.
          \item Business cycle phase.
        \end{itemize}
  \item No single ratio or exhibit explains performance — use a holistic analytical framework.
\end{itemize}


\subsection*{Module 34.1: Inventory Measurement --- Lower of Cost \& NRV/Market}

\subsubsection*{LOS 34.a --- Rules, mechanics, examples, and ratio implications}

\paragraph{Core definitions and tests}
\begin{itemize}[leftmargin=1.2em]
  \item \textbf{IFRS (IAS 2)}: Carry inventory at the \textbf{lower of} \emph{cost} or \emph{net realisable value (NRV)}.
  \[
    \boxed{\text{NRV} = \text{Expected selling price} - \text{Costs to complete} - \text{Selling costs}}
  \]
  \item \textbf{U.S. GAAP}:
    \begin{itemize}
      \item \underline{All methods except LIFO/retail}: \textbf{lower of cost or NRV} (same as IFRS).
      \item \underline{LIFO or retail method}: \textbf{lower of cost or market (LCM)} where
      \[
      \text{Ceiling}=\text{NRV},\quad \text{Floor}=\text{NRV}-\text{Normal profit margin},\quad
      \text{Market}=\min\{\text{Ceiling},\,\max(\text{Replacement cost},\,\text{Floor})\}.
      \]
    \end{itemize}
  \item \textbf{Recognition}: If test value $<$ cost $\Rightarrow$ \emph{write-down} (IS: separate loss line or added to COGS). Test value becomes new carrying amount.
  \item \textbf{Reversals}:
    \begin{itemize}
      \item \textbf{IFRS}: \emph{Write-up} permitted later, \underline{limited to original cost}.
      \item \textbf{U.S. GAAP}: \emph{No write-up} permitted (even if NRV rises).
    \end{itemize}
  \item \textbf{Mechanics}: Use a \emph{valuation allowance} (contra-inventory) to preserve historical cost ledger and show reduced carrying value.
\end{itemize}

\paragraph{Computation flow (exam ready)}
\begin{enumerate}[leftmargin=1.2em]
  \item Compute \(\text{NRV}\) and (if GAAP---LIFO/retail) \(\text{Ceiling/Floor/RC}\).
  \item Determine test value:
    \[
    \text{IFRS or GAAP(non-LIFO)}:\ \min(\text{Cost},\text{NRV})
    \qquad
    \text{GAAP(LIFO/retail)}:\ \min(\text{Cost},\text{Market}).
    \]
  \item \(\text{Loss}=\text{New carrying value}-\text{Cost}\) (negative if write-down).
\end{enumerate}

\paragraph{Worked example (per unit)}
\begin{table}[H]
\centering
\footnotesize
\begin{tabular}{|l|c|}
\hline
\textbf{Original cost} & \$210 \\
\hline
\textbf{Expected selling price} & \$225 \\
\hline
\textbf{Costs to sell/complete} & \$22 \\
\hline
\textbf{NRV $=225-22$} & \$203 \\
\hline
\textbf{Replacement cost (RC)} & \$197 \\
\hline
\textbf{Normal profit margin} & \$12 \\
\hline
\end{tabular}
\caption{Inputs for lower-of tests}
\end{table}

\noindent\textbf{IFRS / GAAP (non-LIFO/retail):}
\[
\text{Carrying value}=\min(210,203)=\boxed{\$203},\quad \text{IS loss}=203-210=\boxed{-\$7}.
\]

\noindent\textbf{GAAP (LIFO/retail; LCM):}
\[
\text{Ceiling}=203,\ \text{Floor}=203-12=191,\ \text{RC}=197\in[191,203]\Rightarrow \text{Market}=197.
\]
\[
\text{Carrying value}=\min(210,197)=\boxed{\$197},\quad \text{IS loss}=197-210=\boxed{-\$13}.
\]

\paragraph{Recovery scenario (next year: NRV $\uparrow$ \$10 to 213; RC $\uparrow$ \$10 to 207)}
\begin{itemize}[leftmargin=1.2em]
  \item \textbf{IFRS}: Write up to \(\min(\text{NRV}, \text{original cost})=\min(213,210)=\boxed{210}\) \(\Rightarrow\) \(\boxed{+\$7}\) gain (limited to prior write-down).
  \item \textbf{U.S. GAAP}: \emph{No reversal} permitted (even if NRV rises).
\end{itemize}

\paragraph{Special fair-value/NRV exception (both frameworks)}
\begin{itemize}[leftmargin=1.2em]
  \item For certain commodities (agriculture, forest products, mineral ores, precious metals) held by producers/dealers: inventory may be carried at \textbf{NRV/fair value}; \emph{unrealized} gains/losses flow through earnings if an active market/observable prices exist.
\end{itemize}

\subsubsection*{Financial statement and ratio implications}

\paragraph{Immediate period (write-down runs through COGS or separate line)}
\begin{itemize}[leftmargin=1.2em]
  \item \textbf{Balance sheet:} \(\downarrow\) Inventory (CA) $\Rightarrow$ \(\downarrow\) Current assets \& Total assets; \(\downarrow\) Equity via NI.
  \item \textbf{Liquidity:} \(\downarrow\) Current ratio (CA/CL); \textbf{Quick ratio} unchanged (inventory excluded).
  \item \textbf{Activity:} \(\uparrow\) Inventory turnover \(\left(\frac{\text{COGS}}{\text{Avg Inv}}\right)\) \(\Rightarrow\) \(\downarrow\) Days’ inventory on hand, \(\downarrow\) Cash conversion cycle; \(\uparrow\) Total asset turnover.
  \item \textbf{Leverage:} \(\uparrow\) Debt-to-assets and \(\uparrow\) Debt-to-equity (assets/equity \(\downarrow\)).
  \item \textbf{Profitability:} \(\uparrow\) COGS $\Rightarrow$ \(\downarrow\) Gross/Operating/Net margins; NI typically falls by a \emph{larger \%} than assets or equity $\Rightarrow$ \(\downarrow\) ROA, \(\downarrow\) ROE.
\end{itemize}

\paragraph{Subsequent periods (post write-down base is lower)}
\begin{itemize}[leftmargin=1.2em]
  \item Lower carrying cost flowing into COGS may \(\downarrow\) COGS $\Rightarrow$ \(\uparrow\) margins/NI mechanically.
  \item With reduced asset/equity bases, \(\uparrow\) ROA and ROE can result \emph{even without operational improvement} $\Rightarrow$ \textbf{comparability caution}.
\end{itemize}

\subsubsection*{Comparative summary (IFRS vs U.S. GAAP)}
\begin{table}[H]
\centering
\footnotesize
\begin{tabular}{|l|p{5.5cm}|p{6.1cm}|}
\hline
\textbf{Topic} & \textbf{IFRS} & \textbf{U.S.\@ GAAP} \\
\hline
Measurement basis & Lower of \emph{cost or NRV} (all methods) & Lower of \emph{cost or NRV} (non-LIFO/retail); \emph{LCM} (LIFO/retail) with Ceiling/Floor \\
\hline
Write-up on recovery & Allowed up to original cost (gain in IS) & Prohibited (no reversal) \\
\hline
Valuation mechanism & Valuation allowance (contra-asset) & Valuation allowance (contra-asset) \\
\hline
Commodity exception & NRV/fair value permitted; unrealized P\&L & Same concept for qualifying producers/dealers \\
\hline
LIFO effect on impairments & N/A (IFRS prohibits LIFO) & LIFO less likely to impair in inflation (older, lower costs) \\
\hline
\end{tabular}
\caption{IFRS vs U.S. GAAP — inventory measurement at lower-of tests}
\end{table}

\subsubsection*{Analyst checklist}
\begin{itemize}[leftmargin=1.2em]
  \item Identify framework and method (IFRS vs GAAP; LIFO/retail or not) \(\Rightarrow\) choose \(\text{NRV}\) vs \(\text{LCM}\).
  \item For LCM: compute \(\text{Ceiling}=\text{NRV}\), \(\text{Floor}=\text{NRV}-\text{normal profit}\), clamp \(\text{RC}\) to range.
  \item Track valuation allowance changes (footnotes) to quantify recurring vs one-off charges.
  \item Adjust time-series ratios for the mechanical post-impairment boost to margins/ROA/ROE.
  \item Remember: IFRS allows reversals (limit = prior write-down); U.S. GAAP does not.
\end{itemize}

\subsubsection*{Compact formulas}

% Part 1 — IFRS & GAAP (non-LIFO)
\[
\textbf{IFRS \& GAAP (non-LIFO)}:\quad
\boxed{\mathrm{CV}=\min\{\mathrm{Cost},\,\mathrm{NRV}\}}
\]

% Part 2a — U.S. GAAP (LIFO/retail): Market clamp
\[
\textbf{U.S. GAAP (LIFO/retail)}:\quad
\mathrm{Market}=
\begin{cases}
\mathrm{NRV}, & \mathrm{RC}>\mathrm{NRV},\\[2pt]
\mathrm{RC}, & \mathrm{NRV}-\mathrm{NP}\le \mathrm{RC}\le \mathrm{NRV},\\[2pt]
\mathrm{NRV}-\mathrm{NP}, & \mathrm{RC}<\mathrm{NRV}-\mathrm{NP}.
\end{cases}
\]

% Part 2b — U.S. GAAP (LIFO/retail): Carrying value
\[
\boxed{\mathrm{CV}=\min\{\mathrm{Cost},\,\mathrm{Market}\}}
\]


\subsection*{Module 34.2: Inflation Impact on FIFO and LIFO}
\textbf{LOS 34.b:} Calculate and explain how inflation and deflation of inventory costs affect the financial statements and ratios of companies that use different inventory valuation methods.

\subsubsection*{1. Overview and Core Logic}

\begin{itemize}
  \item \textbf{Assumption:} Stable or increasing inventory quantities unless otherwise stated.
  \item \textbf{Key idea:} Cost flow assumptions (FIFO vs LIFO) determine which purchase costs are assigned to COGS and to ending inventory, but \emph{not} which physical units are sold.
  \item \textbf{Inflationary Periods ($\uparrow$ prices):}
    \begin{itemize}
      \item LIFO COGS $>$ FIFO COGS (because latest, higher-cost items are assumed sold first)
      \item LIFO Ending Inventory $<$ FIFO Ending Inventory (older, lower-cost items remain)
      \item $\Rightarrow$ LIFO $\rightarrow$ lower gross profit and lower net income
    \end{itemize}
  \item \textbf{Deflationary Periods ($\downarrow$ prices):} Reverse effects of inflation.
\end{itemize}

\subsubsection*{2. Comparative Table: Inflation vs Deflation Effects}

\begin{center}
\begin{tabular}{|l|c|c|}
\hline
\textbf{Item} & \textbf{Inflation (Rising Prices)} & \textbf{Deflation (Falling Prices)} \\
\hline
COGS (LIFO vs FIFO) & LIFO $>$ FIFO & LIFO $<$ FIFO \\
\hline
Ending Inventory & LIFO $<$ FIFO & LIFO $>$ FIFO \\
\hline
Gross Profit / Net Income & LIFO $<$ FIFO & LIFO $>$ FIFO \\
\hline
Tax Expense & LIFO $<$ FIFO & LIFO $>$ FIFO \\
\hline
Cash Flow (from lower taxes) & LIFO $>$ FIFO & LIFO $<$ FIFO \\
\hline
Inventory Turnover & LIFO $>$ FIFO & LIFO $<$ FIFO \\
\hline
Current Ratio & LIFO $<$ FIFO & LIFO $>$ FIFO \\
\hline
Debt / Equity Ratio & LIFO $>$ FIFO & LIFO $<$ FIFO \\
\hline
\end{tabular}
\end{center}

\subsubsection*{3. Conceptual Visualization}

\textbf{Figure 34.1: LIFO vs FIFO under Rising Prices}
\[
\text{FIFO: COGS uses older (low) prices; Inventory uses newer (high) prices.}
\]
\[
\text{LIFO: COGS uses newer (high) prices; Inventory uses older (low) prices.}
\]

\subsubsection*{4. Economic Interpretation}

\begin{itemize}
  \item \textbf{FIFO inventory} reflects \textit{current replacement cost}, a closer proxy to \textit{economic value}.
  \item \textbf{LIFO COGS} reflects \textit{current cost of sales}, a better measure of current period profitability.
  \item \textbf{Average Cost Method:} Produces values between FIFO and LIFO for both COGS and inventory.
\end{itemize}

\subsubsection*{5. Ratio Analysis Effects (Assuming Rising Prices)}

\begin{itemize}
  \item \textbf{Profitability Ratios}
    \begin{itemize}
      \item LIFO $\rightarrow$ Higher COGS $\rightarrow$ Lower gross, operating, and net margins.
      \item FIFO $\rightarrow$ Lower COGS $\rightarrow$ Higher profitability.
    \end{itemize}
  \item \textbf{Liquidity Ratios}
    \begin{itemize}
      \item FIFO inventory higher $\Rightarrow$ Higher current ratio and working capital.
      \item LIFO inventory lower $\Rightarrow$ Lower liquidity.
    \end{itemize}
  \item \textbf{Activity Ratios}
    \begin{itemize}
      \item Inventory Turnover = $\dfrac{\text{COGS}}{\text{Average Inventory}}$
      \item LIFO: numerator $\uparrow$, denominator $\downarrow$ $\Rightarrow$ higher turnover.
      \item FIFO: opposite effect $\Rightarrow$ lower turnover, higher days inventory on hand.
    \end{itemize}
  \item \textbf{Solvency Ratios}
    \begin{itemize}
      \item FIFO $\Rightarrow$ Higher total assets and equity $\Rightarrow$ Lower debt ratio and debt-to-equity ratio.
      \item LIFO $\Rightarrow$ Lower total assets $\Rightarrow$ Higher leverage measures.
    \end{itemize}
\end{itemize}

\subsubsection*{6. LIFO Reserve and Adjustments}

\[
\text{LIFO Reserve} = \text{FIFO Inventory Value} - \text{LIFO Inventory Value}
\]
\begin{itemize}
  \item Used to reconcile LIFO to FIFO financial statements.
  \item Analysts can adjust reported LIFO financials to FIFO basis:
    \[
    \text{FIFO Inventory} = \text{LIFO Inventory} + \text{LIFO Reserve}
    \]
    \[
    \text{FIFO COGS} = \text{LIFO COGS} - \Delta(\text{LIFO Reserve})
    \]
\end{itemize}

\subsubsection*{7. LIFO Liquidation}

\begin{itemize}
  \item Occurs when inventory quantities \textbf{decline} under LIFO.
  \item Older, cheaper layers of inventory are recognized in COGS $\Rightarrow$ artificially \textbf{lower COGS}, \textbf{higher income}.
  \item \textbf{Effect:} Non-sustainable increase in profit margins (temporary).
  \item May occur intentionally (earnings management) or due to external events (strike, shortage, demand drop).
  \item \textbf{Analyst Action:} Inspect LIFO reserve disclosure:
    \begin{itemize}
      \item Decrease in LIFO reserve $\Rightarrow$ possible LIFO liquidation.
    \end{itemize}
\end{itemize}

\subsubsection*{8. Example: Willock Corporation}

\textbf{Assumptions:}
\begin{itemize}
  \item Purchase prices and sales prices inflating at $5\%$ per year.
  \item Purchases made at start of each year.
\end{itemize}

\begin{table}[H]
\centering
\begin{tabular}{|l|p{4cm}|p{4cm}|p{4cm}|}
\hline
\textbf{Year} & \textbf{Event} & \textbf{Effect (FIFO)} & \textbf{Effect (LIFO)} \\
\hline
1 & No beginning inventory; constant price & Same GP, same margin & Same GP, same margin \\
\hline
2 & Rising prices \& inventory quantities & Higher ending inventory (at \$84) $\Rightarrow$ higher GP & Lower ending inventory (mix of \$80--\$84) $\Rightarrow$ lower GP \\
\hline
3 & Prices still rising but inventory \textbf{declines} (LIFO liquidation) & Normal higher COGS (new purchases) & Older cheap inventory released $\Rightarrow$ artificially higher GP \\
\hline
\end{tabular}
\end{table}

\textbf{Interpretation:}
\begin{itemize}
  \item Year 2: Normal inflation pattern $\Rightarrow$ FIFO reports higher gross margin.
  \item Year 3: LIFO liquidation $\Rightarrow$ lower COGS, higher reported profit, despite inflation.
  \item The increase in margins in Year 3 is temporary and unsustainable.
\end{itemize}

\subsubsection*{9. Summary Table: FIFO vs LIFO Comparison (Rising Prices)}

\begin{center}
\begin{tabular}{|l|c|c|}
\hline
\textbf{Metric} & \textbf{FIFO} & \textbf{LIFO} \\
\hline
COGS & Lower & Higher \\
\hline
Ending Inventory & Higher & Lower \\
\hline
Gross Profit / Net Income & Higher & Lower \\
\hline
Taxes & Higher & Lower \\
\hline
Cash Flow (after tax) & Lower & Higher \\
\hline
Inventory Turnover & Lower & Higher \\
\hline
Current Ratio & Higher & Lower \\
\hline
Debt Ratio & Lower & Higher \\
\hline
Book Value of Equity & Higher & Lower \\
\hline
\end{tabular}
\end{center}

\subsubsection*{10. Key Takeaways}

\begin{itemize}
  \item During \textbf{inflation}, LIFO $\rightarrow$ higher COGS, lower income, lower taxes, higher cash flows.
  \item During \textbf{deflation}, FIFO $\rightarrow$ higher COGS, lower income.
  \item \textbf{Average Cost} method results fall between FIFO and LIFO.
  \item \textbf{LIFO Liquidation} inflates profits temporarily; analysts must adjust for it.
  \item FIFO inventory approximates current value; LIFO COGS approximates current cost.
\end{itemize}


\subsection*{Module 34.3: Presentation and Disclosure of Inventories}
\textbf{LOS 34.c:} Describe the presentation and disclosures relating to inventories and explain issues that analysts should consider when examining a company’s inventory disclosures and other sources of information.

\subsubsection*{1. Required Inventory Disclosures (U.S. GAAP \& IFRS)}

\begin{itemize}
  \item Cost flow method used (e.g., LIFO, FIFO, weighted‐average).
  \item Total carrying value of inventory.
  \item Breakdown by classification (raw materials, work in progress, finished goods), if relevant.
  \item Carrying value of any inventories reported at fair value less costs to sell.
  \item Cost of inventory recognized as expense (COGS) during the period.
  \item Amount of inventory write-downs (valuation allowances).
  \item Reversals of write-downs and related circumstances (IFRS only; U.S. GAAP prohibits reversals).
  \item Carrying value of inventories pledged as collateral for borrowings.
\end{itemize}

\subsubsection*{2. Analytical Uses of Inventory Disclosures}

\begin{itemize}
  \item Evaluate inventory management quality and efficiency.
  \item Adjust financial statements for comparability (e.g., converting LIFO to FIFO).
  \item Infer \textbf{demand expectations}:
    \begin{itemize}
      \item $\uparrow$ Raw materials / WIP $\Rightarrow$ Firm expects higher future demand.
      \item $\uparrow$ Finished goods, but $\downarrow$ WIP / Raw materials $\Rightarrow$ Potential \textbf{falling demand} or risk of \textbf{write-downs}.
    \end{itemize}
  \item Detect potential obsolescence:
    \begin{itemize}
      \item Finished goods increasing faster than sales $\Rightarrow$ declining demand, possible write-offs.
      \item Excess inventory also increases costs (storage, insurance, taxes) and ties up cash.
      \item All categories declined proportionally; no sign of finished-goods buildup.
      \item Hence, not indicative of lower demand or obsolescence.
    \end{itemize}
\end{itemize}

\paragraph{(c) Valuation Allowance (Obsolescence Reserve)}

\begin{center}
\begin{tabular}{|c|c|c|}
\hline
\textbf{Year} & \textbf{Allowance / Cost of FG} & \textbf{Ratio (\%)} \\
\hline
20X6 & 76,000 / 706,000 & 10.7\% \\
\hline
20X7 & 17,000 / 221,000 & 7.7\% \\
\hline
\end{tabular}
\end{center}

\textbf{Interpretation:}
\begin{itemize}
  \item Allowance decreased (10.7\% \rightarrow 7.7\%), indicating \textbf{no increase in expected obsolescence}.
\end{itemize}

\paragraph{(d) Sales Growth and Gross Profit Margin}

\[
\text{Sales Growth} = \frac{S_t}{S_{t-1}} - 1
\qquad
\text{Gross Margin} = \frac{\text{Gross Profit}}{\text{Sales}}
\]

\begin{center}
\renewcommand{\arraystretch}{1.3}
\setlength{\tabcolsep}{8pt}
\begin{tabular}{|c|c|c|c|}
\hline
\textbf{Year} & \textbf{Sales (\$)} & \textbf{Growth (\%)} & \textbf{Gross Margin (\%)} \\
\hline
20X6 & 5,500,000 & +3.8 & 52.7 \\
\hline
20X7 & 7,500,000 & +36.4 & 45.3 \\
\hline
\end{tabular}
\end{center}

\textbf{Interpretation:}
\begin{itemize}
  \item Strong sales growth (+36\%) with falling margin (52.7 \rightarrow 45.3\%) $\Rightarrow$ higher production or input costs.
  \item Declining inventory + rising costs $\Rightarrow$ possible \textbf{supply-side constraints}, not demand weakness.
\end{itemize}

\paragraph{(e) Liquidity Ratios}

\[
\text{Current Ratio} = \frac{\text{Current Assets}}{\text{Current Liabilities}}, \qquad
\text{Quick Ratio} = \frac{\text{Cash + Market Securities + Receivables}}{\text{Current Liabilities}}
\]

\begin{center}
\renewcommand{\arraystretch}{1.3}
\setlength{\tabcolsep}{8pt}
\begin{tabular}{|c|c|c|c|c|}
\hline
\textbf{Year} & \textbf{Current Assets (\$)} & \textbf{Current Liabilities (\$)} & \textbf{Current Ratio} & \textbf{Quick Ratio} \\
\hline
20X6 & 3,670,000 & 866,000 & 4.14 & 3.13 \\
\hline
20X7 & 5,995,000 & 1,505,000 & 3.98 & 3.78 \\
\hline
\end{tabular}
\end{center}

\textbf{Interpretation:}
\begin{itemize}
  \item \textbf{Current ratio} slightly decreased (inventory dropped from current assets).  
  \item \textbf{Quick ratio} improved \rightarrow liquidity strengthened when excluding inventory.
  \item Inventory is illiquid; combining days inventory on hand + DSO approximates cash conversion time.
\end{itemize}

\subsubsection*{5. Analyst Interpretation and Sources of Further Evidence}

\begin{itemize}
  \item Declining inventories amid rising sales \rightarrow not due to demand weakness, but possibly due to \textbf{cost increases} or \textbf{supply bottlenecks}.
  \item Gross margin compression confirms higher input costs.
  \item Analysts should verify via:
    \begin{itemize}
      \item Management Discussion \& Analysis (MD\&A).
      \item Footnote details on inventory valuation.
      \item Industry reports and competitor disclosures.
      \item Media or conference commentary.
    \end{itemize}
\end{itemize}

\subsubsection*{6. Key Takeaways}

\begin{itemize}
  \item Inventory disclosures reveal both \textbf{valuation policy} and \textbf{business trends}.
  \item Analysts must interpret changes in composition, turnover, and allowances carefully:
    \begin{itemize}
      \item Rising inventory \rightarrow potential overproduction or falling demand.
      \item Falling inventory \rightarrow possible supply constraints or strong sales.
    \end{itemize}
  \item Ratio trends must be cross-checked with sales growth and gross margins.
  \item Always combine quantitative ratios with qualitative sources (MD\&A, industry context) for accurate insight.
\end{itemize}


\subsection*{Module 35.1: Intangible Long-Lived Assets}
\subsubsection*{1. Overview and Key Concepts}

\begin{itemize}
  \item \textbf{Definition:} Intangible assets are identifiable non-monetary assets without physical substance that provide future economic benefits (e.g., patents, brands, copyrights, franchises, software, goodwill).
  \item \textbf{Classification by Useful Life:}
    \begin{itemize}
      \item \textbf{Finite-lived} $\Rightarrow$ amortized over useful life.
    \begin{itemize}
      \item \textbf{Finite-lived} $\Rightarrow$ amortized over useful life.
      \item \textbf{Indefinite-lived} $\Rightarrow$ not amortized; tested annually for impairment.
    \end{itemize}
  \item \textbf{Identifiability under IFRS:} An intangible asset is identifiable if it is:
    \begin{enumerate}
      \item Separately transferable or arises from legal/contractual rights.
      \item Controlled by the firm (ability to obtain future benefits).
      \item Expected to generate probable future economic benefits.
      \item Has a cost that can be measured reliably.
    \end{enumerate}
  \item \textbf{Unidentifiable assets:} Cannot be separated from the business (e.g., goodwill).
\end{itemize}

\subsubsection*{2. Classification by Source of Creation}

\begin{center}
\renewcommand{\arraystretch}{1.3}
\setlength{\tabcolsep}{6pt}
\begin{tabular}{|l|p{5.5cm}|p{4cm}|}
\hline
\textbf{Type} & \textbf{Description and Accounting Treatment} & \textbf{Examples} \\
\hline
\textbf{Internally Developed} &
Most costs expensed as incurred. Under IFRS, \textbf{development costs} (after research stage) may be capitalized if specific criteria are met. Under U.S.\@ GAAP, \textbf{R\&D costs} are expensed, except certain \textbf{software development} costs. &
R\&D, internally built software, internally generated brand or customer lists. \\
\hline
\textbf{Purchased Separately} &
Recorded on balance sheet at cost (usually fair value at purchase). If purchased as part of a group, allocate total price by fair value of individual assets. &
Purchased patents, licenses, franchises, or trademarks. \\
\hline
\textbf{Acquired in Business Combination} &
Acquisition method: allocate purchase price to identifiable tangible and intangible assets at fair value. Unidentifiable residual = \textbf{Goodwill}. Only goodwill created in acquisitions is capitalized. &
Acquired brands, customer relationships, patents, goodwill. \\
\hline
\end{tabular}
\end{center}

\subsubsection*{3. Internally Developed Intangibles}

\textbf{Under IFRS:}
\begin{itemize}
  \item \textbf{Research phase:} Expensed as incurred.
  \item \textbf{Development phase:} Capitalized if:
    \begin{itemize}
      \item Technically feasible to complete.
      \item Firm intends and can use or sell the asset.
      \item Future economic benefits are probable.
      \item Adequate resources are available to complete the project.
      \item Costs can be reliably measured.
    \end{itemize}
  \item Example: development of a new drug after discovery stage.
\end{itemize}

\textbf{Under U.S. GAAP:}
\begin{itemize}
  \item \textbf{Research and Development:} Expensed as incurred.
  \item \textbf{Software development:}
    \begin{itemize}
      \item For sale to others:  
        \begin{itemize}
          \item Expensed until \textbf{technological feasibility} is proven.  
          \item Capitalized thereafter.
        \end{itemize}
      \item For internal use:  
        \begin{itemize}
          \item Expensed until \textbf{probable completion} and intended use; then capitalized.
        \end{itemize}
    \end{itemize}
\end{itemize}

\subsubsection*{4. Purchased Intangibles}

\begin{itemize}
  \item Initially recorded at \textbf{cost (fair value)}.
  \item Cost allocation for grouped purchases based on fair value of each asset.
  \item Capitalization impacts:
    \begin{itemize}
      \item Higher initial net income (vs.\ expensing) due to deferral of costs.
      \item Lower future net income due to amortization.
      \item Higher assets, equity, and operating cash flow in early periods.
    \end{itemize}
  \item Analytical implication:
    \begin{itemize}
      \item Firms with internally generated intangibles show lower total assets and equity than firms that purchase them.
      \item Thus, ratios (e.g., ROA, ROE) may not be directly comparable.
    \end{itemize}
\end{itemize}

\subsubsection*{5. Intangibles Acquired in a Business Combination}

\begin{itemize}
  \item Accounted for using the \textbf{Acquisition Method}.
  \item Steps:
    \begin{enumerate}
      \item Determine purchase price (consideration paid).
      \item Identify and value all \textbf{identifiable} tangible and intangible assets at fair value.
      \item Record liabilities assumed at fair value.
      \item Recognize \textbf{Goodwill} as:
      \[
      \text{Goodwill} = \text{Purchase Price} - (\text{Fair Value of Identifiable Net Assets})
      \]
    \end{enumerate}
  \item \textbf{Goodwill:}
    \begin{itemize}
      \item Represents synergies, reputation, or assembled workforce value.
      \item Not amortized; tested annually for impairment.
      \item Only goodwill from business combinations is capitalized; internally generated goodwill is expensed.
    \end{itemize}
\end{itemize}

\begin{center}
\renewcommand{\arraystretch}{1.2}
\begin{tabular}{|l|l|}
\hline
\textbf{Concept} & \textbf{Treatment under Acquisition Method} \\
\hline
Purchase price allocation & To identifiable assets \& liabilities at fair value \\
\hline
Previously unrecorded intangibles & Capitalized if identifiable \\
\hline
Goodwill & Residual = Purchase price $-$ Fair value of identifiable net assets \\
\hline
Internally generated goodwill & \textbf{Not recognized} (expensed) \\
\hline
\end{tabular}
\end{center}

\subsubsection*{6. Financial Statement Effects of Capitalization vs. Expensing}

\begin{center}
\renewcommand{\arraystretch}{1.3}
\setlength{\tabcolsep}{6pt}
\begin{tabular}{|l|p{4.5cm}|p{4.5cm}|}
\hline
\textbf{Item} & \textbf{Capitalizing (Purchased or Development Phase)} & \textbf{Expensing (Research or Internal)} \\
\hline
Income (initial periods) & Higher (cost deferred) & Lower (cost expensed immediately) \\
\hline
Income (later periods) & Lower (due to amortization) & Higher (no future amortization) \\
\hline
Assets / Equity & Higher (asset recognized) & Lower (no asset recorded) \\
\hline
Operating Cash Flow & Higher (capitalized costs not deducted in CFO) & Lower (expenses reduce CFO) \\
\hline
Total Cash Flow & Unchanged (classification difference only) & Unchanged \\
\hline
Analyst adjustment & Improves comparability across firms with different capitalization policies. &  --- \\
\hline
\end{tabular}
\end{center}

\subsubsection*{7. Example: Goodwill Calculation (Business Combination)}

\textbf{Example:}
\[
\text{Purchase Price} = \$1{,}000{,}000, \quad
\text{Fair Value of Identifiable Net Assets} = \$850{,}000
\]
\[
\Rightarrow \text{Goodwill} = \$1{,}000{,}000 - \$850{,}000 = \boxed{\$150{,}000}
\]

\begin{itemize}
  \item This \$150,000 goodwill represents the premium paid for expected synergies, customer relationships, or brand reputation.
  \item Appears as an asset on the acquirer’s balance sheet.
  \item Subject to annual impairment testing.
\end{itemize}

\subsubsection*{8. Summary Table: Comparison Across Intangible Asset Types}

\begin{center}
\renewcommand{\arraystretch}{1.25}
\setlength{\tabcolsep}{4pt}
\begin{tabular}{|l|p{3.2cm}|p{3.2cm}|p{3.2cm}|}
\hline
\textbf{Aspect} & \textbf{Internally Developed} & \textbf{Purchased Separately} & \textbf{Acquired in Business Combination} \\
\hline
Initial Recognition & Mostly expensed (except IFRS development phase) & Capitalized at cost (fair value) & Capitalized at fair value \\
\hline
Amortization & If finite-lived & If finite-lived & If finite-lived \\
\hline
Impairment Test & If indefinite-lived or indicators exist & Same & Same \\
\hline
Goodwill Recognition & Not recognized & N/A & Residual value from acquisition \\
\hline
Impact on Ratios & Low assets, high turnover ratios & Higher assets \& equity & May include goodwill distortion \\
\hline
Reporting Difference & Lower net income in early years & Higher initial income, later amortization & Requires goodwill impairment testing \\
\hline
\end{tabular}
\end{center}

\subsubsection*{9. Key Takeaways}

\begin{itemize}
  \item \textbf{Finite-lived} intangibles: amortized; \textbf{indefinite-lived}: impairment-tested only.
  \item \textbf{IFRS vs. U.S.\@ GAAP:}
    \begin{itemize}
      \item IFRS allows capitalization of \textbf{development} costs; GAAP generally does not.
      \item Both require \textbf{research} costs to be expensed.
    \end{itemize}
  \item Purchased intangibles and goodwill recognized at \textbf{fair value}.
  \item Internally generated goodwill is never capitalized.
  \item Analysts must adjust for capitalization policy differences when comparing firms.
\end{itemize}

\subsection*{Module 35.2: Impairment and Derecognition}
\textbf{LOS 35.b:} Explain and evaluate how impairment and derecognition of property, plant, and equipment and intangible assets affect the financial statements and ratios.

\subsubsection*{1. Key Concepts}

\begin{itemize}
  \item \textbf{Depreciation / Amortization:} Allocation of an asset’s cost over its useful life (expected decline).
  \item \textbf{Impairment:} Unanticipated decline in an asset’s recoverable value below carrying amount.
  \item \textbf{Derecognition:} Removal of a long-lived asset from the balance sheet upon sale, exchange, or abandonment.
\end{itemize}

\subsubsection*{2. Impairments Overview}

\begin{itemize}
  \item Both \textbf{IFRS} and \textbf{U.S. GAAP} require recognition of impairment losses through the income statement.
  \item If material, impairment losses are presented as \textbf{unusual or infrequent items}.
  \item Applies to both tangible (PP\&E) and intangible assets (finite-lived).
\end{itemize}

\subsubsection*{3. Impairment Testing Under IFRS}

\begin{itemize}
  \item \textbf{Assessment:} At least annually, determine whether indicators of impairment exist (e.g., decline in market value, damage, obsolescence).
  \item \textbf{Impairment condition:}
  \[
  \text{Carrying Value} > \text{Recoverable Amount}
  \]
  \item \textbf{Recoverable Amount:}
  \[
  \text{Recoverable Amount} = \max(\text{Fair Value $-$ Selling Costs}, \text{Value in Use})
  \]
  \item \textbf{Value in Use:} Present value (PV) of future cash flows from continued use and disposal.
  \item \textbf{Loss Recognition:}
  \[
  \text{Impairment Loss} = \text{Carrying Value} - \text{Recoverable Amount}
  \]
  \item \textbf{Reversals:} Allowed (up to original impairment loss), except for goodwill.
\end{itemize}

\subsubsection*{4. Impairment Testing Under U.S. GAAP}

\begin{itemize}
  \item \textbf{Trigger:} Only when indicators suggest potential non-recoverability.
  \item \textbf{Two-Step Process:}
    \begin{enumerate}
      \item \textbf{Recoverability Test (Step 1):}  
      \[
      \text{If } \text{Carrying Value} > \text{Undiscounted Future Cash Flows} \Rightarrow \text{Impaired}
      \]
      \item \textbf{Measurement (Step 2):}  
      \[
      \text{Impairment Loss} = \text{Carrying Value} - \text{Fair Value}
      \]
      (Fair value $\approx$ discounted expected cash flows if market value not known.)
    \end{enumerate}
  \item \textbf{Reversals:} \textit{Not permitted} (except for held-for-sale assets).
\end{itemize}

\subsubsection*{5. Example: Brownfield Equipment Impairment}

\textbf{Data:}
\[
\begin{aligned}
\text{Original Cost} &= \$900{,}000 \\
\text{Accumulated Depreciation} &= \$100{,}000 \\
\Rightarrow \text{Carrying Value} &= \$800{,}000
\end{aligned}
\]

\textbf{Under IFRS:}
\[
\text{Recoverable Amount} = \max(785{,}000 \text{ (Value in Use)}, 760{,}000 \text{ (Fair Value - Costs)})
\]
\[
\text{Impairment Loss} = 800{,}000 - 785{,}000 = \boxed{15{,}000}
\]
\textbf{Action:} Write down asset to \$785,000; recognize \$15,000 loss in income statement.

\textbf{Under U.S. GAAP:}
\[
\text{Future Undiscounted Cash Flows} = 795{,}000 < 800{,}000 \Rightarrow \text{Impaired}
\]
\[
\text{Fair Value} = 790{,}000 \Rightarrow \text{Impairment Loss} = 800{,}000 - 790{,}000 = \boxed{10{,}000}
\]
\textbf{Action:} Write down asset to \$790,000; recognize \$10,000 loss.

\begin{center}
\renewcommand{\arraystretch}{1.25}
\begin{tabular}{|l|c|c|}
\hline
\textbf{Method} & \textbf{Test Basis} & \textbf{Reversal Allowed?} \\
\hline
IFRS & PV of future cash flows or fair value less costs & Yes (except goodwill) \\
\hline
U.S.\@ GAAP & Undiscounted cash flows test, loss = FV--CV & No \\
\hline
\end{tabular}
\end{center}

\subsubsection*{6. Financial Statement and Ratio Effects}

\begin{itemize}
  \item \textbf{Immediate Effects (Year of Impairment):}
    \begin{itemize}
      \item Assets $\downarrow$, Equity $\downarrow$, Net Income $\downarrow$.
      \item ROA, ROE $\downarrow$ due to reduced earnings.
      \item Asset turnover $\uparrow$ (lower denominator).
      \item No cash flow impact (non-cash charge, not tax-deductible).
    \end{itemize}
  \item \textbf{Subsequent Periods:}
    \begin{itemize}
      \item Lower depreciation/amortization \rightarrow Net income $\uparrow$.
      \item ROA, ROE $\uparrow$ in future periods.
      \item Total asset base remains smaller.
    \end{itemize}
\end{itemize}

\subsubsection*{7. Analytical Implications}

\begin{itemize}
  \item Impairments suggest prior overstatement of asset values or understated depreciation.
  \item Judgment involved \rightarrow management discretion and potential \textbf{earnings management}.
  \item Common motives:
    \begin{itemize}
      \item Delay recognition during strong years to smooth earnings.
      \item Take “big bath” impairments in weak years to reset future profitability.
      \item New management often records larger impairments for clean slate.
    \end{itemize}
\end{itemize}

\subsubsection*{8. Intangible Assets with Indefinite Lives}

\begin{itemize}
  \item Not amortized.
  \item Tested at least annually for impairment.
  \item Impairment loss = Carrying amount -- Fair value (if carrying amount > fair value).
\end{itemize}

\subsubsection*{9. Long-Lived Assets Held for Sale}

\begin{itemize}
  \item Must be reclassified from \textbf{held for use} \rightarrow \textbf{held for sale} if:
    \begin{itemize}
      \item Sale is probable and asset is available for immediate sale.
    \end{itemize}
  \item No further depreciation or amortization.
  \item Impairment if:
  \[
  \text{Carrying Value} > \text{Fair Value $-$ Selling Costs}
  \]
  \item Loss recognized in income statement.
  \item \textbf{Loss reversal:} Allowed under both IFRS and U.S. GAAP (limited to original impairment).
\end{itemize}

\subsubsection*{10. Derecognition of Long-Lived Assets}

\begin{itemize}
  \item Occurs when an asset is sold, exchanged, or abandoned.
  \item \textbf{Gain/Loss on Sale:}
  \[
  \text{Gain or Loss} = \text{Proceeds} - (\text{Cost} - \text{Accumulated Depreciation} - \text{Impairment})
  \]
  \item \textbf{Cash Flow Classification:}
    \begin{itemize}
      \item Proceeds = Investing inflow.
      \item Gain/loss removed from CFO (indirect method adjustment).
    \end{itemize}
\end{itemize}

\begin{center}
\renewcommand{\arraystretch}{1.2}
\begin{tabular}{|l|l|}
\hline
\textbf{Transaction Type} & \textbf{Accounting Treatment} \\
\hline
Sale & Remove asset; recognize gain/loss = sale proceeds $-$ carrying value \\
\hline
Abandonment & Remove asset; recognize full carrying value as loss (no proceeds) \\
\hline
Exchange & Gain/loss = FV(old) - CV(old); new asset recorded at FV(new) \\
\hline
Spinoff & Transfer division assets to new entity; shareholders receive new shares; no gain/loss recognized \\
\hline
\end{tabular}
\end{center}

\subsubsection*{11. Example: Derecognition Scenarios}

\begin{itemize}
  \item \textbf{Sale:}  
    Cost \$500k, Accum. Dep. \$300k, Sale proceeds \$250k \rightarrow  
    Carrying value = \$200k \rightarrow Gain = \$50k.
  \item \textbf{Abandonment:}  
    Cost \$400k, Accum. Dep. \$250k \rightarrow CV \$150k \rightarrow Recognize \$150k loss.
  \item \textbf{Exchange:}  
    CV(old) \$100k, FV(old) = \$120k, FV(new) = \$120k \rightarrow Gain \$20k; record new asset at \$120k.
\end{itemize}

\subsubsection*{12. Summary Comparison: IFRS vs U.S. GAAP Impairment}

\begin{center}
\renewcommand{\arraystretch}{1.25}
\setlength{\tabcolsep}{4pt}
\begin{tabular}{|l|p{5cm}|p{5cm}|}
\hline
\textbf{Feature} & \textbf{IFRS} & \textbf{U.S.\@ GAAP} \\
\hline
Trigger for test & Annual or when indicators appear & Only when indicators appear \\
\hline
Test basis & Compare CV with recoverable amount (max of FV---costs or PV of CFs) & Compare CV with undiscounted CFs \\
\hline
Measurement of loss & CV $-$ Recoverable amount (discounted basis) & CV $-$ Fair value (discounted basis) \\
\hline
Reversal & Allowed (except goodwill) & Not allowed (except held-for-sale) \\
\hline
Subjectivity & Requires PV estimation (discount rate choice) & Relies on undiscounted CF estimates \\
\hline
Impact on ratios & Assets$\downarrow$, ROA$\downarrow$ (in year), future ROA$\uparrow$, turnover$\uparrow$ & Same pattern \\
\hline
\end{tabular}
\end{center}

\subsubsection*{13. Spinoff Transactions}

\begin{itemize}
  \item \textbf{Definition:} Transfer of a division/subsidiary (spinnee) into a new entity; parent (spinnor) shareholders receive shares in the new company.
  \item \textbf{Accounting:}
    \begin{itemize}
      \item Once probable, assets/liabilities of spinnee \rightarrow reclassified as \textbf{held for sale / distribution}.
      \item No gain/loss recognized on disposal.
      \item Post-spinoff: parent shareholders hold shares in both entities.
    \end{itemize}
\end{itemize}

\subsubsection*{14. Key Takeaways}

\begin{itemize}
  \item Impairments signal that prior depreciation/amortization may have been insufficient.
  \item IFRS allows reversal of impairment (except goodwill); GAAP generally prohibits.
  \item Impairment reduces asset values and net income but improves future profitability metrics.
  \item Held-for-sale assets are not depreciated; impairment reversals are allowed up to prior losses.
  \item Derecognition removes assets upon sale, abandonment, exchange, or spinoff, affecting income and investing cash flows.
  \item Analysts should assess impairment timing and motivation to detect earnings management.
\end{itemize}


\subsection*{Module 35.3: Long-Term Asset Disclosures}
\textbf{LOS 35.c:} Analyze and interpret financial statement disclosures regarding property, plant, and equipment (PP\&E) and intangible assets.

\subsubsection*{1. Overview}

\begin{itemize}
  \item Financial statement disclosures on PP\&E and intangibles help analysts:
    \begin{itemize}
      \item Evaluate asset management efficiency.
      \item Estimate average asset age, life, and replacement needs.
      \item Compare depreciation/amortization policies across firms.
    \end{itemize}
  \item Standards differ slightly between \textbf{IFRS} and \textbf{U.S. GAAP}, particularly in revaluation and future amortization disclosures.
\end{itemize}

\subsubsection*{2. IFRS Disclosure Requirements}

\textbf{For each class of PP\&E:}
\begin{itemize}
  \item Basis of measurement (historical cost or revaluation).
  \item Depreciation method.
  \item Depreciation expense for the period.
  \item Useful lives or depreciation rates.
  \item Gross carrying amount and accumulated depreciation:
    \[
    \text{Carrying Value} = \text{Gross PP\&E} - \text{Accumulated Depreciation}
    \]
  \item Reconciliation of carrying amounts (opening to closing balance).
\end{itemize}

\textbf{Additional IFRS PP\&E disclosures:}
\begin{itemize}
  \item Title restrictions or collateralized assets.
  \item Future purchase commitments.
\end{itemize}

\textbf{If using the Revaluation Model:}
\begin{itemize}
  \item Revaluation date.
  \item Method of determining fair value.
  \item Carrying amount under cost model (for comparison).
  \item Revaluation surplus recognized in Other Comprehensive Income (OCI).
\end{itemize}

\textbf{For Intangible Assets:}
\begin{itemize}
  \item Same disclosures as PP\&E.
  \item Must specify whether useful lives are \textbf{finite} or \textbf{indefinite}.
\end{itemize}

\textbf{For Impaired Assets:}
\begin{itemize}
  \item Amounts of impairment losses and reversals by asset class.
  \item Income statement location of losses/reversals.
  \item Circumstances causing impairments or recoveries.
\end{itemize}

\subsubsection*{3. U.S. GAAP Disclosure Requirements}

\textbf{For PP\&E:}
\begin{itemize}
  \item Depreciation expense by period.
  \item Balances of major asset classes (e.g., land, buildings, machinery).
  \item Accumulated depreciation (by class or total).
  \item General description of depreciation methods.
\end{itemize}

\textbf{For Intangible Assets:}
\begin{itemize}
  \item Similar to PP\&E.
  \item Must disclose estimated \textbf{amortization expense for the next five years}.
\end{itemize}

\textbf{For Impaired Assets:}
\begin{itemize}
  \item Description of impaired asset.
  \item Circumstances causing impairment.
  \item Method of determining fair value.
  \item Amount and income statement location of loss.
\end{itemize}

\subsubsection*{4. Presentation of Depreciation and Amortization}

\begin{itemize}
  \item \textbf{IFRS:}
    \begin{itemize}
      \item If \textit{nature-of-expense method} used \rightarrow shown explicitly on income statement.
      \item If \textit{function-of-expense method} used \rightarrow included in COGS or SG\&A.
    \end{itemize}
  \item \textbf{Cash Flow Presentation:}
    \begin{itemize}
      \item Indirect method: shown as non-cash add-back to net income in CFO.
      \item Direct method: excluded from CFO; disclosed in reconciliation (U.S. GAAP requires reconciliation in notes).
    \end{itemize}
  \item \textbf{Investing Activities:}
    \begin{itemize}
      \item Asset purchases = cash outflows.
      \item Asset sales = cash inflows.
    \end{itemize}
\end{itemize}

\subsubsection*{5. Analytical Ratios from Disclosures}

\paragraph{(a) Fixed Asset Turnover}
\[
\text{Fixed Asset Turnover} = \frac{\text{Revenue}}{\text{Average Net PP\&E}}
\]
\begin{itemize}
  \item Measures sales generated per dollar of PP\&E.
  \item High ratio = efficient utilization of fixed assets.
  \item Declining ratio = potential overinvestment or aging assets.
\end{itemize}

\paragraph{(b) Average Age, Total Useful Life, and Remaining Useful Life}
Assuming straight-line depreciation and zero salvage value:

\begin{center}
\renewcommand{\arraystretch}{1.25}
\setlength{\tabcolsep}{5pt}
\begin{tabular}{|l|l|}
\hline
\textbf{Metric} & \textbf{Formula} \\
\hline
Average Age & $\displaystyle \frac{\text{Accumulated Depreciation}}{\text{Depreciation Expense}}$ \\
\hline
Total Useful Life & $\displaystyle \frac{\text{Gross PP\&E}}{\text{Depreciation Expense}}$ \\
\hline
Remaining Useful Life & $\displaystyle \frac{\text{Net PP\&E}}{\text{Depreciation Expense}}$ \\
\hline
\end{tabular}
\end{center}

\textbf{Relationships:}
\[
\text{Remaining Life} = \text{Total Useful Life} - \text{Average Age}
\]

\subsubsection*{6. Example: Estimating Asset Age and Life}

\textbf{Given:}
\[
\begin{aligned}
\text{Gross PP\&E} &= \$3{,}000{,}000 \\
\text{Accumulated Depreciation} &= \$1{,}000{,}000 \\
\text{Depreciation Expense} &= \$500{,}000
\end{aligned}
\]

\begin{center}
\renewcommand{\arraystretch}{1.25}
\begin{tabular}{|l|c|l|}
\hline
\textbf{Metric} & \textbf{Formula} & \textbf{Result (Years)} \\
\hline
Average Age & $1{,}000{,}000 / 500{,}000$ & $= 2$ \\
\hline
Total Useful Life & $3{,}000{,}000 / 500{,}000$ & $= 6$ \\
\hline
Remaining Life & $2$ subtracted from $6$ & $= 4$ \\
\hline
\end{tabular}
\end{center}

\textbf{Interpretation:}
\begin{itemize}
  \item Assets are, on average, 2 years old.
  \item Expected useful life = 6 years $\Rightarrow$ roughly one-third expired.
  \item Remaining useful life = 4 years.
\end{itemize}

\subsubsection*{7. Capital Expenditure vs. Depreciation Ratio}

\[
\text{CapEx-to-Depreciation Ratio} = \frac{\text{Capital Expenditures}}{\text{Depreciation Expense}}
\]

\begin{itemize}
  \item Indicates whether the firm is maintaining, expanding, or shrinking productive capacity.
  \item $\text{Ratio} > 1$ \rightarrow Growth or replacement of capacity.
  \item $\text{Ratio} < 1$ \rightarrow Possible aging assets, underinvestment.
\end{itemize}

\subsubsection*{8. Comparative Summary: IFRS vs U.S. GAAP Disclosures}

\begin{center}
\renewcommand{\arraystretch}{1.25}
\setlength{\tabcolsep}{3pt}
\begin{tabular}{|l|p{5.5cm}|p{5.5cm}|}
\hline
\textbf{Aspect} & \textbf{IFRS} & \textbf{U.S.\@ GAAP} \\
\hline
PP\&E Disclosure Details & Basis, methods, rates, gross/accumulated values, reconciliation & Expense by period, major classes, accumulated dep., methods \\
\hline
Revaluation Model & Allowed; requires date, method, fair value basis, surplus in OCI & Not allowed (cost model only) \\
\hline
Intangibles & Same as PP\&E; specify finite vs indefinite life & Same as PP\&E; include 5-year amortization forecast \\
\hline
Impairment Disclosure & Amounts, reversals, causes, locations in IS & Description, cause, fair value method, loss amount/location \\
\hline
Future Amortization Disclosure & Not required & Required (next 5 years) \\
\hline
Depreciation on Income Statement & Explicit (if nature method used) & Generally in COGS or SG\&A \\
\hline
Cash Flow Reporting & Indirect: add-back noncash charges & Indirect: required reconciliation (footnote if direct method used) \\
\hline
\end{tabular}
\end{center}

\subsubsection*{9. Analytical Uses of Disclosure Data}

\begin{itemize}
  \item Assess efficiency (fixed asset turnover).
  \item Estimate average and remaining asset life to gauge reinvestment needs.
  \item Compare CapEx trends to depreciation for expansion signals.
  \item Identify potential impairment or obsolescence risks.
  \item Detect conservative vs aggressive accounting (e.g., high/low depreciation rates).
\end{itemize}

\subsubsection*{10. Key Takeaways}

\begin{itemize}
  \item IFRS provides more detailed reconciliation and revaluation disclosures.
  \item GAAP emphasizes future amortization and categorical asset breakdowns.
  \item Average age, useful life, and CapEx/depreciation ratios are key tools for analysts to:
    \begin{itemize}
      \item Detect aging assets or expansion strategies.
      \item Predict future capital needs and financing requirements.
    \end{itemize}
  \item Both frameworks require impairment disclosures detailing causes, methods, and impacts.
  \item Footnote details often reveal hidden information about competitiveness, efficiency, and risk.
\end{itemize}


\subsection*{Module 36.1: Leases}
\textbf{LOS 36.a:} Explain the financial reporting of leases from the perspectives of lessors and lessees.

\subsubsection*{1. Overview}

\begin{itemize}
  \item A \textbf{lease} is a contractual agreement granting the right to use an asset for a specified period in exchange for periodic payments.
  \item \textbf{Lessee:} The party that uses the asset.  
  \textbf{Lessor:} The party that owns the asset and receives payments.
  \item Leases are an alternative to purchasing an asset; they affect both the balance sheet and income statement depending on classification.
\end{itemize}

\subsubsection*{2. Requirements for a Contract to Be a Lease}

A contract must:
\begin{enumerate}
  \item Refer to a specific identifiable asset.
  \item Convey substantially all economic benefits of use to the lessee.
  \item Grant the lessee the right to control the use of the asset during the lease term.
\end{enumerate}

\subsubsection*{3. Advantages of Leasing}

\begin{itemize}
  \item Lower initial cash outflow (vs. asset purchase).
  \item Potentially lower financing cost (asset acts as collateral).
  \item Reduced risk of obsolescence (lessor retains ownership risk).
  \item Flexibility in financing structure.
\end{itemize}

\subsubsection*{4. Lease Classification Criteria (IFRS \& U.S. GAAP)}

A lease is classified as a \textbf{Finance Lease} (otherwise Operating Lease) if any of the following conditions apply:

\begin{enumerate}
  \item Ownership transfers to lessee at lease end.
  \item Lessee has a purchase option and is expected to exercise it.
  \item Lease term covers most of the asset’s useful life.
  \item PV of lease payments $\ge$ substantially all of the asset’s fair value.
  \item Asset is specialized, usable only by lessee.
\end{enumerate}

\subsubsection*{5. Lessee Accounting (IFRS \& U.S. GAAP)}

\textbf{Recognition:}
\begin{itemize}
  \item At lease inception, record both:
  \[
  \text{Right-of-Use (ROU) Asset} = \text{Lease Liability} = \text{PV of Lease Payments}
  \]
  \item Exceptions: short-term leases ($\le 12$ months) or low-value assets ($\le USD 5,000$).
\end{itemize}

\textbf{Balance Sheet Effects:}
\begin{itemize}
  \item ROU Asset (intangible) amortized over lease term.
  \item Lease Liability reduced by principal portion of payments.
\end{itemize}

\textbf{Income Statement Effects:}
\begin{itemize}
  \item \textbf{Finance Lease:}  
    Amortization expense + Interest expense (separate).
  \item \textbf{Operating Lease:}  
    Combined lease expense = total lease payment (interest + amortization not shown separately).
\end{itemize}

\textbf{Cash Flow Effects:}
\begin{itemize}
  \item Principal portion \rightarrow Financing outflow.
  \item Interest portion \rightarrow IFRS: either Operating or Financing; GAAP: Operating.
\end{itemize}

\subsubsection*{6. Example: Lessee – Finance Lease (IFRS)}

\textbf{Data:}
\[
\text{Lease Term} = 4\text{ years}, \quad \text{Payment} = 10{,}000 \text{ per year}, \quad i = 5\%.
\]

\textbf{PV of Lease Payments:}
\[
PV = 10{,}000 \times \frac{1 - (1 + 0.05)^{-4}}{0.05} = 35{,}460
\]

\textbf{Accounting:}
\begin{itemize}
  \item Record ROU Asset = Lease Liability = \$35,460.
  \item Amortization (straight-line): \$35,460 / 4 = \$8,865 per year.
\end{itemize}

\textbf{Year 1 Lease Liability Amortization Schedule:}

\begin{center}
\renewcommand{\arraystretch}{1.2}
\setlength{\tabcolsep}{6pt}
\begin{tabular}{|c|c|c|c|c|}
\hline
\textbf{Year} & \textbf{Opening Liability} & \textbf{Interest (5\%)} & \textbf{Payment} & \textbf{Closing Liability} \\
\hline
1 & 35,460 & 1,773 & 10,000 & 27,233 \\
2 & 27,233 & 1,362 & 10,000 & 18,595 \\
3 & 18,595 & 930 & 10,000 & 9,525 \\
4 & 9,525 & 476 & 10,000 & 0 \\
\hline
\end{tabular}
\end{center}

\textbf{Effects:}
\begin{itemize}
  \item IS: Interest + Amortization = 1,773 + 8,865 = 10,638 (Year 1).
  \item BS: ROU Asset ↓ \$8,865 per year; Liability ↓ by principal paid.
  \item CF: \$8,227 financing outflow (principal); \$1,773 operating (interest under GAAP).
\end{itemize}

\subsubsection*{7. Example: Lessee – Operating Lease (U.S. GAAP)}

\textbf{Same data as above.}

\begin{itemize}
  \item Lease liability amortized as for finance lease.
  \item ROU Asset amortized to equal liability each year.
  \item Total lease expense = fixed \$10,000 per year (straight-line).
  \item CF: Entire \$10,000 as operating outflow.
\end{itemize}

\textbf{Comparison of Lessee Accounting:}

\begin{center}
\renewcommand{\arraystretch}{1.25}
\setlength{\tabcolsep}{3pt}
\begin{tabular}{|l|p{5.5cm}|p{5.5cm}|}
\hline
\textbf{Aspect} & \textbf{Finance Lease} & \textbf{Operating Lease} \\
\hline
Balance Sheet & ROU Asset + Lease Liability & Same, but ROU = Liability each period \\
\hline
Income Statement & Interest + Amortization (separate) & Combined single lease expense \\
\hline
Expense Pattern & Front-loaded (higher early years) & Even (straight-line) \\
\hline
Cash Flow & Principal \rightarrow Financing, Interest \rightarrow CFO/Financing & Entire payment \rightarrow CFO \\
\hline
Impact on Ratios & Higher leverage, lower early NI, higher CFO & Lower leverage, higher early NI, lower CFO \\
\hline
\end{tabular}
\end{center}

\subsubsection*{8. Lessor Accounting (IFRS \& GAAP)}

\textbf{Two Types: Finance Lease and Operating Lease.}

\paragraph{Finance Lease (Sales-Type / Direct Financing):}

\begin{itemize}
  \item \textbf{At inception:}
    \begin{itemize}
      \item Remove leased asset from balance sheet.
      \item Record \textbf{Lease Receivable (Net Investment in Lease)}:
        \[
        \text{PV of Lease Payments + PV of Residual Value}
        \]
      \item If manufacturer/dealer: recognize sales revenue and COGS (like a sale).
      \item If finance company: defer any initial gain; recognize interest income over lease term.
    \end{itemize}
  \item \textbf{During lease:}  
    Recognize interest income via the effective interest method.
  \item \textbf{Cash Flow:} Entire inflow as operating activity.
\end{itemize}

\paragraph{Operating Lease:}

\begin{itemize}
  \item Retain leased asset (PP\&E); continue depreciation.
  \item Record lease payments as \textbf{rental income} on straight-line basis.
  \item CF: Entire payment classified as CFO inflow.
\end{itemize}

\subsubsection*{9. Example: Lessor – Finance Lease (Sales-Type)}

\textbf{Data:}
\[
\text{Payments} = 10{,}000 \text{ for 4 years}, \quad i = 5\%, \quad \text{Residual Value} = 2{,}000, \quad \text{Carrying Value} = 30{,}000.
\]
\[
PV(\text{Lease Payments}) = 35{,}460, \quad PV(\text{Residual}) = 1{,}645, \quad \text{Net Investment} = 37{,}105.
\]

\textbf{Step 1 – Derecognition:}
\begin{itemize}
  \item Revenue (Net Investment) = 35,460 + 1,645 = \$37,105.
  \item Cost of Sales (Carrying Value) = \$30,000.
  \item Initial Profit = 37,105 - 30,000 = \$7,105.
\end{itemize}

\textbf{Step 2 – Balance Sheet:}
\[
\text{Lease Receivable} = 37{,}105 \text{ (PV of payments + residual)}
\]
\textbf{Step 3 – Over Time:}
\begin{itemize}
  \item Interest income recognized using effective interest method.
  \item Lease receivable reduced by cash collections.
\end{itemize}

\subsubsection*{10. Example: Lessor – Operating Lease}

\begin{itemize}
  \item Asset remains on balance sheet (PP\&E).
  \item Depreciation expense recorded each year.
  \item Lease payments recognized as \textbf{rental income}.
  \item Cash inflows classified as CFO.
\end{itemize}

\textbf{Depreciation:}
\[
\text{Depreciation Expense} = \frac{(30{,}000 - 2{,}000)}{4} = 7{,}000
\]

\textbf{Income Statement:}
\begin{itemize}
  \item Lease revenue = \$10,000 per year.
  \item Depreciation expense = \$7,000.
  \item Operating profit = \$3,000 per year.
\end{itemize}

\subsubsection*{11. Lessor Comparison Summary}

\begin{center}
\renewcommand{\arraystretch}{1.25}
\setlength{\tabcolsep}{3pt}
\begin{tabular}{|l|p{5.5cm}|p{5.5cm}|}
\hline
\textbf{Aspect} & \textbf{Finance Lease (Sales-Type / Direct Financing)} & \textbf{Operating Lease} \\
\hline
Asset on Balance Sheet & Removed; replaced with lease receivable & Retained; depreciated normally \\
\hline
Revenue Recognition & Upfront (if dealer); interest income over term & Rental income recognized evenly \\
\hline
COGS & Recognized at inception (dealer) & None \\
\hline
Cash Flow Classification & All lease receipts \rightarrow CFO & All lease receipts \rightarrow CFO \\
\hline
Impact on Income & Front-loaded (sales profit + interest) & Straight-line income pattern \\
\hline
\end{tabular}
\end{center}

\subsubsection*{12. Short-Term and Low-Value Leases}

\begin{itemize}
  \item \textbf{IFRS:} Exemption for leases $\le$ 12 months or asset value $\le$ \$5,000.
  \item \textbf{GAAP:} Exemption for leases $\le$ 12 months.
  \item No ROU asset or liability recorded.
  \item Lease expense recognized straight-line over term.
\end{itemize}

\subsubsection*{13. Ratio and Analytical Implications}

\begin{itemize}
  \item \textbf{Finance Lease:}
    \begin{itemize}
      \item Higher assets and liabilities $\Rightarrow$ higher leverage ratios.
      \item Lower net income early, higher later.
      \item CFO higher (interest/principal split); total CF unchanged.
    \end{itemize}
  \item \textbf{Operating Lease:}
    \begin{itemize}
      \item Lower leverage and total assets.
      \item Smoother expense pattern.
      \item Lower CFO (entire payment in operations).
    \end{itemize}
  \item Analysts should adjust for lease capitalization to improve comparability.
\end{itemize}

\subsubsection*{14. Key Takeaways}

\begin{itemize}
  \item Both IFRS and U.S. GAAP classify leases as \textbf{finance or operating}.
  \item Lessees recognize \textbf{ROU Asset + Lease Liability} for nearly all leases.
  \item Finance leases separate interest and amortization; operating leases combine them.
  \item Lessors classify leases as:
    \begin{itemize}
      \item \textbf{Finance Lease:} Asset removed, lease receivable created.
      \item \textbf{Operating Lease:} Asset retained and depreciated.
    \end{itemize}
  \item Both methods yield the same total income over lease life; only timing differs.
  \item Both operating and finance leases increase leverage and should be included in debt analysis.
\end{itemize}

\subsection*{Module 36.2: Deferred Compensation and Disclosures}
\textbf{LOS 36.b:} Explain the financial reporting of defined contribution, defined benefit, and stock-based compensation plans.\\
\textbf{LOS 36.c:} Describe the financial statement presentation of and disclosures relating to long-term liabilities and share-based compensation.

\subsubsection*{1. Overview: Deferred Compensation}

\begin{itemize}
  \item Deferred compensation includes benefits earned in the current period but paid in the future.
  \item Common forms:
    \begin{itemize}
      \item \textbf{Pensions} (defined contribution / defined benefit).
      \item \textbf{Stock-based compensation} (grants, options, SARs).
    \end{itemize}
  \item Accounting involves estimates of future variables: discount rate, turnover, mortality, compensation growth, etc.
\end{itemize}

\subsubsection*{2. Pension Plans Overview}

\begin{center}
\renewcommand{\arraystretch}{1.2}
\setlength{\tabcolsep}{4pt}
\begin{tabular}{|l|p{5.5cm}|p{5.5cm}|}
\hline
\textbf{Type} & \textbf{Defined Contribution (DC)} & \textbf{Defined Benefit (DB)} \\
\hline
Definition & Employer contributes fixed amounts to employee account. & Employer promises periodic benefits after retirement. \\
\hline
Investment Risk & Employee bears risk. & Employer bears risk. \\
\hline
Employer Obligation & Limited to contributions. & Based on actuarial estimate of future payments. \\
\hline
Expense Recognition & Employer’s contribution in period. & Service cost + Interest cost ± Remeasurements. \\
\hline
Balance Sheet Impact & No future liability after payment. & Net pension asset/liability = Plan assets -- PV of obligation. \\
\hline
\end{tabular}
\end{center}

\subsubsection*{3. Defined Contribution Plan Accounting}

\begin{itemize}
  \item Expense = Employer contribution (recorded in income statement).
  \item Once paid, no further liability.
  \item Example: Employer contributes 5\% of salaries; contribution recorded as:
  \[
  \text{Pension Expense (IS)} = \text{Cash Outflow (CFO)}
  \]
\end{itemize}

\subsubsection*{4. Defined Benefit Plan Accounting}

\paragraph{Key Concept: Funded Status}
\[
\text{Funded Status} = \text{Fair Value of Plan Assets} - \text{Present Value of Pension Obligation}
\]
\begin{itemize}
  \item Overfunded $\Rightarrow$ Net pension \textbf{asset}.
  \item Underfunded $\Rightarrow$ Net pension \textbf{liability}.
\end{itemize}

\paragraph{Components of Change in Funded Status (IFRS vs U.S. GAAP)}

\begin{center}
\renewcommand{\arraystretch}{1.25}
\setlength{\tabcolsep}{4pt}
\begin{tabular}{|l|p{5.5cm}|p{5.5cm}|}
\hline
\textbf{Component} & \textbf{IFRS Treatment} & \textbf{U.S.\@ GAAP Treatment} \\
\hline
1. Service Cost & Included in IS (current + past service). & Current service cost in IS; past service cost $\to$ OCI, amortized. \\
\hline
2. Interest Expense / Income & Net pension liability × discount rate. & Same. \\
\hline
3. Expected Return on Assets & Included implicitly via net interest rate. & Reported separately in IS (expected, not actual). \\
\hline
4. Actuarial Gains / Losses & OCI (remeasurements). & OCI (may be amortized over service life). \\
\hline
5. Past Service Cost & IS (immediate). & OCI (amortized). \\
\hline
\end{tabular}
\end{center}

\paragraph{Under IFRS: Three Elements}
\begin{enumerate}
  \item \textbf{Service Cost} (current + past).
  \item \textbf{Net Interest Expense/Income} = Discount rate × Net Pension Asset/Liability.
  \item \textbf{Remeasurements} (actuarial gains/losses + asset return variance) \rightarrow OCI.
\end{enumerate}

\paragraph{Under U.S. GAAP: Five Elements}
\begin{enumerate}
  \item Service cost.
  \item Interest expense/income.
  \item Expected return on plan assets.
  \item Past service cost.
  \item Actuarial gains and losses.
\end{enumerate}

\paragraph{Analytical Notes}
\begin{itemize}
  \item Pension liability behaves like \textbf{debt} (interest-bearing).
  \item Pension expense affects multiple accounts (COGS, SG\&A, OCI).
  \item Analysts should assess pension plan funded status to gauge long-term solvency risk.
\end{itemize}

\subsubsection*{5. Example: Pension Funded Status Calculation}

\[
\begin{aligned}
\text{Fair Value of Assets} &= \$950{,}000 \\
\text{PV of Obligation} &= \$1{,}000{,}000 \\
\Rightarrow \text{Funded Status} &= -\$50{,}000 \ (\text{Underfunded}) \\
\end{aligned}
\]
Balance Sheet: Net Pension Liability \$50,000.  
Income Statement: Includes service cost and net interest expense.  
OCI: Includes remeasurement losses.

\subsubsection*{6. Share-Based Compensation}

\paragraph{Objectives:}
\begin{itemize}
  \item Align management’s interests with shareholders.
  \item Motivate long-term performance.
  \item No immediate cash outflow but causes share dilution.
\end{itemize}

\paragraph{Types:}
\begin{itemize}
  \item \textbf{Stock Grants} – Shares awarded directly, sometimes restricted.
  \item \textbf{Performance Shares} – Vesting conditional on achieving targets (e.g., ROE).
  \item \textbf{Employee Stock Options (ESOs)} – Rights to buy shares at preset price.
  \item \textbf{Stock Appreciation Rights (SARs)} – Cash equivalent of stock appreciation.
\end{itemize}

\subsubsection*{7. Accounting for Stock-Based Compensation}

\paragraph{At Grant Date:}
\begin{itemize}
  \item Measure \textbf{fair value}:
    \begin{itemize}
      \item Stock grants: market price on grant date.
      \item Options: option-pricing model (e.g., Black-Scholes, binomial).
    \end{itemize}
\end{itemize}

\paragraph{Over Vesting (Service) Period:}
\begin{itemize}
  \item Expense recognized in IS on a straight-line basis:
    \[
    \text{Annual Expense} = \frac{\text{Fair Value at Grant}}{\text{Vesting Period}}
    \]
  \item Corresponding increase in equity (APIC / Share-Based Compensation Reserve).
\end{itemize}

\paragraph{On Vesting or Exercise:}
\begin{itemize}
  \item Stock Grants: reclassify reserve to Common Stock and APIC.
  \item Stock Options: cash inflow (exercise price) + recycle equity reserve.
  \item SARs: cash outflow recognized when exercised (no dilution).
\end{itemize}

\paragraph{Financial Statement Effects:}

\begin{center}
\renewcommand{\arraystretch}{1.25}
\begin{tabular}{|l|l|l|}
\hline
\textbf{Effect} & \textbf{Stock Grants / Options} & \textbf{SARs / Phantom Stock} \\
\hline
Income Statement & Expense over vesting period & Liability revalued, expense each period \\
\hline
Balance Sheet & Equity (APIC or reserve) ↑ & Liability ↑ (fair value basis) \\
\hline
Cash Flow & None until exercise & Cash outflow at exercise \\
\hline
EPS Impact & Dilution due to new shares & None (no new shares issued) \\
\hline
\end{tabular}
\end{center}

\paragraph{Analyst Note:}
\begin{itemize}
  \item IFRS and GAAP both require grant-date fair value measurement.
  \item Differences arise only in equity account naming and option valuation assumptions (e.g., volatility).
\end{itemize}

\subsubsection*{8. Lease Disclosures (IFRS 16)}

\textbf{Lessee Disclosures:}
\begin{itemize}
  \item Carrying amount of ROU assets by class.
  \item Total cash outflows for leases.
  \item Interest expense on lease liabilities.
  \item Depreciation expense by asset class.
  \item Variable lease payments expensed during the period.
  \item Additions to ROU assets.
  \item Maturity analysis of lease liabilities (current vs long-term).
  \item Expenses for low-value or short-term leases.
  \item Qualitative disclosures:
    \begin{itemize}
      \item Nature of leasing activities.
      \item Future cash outflows (e.g., guarantees, covenants, sale-and-leaseback).
    \end{itemize}
\end{itemize}

\textbf{Lessor Disclosures:}
\begin{itemize}
  \item \textbf{Finance Leases:}
    \begin{itemize}
      \item Selling profit/loss on derecognition.
      \item Finance income (interest) recognized.
      \item Variable lease income.
      \item Reconciliation of undiscounted payments to net investment.
      \item Maturity analysis of lease receivables.
    \end{itemize}
  \item \textbf{Operating Leases:}
    \begin{itemize}
      \item Lease income (fixed + variable).
      \item Maturity schedule of undiscounted future receipts.
      \item Compliance with IAS 16 (PP\&E) and IAS 36 (Impairment) disclosures.
    \end{itemize}
\end{itemize}

\subsubsection*{9. Pension Disclosures (IAS 19)}

\textbf{Defined Contribution Plans:}
\begin{itemize}
  \item Disclose employer contribution expense separately.
\end{itemize}

\textbf{Defined Benefit Plans:}
\begin{itemize}
  \item Describe plan nature, governance, regulation, and risk exposures.
  \item Reconcile opening and closing funded status:
    \[
    \text{Funded Status}_{t} = \text{Plan Assets}_{t} - \text{PBO}_{t}
    \]
  \item Reconcile plan assets and defined benefit obligation (PBO).
  \item Provide sensitivity analysis for key actuarial assumptions (discount rate, mortality, salary growth).
  \item Disclose plan asset composition by type.
  \item Expected future contributions and maturity profile of obligations.
\end{itemize}

\subsubsection*{10. Share-Based Compensation Disclosures}

\begin{itemize}
  \item Describe nature of plan (grant date, vesting date, service period, settlement method).
  \item Explain valuation method (Black-Scholes, binomial, etc.) and assumptions (volatility, dividend yield, expected term).
  \item Disclose total expense impact on IS and balance sheet (APIC, reserves).
  \item Quantify outstanding and exercisable awards.
  \item Present dilution effect on EPS (basic and diluted).
\end{itemize}

\subsubsection*{11. Key Analytical Insights}

\begin{itemize}
  \item \textbf{Defined Benefit:} High actuarial sensitivity \rightarrow equity volatility; treat underfunded plans as debt.
  \item \textbf{Stock-Based Compensation:} Non-cash expense but dilutive; analysts may adjust EV or EPS accordingly.
  \item \textbf{Disclosures:} Crucial for assessing off-balance obligations and employee-alignment incentives.
  \item \textbf{Analyst Adjustment:}
    \begin{itemize}
      \item Add pension liabilities to debt in leverage ratios.
      \item Include share-based compensation in adjusted operating expense or in total cost of equity capital.
    \end{itemize}
\end{itemize}

\subsubsection*{12. Summary Table: Deferred Compensation Overview}

\begin{center}
\renewcommand{\arraystretch}{1.25}
\setlength{\tabcolsep}{3pt}
\begin{tabular}{|l|p{3.5cm}|p{4cm}|p{3.5cm}|}
\hline
\textbf{Type} & \textbf{Expense Recognition} & \textbf{Balance Sheet Effect} & \textbf{Cash Flow Effect} \\
\hline
Defined Contribution & Expense = Contribution & None (after payment) & CFO outflow in period \\
\hline
Defined Benefit & Service cost + interest ± OCI items & Net Pension Asset/Liability & CFO = contributions; no full match to expense \\
\hline
Stock Grants & Fair value expensed over vesting & APIC / Reserve ↑ & No cash; equity dilution \\
\hline
Stock Options & Fair value expensed; APIC ↑ & Equity ↑; reserve \rightarrow APIC on exercise & Cash inflow = Exercise price \\
\hline
SARs / Phantom & FV-based liability remeasured each period & Liability ↑ (fair value) & Cash outflow at exercise \\
\hline
\end{tabular}
\end{center}

\subsubsection*{13. Key Takeaways}

\begin{itemize}
  \item Defined contribution = simple expense; defined benefit = actuarial complexity.
  \item Pension funded status (asset/liability) must be analyzed as quasi-debt.
  \item Stock-based compensation expensed over service period using grant-date fair value.
  \item Lease, pension, and share-based disclosures are essential for assessing future obligations and shareholder dilution.
  \item Analysts should adjust financial ratios to reflect off-balance obligations and fair-value equity costs.
\end{itemize}

\subsection*{Module 37.1: Differences Between Accounting Profit and Taxable Income}
\textbf{LOS 37.a:} Contrast accounting profit, taxable income, taxes payable, and income tax expense, and distinguish between temporary and permanent differences.

\subsubsection*{1. Overview}

\begin{itemize}
  \item Financial accounting (IFRS/US GAAP) and tax accounting often differ due to distinct objectives and timing.
  \item These differences cause income tax expense (in IS) to differ from taxes payable (in tax return).
  \item Key outcome: creation of \textbf{Deferred Tax Liabilities (DTL)} and \textbf{Deferred Tax Assets (DTA)}.
\end{itemize}

\subsubsection*{2. Key Terminology}

\begin{tabular}{|l|p{5.5cm}|p{5.5cm}|}
\hline
\textbf{Term} & \textbf{Definition (Accounting View)} & \textbf{Definition (Tax View)} \\
\hline
Accounting Profit & Pretax financial income reported in financial statements. & -- \\
\hline
Taxable Income & -- & Income reported on the tax return; basis for tax computation. \\
\hline
Taxes Payable & -- & Tax liability based on taxable income (current tax expense). \\
\hline
Income Tax Expense & Expense in income statement = current tax payable $\pm$ change in deferred tax. & -- \\
\hline
Deferred Tax Liability (DTL) & Future tax payment expected when temporary differences reverse. & Caused by taxable temporary differences. \\
\hline
Deferred Tax Asset (DTA) & Future tax benefit expected from deductible temporary differences or loss carryforwards. & Results from taxes paid early or deductible expenses recognized later. \\
\hline
Valuation Allowance & Contra-account reducing DTA if not probable to be realized. & -- \\
\hline
Tax Base & Net value of an asset or liability for tax purposes. & May differ from carrying value. \\
\hline
Carrying Value & Net book value on balance sheet (after depreciation/amortization). & -- \\
\hline
\end{tabular}

\subsubsection*{3. Relationships Between Measures}

\[
\text{Tax Expense} = \text{Taxes Payable} + \Delta \text{DTL} - \Delta \text{DTA}
\]

\begin{itemize}
  \item Increase in DTL $\Rightarrow$ increases tax expense (future outflow).
  \item Increase in DTA $\Rightarrow$ decreases tax expense (future benefit).
\end{itemize}

\subsubsection*{4. Temporary vs Permanent Differences}

\begin{tabular}{|l|p{5.5cm}|p{5.5cm}|}
\hline
\textbf{Type} & \textbf{Definition} & \textbf{Effect} \\
\hline
Temporary Difference & Timing difference between accounting and tax recognition of income or expenses. & Reverses in future $\Rightarrow$ creates DTA or DTL. \\
\hline
Permanent Difference & Items recognized in accounting but never for tax (or vice versa). & Do not reverse $\Rightarrow$ no DTA/DTL; affects effective tax rate. \\
\hline
\end{tabular}

\begin{itemize}
  \item \textbf{Examples of permanent differences:}
    \begin{itemize}
      \item Tax-exempt interest (e.g., municipal bond income) $\rightarrow$ lower effective tax rate.
      \item Non-deductible expenses (e.g., fines, life-insurance premiums) $\rightarrow$ higher effective tax rate.
      \item Tax credits $\rightarrow$ directly reduce taxes payable.
    \end{itemize}
\end{itemize}

\subsubsection*{5. Deferred Tax Liabilities (DTLs)}

\begin{itemize}
  \item Created when:
    \begin{itemize}
      \item Revenues are recognized earlier in accounting income than for tax.
      \item Expenses are deductible earlier for tax than for accounting.
    \end{itemize}
  \item Represent future \textbf{taxable temporary differences}.
  \item Expected to reverse \rightarrow future cash outflows.
  \item Common example: Accelerated tax depreciation vs. straight-line book depreciation.
\end{itemize}

\paragraph{Example 1: Accelerated Depreciation}

\textbf{Facts:} Asset cost \$30,000; useful life 6 yrs; straight-line (accounting), double-declining (tax).  
\[
\text{SL depreciation} = 30{,}000 / 6 = 5{,}000 \text{ per year.}
\]
\[
\text{DDB depreciation (Year 1)} = 30{,}000 \times 2/6 = 10{,}000.
\]

\textbf{Result:}
\begin{itemize}
  \item Tax depreciation > book depreciation in early years $\Rightarrow$ taxable income lower $\Rightarrow$ DTL created.
  \item In later years, reverses as book depreciation > tax depreciation.
  \item Total depreciation over life equal; only timing differs.
\end{itemize}

\textbf{Journal Logic:}
\[
\begin{aligned}
\text{Year 1:} & \quad \text{Tax expense (IS)} > \text{Taxes payable (return)} \Rightarrow \text{DTL created.}\\
\text{Later:} & \quad \text{Taxes payable} > \text{Tax expense} \Rightarrow \text{DTL reversed.}
\end{aligned}
\]

\subsubsection*{6. Deferred Tax Assets (DTAs)}

\begin{itemize}
  \item Created when:
    \begin{itemize}
      \item Revenue is taxed before it is recognized in accounting.
      \item Expense is recognized in accounting before being deductible for tax.
      \item Tax loss carryforwards exist.
    \end{itemize}
  \item Represent future \textbf{deductible temporary differences}.
  \item Provide future tax savings (future reductions in tax payable).
  \item Typical causes:
    \begin{itemize}
      \item Warranty or bad debt expense (recognized earlier in accounting).
      \item Unearned revenue (taxed now, recognized later).
      \item Post-employment benefits.
      \item Tax loss carryforwards.
    \end{itemize}
\end{itemize}

\paragraph{Example 2: Warranty Expense}
\begin{itemize}
  \item Accounting recognizes estimated warranty expense now.
  \item Tax deduction allowed only when paid.
  \item $\Rightarrow$ Accounting expense earlier \rightarrow taxes payable higher \rightarrow DTA created.
\end{itemize}

\subsubsection*{7. Summary: Taxable vs Deductible Temporary Differences}

\begin{center}
\renewcommand{\arraystretch}{1.2}
\setlength{\tabcolsep}{4pt}
\begin{tabular}{|l|p{4.5cm}|p{3cm}|}
\hline
\textbf{Scenario} & \textbf{Timing Effect} & \textbf{Result} \\
\hline
Tax dep. $>$ Book dep. & Taxable income lower initially & DTL \\
\hline
Book dep. $>$ Tax dep. & Taxable income higher initially & DTA \\
\hline
Revenue in IS before tax & Future taxable revenue & DTL \\
\hline
Revenue taxed before IS & Future deduction benefit & DTA \\
\hline
Expense deductible early & Future taxable income & DTL \\
\hline
Expense in IS before tax & Future deductible amount & DTA \\
\hline
\end{tabular}
\end{center}

\subsubsection*{8. Tax Expense vs Taxes Payable Relationship}

\[
\text{Tax Expense} = \text{Taxes Payable} + \Delta \text{DTL} - \Delta \text{DTA}
\]

\textbf{Interpretation:}
\begin{itemize}
  \item If DTL increases \rightarrow add to tax expense.
  \item If DTA increases \rightarrow subtract from tax expense.
\end{itemize}

\paragraph{Example 3:}
\[
\begin{aligned}
\text{Taxes Payable} &= \$10{,}000 \\
\Delta \text{DTL} &= +\$2{,}000 \\
\Delta \text{DTA} &= +\$500 \\
\Rightarrow \text{Tax Expense} &= 10{,}000 + 2{,}000 - 500 = 11{,}500
\end{aligned}
\]

\subsubsection*{9. Tax Loss Carryforwards and Valuation Allowances}

\begin{itemize}
  \item \textbf{Tax Loss Carryforward:} Past losses offset future taxable income; creates DTA.
  \item \textbf{Valuation Allowance:} If realization of DTA is unlikely, reduce DTA:
  \[
  \text{Net DTA} = \text{Gross DTA} - \text{Valuation Allowance}
  \]
  \item Reduces both assets and equity; does not affect cash.
\end{itemize}

\subsubsection*{10. Changes in Enacted Tax Rate}

\begin{itemize}
  \item DTLs and DTAs must reflect the tax rate expected at reversal.
  \item When statutory rate changes:
    \begin{itemize}
      \item Increase in tax rate $\Rightarrow$ both DTA and DTL increase.
      \item Decrease in tax rate $\Rightarrow$ both decrease.
    \end{itemize}
  \item Adjustment recognized through tax expense in current period.
\end{itemize}

\paragraph{Example 4:}
\[
\begin{aligned}
\text{Existing DTL} &= \$4{,}000 \text{ at 20\%} \\
\text{New rate} &= 25\% \\
\Rightarrow \text{Revised DTL} = 4{,}000 \times \frac{25}{20} = 5{,}000 \\
\text{Increase of \$1,000 recorded as tax expense.}
\end{aligned}
\]

\subsubsection*{11. Analytical Insights}

\begin{itemize}
  \item DTLs and DTAs stem from temporary timing differences; permanent differences affect only tax rate.
  \item DTLs represent expected future tax \textbf{outflows} (liabilities); DTAs represent expected \textbf{inflows/savings}.
  \item Analysts often treat DTLs as debt-like if reversal is probable and recurring.
  \item Deferred tax accounts are sensitive to changes in tax rates and assumptions.
\end{itemize}

\subsubsection*{12. Summary Table: Deferred Tax Overview}

\begin{tabular}{|l|p{5.5cm}|p{5.5cm}|}
\hline
\textbf{Aspect} & \textbf{Deferred Tax Liability (DTL)} & \textbf{Deferred Tax Asset (DTA)} \\
\hline
Cause & Income tax expense $>$ taxes payable & Taxes payable $>$ income tax expense \\
\hline
Typical Case & Accelerated tax depreciation; revenue recognized earlier in IS & Warranty expense; bad debt allowance; unearned revenue; tax loss carryforward \\
\hline
Future Effect & Taxable temporary difference $\Rightarrow$ future outflow & Deductible temporary difference $\Rightarrow$ future benefit \\
\hline
Reversal Timing & In later years when taxable income rises & In later years when taxable income falls \\
\hline
Impact on IS & Increases tax expense in creation period & Reduces tax expense in creation period \\
\hline
\end{tabular}

\subsubsection*{13. Key Takeaways}

\begin{itemize}
  \item \textbf{Taxable Income} \rightarrow Basis for current tax payable.  
  \item \textbf{Accounting Profit} \rightarrow Basis for financial reporting.  
  \item Temporary differences create deferred taxes; permanent differences do not.
  \item Formula:
  \[
  \boxed{\text{Tax Expense} = \text{Taxes Payable} + \Delta \text{DTL} - \Delta \text{DTA}}
  \]
  \item DTLs = future tax payments; DTAs = future tax savings.
  \item Changes in statutory tax rates revalue all deferred tax balances.
  \item Permanent differences affect effective tax rate but never reverse.
\end{itemize}


\subsection*{Module 37.2: Deferred Tax Assets and Liabilities}
\textbf{LOS 37.b:} Explain how deferred tax liabilities and assets are created and the factors that determine how a company’s deferred tax liabilities and assets should be treated for financial analysis.

\subsubsection*{1. Overview}

\begin{itemize}
  \item Deferred tax assets (DTAs) and deferred tax liabilities (DTLs) arise from \textbf{temporary timing differences} between taxable income and accounting profit.
  \item For a difference to be temporary, the total amount recognized over the asset or liability’s life must be the same in both the income statement and tax return—only the \textbf{timing} differs.
  \item These temporary differences cause future taxable or deductible amounts, leading to DTLs or DTAs respectively.
\end{itemize}

\subsubsection*{2. Example: Firebird Corporation (Depreciation and Warranties)}

\paragraph{Facts:}
\begin{itemize}
  \item Statutory tax rate: 30\%.
  \item PP\&E cost: \$40 million.
  \item Accounting depreciation: Straight-line over 5 years.
  \item Tax depreciation: Straight-line over 4 years.
  \item Warranty expense: 5 \% of gross revenues.
\end{itemize}

\paragraph{Observation:}
\begin{itemize}
  \item Depreciation \rightarrow timing difference in asset.
  \item Warranty provision \rightarrow timing difference in liability.
  \item No permanent differences.
\end{itemize}

\subsubsection*{3. Timing Difference 1: Depreciation (Tax Base vs Carrying Value)}

\begin{itemize}
  \item Accounting total depreciation: \$8 million per year for 5 years.  
  \item Tax total depreciation: \$10 million per year for 4 years.
  \item Total depreciation (\$40 million) is the same—only timing differs.
  \item Early years: Tax depreciation > book \rightarrow taxable income < accounting profit \rightarrow DTL created.  
  \item Later years: Timing reverses \rightarrow DTL reverses.
\end{itemize}

\paragraph{Carrying vs Tax Base:}
\[
\text{DTL arises when: } \text{Carrying Value of Asset} > \text{Tax Base of Asset.}
\]
\[
\text{DTL} = (\text{Carrying Value} - \text{Tax Base}) \times \text{Tax Rate}
\]

\textbf{Example: Year 1}

\[
\begin{aligned}
\text{Carrying Value} &= 40 - 8 = 32 \\
\text{Tax Base} &= 40 - 10 = 30 \\
\text{Temporary Difference} &= 2 \\
\Rightarrow \text{DTL} &= 2 \times 30\% = 0.6
\end{aligned}
\]

\textbf{Interpretation:}
\begin{itemize}
  \item Firm paid less tax now (\$0.6 m less).  
  \item Future tax payment expected \rightarrow liability.
\end{itemize}

\subsubsection*{4. Timing Difference 2: Warranty Provision (Liability Side)}

\begin{itemize}
  \item Warranty expense estimated at 5 \% of sales each year.
  \item Accounting: Expense recognized at sale \rightarrow creates warranty liability.
  \item Tax: Expense deductible only when actually incurred \rightarrow tax return shows lower expense initially.
  \item Therefore: Taxes payable > tax expense \rightarrow DTA created.
\end{itemize}

\paragraph{Tax Base of Liability:}
\[
\text{Tax Base of Liability} = \text{Carrying Value} - \text{Amounts deductible in future.}
\]

\begin{itemize}
  \item Since all future warranty costs will be deductible, tax base = 0.
  \item DTA arises because the entire liability will reduce taxable income later.
\end{itemize}

\[
\text{DTA} = (\text{Carrying Value of Liability} - \text{Tax Base}) \times \text{Tax Rate}
\]

\textbf{Example:}
\[
\text{Warranty Liability} = 2.0 \Rightarrow \text{DTA} = 2.0 \times 30\% = 0.6
\]

\subsubsection*{5. Combined Tax Expense Computation}

\[
\text{Tax Expense} = \text{Taxes Payable} + \Delta \text{DTL} - \Delta \text{DTA}
\]

\begin{itemize}
  \item Depreciation difference \rightarrow DTL increases \rightarrow add to tax expense.
  \item Warranty difference \rightarrow DTA increases \rightarrow subtract from tax expense.
\end{itemize}

\paragraph{Example Summary Table:}

\begin{center}
\renewcommand{\arraystretch}{1.2}
\setlength{\tabcolsep}{4pt}
\begin{tabular}{|l|p{4.5cm}|p{3.5cm}|}
\hline
\textbf{Source} & \textbf{Treatment} & \textbf{Result} \\
\hline
Depreciation (asset) & Higher tax dep. early years & DTL (future outflow) \\
\hline
Warranty (liability) & Expense earlier for book & DTA (future benefit) \\
\hline
\end{tabular}
\end{center}

\subsubsection*{6. Realizability of Deferred Tax Assets}

\begin{itemize}
  \item DTAs represent expected future tax savings—but realization depends on \textbf{future taxable income}.
  \item Neither DTLs nor DTAs are discounted to PV; they are recorded at statutory rates.
  \item Each reporting date: management must assess if DTAs are recoverable.
\end{itemize}

\paragraph{IFRS Treatment:}
\begin{itemize}
  \item Reduce DTA directly if realization unlikely \rightarrow increase tax expense.
\end{itemize}

\paragraph{U.S. GAAP Treatment:}
\begin{itemize}
  \item Maintain gross DTA but record a \textbf{valuation allowance} (contra account) for unrecoverable portion.
  \item \textbf{Creating or increasing valuation allowance} $\Rightarrow$ ↑ tax expense, ↓ net income.
\end{itemize}

\[
\text{Net DTA} = \text{Gross DTA} - \text{Valuation Allowance}
\]

\paragraph{Analyst Notes:}
\begin{itemize}
  \item Persistent losses \rightarrow likely valuation allowance needed.
  \item Subjectivity in assessing future profitability \rightarrow possible earnings management.
\end{itemize}

\subsubsection*{7. Analytical Treatment of Deferred Taxes}

\begin{itemize}
  \item \textbf{If DTLs expected to reverse:} Treat as liabilities (future tax payments).  
  \item \textbf{If DTLs not expected to reverse soon:} Treat as equity (adjust ↓DTL, ↑Equity).
  \item \textbf{Decision Rule:} Ask “When will the temporary difference reverse?”  
  \item In high-growth firms with ongoing capex and accelerated tax depreciation, DTLs may never fully reverse.
\end{itemize}

\paragraph{Analyst Adjustment:}
\[
\text{Adjusted Equity} = \text{Reported Equity} + \text{Non-Reversing Portion of DTL}
\]

\subsubsection*{8. Analytical Examples}

\begin{center}
\renewcommand{\arraystretch}{1.2}
\setlength{\tabcolsep}{4pt}
\begin{tabular}{|l|p{4cm}|p{3.5cm}|}
\hline
\textbf{Scenario} & \textbf{Treatment} & \textbf{Reasoning} \\
\hline
DTL from depreciation in mature firm & Treat as liability & Future tax payments \\
\hline
DTL in growing firm with continuous capex & Treat as equity & Perpetually renewed \\
\hline
DTA with low profitability history & Reduce DTA value & Realization uncertain \\
\hline
Large valuation allowance reversal & Earnings mgmt signal & Boosts income \\
\hline
\end{tabular}
\end{center}

\subsubsection*{9. Key Formulas}

\[
\begin{aligned}
\text{DTL} &= (\text{Carrying Value of Asset} - \text{Tax Base}) \times \text{Tax Rate} \\
\text{DTA} &= (\text{Carrying Value of Liability} - \text{Tax Base}) \times \text{Tax Rate} \\
\text{Tax Expense} &= \text{Taxes Payable} + \Delta\text{DTL} - \Delta\text{DTA}
\end{aligned}
\]

\subsubsection*{10. Key Takeaways}

\begin{itemize}
  \item Temporary differences \rightarrow DTLs (future tax outflows) or DTAs (future tax savings).
  \item Depreciation differences usually create DTLs; warranty and bad debt estimates create DTAs.
  \item DTAs require “more-likely-than-not” realizability test.  
  \item IFRS reduces DTAs directly; GAAP uses valuation allowance.
  \item For analysis:
    \begin{itemize}
      \item Reversing DTLs \rightarrow treat as liabilities.  
      \item Non-reversing DTLs \rightarrow treat as equity.
    \end{itemize}
  \item Analysts must evaluate tax assumptions, profitability forecasts, and valuation allowances for earnings quality.
\end{itemize}


\subsection*{Module 37.3: Tax Rates and Disclosures}
\textbf{LOS 37.c:} Calculate, interpret, and contrast an issuer’s effective tax rate, statutory tax rate, and cash tax rate.\\
\textbf{LOS 37.d:} Analyze disclosures relating to deferred tax items and the effective tax rate reconciliation and explain their effects on a company’s financial statements and ratios.

\subsubsection*{1. Overview}

\begin{itemize}
  \item The \textbf{statutory tax rate} is the legally imposed corporate rate in the firm’s jurisdiction.  
  \item The \textbf{effective tax rate (ETR)} is the rate implied by the income statement.  
  \item The \textbf{cash tax rate (CTR)} shows the actual cash outflow for taxes relative to pretax income.  
  \item Differences between these rates stem primarily from \textbf{permanent differences}, tax credits, foreign income rates, or tax holidays—not from temporary timing differences.
\end{itemize}

\subsubsection*{2. Key Tax Rate Formulas}

\[
\begin{aligned}
\text{Effective Tax Rate (ETR)} &= \frac{\text{Income Tax Expense}}{\text{Earnings Before Tax}} \\[4pt]
\text{Cash Tax Rate (CTR)} &= \frac{\text{Cash Taxes Paid}}{\text{Earnings Before Tax}} \\[4pt]
\text{Statutory Tax Rate (STR)} &= \text{Legal corporate rate of domicile}
\end{aligned}
\]

\begin{itemize}
  \item \textbf{ETR $\neq$ STR} if permanent differences or multi-jurisdictional operations exist.  
  \item \textbf{CTR $\neq$ ETR} due to timing of cash flows (deferred taxes).
\end{itemize}

\subsubsection*{3. Causes of Differences Between Statutory and Effective Rates}

\begin{center}
\renewcommand{\arraystretch}{1.2}
\setlength{\tabcolsep}{4pt}
\begin{tabular}{|l|p{4.5cm}|p{3.5cm}|}
\hline
\textbf{Cause} & \textbf{Explanation} & \textbf{Effect on ETR} \\
\hline
Different foreign tax rates & Subsidiaries under higher/lower rates & Raises or lowers ETR \\
\hline
Permanent differences & Tax-exempt income, non-deductible expenses, credits & Divergence from STR \\
\hline
Tax holidays / incentives & Exemption or reduction in certain jurisdictions & Temporarily lowers ETR \\
\hline
Changes in tax laws & Adjustments to rates or deductions & One-time ETR changes \\
\hline
Capital vs. operating income & Different treatment of gains or dividends & Distorts ETR comparison \\
\hline
\end{tabular}
\end{center}

\subsubsection*{4. Example: NDC – Effective Tax Rate Reconciliation}

\paragraph{Facts:}
\begin{itemize}
  \item NDC operates in U.S. and abroad.  
  \item Reconciliation between statutory and effective tax rates is disclosed for 3 years.
\end{itemize}

\paragraph{Analysis:}
\begin{itemize}
  \item ETR trended \textbf{upward} due to:
    \begin{itemize}
      \item Higher state income tax component.
      \item Reduced benefit from foreign income tax advantages.
      \item Offset by one-off losses and special items.
    \end{itemize}
  \item Volatility of “special items” and “other” categories reduces comparability and forecast reliability.
  \item Continuous items (foreign rates, permanent differences) are relevant for forward projections; sporadic ones (asset sales, holidays) should be excluded from normalized ETR estimates.
\end{itemize}

\subsubsection*{5. Analytical Use of Tax Rate Disclosures}

\begin{itemize}
  \item Footnote reconciliations help analysts:
    \begin{itemize}
      \item Identify recurring vs.\ non-recurring drivers of tax expense.
      \item Forecast sustainable after-tax earnings.  
      \item Adjust valuation models (e.g., DCF) with realistic effective rates.  
    \end{itemize}
  \item Persistent low ETR may signal tax optimization or structural advantages; temporary low ETR may indicate tax holidays.
\end{itemize}

\subsubsection*{6. Deferred Tax Item Disclosures (LOS 37.d)}

\paragraph{Purpose:} Explain sources of temporary differences generating DTAs and DTLs and show how changes affect income tax expense.

\paragraph{Common Sources:}

\begin{center}
\renewcommand{\arraystretch}{1.2}
\setlength{\tabcolsep}{4pt}
\begin{tabular}{|l|p{4cm}|p{3.5cm}|}
\hline
\textbf{Item} & \textbf{Timing Difference} & \textbf{Result} \\
\hline
Accelerated tax dep. vs. SL book & Tax dep. higher early & DTL (future outflow) \\
\hline
Asset impairments & Deductible upon sale/disposal & DTA (future deduction) \\
\hline
Restructuring provisions & Expensed now, deductible when paid & DTA (future benefit) \\
\hline
Post-employment benefits & Recognized when earned, paid later & DTA \\
\hline
Unrealized gains on AFS securities & Taxed upon realization & DTL via OCI \\
\hline
Inventory (LIFO vs. FIFO) & Depends on jurisdiction & Possible DTL \\
\hline
\end{tabular}
\end{center}

\paragraph{Disclosure Requirements:}

\begin{itemize}
  \item Components of \textbf{income tax expense}:  
        – Current tax expense (taxes payable)  
        – Deferred tax expense (change in DTL – change in DTA)
  \item \textbf{Detailed schedule} of DTAs, DTLs, valuation allowance, and changes during the year.  
  \item \textbf{Tax loss carryforwards} and expiry dates.  
  \item \textbf{Unrecognized DTLs} for undistributed earnings of subsidiaries/JVs.  
  \item \textbf{Effective vs.\ statutory rate reconciliation.}
\end{itemize}

\subsubsection*{7. Example: WCCO Inc. – Deferred Tax Disclosure Analysis}

\paragraph{Facts:}
\begin{itemize}
  \item Income tax expense \(>\) taxes payable for last 3 years.
  \item DTA from international tax-loss carryforwards and employee benefits.
  \item DTL from PP\&E (accelerated depreciation) and unrealized gains on AFS securities.
  \item Valuation allowance decreased by \$33 million in 20X5.
\end{itemize}

\paragraph{Analysis:}
\begin{itemize}
  \item \textbf{Why expense > payable:}  
        DTLs increased faster than DTAs \rightarrow positive deferred tax expense.
  \item \textbf{Valuation allowance reduction:}  
        \rightarrow Decreases deferred tax expense, increases net income.  
        \rightarrow Signals improved outlook for future taxable income.
  \item Analysts should verify whether management’s optimism is justified; reversal of valuation allowances can be used for earnings management.
\end{itemize}

\subsubsection*{8. Income Tax Expense Relationship}

\[
\boxed{\text{Income Tax Expense} = \text{Taxes Payable} + \Delta\text{DTL} - \Delta\text{DTA}}
\]
\begin{itemize}
  \item When DTL ↑ \rightarrow tax expense ↑.  
  \item When DTA ↑ \rightarrow tax expense ↓.  
  \item Links income statement tax provision with cash flow and balance-sheet items.
\end{itemize}

\subsubsection*{9. Analytical Implications}

\begin{itemize}
  \item Persistent DTL growth $\Rightarrow$ aggressive tax depreciation or capital expansion.  
  \item Large DTAs $\Rightarrow$ potential underutilized tax benefits; check valuation allowance.  
  \item Declining ETR with stable operations $\Rightarrow$ structural tax advantage.  
  \item Large swings in ETR $\Rightarrow$ one-offs or policy changes; normalize for forecasting.  
  \item Deferred tax balances affect leverage and ROE:  
        – Increasing DTLs boost equity if treated as quasi-equity.  
        – Large valuation allowances depress net assets.
\end{itemize}

\subsubsection*{10. Key Takeaways}

\begin{itemize}
  \item \textbf{Statutory Rate (STR):} Legal corporate tax rate.  
  \item \textbf{Effective Rate (ETR):} Income statement rate = Tax Expense / EBT.  
  \item \textbf{Cash Rate (CTR):} Actual cash taxes / EBT (CFO perspective).  
  \item Differences stem mainly from permanent differences, foreign rate variations, and tax holidays.  
  \item Deferred tax disclosures explain future tax obligations and potential earnings adjustments.  
  \item Analysts should:
    \begin{itemize}
      \item Separate recurring vs.\ non-recurring tax items.  
      \item Evaluate valuation-allowance changes as signals of earnings quality.  
      \item Assess whether DTLs are reversing (liability) or perpetual (equity-like).  
    \end{itemize}
\end{itemize}

\subsection*{Module 38.1: Reporting Quality}
\textbf{LOS 38.a–f:} Compare financial reporting quality with the quality of reported results; describe the spectrum of reporting quality, conservative vs. aggressive accounting, managerial motivations, disciplining mechanisms, and presentation choices (including non-GAAP measures).

\subsubsection*{1. Financial Reporting Quality vs. Quality of Reported Results (LOS 38.a)}

\begin{itemize}
  \item \textbf{Financial Reporting Quality (FRQ):} Refers to the characteristics of financial statements themselves—primarily their compliance with GAAP, relevance, and faithful representation.
  \item \textbf{Quality of Reported Results (QRR):} Refers to the \emph{sustainability, accuracy, and adequacy} of reported earnings, cash flows, and balance sheet items.
  \item \textbf{Key Point:} A firm can have high-quality reporting (GAAP-compliant, unbiased) but still low-quality results (unsustainable or inadequate earnings).
\end{itemize}

\paragraph{Decision Usefulness:}
\begin{itemize}
  \item \textbf{Relevance:} Information affects users’ decisions and is material.  
  \item \textbf{Faithful Representation:} Information is complete, neutral, and free from error.
\end{itemize}

\paragraph{Quality of Earnings:}
\begin{itemize}
  \item Depends on \textbf{sustainability} (recurrence) and \textbf{level} (adequacy).  
  \item One-time or volatile sources (e.g., FX gains, asset sales) $\rightarrow$ low sustainability.  
  \item Operational improvements $\rightarrow$ high sustainability.
\end{itemize}

\begin{tabular}{|l|p{5.5cm}|p{5.5cm}|}
\hline
\textbf{Dimension} & \textbf{High Quality} & \textbf{Low Quality} \\
\hline
Reporting & GAAP-compliant, relevant, neutral, error-free & Biased estimates, poor disclosure, misleading presentation \\
\hline
Earnings & Sustainable, sufficient, reflect ongoing operations & One-off, unsustainable, or inadequate to cover costs/investor return \\
\hline
Cash Flows & Stable, supported by earnings & Volatile, inconsistent with accruals \\
\hline
Balance Sheet & Realistic asset values, adequate provisions & Overstated assets, understated liabilities \\
\hline
\end{tabular}

\subsubsection*{2. Spectrum of Financial Reporting Quality (LOS 38.b)}

\paragraph{Quality Spectrum:}

\begin{tabular}{|l|p{4cm}|p{4.5cm}|}
\hline
\textbf{Level} & \textbf{Characteristics} & \textbf{Implication} \\
\hline
1. Highest & GAAP-compliant, sustainable, adequate ROIC & Transparent, reliable \\
\hline
2. Low earnings quality & Unsustainable or inadequate earnings & Watch for temporary performance \\
\hline
3. Biased estimates & Distorted neutrality (aggressive accruals) & Requires adjustments \\
\hline
4. Earnings management & Smoothing within GAAP & Volatility artificially reduced \\
\hline
5. Non-GAAP activity & Reflects operations but breaks rules & Restatement risk \\
\hline
6. Fraudulent & Misrepresentation or fabrication & No analytical value \\
\hline
\end{tabular}

\paragraph{Key Insight:}  
Quality exists on a continuum — from \textbf{faithful representation and sustainability} $\rightarrow$ to \textbf{bias, manipulation, or fraud.}

\subsubsection*{3. Conservative vs. Aggressive Accounting (LOS 38.c)}

\begin{itemize}
  \item \textbf{Conservative Accounting:} Choices that reduce current-period earnings or asset values.  
        $\rightarrow$ Future earnings likely to rise as prior deferrals reverse.
  \item \textbf{Aggressive Accounting:} Choices that boost current earnings or assets.  
        $\rightarrow$ Future earnings will decline when overstatements correct.
  \item \textbf{Neutral Accounting:} Best quality — unbiased, faithfully represents economic activity.
\end{itemize}

\paragraph{Earnings Smoothing:}
\begin{itemize}
  \item Managers alternate between conservative and aggressive choices to reduce volatility.  
  \item “\textbf{Cookie jar reserves}”: Conservative estimates in good periods (higher liabilities) used later to boost earnings in weak periods.
\end{itemize}

\paragraph{Examples (Figure 38.1):}

\begin{tabular}{|l|p{4cm}|p{4cm}|}
\hline
\textbf{Area} & \textbf{Aggressive Accounting} & \textbf{Conservative Accounting} \\
\hline
Revenue Recognition & Early recognition; optimistic completion estimates & Delay until greater certainty \\
\hline
Expense Recognition & Capitalize expenses; reduce allowances & Immediate expensing; larger provisions \\
\hline
Inventory Valuation & Overstate future value; delay write-downs & Write down early when impaired \\
\hline
Estimates & Understate liabilities (warranties, pensions) & Overstate liabilities or reserve accounts \\
\hline
Presentation & Highlight positive metrics; obscure risks & Transparent, balanced disclosure \\
\hline
\end{tabular}

\paragraph{GAAP-Induced Conservatism:}
\begin{itemize}
  \item Higher verification threshold for revenues than for expenses.
  \item Research costs expensed immediately (uncertain future benefit).
  \item Legal liabilities accrued when “probable,” but asset gains recognized only when realized.
\end{itemize}

\paragraph{Analytical Implication:}
\begin{itemize}
  \item Conservatism $\neq$ always “good” - can understate performance.  
  \item Aggressiveness increases risk of restatements and volatility.
\end{itemize}

\subsubsection*{4. Motivations for Low-Quality Reporting (LOS 38.d)}

\paragraph{Primary Motivations:}
\begin{itemize}
  \item Meet or exceed benchmarks:
    \begin{itemize}
      \item Prior guidance from management.
      \item Analyst consensus expectations.
      \item Prior-year earnings.
    \end{itemize}
  \item Career incentives - enhance reputation, promotion prospects.
  \item Stock-based compensation - link to EPS or price performance.
  \item Avoid covenant violations (leverage, coverage ratios).
  \item Maintain credibility with investors, suppliers, and lenders.
\end{itemize}

\paragraph{Behavioral Drivers of Misreporting (Fraud Triangle):}
\begin{itemize}
  \item \textbf{Motivation:} Pressure to achieve targets.  
  \item \textbf{Opportunity:} Weak controls, poor oversight, flexible standards.  
  \item \textbf{Rationalization:} Self-justification (e.g., “I’ll fix it next quarter.”)
\end{itemize}

\paragraph{Conditions Conducive to Misreporting:}
\begin{itemize}
  \item Weak internal controls or ineffective board oversight.
  \item Wide discretion in accounting treatment under GAAP.
  \item Low enforcement penalties or lax auditing.
\end{itemize}

\subsubsection*{5. Mechanisms That Discipline Reporting Quality (LOS 38.e)}

\paragraph{External Mechanisms:}

\begin{tabular}{|l|p{4cm}|p{4cm}|}
\hline
\textbf{Mechanism} & \textbf{Description} & \textbf{Limitations} \\
\hline
Regulatory oversight & SEC, FCA, ESMA, IOSCO & Limited resources; reactive \\
\hline
Disclosure & Periodic reporting, management commentary & Can be biased; needs audit \\
\hline
Independent audit & “reasonable assurance” of fair presentation & Client hires auditor; conflict risk \\
\hline
Internal controls & Management assessment (e.g., SOX 404) & May fail under weak governance \\
\hline
Legal enforcement & Fines, suspensions, criminal penalties & May not fully deter \\
\hline
Private contracts & Covenants in lending/supplier agreements & Only protect specific parties \\
\hline
\end{tabular}

\paragraph{Audit Opinion Types:}
\begin{itemize}
  \item \textbf{Unqualified (“Clean”):} Fairly presented under GAAP.  
  \item \textbf{Qualified:} Material exceptions exist.  
  \item \textbf{Adverse:} Misstatements pervasive.  
  \item \textbf{Disclaimer:} Auditor cannot form an opinion.
\end{itemize}

\paragraph{Key Note:}  
An unqualified opinion $\neq$ guarantee of accuracy; provides only \textbf{reasonable assurance} based on tests.

\subsubsection*{6. Presentation Choices and Non-GAAP Measures (LOS 38.f)}

\paragraph{Purpose of Non-GAAP (or Non-IFRS) Measures:}
\begin{itemize}
  \item Firms adjust earnings to exclude “non-recurring” or “non-operating” items.  
  \item Aim to show “core performance,” but often improve apparent profitability.
\end{itemize}

\paragraph{Common Adjustments:}
\begin{itemize}
  \item Exclude restructuring costs, impairment losses, or non-cash items.  
  \item Modify depreciation or stock-based compensation treatment.  
  \item Remove FX effects or mark-to-market adjustments.
\end{itemize}

\paragraph{Regulatory Requirements (U.S. SEC):}
\begin{itemize}
  \item Present the most comparable GAAP measure with equal prominence.  
  \item Explain management’s rationale for the non-GAAP measure.  
  \item Reconcile differences with GAAP.  
  \item Disclose whether excluded items are likely to recur.
\end{itemize}

\paragraph{IFRS Requirements:}
\begin{itemize}
  \item Define and explain relevance of non-IFRS measures.  
  \item Provide reconciliation to IFRS-compliant measures.
\end{itemize}

\paragraph{Analyst Cautions:}
\begin{itemize}
  \item Non-GAAP measures can obscure recurring costs or exaggerate profitability.  
  \item Evaluate adjustments critically - are they truly non-recurring?  
  \item Compare reconciliations year-over-year for consistency.
\end{itemize}

\subsubsection*{7. Analytical Summary Table}

\begin{tabular}{|l|p{5.5cm}|p{5.5cm}|}
\hline
\textbf{Dimension} & \textbf{High-Quality Reporting} & \textbf{Low-Quality Reporting} \\
\hline
Compliance & Fully GAAP/IFRS-compliant & Non-compliant or partially compliant \\
\hline
Decision Usefulness & Relevant, faithfully represented, neutral & Misleading, incomplete, biased \\
\hline
Earnings Quality & Sustainable, adequate level, recurring sources & One-time, unsustainable, or overstated \\
\hline
Accounting Bias & Neutral & Aggressive or conservative bias distorting neutrality \\
\hline
Motivation & Transparent reporting for investor understanding & Meet targets, conceal weakness, avoid covenants \\
\hline
Discipline & Strong regulatory, audit, internal control environment & Weak governance, limited oversight \\
\hline
Presentation & Clear, GAAP-comparable, reconciled & Non-GAAP emphasis, selective disclosure \\
\hline
\end{tabular}

\subsubsection*{8. Key Takeaways}

\begin{itemize}
  \item \textbf{High-Quality Reporting:} GAAP-compliant, relevant, neutral, decision-useful.  
  \item \textbf{High-Quality Earnings:} Sustainable, adequate, supported by cash flows.  
  \item \textbf{Aggressive vs. Conservative:} Bias affects timing of recognition; neutrality preferred.  
  \item \textbf{Low-Quality Reporting Drivers:} Incentives, weak controls, rationalization.  
  \item \textbf{Enforcement:} Auditors, regulators, and private contracts impose discipline but not perfection.  
  \item \textbf{Non-GAAP Measures:} Require reconciliation and transparency; analysts must adjust for bias.
\end{itemize}

\subsection*{Module 38.2: Accounting Choices and Estimates}
\textbf{LOS 38.g:} Describe accounting methods (choices and estimates) that could be used to manage earnings, cash flow, and balance sheet items.

\subsubsection*{1. Overview}

\begin{itemize}
  \item Management has discretion within GAAP/IFRS to choose among accounting methods and estimates.
  \item These choices affect the \textbf{timing, magnitude, and classification} of reported earnings, cash flows, and assets/liabilities.
  \item Earnings management often exploits flexibility in:
        \begin{itemize}
          \item Revenue recognition timing.
          \item Expense recognition (allowances, depreciation, amortization).
          \item Classification between operating/investing cash flows.
        \end{itemize}
  \item Such actions may be GAAP-compliant yet reduce financial reporting quality.
\end{itemize}

\subsubsection*{2. Revenue Recognition Choices}

\paragraph{Key Mechanisms:}
\begin{itemize}
  \item \textbf{FOB Shipping Point vs.\ Destination:}  
        -- Revenue recognized when goods leave seller (FOB shipping) $\rightarrow$ earlier recognition.  
        -- Revenue recognized upon customer receipt (FOB destination) $\rightarrow$ later recognition.
  \item \textbf{Channel Stuffing:} Ship excess goods to distributors to inflate current-period revenue.  
        $\rightarrow$ Future revenues fall as pipeline is cleared.
  \item \textbf{Bill-and-Hold:} Customer billed but goods not yet delivered.  
        -- Used legitimately when customer requests delayed delivery.  
        -- Fictitious bill-and-hold = premature revenue recognition.
  \item \textbf{Discounts/Financing:} Temporary incentives accelerate orders to meet earnings targets.
\end{itemize}

\paragraph{Analyst Warning:}
\begin{itemize}
  \item Examine revenue growth vs.\ receivables and inventory.  
  \item Sharp increase in receivables with stable sales may signal premature recognition.
\end{itemize}

\begin{tabular}{|l|p{5.5cm}|p{5.5cm}|}
\hline
\textbf{Method} & \textbf{Effect on Earnings} & \textbf{Effect on Future Periods} \\
\hline
FOB Shipping Point & Recognize earlier $\rightarrow$ higher current revenue & Lower next-period revenue \\
\hline
FOB Destination & Recognize later $\rightarrow$ deferred revenue & Higher next-period revenue \\
\hline
Channel Stuffing & Inflates current sales artificially & Future sales fall; potential returns increase \\
\hline
Bill-and-Hold & Records sales before delivery & Lowers later revenue when goods finally shipped \\
\hline
\end{tabular}

\subsubsection*{3. Estimates of Credit Losses and Warranty Reserves}

\paragraph{Bad-Debt Allowance:}
\begin{itemize}
  \item \textbf{Lower allowance} $\rightarrow$ $\downarrow$ expenses, $\uparrow$ income, $\uparrow$ net receivables.  
  \item \textbf{Higher allowance} $\rightarrow$ $\uparrow$ expenses, $\downarrow$ income, $\downarrow$ receivables.
  \item Used to smooth earnings (“cookie jar” reserves).  
        – High-earnings years: increase reserve.  
        – Low-earnings years: decrease reserve to boost profit.
\end{itemize}

\paragraph{Warranty Reserve:}
\begin{itemize}
  \item Similar mechanics to bad-debt estimates.
  \item $\downarrow$ Estimated warranty expense $\rightarrow$ $\uparrow$ earnings now, but risk of future losses.  
  \item $\uparrow$ Estimate $\rightarrow$ $\downarrow$ current income, more conservative.
\end{itemize}

\paragraph{Analytical Signal:}  
Check trend of allowances vs. sales or receivables; abrupt changes indicate potential manipulation.

\subsubsection*{4. Valuation Allowance (Deferred Tax Assets)}

\begin{itemize}
  \item A \textbf{valuation allowance} reduces DTAs to reflect likelihood of realization.  
  \item \textbf{Increase allowance} $\rightarrow$ $\downarrow$ DTA, $\downarrow$ net income (conservative).  
  \item \textbf{Decrease allowance} $\rightarrow$ $\uparrow$ DTA, $\uparrow$ net income (aggressive).  
  \item Can be used to smooth income across periods.
\end{itemize}

\[
\text{Net DTA} = \text{Gross DTA} - \text{Valuation Allowance}
\]

\paragraph{Example:}
\begin{itemize}
  \item 2024: Increase allowance $\rightarrow$ recognize tax expense $\rightarrow$ $\downarrow$ income.  
  \item 2025: Decrease allowance $\rightarrow$ recognize tax benefit $\rightarrow$ $\uparrow$ income.
\end{itemize}

\subsubsection*{5. Depreciation Methods and Estimates}

\paragraph{Depreciation Choice:}
\begin{itemize}
  \item \textbf{Accelerated methods} (DDB, SYD) $\rightarrow$ higher early-year expense, lower income initially.  
  \item \textbf{Straight-line} $\rightarrow$ stable expense, smoother earnings.
\end{itemize}

\paragraph{Key Estimates:}
\begin{itemize}
  \item \textbf{Useful life:} Longer life $\rightarrow$ $\downarrow$ depreciation, $\uparrow$ income early.  
  \item \textbf{Salvage value:} Higher salvage $\rightarrow$ $\downarrow$ expense, $\uparrow$ asset value.  
  \item Misestimated salvage $\rightarrow$ gain/loss on sale at disposal.
\end{itemize}

\paragraph{Analytical Effect:}

\begin{center}
\renewcommand{\arraystretch}{1.2}
\setlength{\tabcolsep}{4pt}
\begin{tabular}{|l|p{5.5cm}|p{5.5cm}|}
\hline
\textbf{Choice} & \textbf{Short-Term Effect} & \textbf{Long-Term Effect} \\
\hline
Accelerated depreciation & $\downarrow$ earnings, $\downarrow$ assets & $\uparrow$ future earnings as expense declines \\
\hline
Longer useful life & $\uparrow$ earnings now & $\downarrow$ later as remaining life shortened \\
\hline
Higher salvage value & $\uparrow$ current earnings & $\downarrow$ future gains when asset sold below value \\
\hline
\end{tabular}
\end{center}

\subsubsection*{6. Amortization and Impairment}

\begin{itemize}
  \item Amortization of intangibles parallels depreciation mechanics.  
  \item Goodwill \textbf{not amortized} — tested for impairment annually.
  \item \textbf{Ignoring or delaying impairment} $\rightarrow$ $\uparrow$ current earnings, overstated assets.  
  \item \textbf{Timely recognition} $\rightarrow$ $\downarrow$ income now, but more faithful reporting.
\end{itemize}

\paragraph{Analytical Tip:}  
Watch for firms with goodwill > 30 % of assets but no impairments for several years $\rightarrow$ possible aggressive bias.

\subsubsection*{7. Inventory Cost Flow Assumptions}

\paragraph{Choice Impact (Rising Prices):}

\begin{center}
\renewcommand{\arraystretch}{1.2}
\setlength{\tabcolsep}{4pt}
\begin{tabular}{|l|p{5.5cm}|p{5.5cm}|}
\hline
\textbf{Method} & \textbf{Income Statement Effects} & \textbf{Balance Sheet Effects} \\
\hline
FIFO & Lower COGS $\rightarrow$ $\uparrow$ gross profit, $\uparrow$ NI & Ending inventory closer to current cost ($\uparrow$ value) \\
\hline
Weighted-Average & Higher COGS $\rightarrow$ $\downarrow$ profit & Lower inventory valuation \\
\hline
\end{tabular}
\end{center}

\paragraph{Choice Impact (Falling Prices):} Effects reverse — FIFO produces lower income and inventory.  

\paragraph{Information Quality:}
\begin{itemize}
  \item FIFO $\rightarrow$ more relevant balance sheet (current prices).  
  \item Weighted-average $\rightarrow$ more relevant income statement (current COGS).
\end{itemize}

\paragraph{Analytical Focus:}
\begin{itemize}
  \item Transparency of disclosure regarding inventory method.  
  \item Adjust comparables to consistent method (e.g., LIFO reserve adjustment).
\end{itemize}

\subsubsection*{8. Related-Party Transactions}

\begin{itemize}
  \item If management controls both buyer and seller entities, transfer prices can shift earnings.  
  \item Overpricing supplies $\rightarrow$ $\downarrow$ public-firm earnings; underpricing $\rightarrow$ $\uparrow$ earnings.  
  \item Must review footnotes for related-party disclosures under IAS 24 or ASC 850.
\end{itemize}

\subsubsection*{9. Capitalization Choices}

\paragraph{General Rule:}
\begin{itemize}
  \item Capitalizing expenditures defers expense recognition, creating an asset.  
  \item Example: Capitalize \$1.5 m marketing cost over 3 yrs $\rightarrow$ current expense \$0.5 m (instead of \$1.5 m).  
  \item Immediate effect: $\uparrow$ current earnings, $\uparrow$ assets; future periods bear amortization expense.
\end{itemize}

\paragraph{Cash Flow Reclassification:}
\begin{itemize}
  \item Expense $\rightarrow$ Operating outflow.  
  \item Capitalization $\rightarrow$ Investing outflow $\rightarrow$ increases CFO.
\end{itemize}

\paragraph{Analytical Implication:}
\begin{itemize}
  \item Review ratio of capitalized costs to total assets or R\&D to detect earnings management.  
  \item Over-capitalization inflates asset base and margins.
\end{itemize}

\subsubsection*{10. Cash Flow Classification Management}

\paragraph{Stretching Payables:}
\begin{itemize}
  \item Delay payments to suppliers $\rightarrow$ temporarily boosts CFO.  
  \item No immediate earnings impact but reverses next period.
\end{itemize}

\paragraph{Capitalized Interest:}
\begin{itemize}
  \item Interest cost added to asset $\rightarrow$ reduces current CFS from investing, raises CFO.  
  \item Expense spread via depreciation $\rightarrow$ defers cost.
\end{itemize}

\paragraph{IFRS Flexibility:}
\begin{itemize}
  \item IFRS allows classification of interest/dividends as CFO or CFI/CFF.  
  \item Management can choose categories to improve reported CFO.
\end{itemize}

\begin{tabular}{|l|p{4cm}|p{4cm}|}
\hline
\textbf{Technique} & \textbf{Effect on CFO} & \textbf{Analyst Adjustment} \\
\hline
Stretch payables & Increases temporarily & Normalize by adjusting for delayed outflows \\
\hline
Capitalize interest & Increases CFO (shifts to investing) & Reclassify to CFO for comparability \\
\hline
IFRS reclassification (interest/dividends) & Choice can raise or lower CFO & Ensure consistent treatment across firms \\
\hline
\end{tabular}

\subsubsection*{11. Analytical Summary}

\begin{tabular}{|l|p{5.5cm}|p{5.5cm}|}
\hline
\textbf{Area} & \textbf{Aggressive Choice} & \textbf{Conservative Choice} \\
\hline
Revenue & Early recognition, channel stuffing & Defer until delivery/collection \\
\hline
Receivables / Reserves & Reduce allowances ($\uparrow$ NI) & Increase allowances ($\downarrow$ NI) \\
\hline
Valuation Allowance (DTA) & Decrease allowance ($\uparrow$ NI) & Increase allowance ($\downarrow$ NI) \\
\hline
Depreciation & Straight-line, long life, high salvage & Accelerated, short life, low salvage \\
\hline
Intangibles & Delay impairment, minimal amortization & Timely impairment recognition \\
\hline
Inventory & FIFO in rising prices ($\uparrow$ NI) & Weighted-avg or LIFO ($\downarrow$ NI) \\
\hline
Capitalization & Capitalize R\&D or marketing costs & Expense immediately \\
\hline
Cash Flow Classification & Capitalize interest / stretch payables & Consistent classification \\
\hline
\end{tabular}

\subsubsection*{12. Key Takeaways}

\begin{itemize}
  \item Accounting discretion allows timing shifts that affect reported performance without changing cash economics.  
  \item Analysts must:
        \begin{itemize}
          \item Compare policies to peers and historical patterns.  
          \item Adjust for changes in estimates (e.g., useful life, allowances).  
          \item Focus on sustainable cash flows, not temporary accruals.  
        \end{itemize}
  \item Warning signs: rising receivables, declining CFO/NI ratio, frequent policy changes, or large non-cash gains.
\end{itemize}

\subsection*{Module 38.3: Warning Signs}
\textbf{LOS 38.h:} Describe accounting warning signs and methods for detecting manipulation of information in financial reports.

\subsubsection*{1. Overview}

\begin{itemize}
  \item The presence of one or more warning signs does not prove manipulation or fraud, but it \textbf{requires deeper analysis}.
  \item Patterns inconsistent with economic reality, industry norms, or peer behavior may signal \textbf{earnings management or accounting distortion}.
  \item Analysts must investigate \textbf{changes in estimates, accounting policies, and cash flow mismatches}.
\end{itemize}

\paragraph{Common Manipulation Objectives:}
\begin{itemize}
  \item Inflate revenue to meet earnings targets.
  \item Delay expense recognition to boost profit.
  \item Capitalize costs to enhance asset values.
  \item Smooth income to reduce perceived volatility.
\end{itemize}

\subsubsection*{2. Revenue Recognition Warning Signs}

\paragraph{Key Red Flags:}
\begin{itemize}
  \item Frequent or unexplained \textbf{changes in revenue recognition methods}.
  \item Use of \textbf{bill-and-hold} or \textbf{barter} transactions (nonstandard recognition practices).
  \item Aggressive use of \textbf{rebate programs} requiring subjective estimation of rebate impact.
  \item Lack of transparency in multi-element sales (bundled goods/services).
  \item Revenue growth significantly outpacing industry peers.
  \item \textbf{Receivables turnover} declining over several periods.
  \item Decreasing \textbf{total asset turnover}, especially after acquisitions.
  \item Inclusion of \textbf{nonoperating or one-time sales} in revenue.
\end{itemize}

\begin{tabular}{|l|p{5.5cm}|p{5.5cm}|}
\hline
\textbf{Warning Sign} & \textbf{Possible Manipulation} & \textbf{Analyst Detection} \\
\hline
Change in revenue policy & Timing shift in recognition & Compare to prior year notes and footnotes \\
\hline
Bill-and-hold transactions & Premature revenue recognition & Inspect delivery terms and customer acceptance clauses \\
\hline
Channel stuffing or rebates & Inflated current sales & Check receivables growth vs. sales \\
\hline
Revenue > peer growth & Overstatement of revenue & Compare with industry sales and macro demand \\
\hline
Falling receivable turnover & Delayed collections / fake sales & Review aging schedules, DSO trends \\
\hline
\end{tabular}

\paragraph{Analyst Focus:}
\begin{itemize}
  \item Review consistency between \textbf{sales growth, receivables, and cash flow}.
  \item Persistent divergence between revenue and CFO → potential manipulation.
\end{itemize}

\subsubsection*{3. Inventory and Cost Recognition Warning Signs}

\paragraph{Indicators:}
\begin{itemize}
  \item Declining \textbf{inventory turnover} (COGS / average inventory).  
        → May indicate obsolete inventory, overproduction, or inflated asset value.
  \item \textbf{LIFO liquidation} (U.S. GAAP only): Drawing down old, low-cost layers to reduce COGS and inflate profit.
  \item Inventory buildup without corresponding sales growth.
\end{itemize}

\begin{tabular}{|l|p{5.5cm}|p{5.5cm}|}
\hline
\textbf{Warning Sign} & \textbf{Effect} & \textbf{Analyst Response} \\
\hline
Falling inventory turnover & Possible overstatement of assets & Compare inventory growth vs. sales growth \\
\hline
LIFO liquidation & Artificial boost to earnings & Adjust for nonrecurring gain in analysis \\
\hline
Inventory buildup & Channel stuffing / weak demand & Check write-downs or obsolescence provisions \\
\hline
\end{tabular}

\subsubsection*{4. Capitalization Policies}

\begin{itemize}
  \item Capitalizing costs not typically capitalized by peers inflates assets and defers expense recognition.
  \item Examples: advertising, startup, training, or routine maintenance costs.
  \item Review footnotes and compare capitalization policies with industry norms.
\end{itemize}

\paragraph{Analyst Test:}
\begin{itemize}
  \item Calculate \textbf{capitalized costs / total assets} and \textbf{CFO/NI ratio}.
  \item A rising asset base without matching cash flows indicates possible capitalization bias.
\end{itemize}

\subsubsection*{5. Relationship Between Revenue and Cash Flow}

\begin{itemize}
  \item Ratio of \textbf{CFO to Net Income} persistently $<$ 1 or declining over time → red flag.  
        Indicates accrual-based revenue not supported by cash receipts.
\end{itemize}

\[
\text{CFO / Net Income} < 1 \Rightarrow \text{Possible Aggressive Accrual Accounting}
\]

\begin{tabular}{|l|p{4cm}|p{4cm}|}
\hline
\textbf{Observation} & \textbf{Interpretation} & \textbf{Analytical Check} \\
\hline
CFO $<$ NI over several years & Overstated accruals or early revenue & Compare CFO/NI ratio, track AR trends \\
\hline
CFO volatility $>$ NI volatility & Real cash cycles fluctuate more than reported income & Analyze changes in working capital \\
\hline
\end{tabular}

\subsubsection*{6. Other Common Warning Signs}

\paragraph{Accounting Estimates and Policies:}
\begin{itemize}
  \item Depreciation methods or useful lives differ significantly from industry norms.
  \item Estimated salvage values unusually high → lower depreciation.
  \item Changes in estimates without clear rationale.
\end{itemize}

\paragraph{Earnings Patterns:}
\begin{itemize}
  \item Unusual fourth-quarter patterns — “big bath” or “earnings smoothing.”
  \item Frequent one-time charges labeled “nonrecurring” but recurring annually.
  \item Inconsistent profit seasonality vs. revenue patterns.
\end{itemize}

\paragraph{Related-Party Transactions:}
\begin{itemize}
  \item Deals with affiliates controlled by management can shift income.  
  \item Often priced non-arm’s-length to manipulate reported results.
  \item Must review \textbf{IAS 24 / ASC 850 disclosures.}
\end{itemize}

\paragraph{Profitability Ratios:}
\begin{itemize}
  \item Gross or operating margins significantly higher than industry averages.  
        → Could reflect misclassification or underreported expenses.
\end{itemize}

\paragraph{Disclosure Quality:}
\begin{itemize}
  \item Minimal detail in financial statements.  
  \item Heavy reliance on non-GAAP metrics, or aggressive use of “adjusted EBITDA.”  
  \item Overemphasis on positive trends, little discussion of risks.
\end{itemize}

\begin{tabular}{|l|p{5.5cm}|p{5.5cm}|}
\hline
\textbf{Category} & \textbf{Warning Indicator} & \textbf{Analyst Action} \\
\hline
Estimates & Changes in depreciation life or salvage value & Compare with peers, compute depreciation rate \\
\hline
Earnings pattern & Repeated “nonrecurring” charges & Adjust normalized earnings \\
\hline
Related-party & Transactions with affiliates & Evaluate economic substance and pricing \\
\hline
Margins & Unusually high vs. peers & Review SG\&A, COGS allocation, reclassifications \\
\hline
Disclosure & Minimal transparency, excessive non-GAAP use & Cross-check with audited filings \\
\hline
\end{tabular}

\subsubsection*{7. Growth Through Acquisition}

\paragraph{Manipulation Opportunities:}
\begin{itemize}
  \item Purchase price allocation allows flexibility in valuing acquired assets and goodwill.  
  \item Creates future leeway for amortization and impairment adjustments.  
  \item Restructuring reserves can be used as “cookie jar” for future periods.
\end{itemize}

\paragraph{Analyst Caution:}
\begin{itemize}
  \item Compare pre- and post-acquisition margins and depreciation/amortization patterns.  
  \item Reassess goodwill and intangible asset growth relative to sales.
\end{itemize}

\subsubsection*{8. Restructuring and Impairment Adjustments}

\paragraph{Typical Pattern:}
\begin{itemize}
  \item Large one-time restructuring charges recognized in “bad” years → improve future earnings comparability.  
  \item Analysts should restate prior periods to reflect more realistic historical expenses.
\end{itemize}

\paragraph{Analytical Approach:}
\begin{itemize}
  \item Spread restructuring costs or impairments across prior years to normalize trend.  
  \item Avoid interpreting big bath write-downs as “positive resets.”
\end{itemize}

\subsubsection*{9. Practical Analyst Checklist}

\begin{itemize}
  \item Compare growth in \textbf{revenue vs. receivables vs. CFO}.
  \item Examine inventory and asset turnover trends.  
  \item Track recurring “special” or “one-time” items.  
  \item Compare accounting estimates (lives, salvage, allowances) to industry norms.  
  \item Review related-party disclosures and audit opinions.  
  \item Adjust for large acquisitions, restructurings, or policy changes.  
  \item Compute key ratios:
        \[
        \text{CFO/NI}, \quad \text{Inventory Turnover}, \quad \text{Receivable Turnover}, \quad \text{Gross Margin vs. Peers}.
        \]
\end{itemize}

\subsubsection*{10. Key Takeaways}

\begin{itemize}
  \item Warning signs \textbf{do not confirm fraud} but demand deeper forensic review.  
  \item Persistent discrepancies among revenue, receivables, and cash flows are strong red flags.  
  \item Analysts should benchmark accounting choices against peers and monitor disclosure consistency.  
  \item Frequent “nonrecurring” adjustments and acquisitions often mask earnings volatility.  
  \item Adjust historical earnings to reflect economic rather than reported performance for fair valuation.
\end{itemize}

\subsection*{Module 39.1: Introduction to Financial Ratios}
\textbf{LOS 39.a:} Describe tools and techniques used in financial analysis, including their uses and limitations.

\subsubsection*{1. Overview of Analytical Tools}

\begin{itemize}
  \item Financial analysis converts accounting data into decision-useful information.
  \item Techniques help identify trends, relationships, and anomalies — but must be interpreted contextually.
  \item Core analytical methods:
        \begin{itemize}
          \item \textbf{Ratio Analysis}
          \item \textbf{Common-Size Analysis} (vertical and horizontal)
          \item \textbf{Graphical Analysis}
          \item \textbf{Regression Analysis}
        \end{itemize}
  \item These tools correspond to \textbf{Step 3} of the financial analysis framework:
        \[
        \text{“Adjust financial statements, compute ratios, and prepare exhibits.”}
        \]
\end{itemize}

\subsubsection*{2. Ratio Analysis}

\paragraph{Purpose and Usefulness:}
\begin{itemize}
  \item Expresses relationships among financial variables.
  \item Provides quick insight into performance, efficiency, liquidity, solvency, and profitability.
  \item Useful for:
        \begin{itemize}
          \item Projecting future earnings and cash flows.
          \item Assessing flexibility (ability to meet obligations under stress).
          \item Evaluating management performance.
          \item Comparing firm and industry trends.
          \item Benchmarking against peers and historical results.
        \end{itemize}
\end{itemize}

\paragraph{Analyst Objective:}
\begin{itemize}
  \item Ratios raise questions rather than provide final answers.
  \item Must be interpreted together — no single ratio suffices.
\end{itemize}

\begin{tabular}{|l|p{5.5cm}|p{5.5cm}|}
\hline
\textbf{Analytical Goal} & \textbf{Related Ratio Type} & \textbf{Interpretation Focus} \\
\hline
Profitability & Net margin, ROA, ROE & Earnings generation vs. resources used \\
\hline
Liquidity & Current, Quick, Cash ratios & Short-term solvency \\
\hline
Leverage & Debt-to-assets, Debt-to-equity & Long-term solvency and capital structure \\
\hline
Efficiency & Inventory turnover, Receivable turnover & Asset utilization effectiveness \\
\hline
Valuation & P/E, EV/EBITDA, P/B & Market perception and expectations \\
\hline
\end{tabular}

\paragraph{Limitations:}
\begin{itemize}
  \item Ratios are meaningless in isolation — require benchmarking.
  \item Accounting methods differ (e.g., IFRS vs. U.S. GAAP → inconsistent comparability).
  \item Conglomerates complicate peer comparison (multi-industry operations).
  \item Ratios vary across industries — what’s “strong” in one sector may be “weak” in another.
  \item Definitions differ across analysts (e.g., “debt” may or may not include leases).
  \item Requires contextual analysis:
        \begin{itemize}
          \item Prior-period trends.
          \item Business cycle stage.
          \item Company strategy and expectations.
        \end{itemize}
\end{itemize}

\paragraph{Analytical Tip:}
\begin{itemize}
  \item Consistency of calculation method is crucial.
  \item Always specify formula and inputs (e.g., average vs. ending balances).
\end{itemize}

\subsubsection*{3. Common-Size Analysis}

\paragraph{Purpose:}
\begin{itemize}
  \item Standardizes financial statements for comparison across time or peers.
  \item Removes size effect — ideal for multi-period and cross-sectional analysis.
\end{itemize}

\paragraph{Types:}
\begin{itemize}
  \item \textbf{Vertical Common-Size Statements:}
        \begin{itemize}
          \item Express each item as a \% of a key total.
          \item \textbf{Balance Sheet:} Each item ÷ Total Assets.
                \[
                \text{Common-size BS ratio} = \frac{\text{Balance Sheet Item}}{\text{Total Assets}}
                \]
          \item \textbf{Income Statement:} Each item ÷ Sales.
                \[
                \text{Common-size IS ratio} = \frac{\text{Income Statement Item}}{\text{Sales}}
                \]
        \end{itemize}
  \item \textbf{Horizontal Common-Size Statements:}
        \begin{itemize}
          \item Express each item relative to a base-year value (set = 1.0).
          \item Useful for trend and growth analysis over multiple periods.
        \end{itemize}
\end{itemize}

\begin{tabular}{|l|p{5.5cm}|p{5.5cm}|}
\hline
\textbf{Type} & \textbf{Divisor / Base} & \textbf{Analytical Use} \\
\hline
Vertical BS & Total Assets & Capital structure, liquidity mix \\
\hline
Vertical IS & Sales (Revenue) & Margin structure, cost efficiency \\
\hline
Horizontal BS/IS & First-year values (index = 1.0) & Trend growth and volatility \\
\hline
\end{tabular}

\paragraph{Advantages:}
\begin{itemize}
  \item Facilitates identification of cost drivers and margin trends.
  \item Enables structural comparison (e.g., asset mix, leverage composition).
  \item Reveals operating leverage through relative expense movement.
\end{itemize}

\paragraph{Example Interpretation:}
\begin{itemize}
  \item Suppose net profit margin rises from 7\% → 12\%.  
        → Analyst investigates whether this stems from:
        \begin{itemize}
          \item Lower amortization (noncash → temporary effect).  
          \item Lower interest expense (improved capital efficiency).  
          \item Permanent operational gains vs. one-time savings.
        \end{itemize}
  \item Common-size analysis identifies areas to \textbf{investigate further} — not final conclusions.
\end{itemize}

\paragraph{Analyst Caution:}
\begin{itemize}
  \item Presentation format can differ — some show latest year leftmost (as in CFA examples).
  \item Always check which year is the “base” for horizontal statements.
\end{itemize}

\subsubsection*{4. Graphical Analysis}

\paragraph{Purpose:}
\begin{itemize}
  \item Visualizes relationships and time trends for easier pattern recognition.
  \item Useful for presentations and quick diagnostics.
\end{itemize}

\paragraph{Common Formats:}
\begin{itemize}
  \item \textbf{Stacked Column (Bar) Graph:}  
        – Shows composition (e.g., asset categories) over multiple years.  
        – Reveals shifts in structure (e.g., rise in payables, fall in cash).
  \item \textbf{Line Graph:}  
        – Tracks item trends (e.g., revenue, margins, leverage ratios).  
        – Highlights anomalies (e.g., diverging growth between assets and sales).
\end{itemize}

\paragraph{Example Interpretation:}
\begin{itemize}
  \item Rising trade payables and declining cash → possible liquidity issues.
  \item Rapid growth in receivables vs. flat sales → possible aggressive revenue recognition.
\end{itemize}

\begin{tabular}{|l|p{5.5cm}|p{5.5cm}|}
\hline
\textbf{Graph Type} & \textbf{Best For} & \textbf{Analyst Insight} \\
\hline
Stacked Bar & Composition over time & Balance sheet structure shifts \\
\hline
Line Graph & Trend analysis & Growth, seasonality, or volatility patterns \\
\hline
Pie Chart & Cross-sectional composition & Segment contribution to total \\
\hline
\end{tabular}

\subsubsection*{5. Regression Analysis}

\paragraph{Purpose:}
\begin{itemize}
  \item Quantitative tool linking dependent (e.g., sales) and independent variables (e.g., GDP, advertising).
  \item Used for \textbf{forecasting and scenario analysis.}
\end{itemize}

\[
\text{Sales}_t = \alpha + \beta \times \text{GDP}_t + \varepsilon_t
\]

\paragraph{Analytical Application:}
\begin{itemize}
  \item Identify drivers of performance (macroeconomic sensitivity).  
  \item Forecast future metrics (e.g., revenues, margins, default probabilities).  
  \item Evaluate consistency of management forecasts.
\end{itemize}

\paragraph{Limitations:}
\begin{itemize}
  \item Correlation ≠ causation.  
  \item Historical relationships may not persist.  
  \item Sensitive to outliers and structural breaks (e.g., pandemic, crisis).
\end{itemize}

\subsubsection*{6. Comparative Overview of Analytical Tools}

\begin{tabular}{|l|p{5.5cm}|p{5.5cm}|}
\hline
\textbf{Tool} & \textbf{Primary Use} & \textbf{Limitation / Risk} \\
\hline
Ratio Analysis & Quick diagnostics, cross-firm comparison & Accounting differences, context dependency \\
\hline
Common-Size Analysis & Normalized structure, cost trend detection & No insight into absolute scale or cash flows \\
\hline
Graphical Analysis & Visualization of trends & Potential for oversimplification \\
\hline
Regression Analysis & Forecasting relationships & Model risk, multicollinearity, data sensitivity \\
\hline
\end{tabular}

\subsubsection*{7. Analytical Insights for Interpretation}

\begin{itemize}
  \item Always interpret ratios and common-size trends together:
        \[
        \text{E.g., Rising ROE but falling CFO/NI → possible accrual-based boost.}
        \]
  \item Combine horizontal trends (growth) with vertical proportions (structure).
  \item Investigate anomalies:
        \begin{itemize}
          \item Decline in gross margin with stable sales → cost inflation or product mix change.
          \item Decrease in total asset turnover → inefficient asset use or overinvestment.
        \end{itemize}
  \item Verify if improvements are operational (sustainable) or accounting-based (temporary).
\end{itemize}

\subsubsection*{8. Key Takeaways}

\begin{itemize}
  \item \textbf{Ratio analysis} provides relationships but requires context.  
  \item \textbf{Common-size analysis} standardizes data for comparability.  
  \item \textbf{Graphical tools} reveal trends intuitively.  
  \item \textbf{Regression analysis} quantifies predictive relationships.  
  \item Always evaluate results within:
        \begin{itemize}
          \item Historical trend.
          \item Peer and industry context.
          \item Business cycle phase.
        \end{itemize}
  \item No single ratio or exhibit explains performance — use a holistic analytical framework.
\end{itemize}

\subsection*{Module 39.2: Financial Ratios, Part 1}
\textbf{LOS 39.b:} Calculate and interpret activity, liquidity, solvency, and profitability ratios.

\subsubsection*{1. Overview}
\begin{itemize}
  \item Financial ratios are grouped by the type of insight they provide:
        \begin{itemize}
            \item \textbf{Activity ratios (Asset utilization ratios)} – efficiency of asset management.
            \item \textbf{Liquidity ratios} – ability to meet short-term obligations.
            \item \textbf{Solvency ratios} – long-term leverage and financial stability.
            \item \textbf{Profitability ratios} – ability to generate profits from sales and assets.
        \end{itemize}
  \item Categories overlap: e.g., Payables turnover is both an activity and liquidity indicator.
  \item Always compare ratios against industry peers, historical trends, and strategic context.
\end{itemize}


\subsubsection*{2. Activity Ratios (Asset Utilization or Operating Efficiency)}

\paragraph{Purpose:}
\begin{itemize}
  \item Evaluate how efficiently a firm uses its assets to generate sales and revenue.  
  \item High turnover ratios imply greater efficiency—but may also signal under-investment in assets.  
  \item Use \textbf{average balances} (beginning + ending)/2 for balance sheet items to smooth seasonality.
\end{itemize}

\paragraph{Key Ratios and Interpretations:}
\begin{itemize}
  \item \textbf{Receivables Turnover:} 
  $\displaystyle \frac{\text{Net Credit Sales}}{\text{Average Accounts Receivable}}$  
  → High = efficient collection or strict credit terms; Low = poor collections or loose credit.

  \item \textbf{Days Sales Outstanding (DSO):} 
  $\displaystyle \frac{365}{\text{Receivables Turnover}}$  
  → Average days to collect cash; Lower = better liquidity; Higher = potential credit risk.

  \item \textbf{Inventory Turnover:} 
  $\displaystyle \frac{\text{COGS}}{\text{Average Inventory}}$  
  → High = efficient inventory use or low stock; Low = obsolete or slow-moving inventory.

  \item \textbf{Days Inventory on Hand (DOH):} 
  $\displaystyle \frac{365}{\text{Inventory Turnover}}$  
  → Shorter = efficient management; too short = risk of stockouts.

  \item \textbf{Payables Turnover:} 
  $\displaystyle \frac{\text{Purchases}}{\text{Average Accounts Payable}}$  
  where Purchases = COGS + Ending Inventory – Beginning Inventory.  
  → High = paying too quickly or missing credit terms; Low = taking longer to pay or cash strain.

  \item \textbf{Days Payables Outstanding (DPO):} 
  $\displaystyle \frac{365}{\text{Payables Turnover}}$  
  → Long = good supplier terms or delayed payments; Short = not using credit fully.

  \item \textbf{Total Asset Turnover:} 
  $\displaystyle \frac{\text{Revenue}}{\text{Average Total Assets}}$  
  → Overall efficiency; High = good utilization; Low = excess or idle assets.

  \item \textbf{Fixed Asset Turnover:} 
  $\displaystyle \frac{\text{Revenue}}{\text{Average Net Fixed Assets}}$  
  → Low = underused or new assets; Very high = aging or over-stressed capacity.

  \item \textbf{Working Capital Turnover:} 
  $\displaystyle \frac{\text{Revenue}}{\text{Average Working Capital}}$  
  where Working Capital = Current Assets – Current Liabilities.  
  → High = efficient operations but may reflect low working capital.
\end{itemize}

\paragraph{Analytical Relationships:}
\[
\text{Operating Cycle} = \text{DSO} + \text{DOH}, \qquad
\text{Cash Conversion Cycle (CCC)} = \text{DSO} + \text{DOH} - \text{DPO}
\]
\begin{itemize}
  \item Shorter CCC = faster cash recovery and stronger liquidity.  
  \item Negative CCC (e.g., supermarkets) = collect cash before paying suppliers.
\end{itemize}

\paragraph{Analyst Notes:}
\begin{itemize}
  \item Declining turnover ratios → inefficient operations or over-investment in assets.  
  \item Compare all ratios against peers and over time for consistency.  
  \item Integrate findings with liquidity ratios to evaluate working-capital management.
\end{itemize}


\subsubsection*{3. Liquidity Ratios}

\paragraph{Purpose:}
\begin{itemize}
  \item Assess short-term solvency—ability to meet obligations with current assets.  
  \item Focus on availability, speed, and reliability of cash conversion.
\end{itemize}

\paragraph{Key Ratios and Interpretations:}
\begin{itemize}
  \item \textbf{Current Ratio:} 
  $\displaystyle \frac{\text{Current Assets}}{\text{Current Liabilities}}$  
  → Basic liquidity test; >1 = healthy; <1 = possible liquidity risk; too high = idle assets.

  \item \textbf{Quick (Acid-Test) Ratio:} 
  $\displaystyle \frac{\text{Cash + Marketable Securities + Receivables}}{\text{Current Liabilities}}$  
  → Excludes inventory; gauges true near-cash coverage.

  \item \textbf{Cash Ratio:} 
  $\displaystyle \frac{\text{Cash + Marketable Securities}}{\text{Current Liabilities}}$  
  → Most conservative; measures immediate liquidity capacity.

  \item \textbf{Defensive Interval Ratio (DIR):} 
  $\displaystyle \frac{\text{Cash + Marketable Securities + Receivables}}{\text{Daily Cash Expenditures}}$  
  where Daily Cash Expenditures = (COGS + SG\&A + R\&D – Depreciation)/365.  
  → Indicates how many days operations can run using only liquid assets.

  \item \textbf{Cash Conversion Cycle (CCC):} 
  $\displaystyle \text{DSO} + \text{DOH} - \text{DPO}$  
  → Shorter/negative cycle = better cash efficiency; longer cycle = more working-capital financing required.
\end{itemize}

\paragraph{Interpretive Highlights:}
\begin{itemize}
  \item \textbf{Current Ratio < 1:} Possible liquidity stress or very fast turnover.  
  \item \textbf{Large Current–Quick Gap:} Heavy reliance on inventory for liquidity.  
  \item \textbf{High DIR + Short CCC:} Excellent short-term cash management.  
  \item \textbf{CCC > Industry Average:} Excess capital tied in receivables/inventory.  
  \item Combine liquidity and activity analysis to diagnose working-capital efficiency.
\end{itemize}



\subsubsection*{4. Analytical Connections Between Activity and Liquidity}

\begin{itemize}
  \item Activity ratios explain why liquidity may be tight or strong:
        \begin{itemize}
            \item Slow receivables turn → high DSO → lower CFO → weaker liquidity ratios.
            \item Excess inventory → low turnover → cash locked in working capital.
            \item High payables turn → short DPO → reduced supplier financing.
        \end{itemize}
  \item Trend analysis of DSO, DOH, and DPO together offers a complete view of cash flow timing.
  \item Combine these ratios to understand the firm’s \textbf{operating cycle and working-capital efficiency.}
\end{itemize}


\subsubsection*{5. Illustrative Example: Cash Conversion Cycle}

\begin{itemize}
  \item \textbf{Given:} DSO = 40 days, DOH = 50 days, DPO = 30 days.  
  \item \textbf{Compute:}
        \[
        \text{CCC} = 40 + 50 – 30 = 60 \text{ days.}
        \]
  \item Interpretation: Firm recovers cash 60 days after paying suppliers. Shortening DSO or DOH improves liquidity.
\end{itemize}


\subsubsection*{6. Key Analyst Takeaways}

\begin{itemize}
  \item High activity ratios signal efficiency but may hide capacity risk or credit pressure.
  \item Liquidity ratios must be analyzed alongside cash flows and credit facilities.
  \item Trends matter more than levels – evaluate directional changes over time.
  \item Integrate activity and liquidity metrics to assess operating efficiency and financial flexibility.
  \item Cross-industry comparison requires adjusting for business model differences (e.g., retail vs. manufacturing).
\end{itemize}

\subsection*{Module 39.3: Financial Ratios, Part 2}
\textbf{LOS 39.b–c:} Calculate and interpret solvency and profitability ratios; describe relationships among ratios and evaluate a company using ratio analysis.

%=========================================================
\subsubsection*{1. Overview}

\begin{itemize}
  \item \textbf{Solvency ratios} measure long-term leverage and ability to meet debt obligations.  
  \item \textbf{Profitability ratios} assess how effectively a company generates earnings from sales, assets, and equity.  
  \item Combined interpretation provides a holistic view of financial risk and performance.
  \item Consistency of formulas and cross-sectional benchmarking is essential.
\end{itemize}

%=========================================================
\subsubsection*{2. Solvency Ratios}

\paragraph{Purpose:}
\begin{itemize}
  \item Assess a firm’s long-term financial leverage and its ability to meet debt obligations.  
  \item Indicate the sustainability of the capital structure and the adequacy of earnings to cover fixed charges.
\end{itemize}

\paragraph{Key Ratios and Interpretations:}
\begin{itemize}
  \item \textbf{Debt-to-Equity:} 
  $\displaystyle \frac{\text{Total Debt}}{\text{Total Shareholders' Equity}}$  
  → Measures reliance on debt vs. equity. High = greater risk; low = conservative structure.  
  *Exclude leases per CFA convention (Level I).*

  \item \textbf{Debt-to-Capital:} 
  $\displaystyle \frac{\text{Total Debt}}{\text{Total Debt + Preferred + Common Equity}}$  
  → Proportion of debt in total permanent financing; shows capital mix.

  \item \textbf{Debt-to-Assets:} 
  $\displaystyle \frac{\text{Total Debt}}{\text{Total Assets}}$  
  → Indicates percentage of assets financed by debt. Higher = more leverage and risk.

  \item \textbf{Financial Leverage Ratio (Leverage Multiplier):} 
  $\displaystyle \frac{\text{Average Total Assets}}{\text{Average Total Equity}}$  
  → Shows asset financing by equity. Ratio = 1 → all-equity firm. Higher → more leverage and higher ROE sensitivity.

  \item \textbf{Interest Coverage (Times Interest Earned):} 
  $\displaystyle \frac{\text{EBIT}}{\text{Interest Expense}}$  
  → Measures ability to meet interest payments. <2.5× = risk of default; higher = strong coverage.

  \item \textbf{Debt-to-EBITDA:} 
  $\displaystyle \frac{\text{Total Debt}}{\text{EBITDA}}$  
  → Approximate years to repay total debt from operating cash flow. Lower = stronger solvency.

  \item \textbf{Fixed-Charge Coverage:} 
  $\displaystyle \frac{\text{EBIT + Lease Payments}}{\text{Interest + Lease Payments}}$  
  → Incorporates lease obligations; important for airlines/logistics with high lease exposure. Low ratio = potential liquidity stress.
\end{itemize}

\paragraph{Analyst Notes:}
\begin{itemize}
  \item Firms with stable cash flows can safely carry more debt.  
  \item Focus on both \textbf{trend and level} of leverage and coverage ratios.  
  \item Compare across peers—treatment of leases and short-term debt affects comparability.  
  \item \textbf{Net Debt} = Total Debt – (Cash + Marketable Securities) → better measure of true leverage.  
  \item Combine solvency with profitability to assess value creation (ROIC vs. WACC).
\end{itemize}


%=========================================================
\subsubsection*{3. Profitability Ratios}

\paragraph{Purpose:}
\begin{itemize}
  \item Evaluate the firm’s ability to generate profits relative to sales, assets, and equity.  
  \item Capture management efficiency, pricing power, and cost control.  
  \item Support valuation and performance benchmarking.
\end{itemize}

\paragraph{Key Ratios and Interpretations:}
\begin{itemize}
  \item \textbf{Gross Profit Margin:} 
  $\displaystyle \frac{\text{Sales – COGS}}{\text{Sales}}$  
  → Measures production efficiency and pricing power. Low margin = cost or pricing pressure.

  \item \textbf{Operating Profit Margin:} 
  $\displaystyle \frac{\text{EBIT}}{\text{Sales}}$  
  → Operating efficiency before financing costs. Analysts may also use $\text{EBITDA}/\text{Sales}$ for cash-based view.

  \item \textbf{Pretax Profit Margin (EBT Margin):} 
  $\displaystyle \frac{\text{EBT}}{\text{Sales}} = \frac{\text{EBIT – Interest}}{\text{Sales}}$  
  → Isolates performance before taxes and highlights effect of financing structure.

  \item \textbf{Net Profit Margin:} 
  $\displaystyle \frac{\text{Net Income}}{\text{Sales}}$  
  → Overall profitability including all expenses and taxes. Focus on continuing operations for sustainability.

  \item \textbf{Return on Assets (ROA):} 
  $\displaystyle \frac{\text{Net Income}}{\text{Average Total Assets}}$  
  → Efficiency of asset use. Ignores financing mix; low ROA = poor efficiency.

  \item \textbf{Adjusted ROA (Before Interest):} 
  $\displaystyle \frac{\text{Net Income + Interest(1 – Tax Rate)}}{\text{Average Total Assets}}$  
  → Removes leverage distortion; reflects returns to all capital providers.

  \item \textbf{Operating ROA:} 
  $\displaystyle \frac{\text{EBIT}}{\text{Average Total Assets}}$  
  → Measures pure operating performance, unaffected by tax or capital structure.

  \item \textbf{Return on Invested Capital (ROIC):} 
  $\displaystyle \frac{\text{EBIT (1 – Tax Rate)}}{\text{Debt + Equity + Preferred}}$  
  → Return generated on long-term capital base. ROIC > WACC → value creation.

  \item \textbf{Return on Equity (ROE):} 
  $\displaystyle \frac{\text{Net Income}}{\text{Average Total Equity}}$  
  → Profitability to shareholders. High ROE may stem from leverage; assess with ROA.

  \item \textbf{Return on Common Equity (ROCE):} 
  $\displaystyle \frac{\text{Net Income – Preferred Dividends}}{\text{Average Common Equity}}$  
  → Profitability available exclusively to common shareholders. Used in DuPont analysis.
\end{itemize}

\paragraph{Interpretive Highlights:}
\begin{itemize}
  \item Declining margins → weaker cost control or competition pressure.  
  \item High ROE but weak ROA → excessive leverage driving equity returns.  
  \item Stable gross margin + falling net margin → rising non-operating or interest costs.  
  \item Use \textbf{ROIC vs. WACC} to assess whether the firm creates or destroys value.  
  \item Cross-check profitability trends with solvency and efficiency ratios for holistic insight.
\end{itemize}


%=========================================================
\subsubsection*{4. Example: Evaluating Sedgwick Company}

\paragraph{Given Ratios (Current Year):}
\begin{itemize}
  \item Current ratio ↓ but > industry average → liquidity adequate.  
  \item Total asset turnover ↓ → less efficient asset utilization.  
  \item Net profit margin ↓ and below industry → profitability weakness.  
  \item ROE ↓ but still > industry → leverage-driven performance.  
  \item Debt-to-equity ↓ but still >2× industry → gradual deleveraging trend.  
\end{itemize}

\paragraph{Analyst Conclusion:}
\begin{itemize}
  \item Sedgwick improving capital structure but still highly leveraged.  
  \item Profitability and efficiency deteriorating → possibly due to cost control or asset growth.  
  \item ROE outperformance stems from leverage, not operational superiority.  
  \item Analyst should monitor debt reduction progress and margin recovery.  
\end{itemize}

%=========================================================
\subsubsection*{5. Relationships Among Ratios (LOS 39.c)}

\paragraph{Example: Interpreting Trends Together}
\begin{itemize}
  \item Current ratio ↑ but quick ratio ↓ → inventory buildup.  
  \item Days Inventory ↑ → supports the inventory explanation.  
  \item DSO ↓ → faster cash collections.  
  \item Combined inference → firm offsetting poor inventory management by accelerating receivable collections.
\end{itemize}

\paragraph{Cross-Ratio Insights:}
\begin{itemize}
  \item \textbf{ROE = Net Profit Margin × Asset Turnover × Financial Leverage} (DuPont).  
  \item Decline in net margin or turnover may be offset by higher leverage → risky sustainability.  
  \item Relationship logic:
        \[
        \text{Profitability (ROA)} \times \text{Leverage (A/E)} = \text{ROE.}
        \]
  \item Evaluate whether ROE improvements stem from operations or leverage.
\end{itemize}

\paragraph{Comprehensive Evaluation Steps:}
\begin{enumerate}
  \item Compare ratios year-over-year and against industry benchmarks.  
  \item Analyze ratio interdependencies (e.g., liquidity → operations → solvency → profitability).  
  \item Adjust for nonrecurring items (discontinued operations, one-time gains).  
  \item Review consistency of ratio definitions and accounting methods.  
\end{enumerate}

%=========================================================
\subsubsection*{6. Key Analytical Takeaways}

\begin{itemize}
  \item Solvency ratios reveal risk exposure from leverage; coverage ratios test repayment capacity.  
  \item Profitability ratios identify efficiency and value creation across income layers.  
  \item Ratios must be interpreted together — single ratio analysis is insufficient.  
  \item Consider trends, peer benchmarks, and firm-specific strategy.  
  \item Declining efficiency + stable ROE → leverage-driven illusion of performance.  
  \item Analysts should adjust reported data for comparability and consistency across time.  
\end{itemize}

\subsection*{Module 39.4: DuPont Analysis}
\textbf{LOS 39.d:} Demonstrate the application of DuPont analysis of ROE and interpret effects of changes in its components.

%=========================================================
\subsubsection*{1. Concept Overview}

\begin{itemize}
  \item \textbf{Objective:} Decompose Return on Equity (ROE) into component ratios to identify drivers of profitability.  
  \item \textbf{Purpose:} Helps analysts determine whether changes in ROE stem from:
        \begin{itemize}
          \item Operational efficiency (profit margin),
          \item Asset utilization (turnover), or
          \item Financial leverage.
        \end{itemize}
  \item \textbf{Definition:} 
        \[
        \text{ROE} = \frac{\text{Net Income}}{\text{Average Shareholders’ Equity}}
        \]
  \item \textbf{Note:} DuPont is an analytical breakdown of ROE, not a separate calculation method.
\end{itemize}

%=========================================================
\subsubsection*{2. Original (3-Part) DuPont Equation}

\paragraph{Derivation:}
\[
\text{ROE} 
= \frac{\text{Net Income}}{\text{Sales}} 
\times \frac{\text{Sales}}{\text{Average Total Assets}} 
\times \frac{\text{Average Total Assets}}{\text{Average Equity}}
\]
\[
\text{ROE} = \text{Net Profit Margin} \times \text{Total Asset Turnover} \times \text{Financial Leverage}
\]

\paragraph{Interpretation:}
\begin{itemize}
  \item \textbf{Net Profit Margin (NI / Sales):} Measures profitability of each sales dollar.  
  \item \textbf{Total Asset Turnover (Sales / Assets):} Measures efficiency in asset utilization.  
  \item \textbf{Financial Leverage (Assets / Equity):} Captures the effect of financing mix on ROE.  
\end{itemize}

\paragraph{Analyst Insights:}
\begin{itemize}
  \item Decline in ROE must arise from one or more of:
        \begin{enumerate}
          \item Decrease in profit margin → lower profitability.  
          \item Decrease in asset turnover → inefficient use of assets.  
          \item Decrease in leverage → less debt financing (reduces risk but lowers ROE).
        \end{enumerate}
  \item This 3-factor decomposition reveals the firm’s profitability structure:
        \[
        \text{ROE} = \text{Operating Performance} \times \text{Efficiency} \times \text{Leverage}.
        \]
\end{itemize}

%=========================================================
\subsubsection*{3. Example: Original DuPont Analysis (Staret Inc.)}

\paragraph{Given:}
\begin{itemize}
  \item Stable ROE ≈ 18\% (over three years)
  \item Components (rounded):
        \begin{align*}
        20X3: &\quad 7.0\% \times 1.33 \times 1.93 = 18.1\%\\
        20X4: &\quad 6.4\% \times 1.21 \times 2.34 = 18.0\%\\
        20X5: &\quad 5.3\% \times 1.17 \times 2.78 = 17.4\%
        \end{align*}
\end{itemize}

\paragraph{Interpretation:}
\begin{itemize}
  \item Net margin ↓ and asset turnover ↓ → operational deterioration.  
  \item Leverage ↑ → compensates for declining margins to maintain ROE.  
  \item Higher leverage increases financial risk → analyst should evaluate debt capacity and cost of capital.  
  \item Trend: modest decline in ROE despite aggressive leveraging → potential warning signal.
\end{itemize}

%=========================================================
\subsubsection*{4. Extended (5-Part) DuPont Equation}

\paragraph{Decomposition:}
\[
\text{ROE} 
= \frac{\text{Net Income}}{\text{EBT}} 
\times \frac{\text{EBT}}{\text{EBIT}} 
\times \frac{\text{EBIT}}{\text{Sales}} 
\times \frac{\text{Sales}}{\text{Assets}} 
\times \frac{\text{Assets}}{\text{Equity}}
\]

\[
\text{ROE} = \text{Tax Burden} \times \text{Interest Burden} \times \text{EBIT Margin} \times \text{Asset Turnover} \times \text{Financial Leverage}
\]

\paragraph{Components:}
\begin{itemize}
  \item \textbf{Tax Burden:} $\displaystyle \frac{\text{Net Income}}{\text{EBT}} = (1 - \text{Tax Rate})$  
        → Lower ratio (higher taxes) ↓ ROE.
  \item \textbf{Interest Burden:} $\displaystyle \frac{\text{EBT}}{\text{EBIT}}$  
        → Measures effect of interest expense. Ratio ↓ when debt cost ↑.
  \item \textbf{EBIT Margin:} $\displaystyle \frac{\text{EBIT}}{\text{Sales}}$  
        → Operating profitability.
  \item \textbf{Asset Turnover:} $\displaystyle \frac{\text{Sales}}{\text{Assets}}$  
        → Asset utilization efficiency.
  \item \textbf{Financial Leverage:} $\displaystyle \frac{\text{Assets}}{\text{Equity}}$  
        → Magnifies both profits and losses.
\end{itemize}

\paragraph{Analyst Notes:}
\begin{itemize}
  \item Each component identifies a distinct source of ROE change:
        \begin{itemize}
          \item Lower \textbf{Tax Burden} → higher taxes → ↓ ROE.
          \item Lower \textbf{Interest Burden} → higher interest cost → ↓ ROE.
          \item Higher \textbf{Operating Margin or Turnover} → ↑ ROE.
        \end{itemize}
  \item High leverage does not always increase ROE—interest burden may offset benefits.
  \item Extended DuPont highlights how tax and financing efficiency affect shareholder returns.
\end{itemize}

%=========================================================
\subsubsection*{5. Example: Extended DuPont (Company A vs. Company B)}

\paragraph{Given Data:}
\begin{itemize}
  \item Company A:  
        \begin{itemize}
          \item ROE = 13.3\%, EBIT Margin = 7.0\%, Asset Turnover = 2.0,  
          \item Interest Burden = 85.7\%, Tax Burden = 66.7\%, Financial Leverage = 1.67.
        \end{itemize}
  \item Company B:  
        \begin{itemize}
          \item ROE = 24.0\%, EBIT Margin = 11.1\%, Asset Turnover = 3.0,  
          \item Interest Burden = 100\%, Tax Burden = 60.0\%, Financial Leverage = 1.2.
        \end{itemize}
\end{itemize}

\paragraph{Step-by-Step Analysis:}
\begin{itemize}
  \item \textbf{Profitability:} Company B has stronger EBIT margin (11.1\% vs. 7.0\%) → better cost control or pricing.  
  \item \textbf{Efficiency:} Higher asset turnover (3.0 vs. 2.0) → superior asset use or inventory management.  
  \item \textbf{Leverage:} Company B uses less leverage (1.2 vs. 1.67) but achieves higher ROE → operational drivers dominate.  
  \item \textbf{Interest Burden:} Company A’s 85.7\% vs. 100\% → A faces more debt cost pressure.  
  \item \textbf{Tax Burden:} Slightly lower for B (60\%) → marginal drag, but outweighed by stronger operations.  
\end{itemize}

\paragraph{Interpretation:}
\begin{itemize}
  \item Company B’s higher ROE (24\%) arises from better margins and turnover—not leverage.  
  \item Company A relies on leverage to offset weaker profitability—riskier structure.  
  \item Extended DuPont shows how \textbf{ROE quality} depends on operating fundamentals rather than leverage alone.
\end{itemize}

%=========================================================
\subsubsection*{6. Analytical Summary}

\paragraph{Core Insights:}
\begin{itemize}
  \item \textbf{3-Part DuPont:} Simplifies ROE into profitability × efficiency × leverage.  
  \item \textbf{5-Part DuPont:} Expands to reveal tax, financing, and operational impacts separately.
\end{itemize}

\paragraph{Interpretive Takeaways:}
\begin{itemize}
  \item \textbf{High ROE ≠ high quality.} Analysts must check whether growth stems from margin strength or excessive leverage.  
  \item \textbf{Declining margins or turnover} offset by rising leverage → increased financial risk.  
  \item \textbf{Sustainable ROE} comes from stable operating margin and efficient asset management, not from borrowing.  
  \item \textbf{Analytical Focus:}  
        \begin{enumerate}
          \item Monitor trends in component ratios year over year.  
          \item Compare with industry peers.  
          \item Assess risk-return balance—especially interest burden sensitivity.  
        \end{enumerate}
\end{itemize}

\paragraph{Formula Summary:}
\[
\boxed{
\begin{aligned}
\text{ROE (3-part)} &= 
\text{Net Profit Margin} \times \text{Asset Turnover} \times \text{Financial Leverage} \\
\text{ROE (5-part)} &= 
\text{Tax Burden} \times \text{Interest Burden} \times \text{EBIT Margin} \\
&\quad \times \text{Asset Turnover} \times \text{Financial Leverage}
\end{aligned}
}
\]


\subsection*{Module 39.5: Industry-Specific Financial Ratios and Forecasting}
\textbf{LOS 39.e–f:} Describe the use of industry-specific ratios and explain how ratio analysis can support forecasting and pro forma modeling.

%=========================================================
\subsubsection*{1. Industry-Specific Ratios Overview}

\paragraph{Purpose:}
\begin{itemize}
  \item Different industries emphasize different performance metrics due to varying operating models, asset bases, and risk profiles.
  \item Analysts identify key ratios most relevant to each sector’s value creation drivers (e.g., margins, efficiency, utilization).
  \item Comparative analysis must always be done within the same industry context.
\end{itemize}

\paragraph{General Insight:}
\begin{itemize}
  \item Ratios outside industry norms → potential inefficiency, mismanagement, or distinct business strategy.
  \item Understanding industry-specific KPIs is critical for accurate valuation and peer comparison.
\end{itemize}

%=========================================================
\subsubsection*{2. Examples of Industry-Specific Ratios}

\paragraph{Service and Consulting Firms:}
\begin{itemize}
  \item \textbf{Sales per Employee} = Revenue / Number of Employees  
        → Measures productivity and human capital efficiency.  
  \item \textbf{Net Income per Employee} = Net Income / Employees  
        → Reflects profitability per worker; useful in labor-intensive industries.
\end{itemize}

\paragraph{Retail and Restaurant Industry:}
\begin{itemize}
  \item \textbf{Same-Store Sales Growth (SSSG):}  
        Measures revenue growth excluding new stores → reflects customer retention and organic performance.  
        \[
        \text{SSSG} = \frac{\text{Current-Year Sales of Existing Stores} - \text{Prior-Year Sales of Same Stores}}{\text{Prior-Year Sales of Same Stores}}
        \]
  \item \textbf{Sales per Square Foot:}  
        \[
        \text{Sales per Square Foot} = \frac{\text{Total Sales}}{\text{Selling Area (sq. ft.)}}
        \]
        → Gauges store efficiency and space utilization.
  \item Decline in SSSG with store expansion may indicate \textit{cannibalization} among locations.
\end{itemize}

\paragraph{Hotel and Hospitality Industry:}
\begin{itemize}
  \item \textbf{Average Daily Rate (ADR):}  
        \[
        \text{ADR} = \frac{\text{Room Revenue}}{\text{Rooms Sold}}
        \]
        → Indicates pricing power and yield management efficiency.
  \item \textbf{Occupancy Rate:}  
        \[
        \text{Occupancy Rate} = \frac{\text{Rooms Sold}}{\text{Rooms Available}}
        \]
        → Measures utilization level of hotel capacity.
  \item \textbf{Revenue per Available Room (RevPAR):}  
        \[
        \text{RevPAR} = \text{ADR} \times \text{Occupancy Rate}
        \]
        → Integrates pricing and occupancy into one performance indicator.
\end{itemize}

\paragraph{Subscription-Based Businesses (Streaming, Telecom, SaaS):}
\begin{itemize}
  \item \textbf{Average Revenue per User (ARPU):}  
        \[
        \text{ARPU} = \frac{\text{Total Revenue}}{\text{Average Number of Active Users}}
        \]
        → Tracks monetization efficiency and pricing trends.
  \item \textbf{Churn Rate:}  
        \[
        \text{Churn Rate} = \frac{\text{Customers Lost}}{\text{Total Customers at Start of Period}}
        \]
        → Measures customer retention and lifetime value risk.
\end{itemize}

\paragraph{Banking and Financial Institutions:}
\begin{itemize}
  \item \textbf{Capital Adequacy Ratio (CAR):}  
        \[
        \text{CAR} = \frac{\text{Capital}}{\text{Risk-Weighted Assets}}
        \]
        → Ensures sufficient capital buffer to absorb losses (Basel III standard).
  \item \textbf{Liquidity Coverage Ratio (LCR):}  
        \[
        \text{LCR} = \frac{\text{High-Quality Liquid Assets}}{\text{Net Cash Outflows (30 days)}}
        \]
        → Indicates short-term liquidity resilience.
  \item \textbf{Net Interest Margin (NIM):}  
        \[
        \text{NIM} = \frac{\text{Interest Income – Interest Expense}}{\text{Average Interest-Earning Assets}}
        \]
        → Measures spread profitability of lending operations.
  \item \textbf{Loan-to-Deposit Ratio (LDR):}  
        \[
        \text{LDR} = \frac{\text{Total Loans}}{\text{Total Deposits}}
        \]
        → Evaluates liquidity and lending aggressiveness.
\end{itemize}

\paragraph{Insurance Companies:}
\begin{itemize}
  \item \textbf{Loss Ratio:} $\displaystyle \frac{\text{Claims Paid}}{\text{Premiums Earned}}$ → Measures underwriting risk.  
  \item \textbf{Expense Ratio:} $\displaystyle \frac{\text{Underwriting Expenses}}{\text{Premiums Earned}}$ → Operating efficiency.  
  \item \textbf{Combined Ratio:} = Loss Ratio + Expense Ratio  
        → <100\% = underwriting profit; >100\% = underwriting loss.
\end{itemize}

\paragraph{Energy and Utilities:}
\begin{itemize}
  \item \textbf{Reserve Replacement Ratio:} $\displaystyle \frac{\text{Additions to Reserves}}{\text{Production}}$ → Indicates sustainability of resource base.  
  \item \textbf{Load Factor (Utilities):} $\displaystyle \frac{\text{Average Load}}{\text{Peak Load}}$ → Measures utilization efficiency.
\end{itemize}

\paragraph{Capital Adequacy and Risk Management:}
\begin{itemize}
  \item Regulators impose \textbf{minimum capital ratios} and \textbf{reserve requirements} to reduce contagion risk.
  \item \textbf{Value at Risk (VaR):}  
        Statistical measure estimating potential loss over a given period at a specific confidence level.  
        → Example: “99\% one-day VaR = \$10 million” means a 1\% chance of losing more than \$10M in one day.
\end{itemize}

%=========================================================
\subsubsection*{3. Business Risk and Volatility Measures}

\paragraph{Purpose:}
\begin{itemize}
  \item Quantify variability in revenue, operating income, or net income → indicates earnings stability and risk exposure.
\end{itemize}

\paragraph{Key Metric – Coefficient of Variation (CV):}
\[
\text{CV} = \frac{\text{Standard Deviation}}{\text{Expected Value}}
\]
\begin{itemize}
  \item Standard deviation measures volatility; CV adjusts for firm size → makes comparisons meaningful.  
  \item Lower CV → more stable, predictable performance.  
  \item Compare CV across time or among peers to assess relative business risk.
\end{itemize}

%=========================================================
\subsubsection*{4. Ratio Analysis in Forecasting and Modeling (LOS 39.f)}

\paragraph{Purpose:}
\begin{itemize}
  \item Use ratios and common-size data to prepare \textbf{pro forma financial statements} for forecasting future performance.  
  \item Link projected income statement and balance sheet values through stable ratio relationships.
\end{itemize}

\paragraph{Steps for Forecasting:}
\begin{enumerate}
  \item \textbf{Estimate Sales:} Base projection on market outlook or trend growth.  
  \item \textbf{Apply Historical Ratios:}  
        \begin{itemize}
          \item COGS/Sales → forecast cost of goods sold.  
          \item Operating Margin → forecast EBIT.  
          \item Interest Coverage or Tax Rate → estimate net income.  
        \end{itemize}
  \item \textbf{Forecast Balance Sheet:}  
        Use turnover ratios (e.g., receivables, inventory) to project asset and liability levels.
  \item \textbf{Prepare Pro Forma Statements:}  
        Integrate assumptions into forward-looking financials.
\end{enumerate}

\paragraph{Analytical Techniques for Forecast Variability:}
\begin{itemize}
  \item \textbf{Sensitivity Analysis:}  
        “What if” analysis — changes one input at a time to test impact on outcome.  
        Example: What if sales growth = 3\% instead of 5\%?
  \item \textbf{Scenario Analysis:}  
        Evaluates predefined scenarios (e.g., optimistic, base, pessimistic) for multiple variable combinations.
  \item \textbf{Simulation (Monte Carlo):}  
        Generates a probability distribution of outcomes by repeatedly sampling from input distributions.  
        → Provides full range of possible earnings forecasts.
\end{itemize}

\paragraph{Interpretive Notes:}
\begin{itemize}
  \item Stable ratio assumptions simplify forecasts but may ignore structural changes.  
  \item Ratio analysis integrates accounting data with economic expectations to create dynamic projections.  
  \item Pro forma forecasts aid valuation models (e.g., DCF, residual income, EVA).
\end{itemize}

%=========================================================
\subsubsection*{5. Key Concept Recap}

\begin{itemize}
  \item \textbf{Industry Ratios:} Tailored to the sector’s economics and operating model.  
  \item \textbf{Financial Institutions:} Focus on capital adequacy, liquidity, and interest spreads.  
  \item \textbf{Business Risk:} Coefficient of Variation measures volatility per unit of expected value.  
  \item \textbf{Forecasting:} Ratio analysis underpins pro forma modeling and earnings sensitivity evaluation.  
  \item \textbf{Analytical Integration:} Combine operational, financial, and risk metrics to evaluate performance consistency and future profitability.
\end{itemize}

\paragraph{Formula Summary:}
\[
\boxed{
\begin{aligned}
\text{Same-Store Sales Growth} &= \frac{\text{Sales}_{t} - \text{Sales}_{t-1}}{\text{Sales}_{t-1}} \\
\text{ADR} &= \frac{\text{Room Revenue}}{\text{Rooms Sold}}, \quad
\text{RevPAR} = \text{ADR} \times \text{Occupancy Rate} \\
\text{ARPU} &= \frac{\text{Revenue}}{\text{Active Users}}, \quad
\text{CAR} = \frac{\text{Capital}}{\text{Risk-Weighted Assets}} \\
\text{Net Interest Margin} &= \frac{\text{Interest Income – Interest Expense}}{\text{Interest-Earning Assets}} \\
\text{Coefficient of Variation} &= \frac{\sigma}{\mu}
\end{aligned}
}
\]


\subsection*{Module 40.1: Financial Statement Modeling}
\textbf{LOS 40.a–40.e:} Develop a sales-based pro forma model; explain behavioral forecasting biases; describe competitive forces, inflation effects on costs, and forecast horizon selection.

%=========================================================
\subsubsection*{1. Sales-Based Pro Forma Company Model (LOS 40.a)}

\paragraph{Purpose:}
\begin{itemize}
  \item A \textbf{pro forma model} projects future financial statements based on forecasted \textbf{sales growth}.
  \item Useful for valuation, scenario testing, and credit analysis.
\end{itemize}

\paragraph{Step-by-Step Process:}
\begin{enumerate}
  \item \textbf{Estimate Revenue Growth:}  
        Based on market share, GDP trends, or sector growth expectations.  
        \(\text{Future Sales} = \text{Current Sales} \times (1 + g_{\text{sales}})\)
  \item \textbf{Estimate COGS:}  
        Typically as a fixed \% of sales or based on input cost trends.
  \item \textbf{Estimate SG\&A:}  
        Treated as fixed, semi-variable, or scaling with sales.
  \item \textbf{Estimate Financing Costs:}  
        Interest expense = debt × average interest rate.  
        Adjust for capital structure changes or new capex funding.
  \item \textbf{Estimate Taxes:}  
        Use historical effective rates adjusted for jurisdictional mix and deferred taxes.
  \item \textbf{Model Balance Sheet:}  
        Derive \textbf{working capital accounts} (A/R, inventory, payables) from turnover ratios:
        \[
        \begin{aligned}
        \text{A/R} &= \frac{\text{Sales}}{\text{Receivables Turnover}}, \quad
        \text{Inventory} = \frac{\text{COGS}}{\text{Inventory Turnover}}, \\
        \text{A/P} &= \frac{\text{COGS}}{\text{Payables Turnover}}
        \end{aligned}
        \]

  \item \textbf{Model Capital Expenditures \& Depreciation:}  
        \[
        \text{Net PP\&E}_{t} = \text{PP\&E}_{t-1} + \text{CapEx} - \text{Depreciation}
        \]
  \item \textbf{Build Pro Forma Cash Flow Statement:}  
        Derived from projected income statement and balance sheet changes.  
        Check \(\Delta \text{Cash} = \text{CFO} + \text{CFI} + \text{CFF}\)
\end{enumerate}

\paragraph{Analyst Insights:}
\begin{itemize}
  \item Ensure internal consistency between income statement, balance sheet, and cash flow.
  \item Forecast drivers (sales, margins, working capital days) rather than line items.
  \item Validate with historical ratios and peer benchmarks.
\end{itemize}

%=========================================================
\subsubsection*{2. Behavioral Factors Affecting Analyst Forecasts (LOS 40.b)}

\paragraph{1. Overconfidence Bias}
\begin{itemize}
  \item Analysts overestimate accuracy and underestimate uncertainty.  
  \item Leads to overly narrow forecast ranges.  
  \item \textbf{Remedy:}  
        \begin{itemize}
          \item Perform post-analysis of forecast errors.  
          \item Use \textbf{scenario analysis} with wide confidence intervals.  
