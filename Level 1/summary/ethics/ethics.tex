\documentclass[12pt]{article}
\usepackage{amsmath}
\usepackage{geometry}
\usepackage{graphicx} % for including images and figures
\usepackage{booktabs}
\usepackage{caption}
\usepackage{titlesec}
\usepackage{float}
\usepackage{makecell}
\usepackage{tabularx}
\usepackage{enumitem}
\usepackage[utf8]{inputenc}
\usepackage{textcomp}
\usepackage{adjustbox}  % put in preamble
\usepackage{array}      % for >{\raggedright\arraybackslash}X

\geometry{margin=1in}

% Define an X column that is ragged-right and allows line breaks
\newcolumntype{Y}{>{\raggedright\arraybackslash}X}

\title{Ethics}
\author{}
\date{}

\begin{document}
\maketitle

\section*{MODULE 89.1: Ethics and Trust}

\subsection*{LOS 89.a: Explain Ethics}

\begin{itemize}
    \item \textbf{Definition:}  
    Ethics refers to a set of shared beliefs about what behavior is \textbf{good or acceptable} and what is \textbf{bad or unacceptable.}

    \item \textbf{Ethical Conduct:}  
    Behavior that:
    \begin{enumerate}
        \item Follows moral principles and social expectations.  
        \item Improves outcomes for stakeholders (clients, coworkers, employers, and the investment community).  
        \item Balances self-interest with the interests of others.
    \end{enumerate}

    \item \textbf{Stakeholder-Centric View:}  
    Ethical behavior aims to enhance long-term trust and outcomes for all parties affected by one’s decisions.
\end{itemize}

---

\subsection*{LOS 89.b: Role of a Code of Ethics in Defining a Profession}

\begin{itemize}
    \item \textbf{Code of Ethics:} A written set of moral principles that:
    \begin{enumerate}
        \item Defines acceptable behavior.  
        \item Communicates the values and expectations of a group or organization.  
        \item Provides general guidance for decision-making.
    \end{enumerate}

    \item Some codes include \textbf{rules or standards} requiring a minimum level of ethical conduct.

    \item \textbf{Profession:}  
    A group of individuals with specialized knowledge who:
    \begin{itemize}
        \item Serve others honestly and competently.  
        \item Agree to abide by a professional code of ethics.
    \end{itemize}

    \item \textbf{Purpose:}  
    Codes of ethics help:
    \begin{itemize}
        \item Enhance public confidence in the profession.  
        \item Signal commitment to integrity and competence.  
        \item Encourage consistent behavior among members.
    \end{itemize}
\end{itemize}

---

\subsection*{LOS 89.c: Professions and How They Establish Trust}

\begin{itemize}
    \item \textbf{Definition:}  
    A profession is an occupational group (e.g., doctors, lawyers, investment managers) that:
    \begin{enumerate}
        \item Requires specialized expert knowledge.  
        \item Emphasizes ethical conduct and service to society.
    \end{enumerate}

    \item \textbf{Key Characteristics:}
    \begin{itemize}
        \item Written \textbf{code of ethics} and professional standards.  
        \item Regulatory or governing body.  
        \item Focus on \textbf{client interests and societal benefit.}  
        \item Continuing education requirements.  
        \item Emphasis on integrity and public trust.
    \end{itemize}

    \item \textbf{How Professions Build Trust:}
    \begin{enumerate}
        \item Set and enforce high standards of conduct.  
        \item Monitor professional behavior.  
        \item Maintain competence through ongoing education.  
        \item Prioritize client interests.  
        \item Mentor others and promote professional excellence.
    \end{enumerate}
\end{itemize}

---

\subsection*{LOS 89.d: Need for High Ethical Standards in Investment Management}

\begin{itemize}
    \item \textbf{Special Responsibility:}  
    Investment professionals are \textbf{entrusted with clients’ wealth} and must:
    \begin{itemize}
        \item Protect and grow client assets responsibly.  
        \item Provide trustworthy, transparent information.
    \end{itemize}

    \item \textbf{Why Ethics Matter More in Finance:}
    \begin{itemize}
        \item Services are \textbf{intangible}, so clients cannot easily assess quality.  
        \item Trust is critical for maintaining confidence and long-term relationships.
    \end{itemize}

    \item \textbf{Consequences of Unethical Behavior:}
    \begin{enumerate}
        \item Damages client wealth and confidence.  
        \item Erodes firm and industry reputation.  
        \item Reduces capital flow and increases cost of capital.  
        \item Causes misallocation of capital, reducing economic growth.  
        \item Adds perceived risk, lowering overall market efficiency.
    \end{enumerate}
\end{itemize}

---

\subsection*{LOS 89.e: Professionalism in Investment Management}

\begin{itemize}
    \item \textbf{Definition:}  
    Acting with integrity, competence, diligence, and respect — using specialized expertise to serve clients’ best interests.

    \item \textbf{Importance:}  
    Clients often lack the financial knowledge to assess recommendations or fees. Professionals must therefore act ethically and transparently.

    \item \textbf{Two Standards:}
    \begin{center}
    \begin{tabular}{|l|p{10cm}|}
    \hline
    \textbf{Standard} & \textbf{Description} \\
    \hline
    Suitability Standard & Requires recommendations suitable for client’s risk tolerance and return objectives. \\
    \hline
    Fiduciary Standard & Requires acting in the \textbf{best interest of the client}, prioritizing client welfare above self-interest. \\
    \hline
    \end{tabular}
    \end{center}

    \item \textbf{Ethical professionalism} demands fiduciary-like care, even when the legal requirement is only suitability.
\end{itemize}

---

\subsection*{LOS 89.f: Challenges to Ethical Behavior}

\begin{itemize}
    \item \textbf{Overconfidence in One’s Own Ethics:}  
    Individuals often overrate their moral integrity.

    \item \textbf{Situational Influences (External Factors):}
    \begin{itemize}
        \item Social pressure or loyalty to employer/supervisor.  
        \item Short-term rewards or prestige.  
        \item Organizational culture focused solely on compliance (“what can I do”) rather than principles (“what should I do”).
    \end{itemize}

    \item \textbf{Internal vs. External Influences:}
    \begin{center}
    \begin{tabular}{|l|l|}
    \hline
    \textbf{Internal (Personal Traits)} & \textbf{External (Situational Factors)} \\
    \hline
    Values, personality, moral beliefs & Corporate culture, peer pressure, incentives \\
    \hline
    \end{tabular}
    \end{center}

    \item \textbf{Key Challenge:}  
    Acting ethically despite pressures or incentives to the contrary.
\end{itemize}

---

\subsection*{LOS 89.g: Ethical Standards vs. Legal Standards}

\begin{itemize}
    \item \textbf{Distinction:}
    \begin{center}
    \begin{tabular}{|l|l|}
    \hline
    \textbf{Ethical Standards} & \textbf{Legal Standards} \\
    \hline
    Based on moral principles and societal values & Based on codified laws and regulations \\
    \hline
    Often set a \textbf{higher} standard than law & Define minimum acceptable behavior \\
    \hline
    Requires judgment and moral reasoning & Requires compliance and documentation \\
    \hline
    \end{tabular}
    \end{center}

    \item \textbf{Examples:}
    \begin{itemize}
        \item Whistle-blowing may be \textit{illegal} but \textbf{ethical}.  
        \item Promoting a relative’s firm without disclosure may be \textit{legal} but \textbf{unethical}.
    \end{itemize}

    \item \textbf{Ethics → Law Relationship:}
    \begin{itemize}
        \item Laws often follow major ethical failures (e.g.,  
        1933 Securities Act after 1929 crash,  
        Sarbanes–Oxley after Enron,  
        Dodd–Frank after 2008 crisis).  
        \item Laws evolve, but ethics require continuous moral judgment.
    \end{itemize}
\end{itemize}

---

\subsection*{LOS 89.h: Framework for Ethical Decision-Making}

\textbf{Purpose:}  
Integrate ethics into all decision processes to improve outcomes, accountability, and consistency.

\textbf{Four-Step Framework:}

\begin{enumerate}
    \item \textbf{Identify:}
    \begin{itemize}
        \item Relevant facts.  
        \item Stakeholders and duties owed.  
        \item Ethical principles and conflicts of interest.
    \end{itemize}
    \item \textbf{Consider:}
    \begin{itemize}
        \item Situational influences and personal biases.  
        \item Possible alternative actions.  
        \item Seek guidance (e.g., mentors, compliance, or legal advisors).
    \end{itemize}
    \item \textbf{Decide and Act:}
    \begin{itemize}
        \item Choose the most ethical and appropriate action.  
        \item Evaluate both short- and long-term effects.
    \end{itemize}
    \item \textbf{Reflect:}
    \begin{itemize}
        \item Evaluate the outcome — was it as anticipated?  
        \item Analyze why or why not; adjust future behavior.
    \end{itemize}
\end{enumerate}

\textbf{Benefits of Using a Framework:}
\begin{itemize}
    \item Clarifies conflicts of interest.  
    \item Encourages broader perspective and deliberation.  
    \item Reduces unintended ethical consequences.  
    \item Promotes consistency and transparency in decision-making.
\end{itemize}

---

\subsection*{Summary Table: Ethics and Trust in Investment Management}

\begin{center}
\begin{tabular}{|l|p{10cm}|}
\hline
\textbf{Concept} & \textbf{Key Points and Examples} \\
\hline
\textbf{Ethics} & Shared beliefs defining acceptable vs. unacceptable conduct. Ethical behavior benefits all stakeholders. \\
\hline
\textbf{Code of Ethics} & Written moral guide; communicates professional values and minimum standards. \\
\hline
\textbf{Profession} & Specialized knowledge, ethical standards, service to society, continuing education. \\
\hline
\textbf{Trust in Professions} & Built through expertise, ethical standards, client-first focus, oversight, and mentorship. \\
\hline
\textbf{High Standards in Finance} & Needed due to client trust, intangibility of services, and economy-wide consequences of unethical acts. \\
\hline
\textbf{Professionalism} & Suitability vs. fiduciary standard — fiduciary duty requires acting in clients’ best interests. \\
\hline
\textbf{Challenges to Ethics} & Overconfidence, social pressure, monetary incentives, short-term focus. \\
\hline
\textbf{Ethics vs. Law} & Ethics are broader; laws follow ethical lapses. Some actions may be legal yet unethical or vice versa. \\
\hline
\textbf{Decision Framework} & Identify → Consider → Decide → Reflect; integrates ethics into organizational culture. \\
\hline
\end{tabular}
\end{center}

---

\subsection*{Key Takeaways}

\begin{itemize}
    \item Ethics underpin professionalism, trust, and long-term market integrity.
    \item Ethical standards go beyond legal compliance; they require moral reasoning.
    \item Investment professionals must balance self-interest with duty to clients, employers, and society.
    \item Establishing an ethical decision-making framework fosters consistency, transparency, and accountability.
    \item Trust, once lost, is difficult to regain — maintaining it is a fundamental ethical obligation.
\end{itemize}

\section*{MODULE 90.1: The CFA Institute Code and Standards}

\subsection*{LOS 90.a: The CFA Institute Professional Conduct Program (PCP)}

\textbf{Overview:}
\begin{itemize}
    \item The \textbf{CFA Institute Professional Conduct Program (PCP)} enforces the Code of Ethics and Standards of Professional Conduct.
    \item It is governed by:
    \begin{enumerate}
        \item The \textbf{CFA Institute Bylaws}, and
        \item The \textbf{Rules of Procedure for Proceedings Related to Professional Conduct}.
    \end{enumerate}
    \item Core principles:
    \begin{itemize}
        \item \textbf{Fairness} — ensuring due process for all members and candidates.
        \item \textbf{Confidentiality} — maintaining privacy throughout investigations.
    \end{itemize}
\end{itemize}

\textbf{Governance Structure:}
\begin{itemize}
    \item The \textbf{CFA Institute Board of Governors} has overall responsibility.
    \item The \textbf{Disciplinary Review Committee (DRC)} enforces the Code and Standards.
    \item The \textbf{Professional Conduct staff} carries out inquiries and investigations.
\end{itemize}

\textbf{Triggers for Inquiry:}
\begin{enumerate}
    \item \textbf{Self-disclosure:} Members/candidates report civil litigation, criminal investigations, or written complaints in their annual Professional Conduct Statement.
    \item \textbf{Written complaints:} Submitted by clients, employers, or other third parties.
    \item \textbf{Public information:} Reports of misconduct from media or public sources.
    \item \textbf{Exam violations:} Reported by CFA exam proctors.
    \item \textbf{Social media or exam analysis:} Detected by CFA Institute’s internal monitoring.
\end{enumerate}

\textbf{Investigation Process:}
\begin{itemize}
    \item The Professional Conduct staff may:
    \begin{enumerate}
        \item Request a written explanation from the member or candidate.
        \item Interview the subject, complainant, or third parties.
        \item Collect relevant records and documentation.
    \end{enumerate}

    \item Possible outcomes:
    \begin{enumerate}
        \item No disciplinary action.
        \item Issue of a \textbf{cautionary letter}.
        \item Proposal of disciplinary sanctions.
    \end{enumerate}
\end{itemize}

\textbf{Hearing and Sanctions:}
\begin{itemize}
    \item If a sanction is proposed, the member/candidate can:
    \begin{itemize}
        \item \textbf{Accept} the sanction (case closed), or
        \item \textbf{Reject} it — leading to a hearing before a \textbf{Disciplinary Review Panel} composed of CFA Institute members.
    \end{itemize}
    \item Possible sanctions include:
    \begin{itemize}
        \item Public censure or condemnation.
        \item Suspension or revocation of membership.
        \item Suspension from the CFA Program.
    \end{itemize}
\end{itemize}

---

\subsection*{LOS 90.b: The Six Components of the Code of Ethics and the Seven Standards of Professional Conduct}

\textbf{Six Components of the CFA Institute Code of Ethics:}

\begin{enumerate}
    \item \textbf{Act with integrity, competence, diligence, respect, and in an ethical manner} with the public, clients, employers, colleagues, and market participants.
    \item \textbf{Place the integrity of the investment profession and clients’ interests above personal interests.}
    \item \textbf{Use reasonable care and independent professional judgment} in analysis, recommendations, and actions.
    \item \textbf{Practice and encourage others to practice} in a professional and ethical manner to enhance the reputation of the profession.
    \item \textbf{Promote the integrity and viability of global capital markets} for the benefit of society.
    \item \textbf{Maintain and improve professional competence} and strive to improve the competence of other investment professionals.
\end{enumerate}

\bigskip

\textbf{Seven Standards of Professional Conduct:}

\begin{center}
\begin{tabular}{|p{5cm}|p{10cm}|}
\hline
\textbf{Standard} & \textbf{Description} \\
\hline
I. Professionalism & Laws, independence, honesty, and objectivity in all conduct. \\
\hline
II. Integrity of Capital Markets & Avoid acting on material nonpublic information and avoid market manipulation. \\
\hline
III. Duties to Clients & Prioritize client interests, ensure fair dealing, suitability, accurate performance reporting, and confidentiality. \\
\hline
IV. Duties to Employers & Loyalty to employer, proper compensation disclosure, and supervisory responsibilities. \\
\hline
V. Investment Analysis, Recommendations, and Actions & Diligent, independent research; transparent communication; and record retention. \\
\hline
VI. Conflicts of Interest & Full disclosure, transaction priority, and referral fee transparency. \\
\hline
VII. Responsibilities as a CFA Member or Candidate & Uphold CFA Institute reputation; accurate references to CFA designation or program. \\
\hline
\end{tabular}
\end{center}

---

\subsection*{LOS 90.c: Ethical Responsibilities Under the Code and Standards}

\textbf{Detailed Breakdown of Standards and Key Ethical Responsibilities:}

% Replace the table with this structured, non-table LaTeX version.
% Uses enumitem for compact nested lists (add \usepackage{enumitem} if not already in the preamble).
\subsection*{CFA Code of Ethics and Standards — Key Requirements}

\begin{enumerate}[label=\Roman*., leftmargin=*, noitemsep, topsep=6pt]
  \item \textbf{PROFESSIONALISM}
    \begin{enumerate}[label=\Alph*., leftmargin=*, noitemsep, topsep=3pt]
      \item \textbf{Knowledge of the Law}
        \begin{itemize}[noitemsep, topsep=0pt, leftmargin=*]
          \item Comply with all applicable laws, regulations, and the Code and Standards.
          \item When conflicts exist, follow the stricter law or regulation.
          \item Do not knowingly participate in violations; dissociate from illegal or unethical activities.
        \end{itemize}

      \item \textbf{Independence and Objectivity}
        \begin{itemize}[noitemsep, topsep=0pt, leftmargin=*]
          \item Maintain objectivity and professional independence.
          \item Avoid offering, soliciting, or accepting gifts that could impair integrity (e.g., expensive trips from brokers).
          \item Disclose potential conflicts to employers.
        \end{itemize}

      \item \textbf{Misrepresentation}
        \begin{itemize}[noitemsep, topsep=0pt, leftmargin=*]
          \item Do not misstate or omit material facts in research, performance data, or marketing.
          \item Avoid plagiarism and exaggeration.
        \end{itemize}

      \item \textbf{Misconduct}
        \begin{itemize}[noitemsep, topsep=0pt, leftmargin=*]
          \item No dishonesty, fraud, or deceit in any professional capacity.
          \item Avoid personal conduct that reflects poorly on professional reputation.
        \end{itemize}
    \end{enumerate}

  \item \textbf{INTEGRITY OF CAPITAL MARKETS}
    \begin{enumerate}[label=\Alph*., leftmargin=*, noitemsep, topsep=3pt]
      \item \textbf{Material Nonpublic Information}
        \begin{itemize}[noitemsep, topsep=0pt, leftmargin=*]
          \item Do not act or cause others to act on material nonpublic information (insider trading).
          \item Example: trading on leaked earnings information violates this standard.
        \end{itemize}

      \item \textbf{Market Manipulation}
        \begin{itemize}[noitemsep, topsep=0pt, leftmargin=*]
          \item Avoid practices that distort market prices or volumes (e.g., wash trades, rumor-spreading).
          \item Manipulation is unethical even if temporarily profitable.
        \end{itemize}
    \end{enumerate}

  \item \textbf{DUTIES TO CLIENTS}
    \begin{enumerate}[label=\Alph*., leftmargin=*, noitemsep, topsep=3pt]
      \item \textbf{Loyalty, Prudence, and Care}
        \begin{itemize}[noitemsep, topsep=0pt, leftmargin=*]
          \item Act for clients' benefit with diligence and prudence; prioritize client interests over employer and self.
          \item Example: allocate IPO shares fairly among clients, not to personal accounts.
        \end{itemize}

      \item \textbf{Fair Dealing}
        \begin{itemize}[noitemsep, topsep=0pt, leftmargin=*]
          \item Deal fairly and objectively with all clients; distribute recommendations and opportunities fairly.
        \end{itemize}

      \item \textbf{Suitability}
        \begin{itemize}[noitemsep, topsep=0pt, leftmargin=*]
          \item Understand clients' risk tolerance, financial situation, and objectives.
          \item Recommend only suitable investments in the context of the total portfolio; for discretionary accounts, follow client mandates or written objectives.
        \end{itemize}

      \item \textbf{Performance Presentation}
        \begin{itemize}[noitemsep, topsep=0pt, leftmargin=*]
          \item Present performance data that is fair, accurate, and complete; avoid cherry‑picking or misleading composites.
        \end{itemize}

      \item \textbf{Preservation of Confidentiality}
        \begin{itemize}[noitemsep, topsep=0pt, leftmargin=*]
          \item Maintain client confidentiality unless: (1) information involves illegal acts, (2) disclosure is required by law, or (3) the client consents.
        \end{itemize}
    \end{enumerate}

  \item \textbf{DUTIES TO EMPLOYERS}
    \begin{enumerate}[label=\Alph*., leftmargin=*, noitemsep, topsep=3pt]
      \item \textbf{Loyalty}
        \begin{itemize}[noitemsep, topsep=0pt, leftmargin=*]
          \item Act for the employer's benefit; avoid harming the firm and do not misuse confidential information.
          \item Disclose independent practice or outside activities to the employer.
        \end{itemize}

      \item \textbf{Additional Compensation Arrangements}
        \begin{itemize}[noitemsep, topsep=0pt, leftmargin=*]
          \item Do not accept gifts, benefits, or compensation that could create conflicts without written consent from all parties (employer and client).
        \end{itemize}

      \item \textbf{Responsibilities of Supervisors}
        \begin{itemize}[noitemsep, topsep=0pt, leftmargin=*]
          \item Ensure subordinates comply with laws, regulations, and the Code and Standards; implement and enforce compliance procedures.
        \end{itemize}
    \end{enumerate}

  \item \textbf{INVESTMENT ANALYSIS, RECOMMENDATIONS, AND ACTIONS}
    \begin{enumerate}[label=\Alph*., leftmargin=*, noitemsep, topsep=3pt]
      \item \textbf{Diligence and Reasonable Basis}
        \begin{itemize}[noitemsep, topsep=0pt, leftmargin=*]
          \item Use thorough research and adequate due diligence before making recommendations; maintain independence and objectivity in analysis.
        \end{itemize}

      \item \textbf{Communication with Clients}
        \begin{itemize}[noitemsep, topsep=0pt, leftmargin=*]
          \item Disclose the investment process, significant risks, and limitations; distinguish clearly between facts and opinions.
          \item Update clients about material changes in methods or significant assumptions.
        \end{itemize}

      \item \textbf{Record Retention}
        \begin{itemize}[noitemsep, topsep=0pt, leftmargin=*]
          \item Maintain records that support analysis, recommendations, and actions; retain records for the period required by law or at least seven years (best practice).
        \end{itemize}
    \end{enumerate}

  \item \textbf{CONFLICTS OF INTEREST}
    \begin{enumerate}[label=\Alph*., leftmargin=*, noitemsep, topsep=3pt]
      \item \textbf{Disclosure of Conflicts}
        \begin{itemize}[noitemsep, topsep=0pt, leftmargin=*]
          \item Disclose all matters that could impair independence or objectivity using clear, plain language.
        \end{itemize}

      \item \textbf{Priority of Transactions}
        \begin{itemize}[noitemsep, topsep=0pt, leftmargin=*]
          \item Client and employer trades take priority over personal trades; personal trades allowed only if they do not disadvantage clients.
        \end{itemize}

      \item \textbf{Referral Fees}
        \begin{itemize}[noitemsep, topsep=0pt, leftmargin=*]
          \item Disclose to employer, client, and prospective clients any compensation for product or service referrals.
        \end{itemize}
    \end{enumerate}

  \item \textbf{RESPONSIBILITIES AS CFA INSTITUTE MEMBER OR CANDIDATE}
    \begin{enumerate}[label=\Alph*., leftmargin=*, noitemsep, topsep=3pt]
      \item \textbf{Conduct as Participants in CFA Programs}
        \begin{itemize}[noitemsep, topsep=0pt, leftmargin=*]
          \item Do not engage in behavior that compromises CFA Institute integrity, the CFA designation, or exam security (e.g., cheating, sharing exam content, falsifying records).
        \end{itemize}

      \item \textbf{Reference to CFA Institute, Designation, and Program}
        \begin{itemize}[noitemsep, topsep=0pt, leftmargin=*]
          \item Do not misrepresent the CFA designation or candidacy.
          \item Correct usage: ``CFA charterholder,'' ``Level II CFA Program candidate.'' Avoid incorrect phrasing such as ``CFA‑certified'' or ``CFA Level II graduate.''
        \end{itemize}
    \end{enumerate}
\end{enumerate}

---

\subsection*{Key Takeaways and Examples}

\begin{itemize}
    \item \textbf{Code vs. Standards:}  
    The \textbf{Code of Ethics} sets broad principles; the \textbf{Standards of Professional Conduct} specify practical rules of behavior.

    \item \textbf{Example – Knowledge of the Law:}  
    A CFA charterholder learns that a colleague is front-running trades. The member must not participate and must dissociate (e.g., refuse to execute, notify compliance).

    \item \textbf{Example – Independence:}  
    An analyst is offered an all-expenses-paid trip to inspect a company. Must either decline or disclose and ensure no compromise of objectivity.

    \item \textbf{Example – Suitability:}  
    Recommending speculative small-cap stocks to a retiree with low risk tolerance breaches suitability.

    \item \textbf{Example – Misrepresentation:}  
    Reusing another analyst’s research report without attribution constitutes plagiarism.

    \item \textbf{Example – Referral Fees:}  
    An advisor receives 1\% of client investments for referring them to an insurance firm — must disclose to both employer and clients.

    \item \textbf{Ethical Rationale:}  
    The goal is to maintain \textbf{trust, transparency, and fairness} in client relationships and global capital markets.
\end{itemize}

\section*{MODULE 91.1: Guidance for Standards I(A) and I(B)}

\subsection*{LOS 91.a–91.c: Application, Prevention, and Identification of Conduct}

\textbf{Focus:}
\begin{itemize}
  \item Demonstrate ethical application of the Code and Standards in professional integrity issues.
  \item Recommend practices that prevent violations.
  \item Identify conduct that conforms to or violates the Code and Standards.
\end{itemize}

\textbf{Context:}
\begin{itemize}
  \item Standard I of the CFA Institute Standards of Professional Conduct covers \textbf{Professionalism}.
  \item Subsections include:
  \begin{enumerate}
      \item I(A) Knowledge of the Law
      \item I(B) Independence and Objectivity
  \end{enumerate}
  \item The 11th Edition of the \textit{Standards of Practice Handbook} (2014) provides detailed examples, which are often tested in CFA Level I.
\end{itemize}

\textbf{Exam Reminder:}
\begin{itemize}
  \item You must recognize violations, not quantify what is “reasonable” or “token.”  
  \item Key interpretive terms such as “reasonable,” “adequate,” and “token” are clarified by context — the question will usually make clear whether conduct violates the Standard.
\end{itemize}

---

\subsection*{Standard I(A): Knowledge of the Law}

\textbf{Core Requirement:}
\begin{itemize}
  \item Members and Candidates must:
  \begin{enumerate}
    \item Understand and comply with all applicable laws, rules, and regulations — including those of CFA Institute, regulators, and professional bodies.
    \item When multiple jurisdictions apply, follow the \textbf{most strict} law or regulation.
    \item Not knowingly participate in, or assist with, violations of any law, rule, or regulation.
    \item Dissociate from any known violations.
  \end{enumerate}
\end{itemize}

\textbf{Guidance:}
\begin{itemize}
  \item “Applicable laws” include local, national, and CFA Institute Standards.
  \item A violation of the Code or Standards also constitutes a violation of this subsection.
  \item Members must remain informed about legal and regulatory changes that affect their professional duties.
\end{itemize}

\textbf{Actions that Violate Standard I(A):}
\begin{itemize}
  \item Ignoring known violations by colleagues or clients.
  \item Participating in misleading or fraudulent activities (e.g., using false marketing materials).
  \item Failing to dissociate from illegal practices such as insider trading or false performance reporting.
\end{itemize}

\textbf{Example:}
\begin{itemize}
  \item An analyst discovers that her firm misallocates trades in favor of certain accounts.  
  She must report it to compliance and \textbf{dissociate} from the activity; if unresolved, resignation may be required.
\end{itemize}

\textbf{Best Practices — for Members:}
\begin{enumerate}
  \item Stay current with laws and regulations through continuing education.
  \item Consult compliance officers or legal counsel when in doubt.
  \item Maintain written records documenting suspected violations and dissociation efforts.
  \item Encourage firmwide ethics and compliance programs.
  \item Although not required, members are strongly encouraged to report violations to the CFA Institute Professional Conduct Program (PCP).
\end{enumerate}

\textbf{Best Practices — for Firms:}
\begin{enumerate}
  \item Adopt a formal \textbf{Code of Ethics}.
  \item Provide staff with regular compliance training and regulatory updates.
  \item Establish internal reporting and whistle-blower mechanisms.
  \item Periodically review compliance procedures.
\end{enumerate}

\textbf{Summary Table: Knowledge of the Law}
\begin{center}
\begin{tabular}{|l|p{10cm}|}
\hline
\textbf{Aspect} & \textbf{Guidance and Examples} \\
\hline
Main Duty & Comply with the strictest applicable rule or regulation. \\
\hline
If Conflict Exists & Follow the rule offering the \textbf{highest ethical standard}. \\
\hline
If Violation by Others & Dissociate; report internally; resign if necessary. \\
\hline
Reporting Requirement & Encouraged (not mandatory) to report to CFA Institute PCP. \\
\hline
Example & Refusing to market a misleading prospectus and documenting dissociation. \\
\hline
\end{tabular}
\end{center}

---

\subsection*{Standard I(B): Independence and Objectivity}

\textbf{Core Requirement:}
\begin{itemize}
  \item Members and Candidates must use \textbf{reasonable care and judgment} to achieve and maintain independence and objectivity in all professional activities.
  \item Must not offer, solicit, or accept any gift, benefit, compensation, or consideration that could reasonably be expected to compromise independence or objectivity.
\end{itemize}

\textbf{Purpose:}
\begin{itemize}
  \item To ensure that professional analysis, recommendations, and actions are based on unbiased judgment — free from undue influence, pressure, or inducement.
\end{itemize}

\textbf{Common Threats to Independence:}
\begin{enumerate}
  \item \textbf{Pressure from issuers or employers} to produce favorable research or ratings.
  \item \textbf{Inducements or gifts} from brokers, clients, or vendors seeking favorable treatment.
  \item \textbf{Allocation of IPO shares} to personal or favored accounts.
  \item \textbf{Issuer-paid research} where compensation is tied to positive outcomes.
\end{enumerate}

\textbf{Examples of Violations:}
\begin{itemize}
  \item Accepting an all-expenses-paid luxury trip from a company being analyzed.
  \item Allowing investment banking interests to alter research conclusions.
  \item Producing “independent” research funded by an issuer without full disclosure.
  \item Allocating oversubscribed IPOs to personal or friends’ accounts.
\end{itemize}

\textbf{Examples of Acceptable Conduct:}
\begin{itemize}
  \item Accepting a modest, routine business meal that is not intended to influence analysis.
  \item Accepting a client gift of small or “token” value (e.g., a holiday basket), provided it is disclosed to the employer.
  \item Preparing issuer-paid research reports if:
  \begin{itemize}
      \item Compensation is a fixed fee, not contingent on report conclusions, and
      \item The report discloses that it is issuer-paid.
  \end{itemize}
\end{itemize}

\textbf{Guidelines for Gifts and Entertainment:}
\begin{center}
\begin{tabular}{|p{3.5cm}|p{9.5cm}|}
\hline
\textbf{Type of Gift or Benefit} & \textbf{Ethical Treatment under Standard I(B)} \\
\hline
\textbf{From Clients} & Allowed if modest and disclosed to employer (before acceptance if possible). \\
\hline
\textbf{From Companies or Brokers} & Not allowed if could compromise objectivity; “token” gifts acceptable. \\
\hline
\textbf{Lavish Entertainment or Luxury Trips} & Violation — expected to influence independence. \\
\hline
\textbf{Issuer-Paid Research Compensation} & Allowed only as flat-fee arrangement with full disclosure. \\
\hline
\end{tabular}
\end{center}

\textbf{Recommended Procedures for Members:}
\begin{itemize}
  \item Pay personal or firm travel expenses to company visits or conferences whenever feasible.
  \item Avoid corporate aircraft unless no commercial alternative exists.
  \item Decline or disclose any benefit that may affect judgment.
  \item Keep detailed records of any gifts, hospitality, or compensation received.
\end{itemize}

\textbf{Recommended Procedures for Firms:}
\begin{enumerate}
  \item \textbf{Establish written policies} on gifts, travel, research independence, and conflicts.
  \item Require \textbf{pre-approval} for IPO or private-placement participation.
  \item Assign a \textbf{compliance officer} to monitor independence and objectivity.
  \item Define a clear threshold for what constitutes a “token” gift.
  \item Require disclosure of all client gifts to supervisors.
\end{enumerate}

\textbf{Practical Examples:}
\begin{itemize}
  \item \textbf{Example 1 (Violation):}  
  An analyst accepts Super Bowl tickets from a company whose stock she covers.  
  → Likely to influence objectivity — violation.

  \item \textbf{Example 2 (Compliant):}  
  A fund manager accepts a \$50 gift card from a long-term client and reports it to compliance.  
  → Token and disclosed — not a violation.

  \item \textbf{Example 3 (Violation):}  
  An analyst accepts payment per published favorable rating.  
  → Compensation tied to outcome — violation.

  \item \textbf{Example 4 (Compliant):}  
  Analyst pays for own travel to company site visit and discloses all material interactions.  
  → Independence preserved.
\end{itemize}

---

\subsection*{Exam Strategy and Key Interpretations}

% Requires \usepackage{enumitem} in the preamble (optional)
\begin{itemize}[leftmargin=*,noitemsep,topsep=0pt]
  \item \textbf{Strictest standard:} Apply the most stringent applicable rule or ethical interpretation when multiple rules conflict.
  \item \textbf{Intent:} Intent does not excuse prohibited conduct—if a Standard prohibits the action, motivation is irrelevant.
  \item \textbf{Reasonable:} ``Reasonable'' means consistent with objective professional judgment, not personal convenience.
  \item \textbf{Token:} ``Token'' denotes low monetary value and low potential to influence behavior; determine meaning from context.
  \item \textbf{If in doubt:} Disclose, document, and distance — the ``3 Ds.''
\end{itemize} 

---

\subsection*{Key Takeaways Summary Table}

\begin{center}
\begin{tabular}{|p{4cm}|p{11cm}|}
\hline
\textbf{Standard} & \textbf{Summary and Prevention Practices} \\
\hline
I(A) Knowledge of the Law &
\begin{itemize}
  \item Comply with all applicable laws and regulations — follow the strictest rule.  
  \item Do not participate in or ignore violations.  
  \item Dissociate and document efforts.  
  \item Encourage firms to maintain compliance systems and education programs.
\end{itemize} \\
\hline
I(B) Independence and Objectivity &
\begin{itemize}
  \item Maintain unbiased judgment and professional autonomy.  
  \item Avoid accepting or offering inducements that may compromise integrity.  
  \item Accept only token gifts; disclose all client gifts.  
  \item Establish firm policies limiting gifts, IPO participation, and research funding.  
  \item Firms should appoint compliance officers and define reporting channels.
\end{itemize} \\
\hline
\end{tabular}
\end{center}

---

\subsection*{Essential Principles to Remember}

\begin{itemize}
  \item Ethical professionalism requires strict compliance with law and unwavering independence of thought.  
  \item Members must actively prevent, not merely avoid, unethical conduct.  
  \item Transparency, documentation, and disclosure protect both clients and professionals.  
  \item Firm policies should institutionalize ethics — codes, training, monitoring, and reporting systems.  
  \item On the exam, focus on:  
  \begin{enumerate}
      \item Recognizing clear violations,  
      \item Identifying compliant behavior, and  
      \item Recommending firm-level controls to prevent future breaches.
  \end{enumerate}
\end{itemize}

\section*{MODULE 91.2: Guidance for Standards I(C) and I(D)}

\subsection*{Overview}
\begin{itemize}
  \item These standards fall under \textbf{Standard I: Professionalism.}
  \item They focus on honesty, accuracy, and integrity in communication and conduct.
  \item The two subsections are:
  \begin{enumerate}
      \item I(C) Misrepresentation  
      \item I(D) Misconduct
  \end{enumerate}
  \item The core goal is to preserve \textbf{trust and transparency} in the investment profession by ensuring members neither mislead others nor engage in deceitful or unethical behavior.
\end{itemize}

---

\subsection*{Standard I(C): Misrepresentation}

\textbf{Main Requirement:}  
\begin{quote}
Members and Candidates must not knowingly make any misrepresentations relating to investment analysis, recommendations, actions, or other professional activities.
\end{quote}

\textbf{Key Principle:}  
All communications — written, oral, or digital — must be \textbf{accurate, complete, and truthful}.  
Misrepresentation can occur through:
\begin{itemize}
  \item False statements or exaggeration.
  \item Omitting material facts.
  \item Misleading data presentation.
  \item Plagiarism (use of others’ work without proper attribution).
\end{itemize}

---

\subsubsection*{1. What Constitutes Misrepresentation}

\begin{itemize}
  \item \textbf{Misleading Statements:}  
  - Exaggerating qualifications or firm capabilities.  
  - Claiming expertise or resources that do not exist.  
  - Guaranteeing investment returns where none are contractually or legally guaranteed.

  \item \textbf{Omissions:}  
  - Failing to disclose key assumptions, risks, or limitations in analysis.  
  - Selectively presenting data to make performance appear stronger.

  \item \textbf{Plagiarism:}  
  - Using reports, models, forecasts, or charts prepared by others without credit.  
  - Presenting third-party research as one’s own work.  
  - Copying text from research reports, spreadsheets, or articles without acknowledgment.

  \item \textbf{Improper Benchmarking:}  
  - Using a performance benchmark that is not comparable to the investment strategy employed.

  \item \textbf{False Marketing:}  
  - Using promotional materials from other firms or advisors that are misleading or unverified.  
  - Selecting valuation services solely because they yield the highest values.
\end{itemize}

---

\subsubsection*{2. Examples of Violations}

\begin{center}
\begin{tabular}{|p{5cm}|p{10cm}|}
\hline
\textbf{Example} & \textbf{Why It Violates the Standard} \\
\hline
Guaranteeing a 12\% return on an equity portfolio & No explicit guarantee exists — creates false expectations. \\
\hline
Using another firm’s report without attribution & Constitutes plagiarism and misrepresentation of intellectual property. \\
\hline
Reporting performance that excludes underperforming accounts & Misleading by omission — performance data must be complete and fair. \\
\hline
Advertising an “award-winning research team” when no such recognition exists & Misrepresentation of firm capabilities. \\
\hline
Using misleading third-party marketing materials & Member remains responsible for accuracy of distributed materials. \\
\hline
\end{tabular}
\end{center}

---

\subsubsection*{3. Plagiarism Rules and Acceptable Use}

\textbf{Permitted:}
\begin{itemize}
  \item Using publicly available data, statistics, or tables from recognized financial and statistical services (e.g., Bloomberg, IMF, World Bank, or government sources) without citation.
  \item Referring to firm-internal research or models (from colleagues or former employees) without listing all prior contributors — provided that the work is firm property.
\end{itemize}

\textbf{Not Permitted:}
\begin{itemize}
  \item Copying another analyst’s model, idea, or report without credit.
  \item Republishing or summarizing third-party research without citing the original source.
\end{itemize}

---

\subsubsection*{4. Recommendations for Members}

\begin{enumerate}
  \item \textbf{Maintain transparency:}  
  Always cite sources of external data, models, or research.
  \item \textbf{Keep records:}  
  Retain copies of all reports, data, and references used in preparing analysis.
  \item \textbf{Verify materials:}  
  Review third-party marketing materials for accuracy before distributing.
  \item \textbf{Clarify services:}  
  Provide written descriptions of your and your firm’s qualifications and offerings.
  \item \textbf{Regular review:}  
  Periodically check all public communications for misrepresentation.
\end{enumerate}

\textbf{Recommendations for Firms:}
\begin{itemize}
  \item Implement compliance procedures for verifying third-party marketing materials.
  \item Maintain records of all published communications.
  \item Conduct internal audits to ensure performance reporting is fair and complete.
  \item Provide clear guidelines on citation and source acknowledgment.
\end{itemize}

---

\textbf{Summary Table: Standard I(C) — Misrepresentation}

\begin{center}
\begin{tabular}{|l|p{10cm}|}
\hline
\textbf{Aspect} & \textbf{Key Points and Examples} \\
\hline
Objective & Ensure honesty, accuracy, and transparency in all professional communications. \\
\hline
Major Violations & Plagiarism, data omission, false claims, misleading marketing, improper benchmarks. \\
\hline
Example of Violation & Presenting a firm's capabilities or client list that does not exist. \\
\hline
Permitted Practice & Using government or recognized statistical data without citation. \\
\hline
Best Practice & Retain all supporting materials; disclose sources; cite external research. \\
\hline
\end{tabular}
\end{center}

---

\subsection*{Standard I(D): Misconduct}

\textbf{Main Requirement:}
\begin{quote}
Members and Candidates must not engage in any professional conduct involving dishonesty, fraud, or deceit, or commit any act that reflects adversely on their professional reputation, integrity, or competence.
\end{quote}

\textbf{Essence of the Standard:}
\begin{itemize}
  \item Prohibits unethical, fraudulent, or deceitful acts in both professional and personal contexts.
  \item Covers not only illegal acts but also actions damaging to one’s professional reputation or the CFA designation.
\end{itemize}

\textbf{Key Points:}
\begin{itemize}
  \item Applies to both \textbf{professional} and \textbf{personal} conduct — even outside the workplace.
  \item The act must reflect adversely on integrity, competence, or reputation.
  \item Examples include: substance abuse affecting work, falsifying credentials, or engaging in fraud.
  \item Members must not misuse this Standard to attack another member for personal or political reasons unrelated to ethics or competence.
\end{itemize}

---

\subsubsection*{1. Examples of Violations}

\begin{center}
\begin{tabular}{|p{5cm}|p{10cm}|}
\hline
\textbf{Example} & \textbf{Reason for Violation} \\
\hline
Engaging in insider trading or market fraud & Dishonesty and deceit in professional conduct. \\
\hline
Falsifying academic or professional credentials on a resume & Misrepresentation and fraud. \\
\hline
Substance abuse that affects work performance & Reflects adversely on professional competence. \\
\hline
Misappropriating client or employer funds & Clear breach of honesty and integrity. \\
\hline
Attempting to use this Standard to harm another member personally & Abuse of ethical enforcement — unethical behavior in itself. \\
\hline
\end{tabular}
\end{center}

---

\subsubsection*{2. Examples of Compliant Conduct}

\begin{itemize}
  \item Reporting own minor regulatory infraction proactively and cooperating with authorities.  
  \item Seeking treatment for personal issues before they impair professional work.  
  \item Refusing to engage in dishonest practices despite external pressure.
\end{itemize}

---

\subsubsection*{3. Recommendations for Firms}

\begin{enumerate}
  \item Develop and publicize a clear \textbf{Code of Ethics} prohibiting dishonest or fraudulent acts.
  \item Provide staff with examples of potential violations and disciplinary actions.
  \item Perform thorough background and reference checks for all new hires.
  \item Establish whistleblower protections to encourage ethical reporting.
  \item Ensure consistent enforcement and consequences for ethical breaches.
\end{enumerate}

---

\textbf{Summary Table: Standard I(D) — Misconduct}

\begin{center}
\begin{tabular}{|l|p{10cm}|}
\hline
\textbf{Aspect} & \textbf{Key Points and Examples} \\
\hline
Objective & Prevent any conduct (professional or personal) involving fraud, deceit, or dishonesty. \\
\hline
Scope & Applies to all conduct that reflects on professional integrity or competence, not just job-related. \\
\hline
Examples of Violations & Fraud, theft, falsifying records, substance abuse impairing work. \\
\hline
Abuse of the Standard & Using it to attack others for non-professional reasons is itself unethical. \\
\hline
Firm Practices & Code of ethics, clear sanctions, employee education, reference checks. \\
\hline
\end{tabular}
\end{center}

---

\subsection*{Key Comparative Overview: Standards I(C) vs. I(D)}

\begin{center}
\begin{tabular}{|l|p{5.5cm}|p{5.5cm}|}
\hline
\textbf{Dimension} & \textbf{Standard I(C) Misrepresentation} & \textbf{Standard I(D) Misconduct} \\
\hline
Primary Focus & Accuracy and truthfulness in communication and representation. & Integrity and honesty in conduct and behavior. \\
\hline
Typical Violations & False statements, omissions, plagiarism, misleading data. & Fraud, deceit, dishonesty, acts harming reputation. \\
\hline
Scope & Investment analysis, marketing, and professional activities. & Both professional and personal conduct affecting reputation. \\
\hline
Intent Requirement & Must be “knowingly” misrepresenting or misleading. & Includes negligent or reckless behavior damaging reputation. \\
\hline
Preventive Measures & Citation, documentation, communication review. & Codes of ethics, background checks, training. \\
\hline
\end{tabular}
\end{center}

---

\subsection*{Essential Takeaways}

\begin{itemize}
  \item Misrepresentation = misleading others through words, omissions, or plagiarism.  
  \item Misconduct = dishonesty, fraud, or any act damaging reputation or integrity.  
  \item Accuracy, honesty, and transparency are non-negotiable.  
  \item Firms should institutionalize ethics through policies, education, and compliance checks.  
  \item Members must take personal responsibility for ensuring their communications and behavior uphold the dignity of the investment profession.
\end{itemize}

\section*{MODULE 91.3: GUIDANCE FOR STANDARD II — INTEGRITY OF CAPITAL MARKETS}

\subsection*{Overview}

\begin{itemize}
  \item \textbf{Purpose:} Protect the fairness and transparency of global capital markets.
  \item \textbf{Core Idea:} All participants should have equal access to material information. 
  \item \textbf{Two Subsections:}
  \begin{enumerate}
    \item Standard II(A) — Material Nonpublic Information
    \item Standard II(B) — Market Manipulation
  \end{enumerate}
  \item These rules ensure that \textbf{confidence, fairness, and efficiency} in capital markets are maintained.
\end{itemize}

---

\subsection*{Standard II(A): Material Nonpublic Information}

\textbf{Standard Text:}
\begin{quote}
Members and Candidates who possess material nonpublic information that could affect the value of an investment must not act or cause others to act on the information.
\end{quote}

---

\subsubsection*{1. Core Concepts and Definitions}

\begin{itemize}
  \item \textbf{Material Information:}
  \begin{itemize}
    \item Information is \textbf{material} if disclosure would affect a security’s price, or if a reasonable investor would consider it important before making an investment decision.
    \item Examples of material information:
    \begin{itemize}
        \item Earnings results or profit warnings.
        \item Mergers, acquisitions, or takeovers.
        \item Dividend changes or stock repurchases.
        \item New product launches or regulatory approvals.
        \item Credit rating downgrades or changes in management.
    \end{itemize}
    \item \textbf{Ambiguous information} (unclear price impact) may not be material.
  \end{itemize}

  \item \textbf{Nonpublic Information:}
  \begin{itemize}
    \item Information is \textbf{nonpublic} until it has been widely disseminated to the marketplace.
    \item Private meetings, analyst calls, and selective disclosures do \textbf{not} constitute public dissemination.
  \end{itemize}

  \item \textbf{Acting or Causing Others to Act:}
  \begin{itemize}
    \item Includes trading, recommending, or passing material nonpublic information to others (“tipping”).
    \item Applies to direct investments, derivatives, mutual funds, swaps, and options.
  \end{itemize}
\end{itemize}

---

\subsubsection*{2. The Mosaic Theory}

\textbf{Definition:}
\begin{quote}
Reaching an investment conclusion through perceptive analysis of \textbf{public information} combined with \textbf{non-material nonpublic information} does not violate the Standard.
\end{quote}

\textbf{Explanation:}
\begin{itemize}
  \item Analysts are encouraged to use skill and diligence to piece together various data sources (e.g., industry trends, supplier data, management tone) to form a conclusion.
  \item The key is that none of the individual pieces used are both \textbf{material} and \textbf{nonpublic}.
\end{itemize}

\textbf{Example:}
\begin{itemize}
  \item A semiconductor analyst predicts a firm’s strong quarterly performance based on:
    \begin{itemize}
      \item Publicly available sales reports from suppliers.
      \item Industry shipment data.
      \item A non-material observation that the firm’s parking lot is full.
    \end{itemize}
  → This is permitted under the Mosaic Theory.
\end{itemize}

---

\subsubsection*{3. Examples of Violations}

\begin{center}
\begin{tabular}{|p{5cm}|p{10cm}|}
\hline
\textbf{Situation} & \textbf{Violation Explanation} \\
\hline
Trading on advance knowledge of a pending merger & Material and nonpublic — insider trading. \\
\hline
Receiving confidential earnings data during an analyst call and trading before it’s published & Nonpublic disclosure — violation. \\
\hline
Leaking information about an upcoming credit downgrade to friends or clients & “Tipping” — causing others to act on material nonpublic information. \\
\hline
Employee of investment bank using client deal information to trade for personal benefit & Use of insider information beyond intended purpose — violation. \\
\hline
\end{tabular}
\end{center}

---

\subsubsection*{4. Recommendations for Members}

\begin{enumerate}
  \item Make reasonable efforts to ensure the firm disseminates material information publicly.
  \item Avoid discussions of sensitive information in public or semi-public places.
  \item Maintain confidentiality — even within the firm — on sensitive client data.
  \item Seek guidance from compliance officers when uncertain about information status.
  \item Maintain and respect firm “restricted lists” of securities under blackout.
\end{enumerate}

---

\subsubsection*{5. Recommendations for Firms}

\begin{itemize}
  \item Implement robust \textbf{information barrier systems} (“firewalls”) to prevent insider information flow.
  \item Appoint compliance officers to monitor cross-department communication.
  \item Maintain lists such as:
  \begin{itemize}
    \item \textbf{Watch List:} Securities under review for potential insider exposure.
    \item \textbf{Restricted List:} Securities for which trading is prohibited.
    \item \textbf{Rumor List:} Speculative cases monitored by compliance.
  \end{itemize}
  \item Monitor employee and proprietary trading.
  \item When holding material nonpublic information:
  \begin{itemize}
    \item Restrict trading to only take the opposite side of unsolicited client trades.
    \item Avoid total trading halts that could alert the market to insider information.
  \end{itemize}
\end{itemize}

---

\textbf{Summary Table: Standard II(A) — Material Nonpublic Information}

\begin{center}
\begin{tabular}{|l|p{10cm}|}
\hline
\textbf{Aspect} & \textbf{Explanation and Examples} \\
\hline
Definition of “Material” & Information that would affect price or influence investor decisions. \\
\hline
Definition of “Nonpublic” & Information not yet widely disseminated to the market. \\
\hline
Prohibited Actions & Trading, recommending, or tipping based on material nonpublic information. \\
\hline
Permitted Practice & Mosaic Theory — combining public and immaterial nonpublic data. \\
\hline
Firm Policies & Firewalls, watch lists, restricted lists, employee trade monitoring. \\
\hline
Member Practices & Maintain confidentiality, consult compliance, and promote public dissemination. \\
\hline
\end{tabular}
\end{center}

---

\subsection*{Standard II(B): Market Manipulation}

\textbf{Standard Text:}
\begin{quote}
Members and Candidates must not engage in practices that distort prices or artificially inflate trading volume with the intent to mislead market participants.
\end{quote}

---

\subsubsection*{1. Core Principle}

\begin{itemize}
  \item The Standard prohibits actions intended to \textbf{deceive or mislead} market participants through price or volume manipulation.
  \item Manipulation damages market integrity, investor confidence, and efficiency.
  \item The key factor is \textbf{intent to mislead} — legitimate trading strategies that affect prices incidentally do not violate the Standard.
\end{itemize}

---

\subsubsection*{2. Types of Market Manipulation}

\begin{enumerate}
  \item \textbf{Information-Based Manipulation}
  \begin{itemize}
    \item Spreading false or misleading information to influence prices.
    \item Example: Circulating fake news about a merger to drive up share price.
  \end{itemize}

  \item \textbf{Transaction-Based Manipulation}
  \begin{itemize}
    \item Artificially affecting price or volume through trades designed to mislead.
    \item Examples:
    \begin{itemize}
        \item “Wash trades” — buying and selling the same security to inflate volume.
        \item “Pump-and-dump” — artificially raising prices to sell at profit.
        \item “Marking the close” — trading near market close to influence closing price.
    \end{itemize}
  \end{itemize}
\end{enumerate}

---

\subsubsection*{3. Examples of Violations}

\begin{center}
\begin{tabular}{|p{5cm}|p{10cm}|}
\hline
\textbf{Situation} & \textbf{Violation Explanation} \\
\hline
Spreading false rumors of a takeover to boost stock price & Intentional dissemination of false information — information-based manipulation. \\
\hline
Executing offsetting buy/sell orders to create fake market activity & Transaction-based manipulation — distorts trading volume. \\
\hline
“Marking the close” to improve portfolio valuation & Intentional price distortion for misleading reporting. \\
\hline
Coordinating trades among funds to move market prices in favor of one & Intent to mislead — coordinated manipulation. \\
\hline
\end{tabular}
\end{center}

---

\subsubsection*{4. Examples of Non-Violations}

\begin{itemize}
  \item A large buy order causing a price increase due to legitimate market demand.  
  → No intent to mislead, hence not a violation.
  \item Executing hedging or arbitrage trades that inadvertently affect prices.  
  → Economic rationale exists — not manipulation.
\end{itemize}

---

\subsubsection*{5. Recommendations for Members and Firms}

\begin{itemize}
  \item Maintain written policies against dissemination of false information.
  \item Verify all market-related communications and research findings before release.
  \item Monitor employees for unusual trading patterns or coordinated activity.
  \item Separate proprietary trading from market-making and research functions.
  \item Document rationale for all large or unusual trades.
\end{itemize}

---

\textbf{Summary Table: Standard II(B) — Market Manipulation}

\begin{center}
\begin{tabular}{|l|p{10cm}|}
\hline
\textbf{Aspect} & \textbf{Explanation and Examples} \\
\hline
Core Principle & No intentional distortion of prices or volume to mislead others. \\
\hline
Information-Based Manipulation & False rumors, misleading press releases, fake research. \\
\hline
Transaction-Based Manipulation & Wash trades, marking the close, pump-and-dump. \\
\hline
Key Test & Was there \textbf{intent to mislead} the market? \\
\hline
Legitimate Activities & Large orders based on real investment intent, hedging, arbitrage. \\
\hline
Firm Controls & Communication verification, trade monitoring, compliance audits. \\
\hline
\end{tabular}
\end{center}

---

\subsection*{Combined Summary: Standard II(A) vs. II(B)}

\begin{center}
\begin{tabular}{|l|p{5.5cm}|p{5.5cm}|}
\hline
\textbf{Dimension} & \textbf{II(A): Material Nonpublic Information} & \textbf{II(B): Market Manipulation} \\
\hline
Primary Focus & Prevent insider trading and misuse of confidential material information. & Prevent artificial price or volume distortion to mislead investors. \\
\hline
Nature of Violation & Acting or tipping on material nonpublic information. & Spreading false info or trading deceptively to influence prices. \\
\hline
Key Test & Was the information material and nonpublic? & Was there intent to mislead or deceive market participants? \\
\hline
Examples & Insider trading, selective disclosure, leaking earnings data. & Wash trades, pump-and-dump, marking the close. \\
\hline
Permitted Practice & Mosaic theory using public + non-material nonpublic data. & Legitimate large trades based on genuine market demand. \\
\hline
Firm Controls & Firewalls, restricted lists, compliance oversight. & Communication controls, trade monitoring, documentation. \\
\hline
\end{tabular}
\end{center}

---

\subsection*{Key Takeaways}

\begin{itemize}
  \item \textbf{Standard II(A):}  
  Protects against unfair informational advantage — trading on or disclosing material nonpublic information is unethical and illegal.
  \item \textbf{Standard II(B):}  
  Protects market integrity — any act designed to manipulate prices or mislead participants is prohibited.
  \item \textbf{Mosaic Theory:}  
  Encourages skillful, ethical analysis using publicly available and immaterial data.
  \item \textbf{Firm Role:}  
  Compliance systems (firewalls, watch/restricted lists, monitoring) are essential for preventing violations.
\end{itemize}

\section*{MODULE 91.4: GUIDANCE FOR STANDARDS III(A) AND III(B)}
\subsection*{STANDARD III: DUTIES TO CLIENTS}

\textbf{Core Principle:}  
Members and Candidates owe a duty of loyalty, fairness, and prudence to their clients.  
They must place client interests above their own or their employer’s interests, and must treat all clients fairly when taking investment actions.

---

\subsection*{Standard III(A): Loyalty, Prudence, and Care}

\textbf{Standard Text:}
\begin{quote}
Members and Candidates have a duty of loyalty to their clients and must act with reasonable care and exercise prudent judgment. They must act for the benefit of their clients and place their clients’ interests before their employer’s or their own interests.
\end{quote}

---

\subsubsection*{1. Core Principles}

\begin{itemize}
  \item \textbf{Client Interests First:}  
  Always act in the client’s best interest, even if this conflicts with personal or employer interests.

  \item \textbf{Prudence and Diligence:}  
  Exercise the same care, skill, and judgment that a prudent person familiar with such matters would use.

  \item \textbf{Portfolio Context:}  
  Evaluate investments in the context of the client’s total portfolio, not in isolation.

  \item \textbf{Governance Documents:}  
  Manage client assets according to the terms of governing documents such as trust deeds, investment mandates, or policy statements.

  \item \textbf{Fiduciary Principle:}  
  Although CFA Standards do not automatically impose a fiduciary duty, they require behavior consistent with fiduciary principles of loyalty, prudence, and care.

  \item \textbf{Transparency and Disclosure:}  
  Inform clients of limitations in services or product offerings (e.g., restricted product lists).
\end{itemize}

---

\subsubsection*{2. Key Applications}

\begin{itemize}
  \item \textbf{Proxy Voting:}  
  Vote proxies responsibly and in clients’ best interests.  
  → If costs outweigh benefits, it may be reasonable not to vote, but this must be documented.

  \item \textbf{Soft Dollars / Soft Commissions:}  
  Client brokerage commissions must be used only for research or services that directly benefit the client — not for the manager’s own advantage.

  \item \textbf{Defining “Client”:}  
  - Usually the person or entity to whom the duty is owed (e.g., individual, pension plan, or institution).  
  - For mutual funds or pooled investments, the client is the fund itself, not individual investors.  
  - For investment managers, the “client” may be the investing public as a whole.
\end{itemize}

---

\subsubsection*{3. Examples of Violations}

\begin{center}
\begin{tabular}{|p{5cm}|p{10cm}|}
\hline
\textbf{Situation} & \textbf{Why It Violates the Standard} \\
\hline
Placing employer’s interests ahead of client’s by recommending proprietary funds & Fails to put client first — conflict of interest. \\
\hline
Failing to diversify client portfolio according to mandate & Neglect of prudence and care. \\
\hline
Voting proxies automatically in favor of management without analysis & Not acting in client’s best interest. \\
\hline
Using client commissions to pay for office equipment or entertainment & Misuse of client assets — violation of loyalty. \\
\hline
Failing to disclose limited product offerings & Misleads clients regarding the scope of service. \\
\hline
\end{tabular}
\end{center}

---

\subsubsection*{4. Examples of Compliant Conduct}

\begin{itemize}
  \item Disclosing all potential conflicts of interest to clients before taking action.  
  \item Maintaining detailed investment policy statements (IPS) reflecting client objectives, constraints, and risk tolerance.  
  \item Periodically reviewing client portfolios and updating suitability profiles.  
  \item Ensuring all brokerage commissions are used for client benefit (e.g., legitimate research).  
  \item Providing clients with clear, itemized quarterly statements of holdings and transactions.
\end{itemize}

---

\subsubsection*{5. Recommendations for Members}

\begin{enumerate}
  \item Submit to clients regular (at least quarterly) statements showing securities in custody, debits, credits, and transactions.
  \item Follow applicable laws and fiduciary obligations.
  \item Establish and document client investment objectives and risk tolerance.
  \item Diversify portfolios appropriately.
  \item Deal fairly and transparently with all clients.
  \item Disclose all conflicts of interest and compensation structures.
  \item Vote proxies and exercise ownership rights in clients’ best interests.
  \item Maintain confidentiality and seek best execution for client trades.
\end{enumerate}

---

\subsubsection*{6. Summary Table: Standard III(A) — Loyalty, Prudence, and Care}

\begin{center}
\begin{tabular}{|l|p{10cm}|}
\hline
\textbf{Aspect} & \textbf{Explanation and Examples} \\
\hline
Core Duty & Put client interests first; act with prudence, care, and diligence. \\
\hline
Scope & Applies to discretionary and advisory relationships; across all client accounts. \\
\hline
Proxy Voting & Vote in best interest of client; may abstain if cost > benefit. \\
\hline
Soft Dollar Use & Only for research or services benefiting the client. \\
\hline
Typical Violations & Favoring own firm’s products, failing to diversify, misusing client commissions. \\
\hline
Best Practices & Written policies, IPS creation, quarterly reporting, conflict disclosure. \\
\hline
\end{tabular}
\end{center}

---

\subsection*{Standard III(B): Fair Dealing}

\textbf{Standard Text:}
\begin{quote}
Members and Candidates must deal fairly and objectively with all clients when providing investment analysis, making investment recommendations, taking investment action, or engaging in other professional activities.
\end{quote}

---

\subsubsection*{1. Core Principle}

\begin{itemize}
  \item All clients must be treated fairly and impartially — no favoritism or discrimination.
  \item “Fairly” does not mean “equally”; different service levels are acceptable if disclosed and offered to all clients willing to pay.
  \item The goal is to ensure equal opportunity for all clients to act on recommendations and trades.
\end{itemize}

---

\subsubsection*{2. Key Requirements}

\begin{itemize}
  \item \textbf{Timely Dissemination:}  
  - Ensure all clients have a fair chance to act on recommendations before trading for personal or firm accounts.  
  - Disseminate new or changed recommendations simultaneously to all relevant clients.

  \item \textbf{Trade Allocation:}  
  - Allocate trades equitably, based on pre-established policies.  
  - No client should receive preferential allocation (e.g., favored accounts getting shares in an oversubscribed IPO).

  \item \textbf{Different Service Levels:}  
  - Permitted if disclosed and available to all clients (e.g., “premium” services for higher fees).

  \item \textbf{Order Execution:}  
  - Ensure fairness in timing, pricing, and accuracy of order fills for all accounts.
\end{itemize}

---

\subsubsection*{3. Examples of Violations}

\begin{center}
\begin{tabular}{|p{5cm}|p{10cm}|}
\hline
\textbf{Situation} & \textbf{Why It Violates the Standard} \\
\hline
Providing new recommendations first to large institutional clients before smaller ones & Fails fair dissemination — unequal access. \\
\hline
Taking personal positions before disseminating a “buy” recommendation (front-running) & Unfair advantage — client disadvantage. \\
\hline
Allocating all shares of an oversubscribed IPO to favored clients & Unfair trade allocation. \\
\hline
Delaying recommendation updates for certain clients & Denies equal opportunity to act. \\
\hline
\end{tabular}
\end{center}

---

\subsubsection*{4. Examples of Compliant Conduct}

\begin{itemize}
  \item Disseminating changes in recommendations to all clients at the same time.  
  \item Maintaining a client list with holdings to ensure fair treatment in all actions.  
  \item Providing different service tiers only when properly disclosed (e.g., institutional vs. retail).  
  \item Using automated systems to release recommendations simultaneously via email or client portal.  
  \item Reviewing account allocations regularly to verify fairness and accuracy.
\end{itemize}

---

\subsubsection*{5. Recommendations for Members}

\begin{enumerate}
  \item Encourage firms to create compliance procedures ensuring fair dissemination of investment recommendations.  
  \item Maintain a detailed client list and portfolio records to ensure consistent treatment.  
  \item Document the timing of all recommendations and trades.  
  \item Avoid discussing pending recommendations with anyone before official release.  
  \item Verify all client trades were executed according to policy.
\end{enumerate}

---

\subsubsection*{6. Recommendations for Firms}

\begin{itemize}
  \item \textbf{Access Control:} Limit the number of people aware of pending recommendations.  
  \item \textbf{Rapid Dissemination:} Shorten time between decision and release.  
  \item \textbf{Pre-Dissemination Policies:} Prohibit personnel with prior knowledge from trading or discussing recommendations before release.  
  \item \textbf{Trade Allocation Procedures:}
  \begin{itemize}
    \item Allocate shares fairly among client accounts.  
    \item Document and disclose allocation policies.  
    \item Ensure trades are executed promptly and recorded accurately.
  \end{itemize}
  \item \textbf{Systematic Account Review:}  
  Periodically check for consistency between portfolio actions and client objectives.  
  \item \textbf{Disclosure of Service Levels:}  
  Explain available levels of service and fees to all clients.
\end{itemize}

---

\subsubsection*{7. Summary Table: Standard III(B) — Fair Dealing}

\begin{center}
\begin{tabular}{|l|p{10cm}|}
\hline
\textbf{Aspect} & \textbf{Explanation and Examples} \\
\hline
Core Duty & Treat all clients fairly and objectively in analysis, recommendations, and trades. \\
\hline
Fair vs. Equal & Fair \neq Equal; differences allowed if disclosed and available to all. \\
\hline
Violations & Front-running, preferential dissemination, unfair trade allocation. \\
\hline
Best Practices & Simultaneous release of recommendations, documented allocation policies. \\
\hline
Firm Controls & Pre-dissemination rules, restricted access, compliance oversight, client audits. \\
\hline
\end{tabular}
\end{center}

---

\subsection*{Comparison: Standard III(A) vs. III(B)}

\begin{center}
\begin{tabular}{|l|p{5.5cm}|p{5.5cm}|}
\hline
\textbf{Dimension} & \textbf{III(A): Loyalty, Prudence, and Care} & \textbf{III(B): Fair Dealing} \\
\hline
Primary Focus & Duty to put client interests first and act prudently. & Duty to treat all clients fairly and objectively. \\
\hline
Scope & Portfolio management, client relationships, conflicts of interest. & Communication, dissemination, trade allocation. \\
\hline
Key Obligation & Loyalty, diligence, care, transparency. & Equal opportunity for all clients, no favoritism. \\
\hline
Examples of Violation & Failing to diversify, misusing soft dollars, not disclosing conflicts. & Front-running, selective recommendation release, unfair allocation. \\
\hline
Firm Procedures & Investment policy statements, proxy voting, best execution. & Dissemination protocols, allocation systems, client list maintenance. \\
\hline
\end{tabular}
\end{center}

---

\subsection*{Essential Takeaways}

\begin{itemize}
  \item \textbf{Standard III(A):}  
  Loyalty and prudence demand that the client’s interest comes before any other.  
  Always act with care, diligence, and professionalism, consistent with fiduciary principles.

  \item \textbf{Standard III(B):}  
  Fair dealing requires giving all clients an equal opportunity to benefit from advice and actions.  
  Fair \neq equal — differences must be transparent and non-discriminatory.

  \item \textbf{Core Practice Themes:}  
  - Maintain client trust through transparency and diligence.  
  - Disclose conflicts and maintain independence.  
  - Develop firm policies for recommendation dissemination and trade fairness.  
  - Document everything — fairness, suitability, and prudence are proven through evidence.
\end{itemize}

\section*{MODULE 91.5: GUIDANCE FOR STANDARDS III(C), III(D), AND III(E)}
\subsection*{STANDARD III: DUTIES TO CLIENTS (CONTINUED)}

\textbf{Purpose:}  
These three Standards ensure that CFA members act with diligence, transparency, and discretion in their professional relationships with clients — ensuring suitability, accuracy, and confidentiality.

---

\subsection*{Standard III(C): Suitability}

\textbf{Standard Text:}
\begin{quote}
1. When Members and Candidates are in an advisory relationship with a client, they must:
\begin{itemize}
  \item[a.] Make a reasonable inquiry into the client’s investment experience, risk and return objectives, and financial constraints, and update regularly.
  \item[b.] Determine that an investment is suitable given the client’s financial situation and written objectives.
  \item[c.] Judge the suitability of investments in the context of the client’s total portfolio.
\end{itemize}
2. When Members and Candidates are responsible for managing a portfolio to a specific mandate, they must make recommendations and take actions consistent with the stated objectives and constraints of the portfolio.
\end{quote}

---

\subsubsection*{1. Core Principles}

\begin{itemize}
  \item \textbf{Suitability = “Fit” between client profile and investment actions.}
  \item Client interests, objectives, and constraints must always guide recommendations.
  \item Suitability depends on whether the member:
  \begin{itemize}
    \item Conducts reasonable inquiry into client’s circumstances;
    \item Keeps the IPS (Investment Policy Statement) current;
    \item Evaluates investments at the portfolio level;
    \item Acts within the mandate’s scope.
  \end{itemize}
  \item Use of leverage must be explicitly considered and explained to clients.
\end{itemize}

---

\subsubsection*{2. Investment Policy Statement (IPS)}

\begin{itemize}
  \item The IPS is the foundational document defining:
  \begin{itemize}
    \item Client type (individual/institutional);
    \item Return objectives;
    \item Risk tolerance;
    \item Constraints (liquidity, time horizon, taxes, legal/regulatory factors);
    \item Performance benchmarks.
  \end{itemize}
  \item IPS must be updated periodically to reflect changes in client circumstances.
\end{itemize}

---

\subsubsection*{3. Unsolicited Trade Requests}

\textbf{Scenario:} A client requests a trade that the manager knows is unsuitable under the IPS.

\textbf{Two cases:}
\begin{enumerate}
  \item \textbf{Minimal portfolio impact:}  
  - Manager discusses with the client why it is unsuitable.  
  - May execute per firm policy if client acknowledges and understands the risk.  
  - Document the discussion and client acknowledgment.

  \item \textbf{Material portfolio impact:}  
  - Manager should update the IPS to reflect the changed risk profile.  
  - If client refuses, follow firm policy (e.g., place trade in a separate “client-directed” account).  
  - As a last resort, consider ending the advisory relationship.
\end{enumerate}

---

\subsubsection*{4. Examples of Violations}

\begin{center}
\begin{tabular}{|p{5cm}|p{10cm}|}
\hline
\textbf{Situation} & \textbf{Why It Violates the Standard} \\
\hline
Recommending speculative small-cap stocks to a low-risk client & Inconsistent with IPS and client risk profile. \\
\hline
Failing to update a client’s IPS after major life changes (e.g., retirement) & Lack of due diligence and reassessment. \\
\hline
Using excessive leverage in a conservative client’s portfolio & Inappropriate given client’s objectives. \\
\hline
Investing outside a fund’s stated style or mandate (e.g., growth manager buys distressed debt) & Breach of mandate suitability. \\
\hline
\end{tabular}
\end{center}

---

\subsubsection*{5. Recommendations for Members}

\begin{itemize}
  \item Create a written IPS for each client.  
  \item Document return objectives, risk tolerance, and constraints clearly.  
  \item Review and update IPS regularly.  
  \item Ensure all investment actions align with client goals and constraints.  
  \item Educate clients about leverage and risk implications.  
  \item Reconfirm suitability when market or personal circumstances change.
\end{itemize}

---

\subsubsection*{6. Summary Table: Standard III(C) — Suitability}

\begin{center}
\begin{tabular}{|l|p{10cm}|}
\hline
\textbf{Aspect} & \textbf{Explanation and Examples} \\
\hline
Objective & Match investment recommendations to client’s risk/return profile and constraints. \\
\hline
Tools & Investment Policy Statement (IPS), periodic reviews, documentation. \\
\hline
Unsolicited Trades & Allowed only after client acknowledgment or IPS update. \\
\hline
Violations & Ignoring IPS, excessive leverage, unsuitable recommendations. \\
\hline
Best Practices & Written IPS, ongoing suitability checks, compliance documentation. \\
\hline
\end{tabular}
\end{center}

---

\subsection*{Standard III(D): Performance Presentation}

\textbf{Standard Text:}
\begin{quote}
When communicating investment performance information, Members and Candidates must make reasonable efforts to ensure that it is fair, accurate, and complete.
\end{quote}

---

\subsubsection*{1. Core Principles}

\begin{itemize}
  \item Accuracy and transparency are essential for maintaining client trust.  
  \item Do not misstate, cherry-pick, or overstate past performance.  
  \item Avoid implying that future returns will mirror past results.  
  \item Provide sufficient disclosure to allow clients to interpret performance correctly.
\end{itemize}

---

\subsubsection*{2. Key Guidelines}

\begin{itemize}
  \item Present performance of a \textbf{weighted composite} of similar portfolios — not only the best account.  
  \item Include \textbf{terminated accounts} in historical results (with disclosure).  
  \item Disclose all relevant facts:
    \begin{itemize}
      \item Gross or net of fees;  
      \item Inclusion of model results;  
      \item Use of leverage or derivatives;  
      \item Benchmarks and time periods.  
    \end{itemize}
  \item Brief summaries must state that detailed data are available upon request.
\end{itemize}

---

\subsubsection*{3. Examples of Violations}

\begin{center}
\begin{tabular}{|p{5cm}|p{10cm}|}
\hline
\textbf{Situation} & \textbf{Why It Violates the Standard} \\
\hline
Showing only top-performing accounts in a composite report & Misleads clients — not a fair representation. \\
\hline
Omitting terminated accounts from historical data & Distorts true performance record. \\
\hline
Projecting past performance as guaranteed future results & Creates false expectations — unethical. \\
\hline
Using simulated (“model”) results without disclosure & Misleading — not transparent about methodology. \\
\hline
\end{tabular}
\end{center}

---

\subsubsection*{4. Recommendations for Members}

\begin{itemize}
  \item Encourage firm compliance with \textbf{Global Investment Performance Standards (GIPS®)}.  
  \item Tailor performance presentation to audience sophistication.  
  \item Retain documentation of all data and calculations used.  
  \item Use standardized time periods and comparable benchmarks.  
  \item Make full supporting details available on request.  
\end{itemize}

---

\subsubsection*{5. Summary Table: Standard III(D) — Performance Presentation}

\begin{center}
\begin{tabular}{|l|p{10cm}|}
\hline
\textbf{Aspect} & \textbf{Explanation and Examples} \\
\hline
Goal & Ensure fair, accurate, and complete communication of investment results. \\
\hline
Prohibited Acts & Misleading data, selective performance, omission of terminated accounts. \\
\hline
Best Practice & GIPS compliance, composite reporting, transparent disclosure. \\
\hline
Documentation & Maintain full calculation records and data sources. \\
\hline
Short Reports & Must state that detailed data are available upon request. \\
\hline
\end{tabular}
\end{center}

---

\subsection*{Standard III(E): Preservation of Confidentiality}

\textbf{Standard Text:}
\begin{quote}
Members and Candidates must keep information about current, former, and prospective clients confidential unless:
\begin{enumerate}
  \item The information concerns illegal activities by the client;
  \item Disclosure is required by law; or
  \item The client or prospective client consents to disclosure.
\end{enumerate}
\end{quote}

---

\subsubsection*{1. Core Principles}

\begin{itemize}
  \item Protect client information as a professional obligation.  
  \item Applies to current, former, and prospective clients.  
  \item Confidentiality may be breached only when:
    \begin{itemize}
      \item Legal requirement or subpoena demands it;
      \item Client engages in illegal activities (e.g., money laundering);
      \item Client explicitly permits disclosure.
    \end{itemize}
  \item Cooperating with CFA Institute Professional Conduct Program (PCP) investigations is permitted under this Standard.
\end{itemize}

---

\subsubsection*{2. Examples of Violations}

\begin{center}
\begin{tabular}{|p{5cm}|p{10cm}|}
\hline
\textbf{Situation} & \textbf{Why It Violates the Standard} \\
\hline
Sharing client portfolio data with friends or other clients & Breach of confidentiality — unauthorized disclosure. \\
\hline
Discussing client accounts in public (e.g., elevator, restaurant) & Risk of revealing confidential information. \\
\hline
Disclosing client trades to media without consent & Unauthorized release — violation. \\
\hline
\end{tabular}
\end{center}

---

\subsubsection*{3. Examples of Permissible Disclosures}

\begin{itemize}
  \item Disclosing information when required by a legal authority or court order.  
  \item Reporting suspected criminal activity to authorities (e.g., insider trading).  
  \item Sharing client data internally with authorized colleagues working for the same client.  
  \item Providing data when the client has given written consent.
\end{itemize}

---

\subsubsection*{4. Recommendations for Members}

\begin{enumerate}
  \item Avoid discussing client information with anyone outside the client’s service team.  
  \item Follow firm protocols for electronic data storage and transmission.  
  \item Encourage firms to establish clear confidentiality policies and secure systems.  
  \item Obtain written client consent before sharing any confidential information externally.  
  \item Continue to maintain confidentiality even after the client relationship ends.
\end{enumerate}

---

\subsubsection*{5. Summary Table: Standard III(E) — Preservation of Confidentiality}

\begin{center}
\begin{tabular}{|l|p{10cm}|}
\hline
\textbf{Aspect} & \textbf{Explanation and Examples} \\
\hline
Goal & Protect client information and privacy. \\
\hline
Scope & Applies to current, former, and prospective clients. \\
\hline
Permitted Disclosure & Required by law, illegal activity, or client consent. \\
\hline
Violations & Unauthorized sharing, gossiping, or careless handling of client data. \\
\hline
Best Practices & Follow firm policy, restrict access, secure data, obtain consent. \\
\hline
\end{tabular}
\end{center}

---

\subsection*{Comparison: Standards III(C), III(D), III(E)}

\begin{center}
\begin{tabular}{|p{2.5cm}|p{4.5cm}|p{4.5cm}|p{4.5cm}|}
\hline
\textbf{Dimension} & \textbf{III(C) Suitability} & \textbf{III(D) Performance Presentation} & \textbf{III(E) Confidentiality} \\
\hline
Focus & Align investments with client objectives, risk tolerance, and constraints. & Ensure truthful and transparent reporting of investment results. & Safeguard all client information unless exceptions apply. \\
\hline
Key Tool & Investment Policy Statement (IPS). & GIPS-compliant performance reporting. & Secure client data management and consent documentation. \\
\hline
Main Violations & Unsuitable recommendations, ignoring IPS, excessive leverage. & Cherry-picked performance, omitting terminated accounts. & Sharing private client data or careless handling. \\
\hline
Best Practices & Written IPS, periodic review, documentation. & Full disclosure, composite reporting, record retention. & Strict confidentiality policies, legal awareness. \\
\hline
Ethical Focus & Prudence and care. & Honesty and transparency. & Trust and discretion. \\
\hline
\end{tabular}
\end{center}

---

\subsection*{Key Takeaways}

\begin{itemize}
  \item \textbf{Standard III(C):}  
  Ensure all investment recommendations and actions are suitable for client circumstances; maintain an updated IPS and document all rationale.
  
  \item \textbf{Standard III(D):}  
  Present performance data that are fair, complete, and transparent; follow GIPS® where possible to maintain comparability and credibility.
  
  \item \textbf{Standard III(E):}  
  Uphold confidentiality for all client information; disclose only when legally required, consented to, or necessary to report illegal activity.
  
  \item Together, these Standards promote \textbf{trust, transparency, and fiduciary responsibility} — the foundation of ethical client relationships in investment management.
\end{itemize}

\section*{MODULE 91.6: GUIDANCE FOR STANDARD IV — DUTIES TO EMPLOYERS}

\textbf{Purpose:}  
Standard IV focuses on maintaining professionalism and ethical conduct within employment relationships.  
It ensures that members act loyally toward their employers, avoid conflicts of interest, and maintain strong supervision and compliance systems.

---

\subsection*{Standard IV(A): Loyalty}

\textbf{Standard Text:}
\begin{quote}
In matters related to their employment, Members and Candidates must act for the benefit of their employer and not deprive their employer of the advantage of their skills and abilities, divulge confidential information, or otherwise cause harm to their employer.
\end{quote}

---

\subsubsection*{1. Core Principles}

\begin{itemize}
  \item Loyalty to the employer means acting in the firm’s best interests while maintaining professional integrity.  
  \item Members must not harm their employer through fraud, deceit, or misuse of proprietary information.  
  \item However, \textbf{client interests always take precedence} over employer interests when there is a conflict.  
  \item Loyalty does not mean placing employer interests above family or ethical responsibilities — personal and moral duties must be balanced.  
  \item Independent contractors must abide by the terms of their contracts rather than this employment-based duty.
\end{itemize}

---

\subsubsection*{2. Independent Practice}

\begin{itemize}
  \item Members may engage in outside (independent) business activities only if:
  \begin{enumerate}
    \item They provide \textbf{written notification} to their employer before starting;
    \item They disclose all terms (compensation, duration, nature of service);
    \item The employer gives explicit consent.
  \end{enumerate}
  \item Example: A portfolio manager taking part-time consulting work for another investment firm must first notify and obtain written approval from her employer.
\end{itemize}

---

\subsubsection*{3. Leaving an Employer}

\begin{itemize}
  \item Members must continue to act in their employer’s best interest until their resignation is effective.  
  \item Violations include:
  \begin{itemize}
    \item Misappropriating trade secrets or confidential information;  
    \item Soliciting clients or employees before departure;  
    \item Self-dealing or diverting opportunities;  
    \item Removing or copying client lists or proprietary records.  
  \end{itemize}
  \item After leaving, members may use:
  \begin{itemize}
    \item General knowledge and experience gained;  
    \item Publicly known client names;  
    \item Permitted information under industry agreements (e.g., U.S. “Broker Recruiting Protocol”).  
  \end{itemize}
\end{itemize}

---

\subsubsection*{4. Social Media Conduct}

\begin{itemize}
  \item Members must adhere to employer policies regarding the use of social media.  
  \item During departure or transition, they must not use professional social media accounts to contact clients unless permitted by firm policy.  
  \item Best practice: maintain separate personal and professional accounts.
\end{itemize}

---

\subsubsection*{5. Examples of Violations}

\begin{center}
\begin{tabular}{|p{8cm}|p{8cm}|}
\hline
\textbf{Situation} & \textbf{Why It Violates the Standard} \\
\hline
Taking client contact information before leaving firm & Misappropriation of employer property. \\
\hline
Starting an advisory business while still employed without notice & Competes with employer without consent. \\
\hline
Soliciting clients before resignation & Undermines employer and violates loyalty. \\
\hline
Using social media to announce move to another firm before official notice & Breaches employer policy and causes reputational harm. \\
\hline
\end{tabular}
\end{center}

---

\subsubsection*{6. Recommendations for Members}

\begin{itemize}
  \item Provide employers with a copy of the CFA Institute Code and Standards.  
  \item Use separate personal and professional communication channels.  
  \item Follow firm procedures during resignation; avoid conflicts or client solicitation.  
  \item Document all external engagements and obtain written approval.  
\end{itemize}

---

\subsubsection*{7. Recommendations for Firms}

\begin{itemize}
  \item Establish ethical compensation systems that do not encourage unethical conduct.  
  \item Create clear policies for outside employment, data use, and social media.  
  \item Implement compliance programs emphasizing loyalty, confidentiality, and client-first principles.  
\end{itemize}

---

\subsubsection*{8. Summary Table: Standard IV(A) — Loyalty}

\begin{center}
\begin{tabular}{|l|p{10cm}|}
\hline
\textbf{Aspect} & \textbf{Explanation and Examples} \\
\hline
Goal & Maintain loyalty to employer while prioritizing client interests. \\
\hline
Scope & Applies to all employee relationships; contractors bound by agreements. \\
\hline
Independent Practice & Allowed only with written disclosure and employer consent. \\
\hline
Leaving a Firm & No client solicitation or misappropriation before departure. \\
\hline
Social Media & Follow firm policy; separate personal and professional use. \\
\hline
Best Practices & Written communication, ethical exit behavior, clear compliance policies. \\
\hline
\end{tabular}
\end{center}

---

\subsection*{Standard IV(B): Additional Compensation Arrangements}

\textbf{Standard Text:}
\begin{quote}
Members and Candidates must not accept gifts, benefits, compensation, or consideration that competes with or might reasonably be expected to create a conflict of interest with their employer’s interest unless they obtain written consent from all parties involved.
\end{quote}

---

\subsubsection*{1. Core Principles}

\begin{itemize}
  \item Members must not accept outside compensation that conflicts with employer interests without prior written consent.  
  \item Compensation includes both direct (money) and indirect (benefits, gifts, incentives) forms.  
  \item Written consent may be through formal letter, email, or signed communication.
\end{itemize}

---

\subsubsection*{2. Additional Compensation vs. Gifts}

\begin{itemize}
  \item \textbf{Additional Compensation:}  
  - Offered for future performance (e.g., bonus tied to portfolio returns).  
  - Requires \textbf{written consent} before acceptance.
  
  \item \textbf{Gift:}  
  - Given for past performance.  
  - Requires \textbf{disclosure} to the employer (under Standard I(B) — Independence and Objectivity).
\end{itemize}

---

\subsubsection*{3. Examples of Violations}

\begin{center}
\begin{tabular}{|p{8cm}|p{8cm}|}
\hline
\textbf{Situation} & \textbf{Why It Violates the Standard} \\
\hline
Accepting a client’s offer of a bonus for outperforming the market without informing employer & Creates potential conflict; requires written consent. \\
\hline
Receiving commissions from a third-party fund manager for recommending its funds & Competes with employer’s interest and biases recommendations. \\
\hline
Accepting free luxury trips for future business promotion & Indirect benefit tied to job performance — needs consent. \\
\hline
\end{tabular}
\end{center}

---

\subsubsection*{4. Recommendations for Members}

\begin{itemize}
  \item Report any proposed additional compensation in writing to the employer before acceptance.  
  \item Describe the nature, amount, and duration of compensation clearly.  
  \item If employed part-time, clarify outside work and compensation at hiring.  
\end{itemize}

---

\subsubsection*{5. Recommendations for Firms}

\begin{itemize}
  \item Verify details of external compensation with offering parties.  
  \item Create procedures for reviewing and approving outside benefits.  
  \item Record all disclosures in compliance documentation.
\end{itemize}

---

\subsubsection*{6. Summary Table: Standard IV(B) — Additional Compensation Arrangements}

\begin{center}
\begin{tabular}{|l|p{10cm}|}
\hline
\textbf{Aspect} & \textbf{Explanation and Examples} \\
\hline
Goal & Prevent conflicts of interest between employee and employer. \\
\hline
Requirement & Obtain written consent from all parties before accepting any external benefit. \\
\hline
Key Distinction & Additional compensation (future) vs. gift (past). \\
\hline
Violations & Accepting undisclosed future-based bonuses, third-party incentives. \\
\hline
Best Practices & Written disclosure, verification, firm approval process. \\
\hline
\end{tabular}
\end{center}

---

\subsection*{Standard IV(C): Responsibilities of Supervisors}

\textbf{Standard Text:}
\begin{quote}
Members and Candidates must make reasonable efforts to ensure that anyone subject to their supervision or authority complies with applicable laws, rules, regulations, and the Code and Standards.
\end{quote}

---

\subsubsection*{1. Core Principles}

\begin{itemize}
  \item Supervisors are responsible for both preventing and detecting violations of laws and the CFA Code and Standards.  
  \item Supervisory duties apply once a member has direct or delegated authority over others.  
  \item Reasonable efforts include implementing and maintaining effective compliance systems.
\end{itemize}

---

\subsubsection*{2. Compliance Systems}

\begin{itemize}
  \item Must meet:
  \begin{itemize}
    \item Industry standards;  
    \item Regulatory requirements;  
    \item CFA Institute’s ethical expectations.
  \end{itemize}
  \item Components of an adequate system:
  \begin{itemize}
    \item Clearly written and easy-to-understand procedures;  
    \item Designated compliance officer with authority;  
    \item System of checks and balances;  
    \item Defined scope, reporting procedures, and sanctions.
  \end{itemize}
\end{itemize}

---

\subsubsection*{3. Dealing with Deficient Compliance Systems}

\begin{itemize}
  \item If no adequate system exists, supervisors must:
  \begin{itemize}
    \item Decline supervisory responsibility in writing;  
    \item Inform management and recommend improvements.  
  \end{itemize}
  \item Failure to do so can make the supervisor personally accountable for violations committed by subordinates.
\end{itemize}

---

\subsubsection*{4. Handling Violations}

\begin{itemize}
  \item Respond promptly upon discovery.  
  \item Conduct a thorough investigation.  
  \item Increase supervision or restrict the employee’s activities during the investigation.  
  \item Report outcomes and enforce sanctions appropriately.
\end{itemize}

---

\subsubsection*{5. Recommendations for Members}

\begin{itemize}
  \item Recommend adoption of a firm code of ethics and distribute it widely.  
  \item Educate and train employees regularly.  
  \item Issue reminders and periodic updates.  
  \item Require professional conduct evaluations.  
  \item Review employee actions periodically for compliance.
\end{itemize}

---

\subsubsection*{6. Recommendations for Firms}

\begin{itemize}
  \item Separate compliance procedures from the firm’s code of ethics.  
  \item Limit activities of suspected violators during investigations.  
  \item Structure incentives to discourage unethical behavior.  
  \item Ensure procedures are updated and reflect legal and regulatory changes.
\end{itemize}

---

\subsubsection*{7. Summary Table: Standard IV(C) — Responsibilities of Supervisors}

\begin{center}
\begin{tabular}{|l|p{10cm}|}
\hline
\textbf{Aspect} & \textbf{Explanation and Examples} \\
\hline
Goal & Ensure compliance with laws, regulations, and the CFA Code. \\
\hline
Obligation & Prevent and detect violations by subordinates. \\
\hline
Compliance System & Written, clear, effective, and regularly updated. \\
\hline
Deficient Systems & Decline responsibility in writing until corrected. \\
\hline
Response to Violations & Investigate promptly and increase supervision. \\
\hline
Best Practices & Education, reminders, compliance audits, ethical culture. \\
\hline
\end{tabular}
\end{center}

---

\subsection*{Comparison of Standards IV(A), IV(B), and IV(C)}

\begin{center}
\begin{tabular}{|p{2.5cm}|p{4.5cm}|p{4.5cm}|p{4.5cm}|}
\hline
\textbf{Dimension} & \textbf{IV(A) Loyalty} & \textbf{IV(B) Additional Compensation} & \textbf{IV(C) Responsibilities of Supervisors} \\
\hline
Focus & Duty to act in employer’s best interest. & Avoiding conflicts from external benefits. & Ensuring compliance by subordinates. \\
\hline
Key Requirement & No harm to employer, written notice before outside work. & Written consent for any additional compensation. & Effective supervision and compliance procedures. \\
\hline
Typical Violations & Client solicitation, misuse of info, secret business. & Undisclosed performance-based bonuses or incentives. & Ignoring violations, failing to install compliance systems. \\
\hline
Best Practices & Written approvals, ethical exits, separate accounts. & Transparency, documentation, verification. & Training, monitoring, prompt investigation. \\
\hline
Ethical Emphasis & Loyalty and transparency. & Independence and conflict avoidance. & Accountability and ethical leadership. \\
\hline
\end{tabular}
\end{center}

---

\subsection*{Key Takeaways}

\begin{itemize}
  \item \textbf{Standard IV(A): Loyalty} — Act in the employer’s best interest without harming firm integrity; obtain consent for outside work; avoid misappropriation when leaving.  
  \item \textbf{Standard IV(B): Additional Compensation Arrangements} — Avoid conflicts by disclosing and obtaining written approval before accepting any outside compensation.  
  \item \textbf{Standard IV(C): Responsibilities of Supervisors} — Supervisors must establish and enforce robust compliance systems and respond swiftly to violations.  
  \item Together, these standards promote \textbf{ethical leadership, transparency, and accountability} in professional relationships.
\end{itemize}

\section*{MODULE 91.7: GUIDANCE FOR STANDARD V — INVESTMENT ANALYSIS, RECOMMENDATIONS, AND ACTIONS}

\textbf{Purpose:}  
Standard V emphasizes professional diligence, transparent communication, and responsible recordkeeping in all aspects of the investment process.  
It ensures that analysis and recommendations are based on adequate research, properly communicated, and supported by evidence.

---

\subsection*{Standard V(A): Diligence and Reasonable Basis}

\textbf{Standard Text:}
\begin{quote}
Members and Candidates must:
\begin{enumerate}
    \item Exercise diligence, independence, and thoroughness in analyzing investments, making investment recommendations, and taking investment actions.
    \item Have a reasonable and adequate basis, supported by appropriate research and investigation, for any investment analysis, recommendation, or action.
\end{enumerate}
\end{quote}

---

\subsubsection*{1. Core Principles}

\begin{itemize}
  \item Investment recommendations must be supported by \textbf{sound research and reasonable evidence}.  
  \item Due diligence and independent judgment must guide all analyses and decisions.  
  \item The extent of required research depends on:
  \begin{itemize}
    \item The member’s role (analyst, portfolio manager, etc.);
    \item The investment philosophy of the firm;
    \item The nature and complexity of the security or service.
  \end{itemize}
  \item Members must not rely blindly on external sources without assessing their quality.
\end{itemize}

---

\subsubsection*{2. Factors to Consider Before Making Recommendations}

\begin{itemize}
  \item Global and national economic conditions;  
  \item Company financial performance, operating history, and industry cycle;  
  \item Fee structure and historical results (for funds);  
  \item Assumptions and limitations of quantitative models;  
  \item Appropriateness of peer-group or comparative valuation metrics.
\end{itemize}

---

\subsubsection*{3. Using Third-Party Research}

\begin{itemize}
  \item Members may use external research, but must first evaluate its:
  \begin{itemize}
    \item Assumptions;  
    \item Analytical rigor;  
    \item Timeliness and accuracy;  
    \item Objectivity and independence.
  \end{itemize}
  \item Firms should establish a \textbf{review policy} for the quality and credibility of third-party research.
\end{itemize}

---

\subsubsection*{4. Recommendations for Members}

\begin{itemize}
  \item Advocate for firm policies requiring:
  \begin{itemize}
    \item Substantiated research and documentation of all recommendations;  
    \item Written procedures defining minimum standards for due diligence;  
    \item Measurable performance criteria for research quality;  
    \item Scenario testing and sensitivity analysis for quantitative models;  
    \item Regular evaluation of data vendors and external advisors.
  \end{itemize}
\end{itemize}

---

\subsubsection*{5. Examples of Violations}

\begin{center}
\begin{tabular}{|p{8cm}|p{8cm}|}
\hline
\textbf{Situation} & \textbf{Reason for Violation} \\
\hline
Recommending a stock after reading only a news headline & Lacks adequate research basis and due diligence. \\
\hline
Using a third-party model without reviewing its assumptions & No reasonable investigation of accuracy or reliability. \\
\hline
Failing to adjust valuation for known macroeconomic risks & Insufficient diligence and incomplete analysis. \\
\hline
Allowing external pressure from management to influence analysis outcome & Violates independence and objectivity. \\
\hline
\end{tabular}
\end{center}

---

\subsubsection*{6. Summary Table: Standard V(A) — Diligence and Reasonable Basis}

\begin{center}
\begin{tabular}{|l|p{10cm}|}
\hline
\textbf{Aspect} & \textbf{Explanation and Examples} \\
\hline
Goal & Ensure all investment actions are based on thorough, independent research. \\
\hline
Key Duties & Diligence, independence, reasonable basis, adequate research. \\
\hline
Due Diligence Areas & Economic factors, firm fundamentals, model limitations, peer comparisons. \\
\hline
Violations & Insufficient analysis, blind reliance on third-party data, external pressure influence. \\
\hline
Best Practices & Scenario testing, sensitivity analysis, external research review, documentation. \\
\hline
\end{tabular}
\end{center}

---

\subsection*{Standard V(B): Communication with Clients and Prospective Clients}

\textbf{Standard Text:}
\begin{quote}
Members and Candidates must:
\begin{enumerate}
    \item Disclose to clients and prospective clients the basic format and general principles of their investment processes and promptly disclose any material changes.
    \item Disclose significant limitations and risks associated with the investment process.
    \item Use reasonable judgment in identifying important factors and include them in communications.
    \item Distinguish between fact and opinion in presenting analyses and recommendations.
\end{enumerate}
\end{quote}

---

\subsubsection*{1. Core Principles}

\begin{itemize}
  \item Clear, accurate, and complete communication is essential.  
  \item Applies to all forms of communication — written, oral, electronic, or visual.  
  \item Members must:
  \begin{itemize}
    \item Explain their investment process and changes to it;  
    \item Identify and disclose key risks and assumptions;  
    \item Distinguish facts (data, verifiable info) from opinions or forecasts.  
  \end{itemize}
\end{itemize}

---

\subsubsection*{2. Communicating Risks and Limitations}

\begin{itemize}
  \item Members must disclose all significant risks such as:
  \begin{itemize}
    \item Liquidity (difficulty in exiting positions);  
    \item Capacity (limit to which fund size affects returns);  
    \item Volatility or leverage exposure;  
    \item Model or data limitations.  
  \end{itemize}
  \item Clients must understand potential outcomes in terms of \textbf{total returns}, not just price changes.
\end{itemize}

---

\subsubsection*{3. Communicating Model-Based Recommendations}

\begin{itemize}
  \item Disclose:
  \begin{itemize}
    \item Model assumptions and data inputs;  
    \item Sensitivity to changes in parameters;  
    \item Uncertainty inherent in statistical results.  
  \end{itemize}
  \item Example: A quantitative analyst must explain that “expected return = 8\%” is a model-based estimate, not a guaranteed result.
\end{itemize}

---

\subsubsection*{4. Ongoing Communication}

\begin{itemize}
  \item Clients must be updated regularly about:
  \begin{itemize}
    \item Changes in process, methodology, or risk;  
    \item Significant developments affecting portfolio performance;  
    \item New or emerging investment limitations.  
  \end{itemize}
\end{itemize}

---

\subsubsection*{5. Examples of Violations}

\begin{center}
\begin{tabular}{|p{8cm}|p{8cm}|}
\hline
\textbf{Situation} & \textbf{Reason for Violation} \\
\hline
Failing to disclose a change in investment strategy to clients & Clients misled about process and risks. \\
\hline
Presenting a model output as a guaranteed forecast & Misrepresentation — confusion between fact and opinion. \\
\hline
Not explaining liquidity constraints in a hedge fund & Lack of risk disclosure. \\
\hline
Communicating outdated investment methodology & Misleads clients — incomplete information. \\
\hline
\end{tabular}
\end{center}

---

\subsubsection*{6. Recommendations for Members}

\begin{itemize}
  \item Maintain records of all research and communications to substantiate recommendations.  
  \item Include all material information needed for client understanding.  
  \item Clearly label assumptions, projections, and opinions.  
  \item Explain methodology, especially for structured or complex products.  
  \item Use plain, accessible language for non-professional audiences.
\end{itemize}

---

\subsubsection*{7. Summary Table: Standard V(B) — Communication with Clients}

\begin{center}
\begin{tabular}{|l|p{10cm}|}
\hline
\textbf{Aspect} & \textbf{Explanation and Examples} \\
\hline
Goal & Ensure clarity, transparency, and understanding in client communication. \\
\hline
Key Duties & Disclose process, risks, limitations, and distinguish fact from opinion. \\
\hline
Common Risks & Liquidity, capacity, model assumptions, volatility. \\
\hline
Violations & Failing to disclose changes, overstating certainty, omitting risks. \\
\hline
Best Practices & Recordkeeping, clear labeling, timely updates, full transparency. \\
\hline
\end{tabular}
\end{center}

---

\subsection*{Standard V(C): Record Retention}

\textbf{Standard Text:}
\begin{quote}
Members and Candidates must develop and maintain appropriate records to support their investment analyses, recommendations, actions, and other investment-related communications with clients and prospective clients.
\end{quote}

---

\subsubsection*{1. Core Principles}

\begin{itemize}
  \item Members must keep adequate records to:
  \begin{itemize}
    \item Substantiate the basis for investment actions and advice;  
    \item Support performance data and disclosures;  
    \item Demonstrate compliance with laws and the Code and Standards.  
  \end{itemize}
  \item Records belong to the employer, not the individual.  
  \item Applies to all communications — reports, emails, calls, and text messages.
\end{itemize}

---

\subsubsection*{2. Changing Employers}

\begin{itemize}
  \item Members leaving a firm may not take research or client records.  
  \item They may recreate analyses using \textbf{publicly available data}.  
  \item Memory or confidential materials from the prior employer cannot be used.
\end{itemize}

---

\subsubsection*{3. Record Retention Duration}

\begin{itemize}
  \item Follow local laws and firm policies.  
  \item If none exist, CFA Institute recommends retaining records for at least \textbf{seven years}.
\end{itemize}

---

\subsubsection*{4. Examples of Violations}

\begin{center}
\begin{tabular}{|p{8cm}|p{8cm}|}
\hline
\textbf{Situation} & \textbf{Reason for Violation} \\
\hline
Failing to keep analysis documents supporting a buy recommendation & Cannot substantiate research — non-compliant recordkeeping. \\
\hline
Deleting emails with client trade instructions & Loss of required record — violates Standard V(C). \\
\hline
Using proprietary data from former employer & Misuse of firm property. \\
\hline
\end{tabular}
\end{center}

---

\subsubsection*{5. Recommendations for Members}

\begin{itemize}
  \item Keep written or electronic copies of all research, notes, and client communications.  
  \item Ensure compliance with firm recordkeeping policies.  
  \item Back up records securely and restrict access to authorized personnel.  
  \item Retain records for at least seven years if no regulatory standard applies.  
\end{itemize}

---

\subsubsection*{6. Summary Table: Standard V(C) — Record Retention}

\begin{center}
\begin{tabular}{|l|p{10cm}|}
\hline
\textbf{Aspect} & \textbf{Explanation and Examples} \\
\hline
Goal & Maintain documentation supporting all investment decisions and communications. \\
\hline
Ownership & Records belong to the employer, not the individual. \\
\hline
Retention Period & Minimum of seven years (if no legal standard specified). \\
\hline
Violations & Failing to keep research notes, deleting client instructions, using old firm data. \\
\hline
Best Practices & Centralized storage, compliance oversight, periodic audits. \\
\hline
\end{tabular}
\end{center}

---

\subsection*{Comparison: Standards V(A), V(B), and V(C)}

\begin{center}
\begin{tabular}{|p{2.5cm}|p{4.5cm}|p{4.5cm}|p{4.5cm}|}
\hline
\textbf{Dimension} & \textbf{V(A): Diligence \& Reasonable Basis} & \textbf{V(B): Communication with Clients} & \textbf{V(C): Record Retention} \\
\hline
Focus & Quality and depth of research and analysis. & Transparency in process, risks, and updates. & Documentation supporting all actions and communications. \\
\hline
Key Duty & Independent, evidence-based diligence. & Full, fair, and clear disclosure. & Maintain accurate and accessible records. \\
\hline
Common Violations & Inadequate research, overreliance on external data. & Omitting risks, failing to distinguish fact from opinion. & Destroying or removing firm-owned records. \\
\hline
Best Practices & Scenario testing, model validation, external review. & Clear, factual, updated, and documented communication. & Seven-year retention, secure storage, firm ownership. \\
\hline
Ethical Focus & Integrity in analysis. & Transparency in communication. & Accountability in documentation. \\
\hline
\end{tabular}
\end{center}

---

\subsection*{Key Takeaways}

\begin{itemize}
  \item \textbf{Standard V(A): Diligence and Reasonable Basis} — Perform independent, thorough research with adequate evidence before recommending or acting.  
  \item \textbf{Standard V(B): Communication with Clients} — Disclose processes, risks, and assumptions; distinguish facts from opinions; communicate changes promptly.  
  \item \textbf{Standard V(C): Record Retention} — Maintain complete and accessible documentation supporting all analyses and actions for at least seven years.  
  \item Together, these standards protect client trust and uphold professional accountability in all investment-related conduct.
\end{itemize}

\section*{MODULE 91.8: GUIDANCE FOR STANDARD VI — CONFLICTS OF INTEREST}

\textbf{Purpose:}  
Standard VI ensures transparency, fairness, and integrity when potential conflicts exist between the interests of clients, employers, and members themselves.  
It focuses on disclosure, prioritization of client interests, and openness about referral compensation.

---

\subsection*{Standard VI(A): Disclosure of Conflicts}

\textbf{Standard Text:}
\begin{quote}
Members and Candidates must make full and fair disclosure of all matters that could reasonably be expected to impair their independence and objectivity or interfere with duties to clients, prospective clients, or employers.  
Disclosures must be prominent, delivered in plain language, and communicate relevant information effectively.
\end{quote}

---

\subsubsection*{1. Core Principles}

\begin{itemize}
  \item Full and fair disclosure allows clients and employers to evaluate the objectivity of members’ advice and potential biases.  
  \item Applies to both \textbf{actual} and \textbf{potential} conflicts of interest.  
  \item The disclosure must be:
  \begin{itemize}
    \item \textbf{Prominent} — not hidden or vague;  
    \item \textbf{Plain language} — understandable to all clients;  
    \item \textbf{Effective} — clearly conveying the nature and extent of the conflict.
  \end{itemize}
\end{itemize}

---

\subsubsection*{2. Common Conflicts Requiring Disclosure}

\begin{itemize}
  \item \textbf{Ownership interests:}  
  Holding or owning shares in companies that are subject of research or recommendations.
  \item \textbf{Compensation structures:}  
  Bonus schemes tied to sales volume or short-term results that may bias advice.  
  \item \textbf{Board service:}  
  Serving on a corporate board while analyzing or recommending its securities.  
  \item \textbf{Broker-dealer activities:}  
  Market-making or proprietary trading in securities that clients hold.  
  \item \textbf{Outside business activities:}  
  Other employment or consulting roles that may influence professional judgment.
\end{itemize}

---

\subsubsection*{3. Employer Responsibility}

\begin{itemize}
  \item Members must give their employer enough detail to evaluate the significance of any conflict.  
  \item They should take steps to avoid or mitigate conflicts when possible.  
  \item Prompt reporting is required when conflicts arise unexpectedly.
\end{itemize}

---

\subsubsection*{4. Examples of Violations}

\begin{center}
\begin{tabular}{|p{8cm}|p{8cm}|}
\hline
\textbf{Situation} & \textbf{Why It Violates the Standard} \\
\hline
Recommending a company in which the member owns stock without disclosure & Creates bias that could impair independence. \\
\hline
Receiving extra commissions tied to short-term trading activity without informing clients & Conflicts client’s long-term interest — not disclosed. \\
\hline
Failing to disclose service on the board of a client firm & Dual role compromises objectivity and independence. \\
\hline
Hiding bonus incentives related to fund inflows from client awareness & Misleads clients about advisor’s motivations. \\
\hline
\end{tabular}
\end{center}

---

\subsubsection*{5. Recommendations for Members}

\begin{itemize}
  \item Fully disclose all special compensation, bonus programs, commissions, and incentives.  
  \item Ensure disclosures are updated when compensation structures change.  
  \item Deliver disclosures in writing, using clear and accessible language.  
  \item Encourage firms to establish centralized conflict-of-interest registers.
\end{itemize}

---

\subsubsection*{6. Summary Table: Standard VI(A) — Disclosure of Conflicts}

\begin{center}
\begin{tabular}{|l|p{10cm}|}
\hline
\textbf{Aspect} & \textbf{Explanation and Examples} \\
\hline
Goal & Protect investor and employer trust through transparency. \\
\hline
Scope & Actual and potential conflicts (ownership, board service, bonuses). \\
\hline
Disclosure Standard & Prominent, plain-language, and effective communication. \\
\hline
Violations & Undisclosed ownership, hidden bonus incentives, failure to update disclosures. \\
\hline
Best Practices & Written disclosure, periodic updates, conflict registers, compliance oversight. \\
\hline
\end{tabular}
\end{center}

---

\subsection*{Standard VI(B): Priority of Transactions}

\textbf{Standard Text:}
\begin{quote}
Investment transactions for clients and employers must have priority over investment transactions in which a Member or Candidate is the beneficial owner.
\end{quote}

---

\subsubsection*{1. Core Principles}

\begin{itemize}
  \item Client interests come first — before personal or employer trading.  
  \item Personal transactions must not disadvantage clients or benefit from client trading.  
  \item Applies to direct and indirect ownership (e.g., spouse or family accounts where member is beneficial owner).  
  \item Personal trading may occur only:
  \begin{itemize}
    \item After clients and employer have had adequate opportunity to act;  
    \item Within firm’s trading and pre-clearance policies.
  \end{itemize}
\end{itemize}

---

\subsubsection*{2. Key Considerations}

\begin{itemize}
  \item \textbf{Family Accounts:}  
  Treated as client accounts if managed under the same fiduciary duty.  
  \item \textbf{Front Running:}  
  Buying or selling for personal benefit before client orders — strictly prohibited.  
  \item \textbf{Information Misuse:}  
  Acting on nonpublic knowledge of pending trades violates both Standard VI(B) and Standard II(A).
\end{itemize}

---

\subsubsection*{3. Firm Policies to Prevent Conflicts}

\begin{itemize}
  \item \textbf{Blackout/Restricted Periods:}  
  Employees may not trade before client or employer trades are executed.  
  \item \textbf{Preclearance Procedures:}  
  All personal trades must be reviewed by compliance.  
  \item \textbf{Duplicate Confirmations:}  
  Require brokerage firms to send duplicate trade records for monitoring.  
  \item \textbf{Holdings Disclosure:}  
  Employees must periodically disclose all beneficial ownership positions.
\end{itemize}

---

\subsubsection*{4. Examples of Violations}

\begin{center}
\begin{tabular}{|p{8cm}|p{8cm}|}
\hline
\textbf{Situation} & \textbf{Why It Violates the Standard} \\
\hline
Buying stock before a large client purchase recommendation is released & Front running — personal gain from client trade information. \\
\hline
Selling a personal holding just before a client sale order & Gains advantage at client’s expense. \\
\hline
Failing to disclose ownership in a security recommended to clients & Conflict between personal and client interest. \\
\hline
Allowing family accounts to trade ahead of other clients & Preferential treatment violates fair dealing. \\
\hline
\end{tabular}
\end{center}

---

\subsubsection*{5. Recommendations for Members}

\begin{itemize}
  \item Avoid participation in IPOs or private placements to eliminate perceived bias.  
  \item Follow firm preclearance, blackout, and reporting procedures strictly.  
  \item Disclose all personal holdings and beneficial ownership.  
  \item Do not act on material nonpublic information regarding pending trades.  
\end{itemize}

---

\subsubsection*{6. Recommendations for Firms}

\begin{itemize}
  \item Implement robust personal trading policies including:
  \begin{itemize}
    \item Blackout periods before client trades;  
    \item Restrictions on IPO/private placement participation;  
    \item Trade preclearance systems;  
    \item Duplicate confirmations and periodic reporting.
  \end{itemize}
  \item Ensure supervisory procedures monitor compliance effectively.
\end{itemize}

---

\subsubsection*{7. Summary Table: Standard VI(B) — Priority of Transactions}

\begin{center}
\begin{tabular}{|l|p{10cm}|}
\hline
\textbf{Aspect} & \textbf{Explanation and Examples} \\
\hline
Goal & Ensure client interests take precedence over personal or firm transactions. \\
\hline
Scope & All beneficial ownership transactions, including family accounts. \\
\hline
Prohibited Acts & Front running, self-dealing, misuse of client trade info. \\
\hline
Firm Controls & Preclearance, blackout periods, duplicate confirmations, periodic reporting. \\
\hline
Best Practices & Client-first approach, transparency, compliance monitoring. \\
\hline
\end{tabular}
\end{center}

---

\subsection*{Standard VI(C): Referral Fees}

\textbf{Standard Text:}
\begin{quote}
Members and Candidates must disclose to their employer, clients, and prospective clients any compensation, consideration, or benefit received from or paid to others for the recommendation of products or services.
\end{quote}

---

\subsubsection*{1. Core Principles}

\begin{itemize}
  \item Members must be transparent about any referral arrangements — whether they receive or pay compensation for client introductions or service recommendations.  
  \item Disclosure allows clients and employers to evaluate potential bias and total service costs.  
  \item Applies to both monetary and non-monetary benefits (e.g., gifts, commissions, discounts).
\end{itemize}

---

\subsubsection*{2. Disclosure Requirements}

\begin{itemize}
  \item Disclose to:
  \begin{enumerate}
    \item \textbf{Employer} — to ensure internal approval and compliance;  
    \item \textbf{Client/Prospect} — so they can assess impartiality and costs;  
    \item \textbf{Regulators} — if legally required.  
  \end{enumerate}
  \item Disclosure must include:
  \begin{itemize}
    \item Nature of consideration (cash, commissions, in-kind benefit);  
    \item Value and timing of the benefit;  
    \item Recipient and payer details.
  \end{itemize}
\end{itemize}

---

\subsubsection*{3. Examples of Violations}

\begin{center}
\begin{tabular}{|p{8cm}|p{8cm}|}
\hline
\textbf{Situation} & \textbf{Why It Violates the Standard} \\
\hline
Receiving a fee from a mutual fund company for client referrals without disclosure & Creates hidden incentive — conflicts objectivity. \\
\hline
Paying another advisor for client introductions without informing employer & Undisclosed outgoing referral payment — non-transparent. \\
\hline
Accepting discounted travel or perks for promoting a fund & In-kind referral benefit — must be disclosed. \\
\hline
\end{tabular}
\end{center}

---

\subsubsection*{4. Recommendations for Members}

\begin{itemize}
  \item Maintain written records of all referral fee arrangements.  
  \item Disclose such arrangements to all relevant parties before execution.  
  \item Update employer quarterly or as terms change.  
  \item Encourage firms to develop clear approval processes for referral compensation.
\end{itemize}

---

\subsubsection*{5. Recommendations for Firms}

\begin{itemize}
  \item Require pre-approval of all referral agreements.  
  \item Maintain centralized logs of all referral-related compensation.  
  \item Define policies covering both cash and non-cash benefits.  
  \item Audit periodically to ensure adherence and consistency of disclosures.
\end{itemize}

---

\subsubsection*{6. Summary Table: Standard VI(C) — Referral Fees}

\begin{center}
\begin{tabular}{|l|p{10cm}|}
\hline
\textbf{Aspect} & \textbf{Explanation and Examples} \\
\hline
Goal & Ensure transparency about referral-related compensation or benefits. \\
\hline
Disclosure Requirement & Inform employer, clients, and prospects of any consideration received or paid. \\
\hline
Types of Benefits & Cash, commission, travel perks, discounts, in-kind rewards. \\
\hline
Violations & Undisclosed payments or gifts influencing referrals. \\
\hline
Best Practices & Written disclosure, employer approval, quarterly updates, centralized recordkeeping. \\
\hline
\end{tabular}
\end{center}

---

\subsection*{Comparison of Standards VI(A), VI(B), and VI(C)}

\begin{center}
\begin{tabular}{|p{2.5cm}|p{4.5cm}|p{4.5cm}|p{4.5cm}|}
\hline
\textbf{Dimension} & \textbf{VI(A) Disclosure of Conflicts} & \textbf{VI(B) Priority of Transactions} & \textbf{VI(C) Referral Fees} \\
\hline
Focus & Transparency about personal or professional conflicts of interest. & Prioritizing client and employer trades over personal transactions. & Full disclosure of referral compensation and benefits. \\
\hline
Main Duty & Disclose actual and potential conflicts clearly and promptly. & Ensure client interests are not disadvantaged by personal trading. & Inform all parties of referral-related compensation or consideration. \\
\hline
Typical Violations & Undisclosed ownership, bonus structures, board service. & Front running, self-dealing, misuse of client trade info. & Hidden referral payments or non-cash benefits. \\
\hline
Best Practices & Written disclosure, periodic review, conflict registers. & Preclearance, blackout periods, client-first execution. & Written employer and client disclosure, quarterly updates. \\
\hline
Ethical Focus & Transparency and independence. & Fairness and client priority. & Honesty and disclosure. \\
\hline
\end{tabular}
\end{center}

---

\subsection*{Key Takeaways}

\begin{itemize}
  \item \textbf{Standard VI(A): Disclosure of Conflicts} — Full and fair disclosure of all relationships or incentives that could impair independence or create bias.  
  \item \textbf{Standard VI(B): Priority of Transactions} — Client and employer trades take absolute priority over personal or related-party trades.  
  \item \textbf{Standard VI(C): Referral Fees} — Disclose all compensation or benefits related to client referrals or product recommendations.  
  \item Together, these standards uphold the principles of \textbf{transparency, fairness, and client-first conduct}, ensuring market integrity and trust.
\end{itemize}


\section*{MODULE 91.9: GUIDANCE FOR STANDARD VII — RESPONSIBILITIES AS A CFA INSTITUTE MEMBER OR CFA CANDIDATE}

\textbf{Purpose:}  
Standard VII governs ethical conduct in relation to CFA Institute, its programs, and the CFA designation.  
It ensures that members and candidates protect the integrity, credibility, and prestige of the CFA charter and the examination process.

---

\subsection*{Standard VII(A): Conduct as Participants in CFA Institute Programs}

\textbf{Standard Text:}
\begin{quote}
Members and Candidates must not engage in any conduct that compromises the reputation or integrity of CFA Institute, the CFA designation, or the integrity, validity, or security of CFA Institute programs.
\end{quote}

---

\subsubsection*{1. Core Principles}

\begin{itemize}
  \item The CFA charter’s value depends on the integrity and fairness of the exam and the conduct of its participants.  
  \item Members and candidates must uphold the reputation of CFA Institute and protect the confidentiality of the program.  
  \item Applies to all forms of misconduct, including cheating, misrepresentation, or revealing exam content.  
  \item Covers both candidates (exam-takers) and members (charterholders or volunteers).
\end{itemize}

---

\subsubsection*{2. Prohibited Conduct}

\begin{itemize}
  \item \textbf{Cheating or misconduct during exams:}
  \begin{itemize}
    \item Copying answers, using unauthorized materials, or accessing notes or devices.  
    \item Violating calculator, personal belongings, or Candidate Pledge rules.
  \end{itemize}
  \item \textbf{Disclosing exam content:}
  \begin{itemize}
    \item Sharing or discussing exam questions, topics tested, or required formulas.  
    \item Applies even after the exam administration is complete.
  \end{itemize}
  \item \textbf{Improper use of confidential information:}
  \begin{itemize}
    \item Sharing information about question development, grading, or exam scoring (for CFA volunteers).  
    \item Revealing any CFA Institute internal processes.
  \end{itemize}
  \item \textbf{Misrepresentation to CFA Institute:}
  \begin{itemize}
    \item Providing false information on the Professional Conduct Statement (PCS) or Continuing Education Program.  
    \item Failing to disclose disciplinary or legal violations truthfully.
  \end{itemize}
\end{itemize}

---

\subsubsection*{3. Scope of Applicability}

\begin{itemize}
  \item Applies to all participants in CFA Institute programs — candidates, members, and volunteers.  
  \item Includes conduct \textbf{before, during, and after} the exam, as well as behavior outside the testing environment that harms CFA Institute’s reputation.
\end{itemize}

---

\subsubsection*{4. Examples of Violations}

\begin{center}
\begin{tabular}{|p{8cm}|p{8cm}|}
\hline
\textbf{Situation} & \textbf{Reason for Violation} \\
\hline
A candidate copies from another examinee’s paper. & Cheating — compromises exam integrity. \\
\hline
Posting on social media about exam questions or topics tested. & Reveals confidential information about the exam. \\
\hline
A CFA charterholder discloses internal grading procedures to the public. & Violates confidentiality of CFA Institute processes. \\
\hline
Falsifying information on the annual Professional Conduct Statement. & Dishonesty toward CFA Institute — damages reputation. \\
\hline
Using CFA Institute volunteer access to preview future exam content. & Misuse of privileged information — integrity breach. \\
\hline
\end{tabular}
\end{center}

---

\subsubsection*{5. Permitted Conduct}

\begin{itemize}
  \item Expressing personal opinions about the CFA Program (difficulty, study hours, fairness) is allowed, provided no confidential details are revealed.  
  \item Discussing general study strategies or materials publicly is acceptable.  
  \item Giving feedback to CFA Institute through official channels is encouraged.
\end{itemize}

---

\subsubsection*{6. Recommendations for Members and Candidates}

\begin{itemize}
  \item Review and strictly follow all CFA Program testing rules and policies.  
  \item Avoid any discussion or sharing of exam-specific details, online or offline.  
  \item Keep exam materials, including topics and formulas, confidential.  
  \item Be truthful and accurate when submitting Professional Conduct Statements.  
  \item Maintain professionalism when discussing CFA Institute publicly.
\end{itemize}

---

\subsubsection*{7. Summary Table: Standard VII(A) — Conduct in CFA Institute Programs}

\begin{center}
\begin{tabular}{|l|p{10cm}|}
\hline
\textbf{Aspect} & \textbf{Explanation and Examples} \\
\hline
Goal & Protect the integrity, validity, and reputation of the CFA Program and designation. \\
\hline
Prohibited Actions & Cheating, revealing exam content, falsifying information, violating exam policies. \\
\hline
Scope & All CFA candidates, members, and volunteers. \\
\hline
Examples & Posting exam topics online, disclosing grading processes, lying on PCS form. \\
\hline
Best Practices & Confidentiality, honesty, compliance with all CFA Program rules. \\
\hline
\end{tabular}
\end{center}

---

\subsection*{Standard VII(B): Reference to CFA Institute, the CFA Designation, and the CFA Program}

\textbf{Standard Text:}
\begin{quote}
When referring to CFA Institute, CFA Institute membership, the CFA designation, or candidacy in the CFA Program, Members and Candidates must not misrepresent or exaggerate the meaning or implications of membership, holding the CFA designation, or candidacy in the CFA Program.
\end{quote}

---

\subsubsection*{1. Core Principles}

\begin{itemize}
  \item Members and candidates must represent their relationship with CFA Institute accurately and modestly.  
  \item Must not make promises or implications that the CFA designation guarantees superior investment performance, employment, or ethical behavior.  
  \item The CFA designation is earned and maintained only by:
  \begin{itemize}
    \item Completing all three CFA exams;  
    \item Satisfying work experience requirements;  
    \item Submitting annual PCS and paying membership dues.
  \end{itemize}
  \item There is no “partial CFA” — one is either a candidate, a charterholder, or neither.
\end{itemize}

---

\subsubsection*{2. Proper Use of the CFA Designation}

\begin{itemize}
  \item “John Smith, CFA” — correct usage (the designation follows the name).  
  \item Members must not use the CFA designation as a noun (e.g., “I am a CFA” is incorrect).  
  \item The charter must always be written in capital letters and without periods (“CFA,” not “C.F.A.”).
\end{itemize}

---

\subsubsection*{3. Proper Reference to CFA Program Candidacy}

\begin{itemize}
  \item Correct: “Level II candidate in the CFA Program.”  
  \item Incorrect: “CFA Level II” or “CFA Level II holder.”  
  \item Candidates must not imply partial completion or guaranteed success.
\end{itemize}

---

\subsubsection*{4. Common Misrepresentations to Avoid}

\begin{itemize}
  \item Claiming that the CFA designation guarantees better performance or job outcomes.  
  \item Implying that the CFA charterholder status ensures higher ethical behavior by itself.  
  \item Using the CFA mark for commercial or promotional exaggeration (e.g., “Hire us — all CFAs outperform the market”).  
  \item Stating or implying partial achievement (e.g., “Passed CFA Level I — thus a partial CFA”).  
  \item Continuing to claim active membership after failing to pay dues or sign PCS.
\end{itemize}

---

\subsubsection*{5. Membership Obligations}

\begin{itemize}
  \item Members must:
  \begin{enumerate}
    \item Sign the Professional Conduct Statement (PCS) annually.  
    \item Pay annual CFA Institute membership dues.  
  \end{enumerate}
  \item Failure to meet these obligations results in inactive status — the member may not present themselves as a CFA Institute member or active charterholder.
\end{itemize}

---

\subsubsection*{6. Examples of Violations}

\begin{center}
\begin{tabular}{|p{8cm}|p{8cm}|}
\hline
\textbf{Situation} & \textbf{Reason for Violation} \\
\hline
Claiming to be a “CFA Level III” without passing the exam & Misrepresentation of candidacy. \\
\hline
Using “CFA” as a noun (“I am a CFA”) & Improper use of the designation. \\
\hline
Advertising “Guaranteed higher returns because we are CFAs” & Exaggerates meaning of the designation. \\
\hline
Continuing to use “CFA” after failing to pay annual dues & Misrepresentation of active membership. \\
\hline
Claiming to have earned the charter faster to suggest superior skill & Improper self-promotion — misleading. \\
\hline
\end{tabular}
\end{center}

---

\subsubsection*{7. Recommendations for Members and Candidates}

\begin{itemize}
  \item Use the CFA designation only after officially awarded by CFA Institute.  
  \item Follow official branding guidelines for use of “CFA,” “CFA Institute,” and “CFA Program.”  
  \item Ensure promotional materials accurately reflect membership or candidacy status.  
  \item Update employer and marketing documents if membership becomes inactive.  
  \item Refrain from making comparisons or guarantees based on holding the charter.
\end{itemize}

---

\subsubsection*{8. Summary Table: Standard VII(B) — Reference to CFA Institute, the CFA Designation, and the CFA Program}

\begin{center}
\begin{tabular}{|l|p{10cm}|}
\hline
\textbf{Aspect} & \textbf{Explanation and Examples} \\
\hline
Goal & Ensure accurate, modest, and proper reference to the CFA designation and program. \\
\hline
Proper Use & “John Smith, CFA” or “Level II candidate in the CFA Program.” \\
\hline
Improper Use & “CFA Level II holder,” “I am a CFA,” or implying job superiority. \\
\hline
Membership Requirement & Sign PCS + pay dues annually to maintain active status. \\
\hline
Violations & Misrepresentation, exaggeration, continued use after inactivity. \\
\hline
Best Practices & Follow official usage rules, disclose accurate status, avoid promotional misuse. \\
\hline
\end{tabular}
\end{center}

---

\subsection*{Comparison of Standards VII(A) and VII(B)}

\begin{center}
\begin{tabular}{|l|p{6cm}|p{6cm}|}
\hline
\textbf{Dimension} & \textbf{VII(A): Conduct in CFA Programs} & \textbf{VII(B): Reference to CFA Institute and CFA Designation} \\
\hline
Focus & Protect the integrity of the CFA exam and program. & Prevent misrepresentation of CFA membership or designation. \\
\hline
Prohibited Acts & Cheating, disclosing exam content, falsifying PCS. & Misuse of “CFA,” exaggerating competence, false membership claims. \\
\hline
Scope & Candidates, members, volunteers. & Members, candidates, firms using CFA name. \\
\hline
Examples of Violations & Discussing exam topics online; lying on PCS. & Claiming “CFA Level II holder”; advertising guaranteed success. \\
\hline
Best Practices & Confidentiality, honesty, compliance with CFA Program rules. & Accurate reference, follow CFA branding policy, avoid exaggeration. \\
\hline
Ethical Focus & Integrity and confidentiality. & Truthfulness and professional representation. \\
\hline
\end{tabular}
\end{center}

---

\subsection*{Key Takeaways}

\begin{itemize}
  \item \textbf{Standard VII(A): Conduct in CFA Institute Programs} — Maintain the confidentiality, integrity, and reputation of CFA Institute and its examinations.  
  \item \textbf{Standard VII(B): Reference to CFA Institute and CFA Designation} — Use the CFA designation truthfully and accurately, without exaggeration or misrepresentation.  
  \item Members must fulfill annual obligations (PCS + dues) to maintain active status.  
  \item Together, these standards preserve the trust, credibility, and global prestige of the CFA charter.
\end{itemize}

\section*{MODULE 92.1: INTRODUCTION TO GIPS (GLOBAL INVESTMENT PERFORMANCE STANDARDS)}

\textbf{Purpose:}  
The Global Investment Performance Standards (GIPS) provide a globally accepted framework for calculating and presenting investment performance results.  
They ensure full, fair, and comparable disclosures, preventing performance misrepresentation and building trust between firms, clients, and regulators.

---

\subsection*{LOS 92.a: Why GIPS Were Created — Applicability and Benefits}

\subsubsection*{1. Background and Rationale}

\begin{itemize}
  \item Historically, firms used inconsistent methods to present performance:
  \begin{itemize}
    \item Highlighted only top-performing portfolios;
    \item Excluded terminated or poorly performing accounts;
    \item Selected favorable time periods to show higher returns.
  \end{itemize}
  \item These inconsistencies made it difficult for clients to compare performance across firms and led to misleading claims.
  \item \textbf{GIPS} were created to establish \textbf{uniform standards} for performance calculation and presentation — ensuring:
  \begin{itemize}
    \item Transparency,  
    \item Comparability,  
    \item Credibility.
  \end{itemize}
\end{itemize}

---

\subsubsection*{2. Scope of Applicability}

\begin{itemize}
  \item \textbf{Who can claim compliance:}
  \begin{itemize}
    \item Only \textbf{investment management firms} that manage assets on behalf of clients.  
    \item Compliance must be on a \textbf{firmwide basis} — not by individual departments, strategies, or composites.
  \end{itemize}
  \item \textbf{Who cannot claim compliance:}
  \begin{itemize}
    \item Software developers, data vendors, consultants, or custodians — they may state that they \emph{endorse} GIPS but cannot \emph{claim compliance}.
  \end{itemize}
  \item \textbf{Nature of compliance:}  
  Compliance with GIPS is voluntary but must be \emph{complete} and \emph{firmwide} if claimed.
\end{itemize}

---

\subsubsection*{3. Benefits of GIPS Compliance}

\begin{itemize}
  \item \textbf{For clients and prospects:} Enables meaningful comparison across investment firms.  
  \item \textbf{For firms:} Builds credibility, transparency, and reputation; attracts institutional clients.  
  \item \textbf{For regulators:} Provides consistency and clarity in performance measurement industrywide.
\end{itemize}

---

\subsubsection*{4. Examples of Non-GIPS-Compliant Practices}

\begin{center}
\begin{tabular}{|p{5cm}|p{10cm}|}
\hline
\textbf{Practice} & \textbf{Why It Violates GIPS Principles} \\
\hline
Showing returns for only one high-performing fund & Not representative; misleading portrayal of total performance. \\
\hline
Excluding accounts closed due to poor performance & Creates bias; hides underperformance. \\
\hline
Selecting only the best 3-year period to report & Cherry-picking; lacks consistency and comparability. \\
\hline
Combining different methodologies for different clients & Inconsistent and nonstandardized reporting. \\
\hline
\end{tabular}
\end{center}

---

\subsection*{LOS 92.b: Key Concepts — The Eight Sections of the GIPS Standards}

\begin{center}
\begin{tabular}{|p{6cm}|p{8cm}|}
\hline
\textbf{Section} & \textbf{Description and Key Ideas} \\
\hline
1. \textbf{Fundamentals of Compliance} &
Defines essential requirements:  
(a) Clearly define the firm;  
(b) Provide GIPS-compliant reports to all clients and prospects;  
(c) Follow all applicable laws and regulations;  
(d) Avoid any misleading or false presentations. \\
\hline
2. \textbf{Input Data and Calculation Methodology} &
Data must be accurate, consistent, and comparable.  
Firms must use standard return calculation methodologies (time-weighted or money-weighted).  
Uniformity ensures comparability across firms. \\
\hline
3. \textbf{Composite and Pooled Fund Maintenance} &
Firms must create meaningful composites that group portfolios with the same investment strategy or objective.  
Performance is asset-weighted (not equally weighted).  
All fee-paying, discretionary portfolios must be included in at least one composite. \\
\hline
4. \textbf{Composite Time-Weighted Return Report} &
Specifies required and recommended disclosures for presenting time-weighted composite returns. Suitable when the manager does not control external cash flows. \\
\hline
5. \textbf{Composite Money-Weighted Return Report} &
Specifies required and recommended disclosures for composites where the firm has control over external cash flows. Suitable for private equity or real estate. \\
\hline
6. \textbf{Pooled Fund Time-Weighted Return Report} &
Outlines reporting requirements for time-weighted performance of pooled funds. \\
\hline
7. \textbf{Pooled Fund Money-Weighted Return Report} &
Outlines reporting requirements for money-weighted returns of pooled funds. \\
\hline
8. \textbf{GIPS Advertising Guidelines} &
Defines conditions under which a firm can advertise compliance.  
Only advertisements that explicitly claim GIPS compliance must follow these guidelines. \\
\hline
\end{tabular}
\end{center}

---

\subsubsection*{Interpretation:}
\begin{itemize}
  \item GIPS provides both \textbf{required} and \textbf{recommended} disclosures.  
  \item Firms need not include disclosures that are not applicable.  
  \item Once all requirements are met, firms may include an official \textbf{GIPS compliance statement}.
\end{itemize}

---

\subsection*{LOS 92.c: Purpose and Concept of Composites}

\begin{itemize}
  \item A \textbf{composite} is a group of discretionary portfolios managed according to a similar investment strategy, objective, or mandate.  
  \item Examples of composites:
  \begin{itemize}
    \item Large-cap growth equity portfolios;  
    \item Investment-grade domestic bond portfolios;  
    \item Accounts managed to replicate a specific index.  
  \end{itemize}
  \item Composites demonstrate a firm’s performance \emph{by strategy}, allowing clients to evaluate consistency and style-specific skill.
\end{itemize}

---

\subsubsection*{Composite Construction Principles}

\begin{itemize}
  \item Must include \textbf{all fee-paying, discretionary portfolios} (current and terminated) managed under that strategy.  
  \item Portfolio inclusion should be determined \textbf{before} performance results are known — this prevents cherry-picking.  
  \item Each portfolio must belong to \textbf{at least one composite}.  
  \item Composite returns are \textbf{asset-weighted averages} — larger portfolios have greater influence.  
  \item Pooled funds are included if they meet the composite definition.
\end{itemize}

---

\subsubsection*{Examples of Proper Composite Practices}

\begin{center}
\begin{tabular}{|p{5cm}|p{10cm}|}
\hline
\textbf{Example} & \textbf{Explanation} \\
\hline
Including all discretionary, fee-paying portfolios in a large-cap composite & Ensures fair, representative performance. \\
\hline
Assigning portfolios to composites based on mandate before results are known & Prevents selection bias. \\
\hline
Calculating composite return as asset-weighted average & Reflects size-adjusted portfolio performance. \\
\hline
\end{tabular}
\end{center}

---

\subsection*{LOS 92.d: Fundamentals of Compliance — Defining the Firm and Discretion}

\subsubsection*{1. Definition of the Firm}

\begin{itemize}
  \item The firm is the \textbf{entity presented to clients as a distinct business unit}.  
  \item If multiple geographic divisions share a common brand (e.g., “Bluestone Advisers”), the firm includes all such offices and clients.  
  \item GIPS compliance must cover the \textbf{entire firm}, not a subset.
\end{itemize}

\subsubsection*{2. Definition of Discretion}

\begin{itemize}
  \item “Discretion” refers to the ability to make independent investment decisions consistent with the client’s objectives.  
  \item Portfolios are \textbf{discretionary} if the manager has full authority to act per the strategy.  
  \item Portfolios are \textbf{nondiscretionary} if client-imposed restrictions prevent the manager from implementing the intended strategy.  
  \item Only discretionary portfolios are included in composites.
\end{itemize}

---

\subsubsection*{3. Examples: Discretionary vs. Nondiscretionary Portfolios}

\begin{center}
\begin{tabular}{|p{8cm}|p{8cm}|}
\hline
\textbf{Portfolio Type} & \textbf{Treatment Under GIPS} \\
\hline
Client allows manager full authority to follow investment strategy & Discretionary — must be included in relevant composite. \\
\hline
Client prohibits investments in certain industries, limiting strategy execution & Nondiscretionary — may be excluded from composite. \\
\hline
Client retains veto power over every trade decision & Nondiscretionary — cannot be treated as part of discretionary composite. \\
\hline
\end{tabular}
\end{center}

---

\subsection*{LOS 92.e: Independent Verification of GIPS Compliance}

\subsubsection*{1. Concept}

\begin{itemize}
  \item Verification provides an independent, third-party evaluation of whether the firm’s policies and performance presentations comply with GIPS requirements.  
  \item Verification enhances the credibility of the firm’s claim of compliance but is not mandatory.
\end{itemize}

---

\subsubsection*{2. Characteristics of Verification}

\begin{itemize}
  \item Must be performed by an \textbf{independent third-party verifier} — not internally.  
  \item Verification applies to the \textbf{entire firm}, not to individual composites.  
  \item The verifier attests that:
  \begin{enumerate}
    \item The firm has complied with all GIPS requirements for composite construction;  
    \item The firm’s processes and procedures are designed to ensure GIPS-compliant performance presentation and calculation.
  \end{enumerate}
\end{itemize}

---

\subsubsection*{3. Example Disclosure Language (Post-Verification)}

\begin{quote}
\textit{“[Firm Name] has been verified for the periods [insert dates] by [Verifier Name]. A copy of the verification report is available upon request.”}
\end{quote}

---

\subsubsection*{4. Examples: Verification Scope and Practice}

\begin{center}
\begin{tabular}{|p{8cm}|p{8cm}|}
\hline
\textbf{Scenario} & \textbf{Verification Applicability} \\
\hline
Firm hires an external verifier to check one equity composite only & Not valid — verification must be firmwide. \\
\hline
Firm’s processes for calculating performance and maintaining composites are verified by an independent third party & Compliant — firmwide verification achieved. \\
\hline
Firm self-reviews its own performance reports for GIPS compliance & Not acceptable — verification must be independent. \\
\hline
\end{tabular}
\end{center}

---

\subsection*{Summary Tables by Learning Objective}

\subsubsection*{1. LOS 92.a — Purpose and Scope of GIPS}

\begin{center}
\begin{tabular}{|l|p{10cm}|}
\hline
\textbf{Aspect} & \textbf{Explanation} \\
\hline
Purpose & Standardize performance reporting; prevent misrepresentation. \\
\hline
Applicability & Investment management firms managing client assets. \\
\hline
Nature & Voluntary but must be complete and firmwide. \\
\hline
Benefits & Enhances comparability, credibility, transparency. \\
\hline
\end{tabular}
\end{center}

\subsubsection*{2. LOS 92.b — Eight GIPS Sections}

\begin{center}
\begin{tabular}{|l|p{10cm}|}
\hline
\textbf{Section} & \textbf{Purpose} \\
\hline
1. Fundamentals & Define firm, avoid false information. \\
\hline
2. Input Data & Ensure consistent, fair performance inputs. \\
\hline
3. Composite Maintenance & Establish fair, asset-weighted composites. \\
\hline
4–7. Reporting Sections & Define TWR/MWR requirements for composites and pooled funds. \\
\hline
8. Advertising Guidelines & Govern compliance claims in advertisements. \\
\hline
\end{tabular}
\end{center}

\subsubsection*{3. LOS 92.c — Composites}

\begin{center}
\begin{tabular}{|l|p{10cm}|}
\hline
\textbf{Aspect} & \textbf{Explanation} \\
\hline
Definition & Group of discretionary portfolios with similar objectives. \\
\hline
Requirement & Must include all fee-paying discretionary portfolios. \\
\hline
Purpose & Demonstrate performance by strategy or style. \\
\hline
Calculation & Asset-weighted average of portfolio returns. \\
\hline
\end{tabular}
\end{center}

\subsubsection*{4. LOS 92.d — Fundamentals of Compliance}

\begin{center}
\begin{tabular}{|l|p{10cm}|}
\hline
\textbf{Concept} & \textbf{Explanation} \\
\hline
Definition of Firm & Business entity held out to clients as a single firm. \\
\hline
Discretion & Determines which portfolios belong in composites. \\
\hline
Inclusion Rule & All discretionary, fee-paying portfolios must be included. \\
\hline
\end{tabular}
\end{center}

\subsubsection*{5. LOS 92.e — Independent Verification}

\begin{center}
\begin{tabular}{|l|p{10cm}|}
\hline
\textbf{Aspect} & \textbf{Explanation} \\
\hline
Purpose & Strengthen credibility of firm’s compliance claim. \\
\hline
Performed By & Independent third party (not internal). \\
\hline
Scope & Entire firm, not a single composite. \\
\hline
Disclosure & “Verified for the period [dates] by [verifier].” \\
\hline
\end{tabular}
\end{center}

---

\subsection*{Key Takeaways}

\begin{itemize}
  \item \textbf{GIPS} standardize global performance reporting, enabling comparability and transparency.  
  \item \textbf{Compliance} is voluntary but must be full-firm and comprehensive.  
  \item \textbf{Composites} ensure that all portfolios within a similar strategy are fairly represented.  
  \item \textbf{Firm Definition} and \textbf{Discretion} are central to GIPS integrity.  
  \item \textbf{Independent Verification} adds credibility but is not mandatory.  
  \item Overall, GIPS enhances investor confidence and aligns global reporting practices with ethical transparency.
\end{itemize}

\section*{MODULE 93.1: ETHICS APPLICATION}

\textbf{Purpose:}  
This module applies the \textbf{CFA Institute Code of Ethics and Standards of Professional Conduct (Standards I–VII)} to real-world scenarios.  
It helps identify ethical and unethical conduct, evaluate firm policies, and interpret how each case aligns or violates CFA Standards.

---

\subsection*{LOS 93.a–93.b: Evaluate and Explain Conduct Relative to the CFA Code and Standards}

Members and Candidates must understand not only what violates each Standard but also how to evaluate conduct, firm policies, and ethical decision-making frameworks.

---

\section*{STANDARD I — PROFESSIONALISM}

\subsection*{Standard I(A): Knowledge of the Law}

\textbf{Principle:}  
Members must understand and comply with all applicable laws, rules, and regulations (including CFA Standards).  
When conflicts exist, the strictest standard must be followed.

\subsubsection*{Case Applications}

\begin{itemize}
  \item \textbf{Case 1: Overcharging Clients}  
  - Firm seeks reimbursement for expenses already reimbursed.  
  - Member reports issue; firm corrects only partially.  
  - \textbf{Violation:} Member must disassociate from any continuing client overcharging.  
  - \textit{Ethical lesson:} Reporting is not enough — continued association with unethical practice breaches the Standard.

  \item \textbf{Case 2: Money-Laundering Suspicion Ignored}  
  - Member fails to investigate suspicious transactions for a long-standing client.  
  - \textbf{Violation:} Lack of diligence in preventing legal breaches.  
  - \textit{Key concept:} Loyalty to client never overrides law.

  \item \textbf{Case 3: Forged Customer Signatures}  
  - Member falsifies documents for expediency.  
  - \textbf{Violation:} Direct breach of law and integrity.
\end{itemize}

\subsubsection*{Summary Table: Knowledge of the Law}

\begin{center}
\begin{tabular}{|l|p{10cm}|}
\hline
\textbf{Case} & \textbf{Ethical Interpretation} \\
\hline
1. Overcharging Clients & Must fully disassociate; partial correction is insufficient. \\
\hline
2. Ignoring Suspicious Transactions & Duty to investigate and comply with AML laws. \\
\hline
3. Forging Signatures & Fraudulent conduct; direct legal violation. \\
\hline
\end{tabular}
\end{center}

---

\subsection*{Standard I(B): Independence and Objectivity}

\textbf{Principle:}  
Members must maintain objectivity and independence in analysis and decision-making, avoiding influence from gifts, payments, or political contributions.

\textbf{Case: Political Contributions}  
- Member donates to a politician hoping to gain investment contracts.  
- \textbf{Violation:} Attempts to secure advantage impair independence and objectivity.

---

\subsection*{Standard I(C): Misrepresentation}

\textbf{Principle:}  
Members must not make any false, misleading, or deceptive statements relating to investment analysis, recommendations, actions, or qualifications.

\subsubsection*{Case Applications}

\begin{itemize}
  \item \textbf{Case 1: Guaranteed Returns} — Violation  
  - Member guarantees future fund performance; misleading to clients.

  \item \textbf{Case 2: Omission of Personnel Change} — Violation  
  - Proposal lists key personnel; one leaves before client decision.  
  - Failure to update = misrepresentation by omission.

  \item \textbf{Case 3: False Corporate Disclosure} — Violation  
  - CEO tweets false information (“funding secured”) to manipulate price.  
  - Misrepresentation of material fact to investors.
\end{itemize}

---

\subsection*{Standard I(D): Misconduct}

\textbf{Principle:}  
Members must not engage in dishonesty, fraud, or deceit, nor commit acts that reflect adversely on their professional integrity.

\subsubsection*{Case Applications}

\begin{itemize}
  \item \textbf{Case 1: Civil Disobedience Arrest} — Not a Violation  
  - Conduct unrelated to professional integrity; motivated by personal beliefs.

  \item \textbf{Case 2: Error Correction Abuse} — Violation  
  - Member covers client losses using personal funds to hide poor performance.  
  - Misleading clients — unethical and dishonest.
\end{itemize}

---

\section*{STANDARD II — INTEGRITY OF CAPITAL MARKETS}

\subsection*{Standard II(A): Material Nonpublic Information}

\textbf{Principle:}  
Members must not act or cause others to act on material nonpublic information.

\subsubsection*{Case Applications}

\begin{itemize}
  \item \textbf{Case 1: Overheard Takeover Tip} — Violation  
  - Trading on overheard information about an acquisition = insider trading.

  \item \textbf{Case 2: Selective Disclosure by Company} — Violation  
  - Information shared with select analysts = nonpublic; acting on it breaches Standard.
\end{itemize}

---

\subsection*{Standard II(B): Market Manipulation}

\textbf{Principle:}  
Members must not engage in practices that distort prices or artificially inflate volume with intent to mislead.

\textbf{Case: Fake Shareholders}  
- Member falsifies shareholder list to meet listing requirements.  
- \textbf{Violation:} Misleads market about liquidity and violates market integrity.

---

\section*{STANDARD III — DUTIES TO CLIENTS}

\subsection*{Standard III(A): Loyalty, Prudence, and Care}

\textbf{Principle:}  
Members must act for the benefit of clients, placing client interests above their own or employer’s.

\subsubsection*{Case Applications}

\begin{itemize}
  \item \textbf{Case 1: Limiting Fiduciary Duty in Client Agreement} — Violation  
  - Cannot “opt out” of acting in client’s best interest.

  \item \textbf{Case 2: Self-Directed Margin Account} — Not a Violation  
  - Relationship defined clearly; member acts within negotiated limits.

  \item \textbf{Case 3: Expense Allocation Misuse} — Violation if personal benefit  
  - Allocating personal or unrelated client expenses breaches loyalty.
\end{itemize}

---

\subsection*{Standard III(B): Fair Dealing}

\textbf{Principle:}  
Members must deal fairly and objectively with all clients regarding investment recommendations and actions.

\textbf{Case: Paid Update Service}  
- Member offers additional research updates for a fee; not a violation if all clients are informed and have equal access to the option.  
- \textbf{Violation occurs only if} preferential info is given selectively.

---

\subsection*{Standard III(C): Suitability}

\textbf{Principle:}  
Recommendations must match client risk tolerance, objectives, and constraints.

\subsubsection*{Case Applications}

\begin{itemize}
  \item \textbf{Case 1: Excessive Risk for Tax Advantage} — Violation  
  - Tax benefit does not justify risk beyond client’s tolerance.

  \item \textbf{Case 2: Client-Requested Change Without Investigation} — Violation  
  - Failing to reassess suitability = breach of due diligence.
\end{itemize}

---

\subsection*{Standard III(D): Performance Presentation}

\textbf{Principle:}  
Performance must be fair, accurate, and complete.

\textbf{Case: Pre-Fund Composite Data}  
- Member uses pre-fund portfolio data to imply longer track record.  
- \textbf{Violation:} Misleading representation of fund history.

---

\subsection*{Standard III(E): Preservation of Confidentiality}

\textbf{Principle:}  
Client information must remain confidential unless:  
(1) it involves illegal activity, (2) law requires disclosure, or (3) client consents.

\textbf{Case: Client Data Breach}  
- Member downloads client files to personal device; data hacked.  
- Both member and compliance head violated Standard for inadequate data protection.

---

\section*{STANDARD IV — DUTIES TO EMPLOYERS}

\subsection*{Standard IV(A): Loyalty}

\textbf{Principle:}  
Members must act for the benefit of their employer and not harm it.

\subsubsection*{Case Applications}

\begin{itemize}
  \item \textbf{Case 1: Disparaging Employer While Employed} — Violation.  
  \item \textbf{Case 2: Whistleblowing to Regulators for Client Protection} — Not a Violation.  
  \item \textbf{Case 3: Taking Client Information Upon Leaving Firm} — Violation.  
\end{itemize}

---

\subsection*{Standard IV(B): Additional Compensation Arrangements}

\textbf{Principle:}  
Members must not accept gifts, benefits, or compensation that may create a conflict with employer interests unless written consent is obtained.

\textbf{Case: Bonus from Covered Company}  
- Member offered extra payment by company for coverage selection.  
- Must obtain employer’s written approval before acceptance.

---

\subsection*{Standard IV(C): Responsibilities of Supervisors}

\textbf{Principle:}  
Supervisors must ensure that subordinates comply with laws and CFA Standards.

\subsubsection*{Case Applications}

\begin{itemize}
  \item \textbf{Case 1: Lax Supervision and No Procedures} — Violation.  
  \item \textbf{Case 2: Accepting Role Without Authority or Expertise} — Violation.  
  - Should have declined supervisory responsibility until adequate authority and procedures were established.
\end{itemize}

---

\section*{STANDARD V — INVESTMENT ANALYSIS, RECOMMENDATIONS, AND ACTIONS}

\subsection*{Standard V(A): Diligence and Reasonable Basis}

\textbf{Principle:}  
Recommendations must be based on diligent, independent research with a reasonable basis.

\subsubsection*{Case Applications}

\begin{itemize}
  \item \textbf{Case 1: Inadequate Analysis and Plagiarized Report} — Both members violate the Standard.  
  - Second member cannot rely blindly on unverified work of others.
\end{itemize}

---

\subsection*{Standard V(B): Communication with Clients and Prospective Clients}

\textbf{Principle:}  
Members must disclose changes in investment processes, risks, and methodologies to clients promptly.

\subsubsection*{Case Applications}

\begin{itemize}
  \item \textbf{Case 1: Change in Fee Calculation} — Violation.  
  - Must notify clients prior to implementing fee changes.  
  \item \textbf{Case 2: Undisclosed Rating Methodology Change} — Violation.  
  - Must disclose model/method changes before publication.
\end{itemize}

---

\subsection*{Standard V(C): Record Retention}

\textbf{Principle:}  
Maintain appropriate records to support investment actions and client communications.

\textbf{Case: Outdated Client Profiles} — Violation.  
- Failing to update records violates due diligence even if portfolios are adjusted.

---

\section*{STANDARD VI — CONFLICTS OF INTEREST}

\subsection*{Standard VI(A): Disclosure of Conflicts}

\textbf{Principle:}  
All actual and potential conflicts must be disclosed to clients and employers.

\textbf{Case: Third-Party Subadvisor Payments} — Violation.  
- Undisclosed compensation creates hidden bias.

---

\subsection*{Standard VI(B): Priority of Transactions}

\textbf{Principle:}  
Client trades take precedence over personal or related-party transactions.

\subsubsection*{Case Applications}

\begin{itemize}
  \item \textbf{Case 1: Front Running Own Accounts} — Violation.  
  \item \textbf{Case 2: Sharing Client Order Info with Friends} — Violation.  
  \item \textbf{Case 3: Biased Trade Allocation} — Violation (and breach of fair dealing).  
\end{itemize}

---

\subsection*{Standard VI(C): Referral Fees}

\textbf{Principle:}  
Members must disclose to clients and employers any compensation received for referrals.

\textbf{Case: Undisclosed Gifts to Referrers} — Violation.  
- Must disclose referral-based rewards to all clients and prospects.

---

\section*{STANDARD VII — RESPONSIBILITIES AS A CFA INSTITUTE MEMBER OR CANDIDATE}

\subsection*{Standard VII(A): Conduct as Participants in CFA Programs}

\textbf{Principle:}  
Members must not compromise the integrity or confidentiality of the CFA Program.

\textbf{Case: Post-Exam Party Discussion}  
- Discussing exam difficulty = allowed.  
- Discussing exam topics/questions = violation.

---

\subsection*{Standard VII(B): Reference to CFA Institute, CFA Designation, and CFA Program}

\textbf{Principle:}  
Members must not misrepresent or exaggerate the meaning or implications of the CFA designation or membership.

\textbf{Case: Use of CFA by Inactive Member} — Violation.  
- Member who failed to pay dues may not use the CFA designation.  
- Misrepresenting inactive individuals as “CFA charterholders” violates the Standard.

---

\section*{KEY SUMMARY TABLE: MODULE 93.1}

\begin{center}
\begin{tabular}{|p{3cm}|p{4.5cm}|p{4.5cm}|p{4.5cm}|}
\hline
\textbf{Standard} & \textbf{Core Principle} & \textbf{Example of Violation} & \textbf{Example of Compliance} \\
\hline
I(A) Knowledge of Law & Follow the strictest applicable rule. & Ignoring money-laundering alerts. & Reporting and disassociating from violations. \\
\hline
I(B) Independence & Maintain objectivity. & Political donations for contracts. & Declining gifts that could bias judgment. \\
\hline
I(C) Misrepresentation & No misleading statements. & Guaranteeing returns. & Accurate, balanced disclosures. \\
\hline
I(D) Misconduct & No dishonest or deceitful behavior. & Forgery, manipulation. & Professional civil behavior outside work. \\
\hline
II(A) Material Info & No trading on inside information. & Acting on takeover rumor. & Waiting for public disclosure. \\
\hline
II(B) Market Manipulation & No distortion of prices/volume. & Fake shareholder list. & Honest, transparent trades. \\
\hline
III(A) Loyalty to Clients & Client interests first. & Hidden fees, expense abuse. & Full disclosure and best execution. \\
\hline
III(B) Fair Dealing & Treat all clients equally. & Selective recommendation timing. & Simultaneous communication. \\
\hline
III(C) Suitability & Match investment to client profile. & Ignoring risk tolerance. & IPS-driven decisions. \\
\hline
III(D) Performance & Fair, accurate reporting. & Fake track record. & GIPS-compliant presentation. \\
\hline
III(E) Confidentiality & Protect client data. & Leaking or mishandling data. & Secure systems, consent-based sharing. \\
\hline
IV(A) Loyalty to Employer & No harm or misappropriation. & Badmouthing firm, taking client lists. & Honest resignation, no data theft. \\
\hline
IV(B) Compensation & Written consent for external pay. & Accepting undisclosed bonuses. & Employer-approved agreements. \\
\hline
IV(C) Supervision & Prevent and detect violations. & Inadequate procedures or training. & Strong compliance framework. \\
\hline
V(A) Diligence & Adequate research basis. & Blindly copying analysis. & Independent verification. \\
\hline
V(B) Communication & Full disclosure of changes/risks. & Hiding fee or model changes. & Timely updates to clients. \\
\hline
V(C) Record Retention & Maintain supporting records. & Failing to update client files. & Up-to-date documentation. \\
\hline
VI(A) Disclosure & Reveal all conflicts. & Hidden third-party payments. & Transparent disclosures. \\
\hline
VI(B) Priority & Client trades first. & Front running. & Preclearance and blackout rules. \\
\hline
VI(C) Referral Fees & Disclose referral compensation. & Hidden client gifts or discounts. & Written disclosure to all parties. \\
\hline
VII(A) Conduct & Protect CFA Program integrity. & Discussing exam content. & Respect confidentiality. \\
\hline
VII(B) Reference & Proper use of CFA title. & Using CFA when inactive. & “John Smith, CFA” after dues paid. \\
\hline
\end{tabular}
\end{center}

---

\section*{KEY TAKEAWAYS}

\begin{itemize}
  \item Ethical practice means consistent application of integrity, transparency, diligence, and fairness across all professional interactions.  
  \item The \textbf{intent} behind actions matters — even technically legal acts can violate ethical standards if they mislead or harm clients.  
  \item Members must prioritize:  
  (1) Law and regulation,  
  (2) CFA Standards,  
  (3) Client and market integrity,  
  (4) Employer loyalty,  
  (5) Ongoing professional conduct.
  \item The Code and Standards are not only a rulebook but a \textbf{framework for trust} in global investment practice.
\end{itemize}


\end{document}
