\documentclass[12pt]{article}
\usepackage{amsmath}
\usepackage{geometry}
\usepackage{graphicx} % for including images and figures
\usepackage{booktabs}
\usepackage{caption}
\usepackage{titlesec}
\usepackage{float}
\usepackage{makecell}
\usepackage{tabularx}
\usepackage{enumitem}
\usepackage[utf8]{inputenc}
\usepackage{textcomp}
\usepackage{adjustbox}  % put in preamble


\geometry{margin=1in}

\title{Equity Investment}
\author{}
\date{}

\begin{document}
\maketitle

\section*{Module 41.1: Markets, Assets, and Intermediaries}

\subsection*{LOS 41.a: Main Functions of the Financial System}

\begin{itemize}
    \item \textbf{1. Facilitate saving, borrowing, raising capital, risk management, and trading.}
    \begin{itemize}
        \item Allows entities (individuals, firms, governments) to:
        \begin{itemize}
            \item Save for future consumption (e.g., retirement accounts, deposits).
            \item Borrow for consumption or investment (e.g., mortgages, bonds).
            \item Issue equity to raise capital (e.g., IPOs, private placements).
            \item Manage risks via hedging instruments (e.g., forwards, options).
            \item Exchange assets or currencies.
        \end{itemize}
    \end{itemize}

    \item \textbf{2. Determine the equilibrium return (interest rate).}
    \begin{itemize}
        \item Balances total \textit{savings supply} and total \textit{borrowing demand}.
        \item Low interest rate $\Rightarrow$ encourages borrowing, discourages saving.
        \item High interest rate $\Rightarrow$ encourages saving, discourages borrowing.
        \item Different rates exist for varying levels of \textit{risk, liquidity, and maturity}.
    \end{itemize}

    \item \textbf{3. Allocate capital to its most efficient uses.}
    \begin{itemize}
        \item Investors allocate funds to projects with the best \textit{risk–return trade-off}.
        \item Efficient allocation requires:
        \begin{itemize}
            \item Liquid markets,
            \item Low transaction costs,
            \item Transparent information,
            \item Strong regulation and contract enforcement.
        \end{itemize}
    \end{itemize}
\end{itemize}

\bigskip
\begin{table}[H!]
\centering
\caption*{Exhibit 1: Functions of the Financial System}
\begin{tabular}{|l|p{4cm}|p{4cm}|}
\hline
\textbf{Function} & \textbf{Mechanism} & \textbf{Example} \\
\hline
Saving & Provide vehicles for deferred consumption & Pension funds, CDs, bonds \\
\hline
Borrowing & Transfer funds to those needing capital & Bank loans, corporate bonds \\
\hline
Issuing equity & Raise funds via ownership dilution & IPO by Apple Inc. \\
\hline
Risk management & Hedge risks via derivatives & Forward to hedge FX exposure \\
\hline
Exchanging assets & Trade goods/currencies efficiently & FX market (EUR/USD) \\
\hline
Utilizing information & Profit from mispricing via info & Value investing (undervalued stocks) \\
\hline
\end{tabular}
\end{table}

\bigskip

\subsection*{LOS 41.b: Classifications of Assets and Markets}

\begin{itemize}
    \item \textbf{Asset Types:}
    \begin{itemize}
        \item \textbf{Financial assets:} stocks, bonds, derivatives, currencies.
        \item \textbf{Real assets:} real estate, equipment, commodities.
    \end{itemize}

    \item \textbf{Security Classifications:}
    \begin{itemize}
        \item \textbf{Debt:} fixed claim (interest + principal repayment).
        \item \textbf{Equity:} ownership interest with residual claim.
        \item \textbf{Public vs. Private:} 
        \begin{itemize}
            \item Public — traded on exchanges, regulated.
            \item Private — not publicly traded, illiquid.
        \end{itemize}
        \item \textbf{Derivative contracts:} value derived from underlying (e.g., oil futures, stock options).
    \end{itemize}

    \item \textbf{Market Classifications:}
    \begin{itemize}
        \item \textbf{Spot Market:} immediate delivery.
        \item \textbf{Forward/Futures Market:} future delivery.
        \item \textbf{Primary Market:} new issue sale (IPO).
        \item \textbf{Secondary Market:} resale among investors.
        \item \textbf{Money Market:} short-term debt ($\leq$ 1 year).
        \item \textbf{Capital Market:} long-term debt and equity.
        \item \textbf{Alternative Market:} hedge funds, commodities, real estate, art, etc.
    \end{itemize}
\end{itemize}

\bigskip
\begin{table}[h!]
\centering
\caption*{Exhibit 2: Market and Asset Classification}
\begin{tabular}{|l|l|l|}
\hline
\textbf{Market Type} & \textbf{Instruments} & \textbf{Example} \\
\hline
Money Market & Short-term debt ($\leq$ 1 year) & T-bills, commercial paper \\
\hline
Capital Market & Long-term debt and equity & Bonds, stocks \\
\hline
Primary Market & New issue & IPO of a company \\
\hline
Secondary Market & Trading existing issues & NYSE, NASDAQ \\
\hline
Spot Market & Immediate exchange & FX spot trade \\
\hline
Forward/Futures & Future settlement & Crude oil futures \\
\hline
Alternative & Non-traditional assets & Hedge funds, real estate \\
\hline
\end{tabular}
\end{table}

\bigskip

\subsection*{LOS 41.c: Types of Assets and Their Characteristics}

\paragraph{1. Securities}
\begin{itemize}
    \item \textbf{Fixed-Income (Debt):} promise to repay borrowed funds.
    \begin{itemize}
        \item \textit{Short-term:} $\leq$ 1 year (T-bills, commercial paper).
        \item \textit{Intermediate:} 1–10 years (notes).
        \item \textit{Long-term:} $>$ 10 years (bonds).
        \item \textit{Convertible debt:} can be converted into equity shares.
    \end{itemize}
    \item \textbf{Equity:} ownership interest in a firm.
    \begin{itemize}
        \item \textit{Common stock:} residual claim, variable dividends.
        \item \textit{Preferred stock:} fixed dividend, priority over common stock.
        \item \textit{Warrants:} right to purchase equity at set price before expiry.
    \end{itemize}
    \item \textbf{Pooled Investment Vehicles:}
    \begin{itemize}
        \item \textit{Mutual funds:} open- or closed-end funds.
        \item \textit{ETFs / ETNs:} exchange-traded; track index or portfolio.
        \item \textit{ABS (Asset-Backed Securities):} claims on asset pool (e.g., mortgages).
        \item \textit{Hedge funds:} limited partnerships using complex strategies and leverage.
    \end{itemize}
\end{itemize}

\paragraph{2. Currencies}
\begin{itemize}
    \item Issued by central banks.
    \item \textbf{Reserve currencies:} USD, EUR, GBP, JPY, CHF.
    \item Traded in \textbf{spot markets} for immediate delivery.
\end{itemize}

\paragraph{3. Contracts}
\begin{itemize}
    \item \textbf{Forwards:} private agreement to trade in the future at a fixed price.
    \item \textbf{Futures:} standardized forward traded on exchange.
    \item \textbf{Swaps:} exchange of cash flows (interest, currency, equity returns).
    \item \textbf{Options:} right (not obligation) to buy (call) or sell (put).
    \item \textbf{Insurance:} pays cash upon adverse event.
    \item \textbf{Credit Default Swaps (CDS):} pay if issuer defaults.
\end{itemize}

\paragraph{4. Commodities}
\begin{itemize}
    \item Include metals, energy, agricultural products.
    \item Trade via spot, futures, and forwards.
\end{itemize}

\paragraph{5. Real Assets}
\begin{itemize}
    \item Real estate, machinery, infrastructure.
    \item \textbf{Direct ownership:} illiquid, high management cost.
    \item \textbf{Indirect ownership:} via REITs, MLPs, listed firms.
\end{itemize}

\bigskip
\begin{table}[h!]
\centering
\caption*{Exhibit 3: Summary of Major Asset Classes}
\begin{tabular}{|l|p{5.5cm}|p{5.5cm}|}
\hline
\textbf{Category} & \textbf{Example Instruments} & \textbf{Key Characteristics} \\
\hline
Fixed-Income & Bonds, notes, commercial paper & Pays fixed interest; defined maturity \\
\hline
Equity & Common, preferred shares, warrants & Ownership interest; dividends optional \\
\hline
Derivatives & Forwards, futures, options, swaps & Value derived from underlying asset \\
\hline
Currencies & USD, EUR, JPY, GBP & Medium of exchange; traded spot/futures \\
\hline
Commodities & Gold, oil, wheat & Tangible goods; used for hedging/speculation \\
\hline
Real Assets & Real estate, equipment, infrastructure & Physical; provides income/diversification \\
\hline
\end{tabular}
\end{table}

\bigskip

\subsection*{LOS 41.d: Financial Intermediaries and Their Services}

\paragraph{1. Brokers, Dealers, and Exchanges}
\begin{itemize}
    \item \textbf{Brokers:} match buyers and sellers (earn commission).
    \item \textbf{Block brokers:} handle large trades discreetly to avoid price impact.
    \item \textbf{Investment banks:} underwrite securities, advise on M\&A.
    \item \textbf{Exchanges:} provide trading venues, regulate members.
    \item \textbf{ATS / ECN / MTF:} alternative electronic trading systems (no regulation role).
    \item \textbf{Dark pools:} ATSs that conceal orders.
    \item \textbf{Dealers:} trade from own inventory, earn bid–ask spread.
    \item \textbf{Primary dealers:} transact with central banks in government securities.
\end{itemize}

\paragraph{2. Securitizers}
\begin{itemize}
    \item Pool financial assets (e.g., mortgages) $\Rightarrow$ issue securities backed by pool.
    \item May use SPVs/SPEs to isolate assets from parent firm.
    \item Create tranches (senior vs. junior) with different risk-return levels.
    \item Benefits:
    \begin{itemize}
        \item Increased liquidity,
        \item Lower funding cost,
        \item Diversification.
    \end{itemize}
\end{itemize}

\paragraph{3. Depository Institutions}
\begin{itemize}
    \item Examples: banks, savings \& loans, credit unions.
    \item Accept deposits $\Rightarrow$ make loans.
    \item Earn spread between deposit and lending rates.
    \item Provide diversification, transaction services, credit analysis.
    \item Other lenders: payday lenders, factoring companies.
\end{itemize}

\paragraph{4. Insurance Companies}
\begin{itemize}
    \item Collect premiums to provide risk coverage.
    \item Diversify across policyholders $\Rightarrow$ predictable losses.
    \item Manage:
    \begin{itemize}
        \item \textbf{Moral hazard:} insured takes more risk.
        \item \textbf{Adverse selection:} high-risk buyers dominate.
        \item \textbf{Fraud:} deliberate loss to claim benefit.
    \end{itemize}
\end{itemize}

\paragraph{5. Arbitrageurs}
\begin{itemize}
    \item Profit from price discrepancies across markets.
    \item Provide liquidity, ensure price efficiency.
    \item \textit{Pure arbitrage} is riskless; \textit{risk arbitrage} involves correlated instruments.
    \item \textbf{Replication:} creating equivalent positions using different assets.
\end{itemize}

\paragraph{6. Clearinghouses and Custodians}
\begin{itemize}
    \item \textbf{Clearinghouses:} 
    \begin{itemize}
        \item Act between buyers/sellers,
        \item Provide escrow, settlement, margin control,
        \item Reduce counterparty risk.
    \end{itemize}
    \item \textbf{Custodians:}
    \begin{itemize}
        \item Hold and safeguard client securities,
        \item Prevent fraud or loss from intermediary failure.
    \end{itemize}
\end{itemize}

\bigskip
\begin{table}[h!]
\centering
\caption*{Exhibit 4: Summary of Financial Intermediaries}
\begin{tabular}{|l|l|l|}
\hline
\textbf{Intermediary Type} & \textbf{Primary Function} & \textbf{Example} \\
\hline
Broker & Match buyers and sellers & E\&Trade, Charles Schwab \\
\hline
Dealer & Trade from own account & Goldman Sachs, JPMorgan dealer desk \\
\hline
Exchange & Trading venue, regulation & NYSE, NASDAQ \\
\hline
Securitizer & Pool assets, issue ABS & Fannie Mae, Freddie Mac \\
\hline
Depository Institution & Accept deposits, make loans & Bank of America \\
\hline
Insurance Company & Risk pooling & Allianz, AXA \\
\hline
Arbitrageur & Exploit price inefficiencies & Quant hedge funds \\
\hline
Clearinghouse & Ensure trade settlement & CME Clearing, Euroclear \\
\hline
Custodian & Safekeeping of assets & State Street, BNY Mellon \\
\hline
\end{tabular}
\end{table}

\section*{Module 41.2: Positions and Leverage}

\subsection*{LOS 41.e: Compare Positions an Investor Can Take in an Asset}

\paragraph{1. Long and Short Positions}
\begin{itemize}
    \item \textbf{Long position:}
    \begin{itemize}
        \item Investor \textbf{owns} the asset or has the right/obligation to \textbf{buy}.
        \item Profits when asset price \textbf{increases}.
        \item Examples:
        \begin{itemize}
            \item Buying a stock outright.
            \item Buying a call option (right to buy).
            \item Entering a long forward/futures position.
        \end{itemize}
    \end{itemize}

    \item \textbf{Short position:}
    \begin{itemize}
        \item Investor \textbf{borrows and sells} the asset, intending to buy it back later at a lower price.
        \item Obliged to \textbf{return (cover)} the borrowed asset.
        \item Profits when asset price \textbf{falls}.
        \item The party obliged to deliver an asset under a contract (e.g., forward seller) is also \textbf{short}.
    \end{itemize}
\end{itemize}

\paragraph{2. Hedging via Short Positions}
\begin{itemize}
    \item \textbf{Purpose:} Reduce risk exposure from existing long position.
    \item \textbf{Mechanism:} Take a short position in a correlated asset.
    \item \textbf{Example:}
    \begin{itemize}
        \item A wheat farmer with a long position in the crop takes a short futures position in wheat.
        \item If wheat prices fall, the gain on the short futures offsets the loss on the crop.
    \end{itemize}
    \item \textbf{Rule of thumb:} “Do in the futures market what you must do in the future.”
\end{itemize}

\paragraph{3. Long and Short in Options and Swaps}
\begin{itemize}
    \item \textbf{Options:}
    \begin{itemize}
        \item Long call $\Rightarrow$ benefit if underlying \textbf{price rises}.
        \item Short call $\Rightarrow$ loss if underlying price rises.
        \item Long put $\Rightarrow$ benefit if underlying \textbf{price falls}.
        \item Short put $\Rightarrow$ loss if underlying price falls.
    \end{itemize}
    \item \textbf{Swaps:}
    \begin{itemize}
        \item Each party is both long and short.
        \item The side that benefits when the quoted rate increases is considered \textbf{long}.
    \end{itemize}
    \item \textbf{Currency Contracts:}
    \begin{itemize}
        \item Long one currency and short another.
        \item Example: Long EUR/USD futures $\Rightarrow$ long euro, short dollar.
    \end{itemize}
\end{itemize}

\begin{table}[h!]
\centering
\caption*{Exhibit 1: Summary of Long and Short Positions}
\begin{adjustbox}{max width=\textwidth}
\begin{tabular}{|l|l|l|l|}
\hline
\textbf{Instrument} & \textbf{Long Position} & \textbf{Short Position} & \textbf{Profit When...} \\
\hline
Stock & Buy shares & Borrow and sell shares & Long: price rises \\
\hline
Forward/Future & Obligation to buy later & Obligation to sell later & Short: price falls \\
\hline
Call Option & Right to buy & Obligation to sell if exercised & Long: price rises \\
\hline
Put Option & Right to sell & Obligation to buy if exercised & Long: price falls \\
\hline
Swap & Receive floating rate & Pay floating rate (receive fixed) & Long: quoted rate rises \\
\hline
Currency Future & Buy foreign currency & Sell foreign currency & Long: foreign currency appreciates \\
\hline
\end{tabular}
\end{adjustbox}
\end{table}

\bigskip

\subsection*{Short Sales and Positions}

\paragraph{1. Mechanics of a Short Sale}
\begin{itemize}
    \item \textbf{Steps:}
    \begin{enumerate}
        \item Borrow security and sell it on the market.
        \item Deposit sale proceeds and margin collateral with broker.
        \item Repurchase (cover) later and return security to lender.
    \end{enumerate}
    \item \textbf{Goal:} Profit from \textbf{price decline}.
    \item \textbf{Covering the short:} Repurchasing the borrowed asset to close position.
\end{itemize}

\paragraph{2. Obligations and Costs}
\begin{itemize}
    \item Must pay lender any \textbf{dividends or interest} the lender would have received.
    \item Deposit the \textbf{proceeds as collateral}.
    \item Broker may pay a \textbf{short rebate rate} on these funds:
    \begin{itemize}
        \item Typically $\approx$ overnight interest rate $-$ 0.1\%.
        \item May be negative for hard-to-borrow securities.
    \end{itemize}
    \item Broker earns the difference between market rate and short rebate.
\end{itemize}

\paragraph{3. Margin Requirements in Short Sales}
\begin{itemize}
    \item Short seller must also deposit \textbf{initial margin} (cash or riskless securities).
    \item If asset price rises $\Rightarrow$ equity decreases.
    \item If equity $\downarrow$ below maintenance margin $\Rightarrow$ \textbf{margin call}.
\end{itemize}

\begin{table}[h!]
\centering
\caption*{Exhibit 2: Summary of Short Sale Mechanics}
\begin{tabular}{|l|l|}
\hline
\textbf{Component} & \textbf{Description} \\
\hline
Borrowed asset & Security borrowed from another investor via broker \\
\hline
Sale proceeds & Held by broker as collateral \\
\hline
Payments-in-lieu & Dividends/interest paid to original owner \\
\hline
Short rebate rate & Interest rebate paid on collateral (may be negative) \\
\hline
Margin requirement & Additional collateral to protect broker \\
\hline
Covering the short & Buying back security to close short position \\
\hline
\end{tabular}
\end{table}

\bigskip

\subsection*{LOS 41.f: Leverage, Margin, and Margin Call Calculations}

\paragraph{1. Leverage and Margin Concepts}
\begin{itemize}
    \item \textbf{Leverage:} Use of borrowed funds to magnify returns.
    \item \textbf{Buying on margin:} Investor borrows from broker to purchase securities.
    \item \textbf{Margin loan:} Borrowed funds; interest paid at \textbf{call money rate}.
    \item \textbf{Equity:} Investor’s own funds in position.
    \item \textbf{Initial margin requirement:} Minimum equity at purchase.
    \item \textbf{Maintenance margin:} Minimum equity percentage to maintain position.
\end{itemize}

\paragraph{2. Key Formulas}

\[
\text{Leverage Ratio (LR)} = \frac{\text{Value of Asset}}{\text{Value of Equity}} = \frac{1}{\text{Margin Requirement}}
\]

\[
\text{Return on Equity (ROE)} = \frac{\text{Gain or Loss on Position}}{\text{Initial Equity}}
\]

\[
\text{Equity} = \text{Value of Asset} - \text{Loan Value}
\]

\[
\text{Margin \%} = \frac{\text{Equity}}{\text{Value of Asset}}
\]

\[
\text{Margin Call Price} = P_0 \times \frac{1 - \text{Initial Margin}}{1 - \text{Maintenance Margin}}
\]

\bigskip

\paragraph{3. Example: Margin Transaction}

\textbf{Given:}
\begin{itemize}
    \item Shares purchased: 1,000 at \$100 each.
    \item Initial margin = 40\%.
    \item Interest on margin loan = 4\%.
    \item Dividends = \$2/share.
    \item Commission = \$0.05/share (both purchase and sale).
    \item End of year stock price = \$110.
\end{itemize}

\textbf{Step 1: Compute leverage ratio.}
\[
\text{Leverage Ratio} = \frac{1}{0.40} = 2.5
\]

\textbf{Step 2: Compute initial equity.}
\[
\text{Total cost} = 1{,}000 \times 100 = 100{,}000
\]
\[
\text{Investor equity} = 0.40 \times 100{,}000 = 40{,}000
\]
\[
\text{Commissions} = 1{,}000 \times 0.05 = 50
\]
\[
\text{Total initial outflow} = 40{,}050
\]

\textbf{Step 3: Compute ending inflow.}
\[
\text{Stock value} = 1{,}000 \times 110 = 110{,}000
\]
\[
\text{Dividends} = 1{,}000 \times 2 = 2{,}000
\]
\[
\text{Interest} = 60{,}000 \times 0.04 = 2{,}400
\]
\[
\text{Sale commission} = 50
\]
\[
\text{Net inflow} = 110{,}000 + 2{,}000 - 60{,}000 - 2{,}400 - 50 = 49{,}550
\]

\textbf{Step 4: Compute Return on Equity.}
\[
\text{ROE} = \frac{49{,}550 - 40{,}050}{40{,}050} = 23.72\%
\]

\textbf{Interpretation:}
\begin{itemize}
    \item Asset total return = \(10\% + 2\% = 12\%\)
    \item Expected with 2.5× leverage: \(12\% \times 2.5 = 30\%\)
    \item Actual ROE lower (23.72\%) due to interest and commissions.
\end{itemize}

\begin{table}[h!]
\centering
\caption*{Exhibit 3: Margin Transaction Summary}
\begin{tabular}{|l|r|}
\hline
\textbf{Item} & \textbf{Amount (\$)} \\
\hline
Initial Investment (Equity + Commission) & 40,050 \\
\hline
Borrowed Funds & 60,000 \\
\hline
End-of-Year Stock Value & 110,000 \\
\hline
Dividends Received & 2,000 \\
\hline
Loan Interest & -2,400 \\
\hline
Commissions (buy + sell) & -100 \\
\hline
Net Inflow at Year-End & 49,550 \\
\hline
Gain on Equity & 9,500 \\
\hline
Return on Equity (ROE) & 23.72\% \\
\hline
\end{tabular}
\end{table}

\bigskip

\paragraph{4. Example: Margin Call Price}

\textbf{Given:}
\begin{itemize}
    \item Purchase price = \$40 per share.
    \item Initial margin = 50\%.
    \item Maintenance margin = 25\%.
\end{itemize}

\textbf{Formula:}
\[
P_{\text{margin call}} = P_0 \times \frac{1 - \text{Initial Margin}}{1 - \text{Maintenance Margin}}
\]

\[
P_{\text{margin call}} = 40 \times \frac{1 - 0.50}{1 - 0.25} = 40 \times \frac{0.5}{0.75} = 26.67
\]

\textbf{Interpretation:}
\begin{itemize}
    \item Margin call is triggered if price drops \textbf{below \$26.67}.
    \item Investor must deposit cash or additional securities to restore margin.
\end{itemize}

\begin{table}[h!]
\centering
\caption*{Exhibit 4: Key Margin Formulas Summary}
\begin{tabular}{|l|l|}
\hline
\textbf{Concept} & \textbf{Formula / Description} \\
\hline
Leverage Ratio & \( = \frac{1}{\text{Initial Margin Requirement}} \) \\
\hline
Equity & \( = \text{Asset Value} - \text{Loan Value} \) \\
\hline
Margin \% & \( = \frac{\text{Equity}}{\text{Asset Value}} \) \\
\hline
Margin Call Price & \( = P_0 \times \frac{1 - M_i}{1 - M_m} \) \\
\hline
Return on Equity (ROE) & \( = \frac{\text{Gain on Investment}}{\text{Initial Equity}} \) \\
\hline
Effect of Leverage & Amplifies both gains and losses \\
\hline
\end{tabular}
\end{table}

\section*{Module 41.3: Order Execution and Validity}

\subsection*{LOS 41.g: Compare Execution, Validity, and Clearing Instructions}

\paragraph{1. Order Instructions Overview}
Every trade order contains:
\begin{itemize}
    \item \textbf{Execution Instructions} — how to trade (e.g., market or limit).
    \item \textbf{Validity Instructions} — when the order is valid for execution.
    \item \textbf{Clearing Instructions} — how settlement and delivery occur.
\end{itemize}

\paragraph{2. Execution Instructions}
\begin{itemize}
    \item \textbf{Market Order:}
    \begin{itemize}
        \item Execute immediately at best available price.
        \item \textbf{Pros:} Immediate execution.
        \item \textbf{Cons:} Uncertain execution price — possible unfavorable fills.
        \item \textbf{Example:} Urgent purchase based on private information.
    \end{itemize}
    \item \textbf{Limit Order:}
    \begin{itemize}
        \item Sets a maximum (buy) or minimum (sell) execution price.
        \item \textbf{Pros:} Control over price.
        \item \textbf{Cons:} May not execute if price not reached.
    \end{itemize}
\end{itemize}

\paragraph{3. Types of Limit Orders (Relative to Market)}
\begin{itemize}
    \item \textbf{Marketable / Aggressive:} 
    \begin{itemize}
        \item Buy limit above best ask or sell limit below best bid.
        \item Executes immediately (fully or partially).
    \end{itemize}
    \item \textbf{Inside the Market:} Between best bid and best ask — sets new quotes.
    \item \textbf{At the Market:} Buy at best bid or sell at best ask — may not fill.
    \item \textbf{Behind the Market:} Away from current market — waits for movement.
    \item \textbf{Far from Market:} Unlikely to execute soon.
\end{itemize}

\paragraph{4. Additional Execution Specifications}
\begin{itemize}
    \item \textbf{All-or-Nothing (AON):} Only executed if full size can be filled.
    \item \textbf{Minimum Fill:} Execute only if at least a specified quantity is available.
    \item \textbf{Hidden Orders:} Size hidden from market (broker-only visibility).
    \item \textbf{Iceberg Orders:} Partial display size visible; rest hidden.
    \item \textbf{Purpose:} Conceal large trades to avoid price impact and gauge liquidity.
\end{itemize}

\begin{table}[h!]
\centering
\caption*{Exhibit 1: Execution Instructions Summary}
\begin{tabular}{|l|l|l|l|}
\hline
\textbf{Order Type} & \textbf{Execution Condition} & \textbf{Advantage} & \textbf{Disadvantage} \\
\hline
Market & Immediate at best price & Fast, certain execution & Price uncertainty \\
\hline
Limit & At specific price or better & Price control & No execution guarantee \\
\hline
AON & Only full order filled & Avoids partials & May never execute \\
\hline
Hidden/Iceberg & Partially or fully hidden & Conceals strategy & May reduce liquidity \\
\hline
Aggressive Limit & Crosses spread & Immediate execution & Higher cost \\
\hline
Passive Limit & Behind market & Price improvement & Execution delay \\
\hline
\end{tabular}
\end{table}

\bigskip

\paragraph{5. Validity Instructions}
\begin{itemize}
    \item \textbf{Day Order:} Expires if unfilled by end of day.
    \item \textbf{Good-till-Canceled (GTC):} Stays active until executed or canceled.
    \item \textbf{Immediate-or-Cancel (IOC) / Fill-or-Kill (FOK):} Execute immediately or cancel.
    \item \textbf{Good-on-Close (GOC):} Executed only at market close.
    \item \textbf{Good-on-Open (GOO):} Executed only at market open.
    \item \textbf{Stop Orders:} Triggered once stop price reached.
    \begin{itemize}
        \item \textbf{Stop-Sell:} Protects long positions; triggers sell if price falls.
        \item \textbf{Stop-Buy:} Protects short positions; triggers buy if price rises.
    \end{itemize}
\end{itemize}

\begin{table}[h!]
\centering
\caption*{Exhibit 2: Validity Instruction Types}
\begin{tabular}{|l|l|l|}
\hline
\textbf{Order Type} & \textbf{When Active} & \textbf{Purpose} \\
\hline
Day Order & During current session & Common default \\
\hline
GTC & Until manually canceled & Long-term position entry \\
\hline
IOC / FOK & Immediate execution only & Rapid liquidity demand \\
\hline
GOC / MOC & At market close & Used by mutual funds \\
\hline
Stop-Sell & Below current market & Limit downside on longs \\
\hline
Stop-Buy & Above current market & Limit losses on shorts \\
\hline
\end{tabular}
\end{table}

\paragraph{Example: Stop Order Use}
\begin{itemize}
    \item Investor buys stock at \$40.
    \item Places \textbf{stop-sell order at \$36} (10\% below current).
    \item Order converts to market order if price $\leq 36$.
    \item Prevents further losses but may execute below stop price if price gaps down.
\end{itemize}

\bigskip

\paragraph{6. Clearing Instructions}
\begin{itemize}
    \item Specify how trade is settled and which entities handle delivery/payment.
    \item \textbf{Retail:} Cleared by the investor’s broker.
    \item \textbf{Institutional:} Settled by custodian or prime broker.
    \item Important to distinguish:
    \begin{itemize}
        \item \textbf{Long Sale:} Selling owned securities.
        \item \textbf{Short Sale:} Broker must locate borrowable shares.
    \end{itemize}
\end{itemize}

\bigskip

\subsection*{LOS 41.h: Compare Market and Limit Orders}

\begin{table}[h!]
\centering
\caption*{Exhibit 3: Market vs. Limit Orders Comparison}
\begin{tabular}{|l|l|l|}
\hline
\textbf{Characteristic} & \textbf{Market Order} & \textbf{Limit Order} \\
\hline
Execution Speed & Immediate & Conditional (price-based) \\
\hline
Price Certainty & Low & High \\
\hline
Execution Certainty & High & Low \\
\hline
Typical Use & Urgent information-based trades & Price-sensitive trades \\
\hline
Risk & Adverse price execution & Missed opportunity \\
\hline
Example & Buy immediately at market & Buy if price ≤ \$50 \\
\hline
\end{tabular}
\end{table}

\bigskip

\subsection*{LOS 41.i: Primary vs. Secondary Markets}

\paragraph{1. Primary Market}
\begin{itemize}
    \item \textbf{Definition:} Market for new security issuance.
    \item \textbf{Types of Issues:}
    \begin{itemize}
        \item \textbf{IPO (Initial Public Offering):} First-time public issue.
        \item \textbf{Seasoned Offering (Secondary Issue):} New shares by existing public firm.
    \end{itemize}
\end{itemize}

\paragraph{2. Offering Methods}
\begin{itemize}
    \item \textbf{Underwritten Offering:} Investment bank buys entire issue at fixed price.
    \item \textbf{Best Efforts Offering:} Bank sells issue without guarantee.
    \item \textbf{Book Building:} Collecting indications of interest from investors to set price.
    \item \textbf{Hot Issue:} Oversubscribed IPO expected to trade above offer price.
    \item \textbf{Conflict of Interest:}
    \begin{itemize}
        \item Issuer wants high price; underwriter prefers low price for successful sale.
        \item Results in \textbf{underpricing} of IPOs.
    \end{itemize}
\end{itemize}

\paragraph{3. Alternative Primary Market Mechanisms}
\begin{itemize}
    \item \textbf{Private Placement:} Sale directly to qualified investors.
    \item \textbf{Shelf Registration:} Pre-registered issuance spread over time.
    \item \textbf{Dividend Reinvestment Plan (DRIP):} Dividends reinvested into new shares.
    \item \textbf{Rights Offering:} Existing shareholders buy new shares at discount.
\end{itemize}

\paragraph{4. Secondary Market}
\begin{itemize}
    \item \textbf{Definition:} Trading of existing securities between investors.
    \item \textbf{Purpose:} Provide liquidity and price discovery.
    \item \textbf{Importance:}
    \begin{itemize}
        \item Enhances primary market efficiency.
        \item Lowers firms’ cost of capital.
    \end{itemize}
\end{itemize}

\begin{table}[h!]
\centering
\caption*{Exhibit 4: Primary vs. Secondary Markets}
\begin{tabular}{|l|l|l|}
\hline
\textbf{Feature} & \textbf{Primary Market} & \textbf{Secondary Market} \\
\hline
Participants & Issuers and investors & Investors only \\
\hline
Purpose & Raise new capital & Provide liquidity \\
\hline
Example & IPO, rights issue & NYSE trade between investors \\
\hline
Price Set By & Issuer and underwriter & Supply and demand \\
\hline
Importance & Funds to firms & Price/value info \\
\hline
\end{tabular}
\end{table}

\bigskip

\subsection*{LOS 41.j: Market Structures – Trading Mechanisms}

\paragraph{1. Types of Market Structures}
\begin{itemize}
    \item \textbf{Call Market:} Trading occurs at specific times at single clearing price.
    \item \textbf{Continuous Market:} Trading occurs continuously whenever market is open.
\end{itemize}

\paragraph{2. Three Main Market Types}
\begin{itemize}
    \item \textbf{Quote-Driven Market (Dealer Market):}
    \begin{itemize}
        \item Investors trade with dealers who quote bid–ask prices.
        \item Also known as \textit{price-driven or OTC markets}.
        \item Example: Bond markets, FX markets.
    \end{itemize}
    \item \textbf{Order-Driven Market:}
    \begin{itemize}
        \item Orders matched by trading rules (exchanges, electronic systems).
        \item \textbf{Order Matching Rules:}
        \begin{itemize}
            \item Price priority: highest bid, lowest ask first.
            \item Time priority: earliest order at same price executes first.
        \end{itemize}
        \item \textbf{Trade Pricing Rules:}
        \begin{itemize}
            \item Uniform pricing (all trades at one price).
            \item Discriminatory pricing (based on limit price).
        \end{itemize}
        \item Example: NYSE, NASDAQ order book.
    \end{itemize}
    \item \textbf{Brokered Market:}
    \begin{itemize}
        \item Brokers find counterparties for unique/illiquid assets.
        \item Example: Real estate, artwork, large block trades.
    \end{itemize}
\end{itemize}

\begin{table}[h!]
\centering
\caption*{Exhibit 5: Comparison of Market Structures}
\begin{tabular}{|l|l|l|l|}
\hline
\textbf{Type} & \textbf{Who Trades With Whom} & \textbf{Price Discovery} & \textbf{Examples} \\
\hline
Quote-Driven & Investor $\leftrightarrow$ Dealer & Dealer quotes & Bonds, FX \\
\hline
Order-Driven & Investor $\leftrightarrow$ Investor & Matching rules & Stock exchanges \\
\hline
Brokered & Investor $\leftrightarrow$ Investor (via broker) & Negotiation & Real estate, large blocks \\
\hline
\end{tabular}
\end{table}

\paragraph{3. Market Transparency}
\begin{itemize}
    \item \textbf{Pre-trade transparency:} Quotes/order info available.
    \item \textbf{Post-trade transparency:} Prices and volumes disclosed after trade.
    \item \textbf{Implications:}
    \begin{itemize}
        \item High transparency $\Rightarrow$ better price discovery, lower costs.
        \item Dealers prefer opacity (information advantage).
    \end{itemize}
\end{itemize}

\bigskip

\subsection*{LOS 41.k: Characteristics of a Well-Functioning Financial System}

\paragraph{1. Key Characteristics}
\begin{itemize}
    \item \textbf{Complete Markets:} All needs for saving, borrowing, risk management met.
    \item \textbf{Operational Efficiency:} Low transaction costs.
    \item \textbf{Informational Efficiency:} Prices reflect all available information.
    \item \textbf{Allocational Efficiency:} Capital flows to most productive uses.
\end{itemize}

\paragraph{2. Key Intermediary Roles}
\begin{itemize}
    \item Provide trading venues (exchanges, ATS).
    \item Supply liquidity.
    \item Securitize assets.
    \item Manage depositories, clearinghouses, banks, insurance, advisory services.
\end{itemize}

\paragraph{3. Benefits}
\begin{itemize}
    \item Lowers cost of capital.
    \item Improves investment opportunities.
    \item Facilitates capital formation and risk sharing.
    \item Promotes economic growth.
\end{itemize}

\begin{table}[h!]
\centering
\caption*{Exhibit 6: Types of Market Efficiency}
\begin{tabular}{|l|l|}
\hline
\textbf{Efficiency Type} & \textbf{Definition} \\
\hline
Operational Efficiency & Low transaction and information costs \\
\hline
Informational Efficiency & Prices reflect all known information \\
\hline
Allocational Efficiency & Capital directed to best uses \\
\hline
\end{tabular}
\end{table}

\bigskip

\subsection*{LOS 41.l: Objectives of Market Regulation}

\paragraph{1. Problems Without Regulation}
\begin{itemize}
    \item Fraud, theft, insider trading.
    \item Costly information.
    \item Default risk.
    \item Loss of investor confidence.
\end{itemize}

\paragraph{2. Objectives of Regulation}
\begin{itemize}
    \item Protect unsophisticated investors.
    \item Ensure competency and performance transparency (e.g., CFA Program, GIPS).
    \item Prevent insider exploitation.
    \item Require standardized financial reporting (IASB/IFRS).
    \item Impose capital adequacy for long-term obligations (banks, insurers).
\end{itemize}

\paragraph{3. Types of Regulators}
\begin{itemize}
    \item \textbf{Government Regulators:} Public authorities enforcing laws.
    \item \textbf{Self-Regulating Organizations (SROs):} Industry bodies overseeing members (e.g., exchanges, clearinghouses).
\end{itemize}

\paragraph{4. Outcomes of Effective Regulation}
\begin{itemize}
    \item Maintains market liquidity.
    \item Reduces risk of fraud/default.
    \item Encourages participation and capital formation.
    \item Supports economic growth.
\end{itemize}

\begin{table}[h!]
\centering
\caption*{Exhibit 7: Market Regulation – Summary}
\begin{tabular}{|l|l|}
\hline
\textbf{Regulatory Goal} & \textbf{Purpose} \\
\hline
Fraud/Insider Protection & Preserve market trust \\
\hline
Performance Standards & Ensure fair evaluation of managers \\
\hline
Transparency Rules & Lower info costs and improve pricing \\
\hline
Capital Requirements & Ensure solvency and stability \\
\hline
SRO Oversight & Maintain discipline and compliance \\
\hline
\end{tabular}
\end{table}


\section*{Module 42.1: Index Weighting Methods}

\subsection*{LOS 42.a: Describe a Security Market Index}

\begin{itemize}
    \item \textbf{Definition:}  
    A \textit{security market index} measures the performance of an asset class, market, or segment.
    \item \textbf{Constituent securities:} Individual assets included in the index.
    \item \textbf{Index value:} Numeric representation computed from market prices of constituents.
    \item \textbf{Index return:} Percentage change in index value over a period.
    \[
        R_t = \frac{I_t - I_{t-1}}{I_{t-1}}
    \]
    \item \textbf{Purpose:} Benchmark performance, create investment products (ETFs, futures), and represent markets.
\end{itemize}

\begin{table}[h!]
\centering
\caption*{Exhibit 1: Key Features of a Market Index}
\begin{tabular}{|l|l|}
\hline
\textbf{Feature} & \textbf{Description} \\
\hline
Asset class coverage & Equities, bonds, commodities, REITs, etc. \\
\hline
Constituent count & Broad (S\&P 500) vs. narrow (DJIA 30) \\
\hline
Calculation & Weighted average of constituent prices or values \\
\hline
Usage & Benchmark, performance tracking, portfolio construction \\
\hline
\end{tabular}
\end{table}

\bigskip

\subsection*{LOS 42.b: Calculate and Interpret Index Value, Price Return, and Total Return}

\paragraph{1. Types of Index Returns}
\begin{itemize}
    \item \textbf{Price Index:} Uses only prices of constituents.  
        \(\Rightarrow\) Measures capital gains only.
        \[
            R_P = \frac{P_t - P_{t-1}}{P_{t-1}}
        \]
    \item \textbf{Return Index (Total Return):} Includes price change + income (dividends, coupons).  
        \(\Rightarrow\) Measures total wealth change.
        \[
            R_T = \frac{(P_t + D_t) - P_{t-1}}{P_{t-1}}
        \]
\end{itemize}

\paragraph{2. Linking Returns over Multiple Periods}
\[
R_{\text{period}} = (1 + R_1)(1 + R_2)\cdots(1 + R_n) - 1
\]
\textbf{Example:}
\[
R_P = (1.005)(1.0104) - 1 = 0.0155 = 1.55\%
\]
If starting index \(I_0 = 100\):
\[
I_2 = 100 \times 1.0155 = 101.55
\]

\begin{table}[h!]
\centering
\caption*{Exhibit 2: Comparison – Price Return vs. Total Return}
\begin{tabular}{|l|l|l|}
\hline
\textbf{Aspect} & \textbf{Price Index} & \textbf{Return Index} \\
\hline
Includes dividends/interest? & No & Yes \\
\hline
Reflects capital gain only? & Yes & No (includes income) \\
\hline
Example & S\&P 500 Price Index & S\&P 500 Total Return Index \\
\hline
Use & Market performance & Investor total return tracking \\
\hline
\end{tabular}
\end{table}

\bigskip

\subsection*{LOS 42.c: Choices and Issues in Index Construction and Management}

\paragraph{1. Key Decisions by Index Providers}
\begin{itemize}
    \item Define the \textbf{target market}: broad (e.g., global equities) or narrow (e.g., U.S. small-cap value).
    \item Determine \textbf{constituent selection}: objective rule-based or discretionary.
    \item Choose \textbf{weighting scheme}: price, equal, market cap, float-adjusted, or fundamental.
    \item Set \textbf{rebalancing frequency}: to realign weights to methodology.
    \item Set \textbf{re-examination schedule}: to revise constituents and structure.
\end{itemize}

\paragraph{2. Selection Criteria}
\begin{itemize}
    \item \textbf{Objective:} Size, liquidity, sector, style, or geographic region.
    \item \textbf{Subjective:} Committee review (e.g., S\&P Index Committee).
\end{itemize}

\bigskip

\subsection*{LOS 42.d: Compare Weighting Methods Used in Index Construction}

\paragraph{1. Price-Weighted Index (PWI)}
\begin{itemize}
    \item \textbf{Formula:}
    \[
        I = \frac{\sum P_i}{d}
    \]
    where \(d\) = divisor (adjusted for splits and changes).  
    \item \textbf{Examples:} DJIA (30 U.S. stocks), Nikkei 225.
    \item \textbf{Characteristics:}
    \begin{itemize}
        \item Simple to compute.
        \item High-priced stocks dominate weight.
        \item Sensitive to splits/dividends.
    \end{itemize}
    \item \textbf{Matching portfolio:} Equal number of shares of each stock.
\end{itemize}

\paragraph{Example 1: Price-Weighted Index Return}
\[
I_{Dec31} = \frac{10 + 20 + 60}{3} = 30
\]
\[
I_{Jan31} = \frac{20 + 15 + 40}{3} = 25
\]
\[
R = \frac{25 - 30}{30} = -16.67\%
\]

\paragraph{Example 2: Adjusting for a Stock Split}
\[
I_{Day1} = \frac{10 + 20 + 90}{3} = 40
\]
Stock C splits 2-for-1 $\Rightarrow$ new price = 45.
\[
\frac{10 + 20 + 45}{d} = 40 \Rightarrow d = 1.875
\]

\begin{table}[h!]
\centering
\caption*{Exhibit 3: Price-Weighted Index Summary}
\begin{tabular}{|l|l|}
\hline
\textbf{Advantage} & Simple computation \\
\hline
\textbf{Disadvantages} & Overweights high-price stocks; distorted by splits \\
\hline
\textbf{Replication} & Equal number of shares per stock \\
\hline
\end{tabular}
\end{table}

\bigskip

\paragraph{2. Equal-Weighted Index (EWI)}
\begin{itemize}
    \item \textbf{Formula:}
    \[
        R_{EWI} = \frac{1}{N} \sum_{i=1}^{N} R_i
    \]
    \item \textbf{Characteristics:}
    \begin{itemize}
        \item Equal dollar weight per constituent.
        \item Requires frequent rebalancing → high transaction costs.
        \item Overweights smaller-cap firms; underweights large-caps.
    \end{itemize}
    \item \textbf{Examples:} Value Line Composite Average, FT Ordinary Share Index.
\end{itemize}

\paragraph{Example 3: Equal-Weighted Index Calculation}
If three stocks have returns of 4\%, 2\%, and −1\%:
\[
R_{EWI} = \frac{4 + 2 - 1}{3} = 1.67\%
\]
If starting index = 131:
\[
I_1 = 131 \times (1 + 0.0167) = 133.19
\]

\begin{table}[h!]
\centering
\caption*{Exhibit 4: Equal-Weighted Index Summary}
\begin{tabular}{|l|l|}
\hline
\textbf{Advantage} & Simple, unbiased by price or size \\
\hline
\textbf{Disadvantages} & Requires rebalancing, higher costs \\
\hline
\textbf{Bias} & Toward smaller-cap, high-volatility stocks \\
\hline
\end{tabular}
\end{table}

\bigskip

\paragraph{3. Market-Capitalization-Weighted Index (MCWI)}
\begin{itemize}
    \item \textbf{Formula:}
    \[
        w_i = \frac{P_i Q_i}{\sum P_i Q_i}
    \quad \text{and} \quad
        I = \frac{\sum P_i Q_i}{\sum P_i^0 Q_i^0} \times I_0
    \]
    \item \textbf{Characteristics:}
    \begin{itemize}
        \item Reflects aggregate investor wealth.
        \item Automatically adjusts for splits/dividends.
        \item No need for rebalancing unless constituents change.
    \end{itemize}
    \item \textbf{Disadvantage:} Overweights overvalued, underweights undervalued stocks.
    \item \textbf{Example:} S\&P 500.
\end{itemize}

\paragraph{Example 4: Market-Cap Index Calculation}
\[
\text{Base total MV} = \$80\,\text{million}; \quad \text{End MV} = \$95\,\text{million}
\]
\[
I = \frac{95}{80} \times 100 = 118.75
\Rightarrow R = 18.75\%
\]

\begin{table}[h!]
\centering
\caption*{Exhibit 5: Market-Cap-Weighted Index Summary}
\begin{tabular}{|l|l|}
\hline
\textbf{Advantage} & Represents total market wealth; low turnover \\
\hline
\textbf{Disadvantage} & Momentum bias; overweights expensive stocks \\
\hline
\textbf{Example} & S\&P 500, MSCI World \\
\hline
\end{tabular}
\end{table}

\bigskip

\paragraph{4. Float-Adjusted Market-Cap-Weighted Index (FAMCI)}
\begin{itemize}
    \item Adjusts \(Q_i\) to include only publicly tradable (float) shares.  
    Excludes shares held by insiders, governments, or cross-holdings.
    \item \textbf{Free Float:} Excludes shares restricted from foreign ownership.
    \item \textbf{Effect:} Reduces weight of firms with large controlling shareholders.
    \item \textbf{Used by:} MSCI indices, FTSE indices.
\end{itemize}

\paragraph{5. Fundamental-Weighted Index (FWI)}
\begin{itemize}
    \item \textbf{Weights:} Based on firm fundamentals (e.g., earnings, book value, cash flow, dividends).
    \item \textbf{Advantages:}
    \begin{itemize}
        \item Reduces price bias; avoids overweighting overvalued stocks.
        \item Implicit \textit{value tilt} — overweights high book-to-market, high earnings-yield firms.
    \end{itemize}
    \item \textbf{Disadvantages:}
    \begin{itemize}
        \item Requires accounting data updates.
        \item Higher rebalancing cost.
    \end{itemize}
\end{itemize}

\begin{table}[h!]
\centering
\caption*{Exhibit 6: Comparison of Weighting Methods}
\begin{adjustbox}{max width=\textwidth}
\begin{tabular}{|l|l|l|}
\hline
\textbf{Method} & \textbf{Basis of Weight} & \textbf{Main Implication} \\
\hline
Price-Weighted & Stock price & High-priced stocks dominate \\
\hline
Equal-Weighted & Equal dollar per stock & Small-cap bias, frequent rebalancing \\
\hline
Market-Cap-Weighted & Market value (price × shares) & Reflects market wealth; momentum bias \\
\hline
Float-Adjusted MCW & Publicly tradable market cap & Excludes non-tradable shares \\
\hline
Fundamental-Weighted & Fundamentals (earnings, book value) & Value bias; avoids price distortion \\
\hline
\end{tabular}
\end{adjustbox}
\end{table}

\bigskip

\subsection*{LOS 42.e: Calculate and Analyze Index Value and Return by Weighting Method}

\paragraph{1. Price-Weighted Index Example}
\begin{itemize}
    \item Initial prices: \(10, 20, 60\) → Index = \((10 + 20 + 60)/3 = 30\)
    \item Next month: \(20, 15, 40\) → Index = 25
    \[
    R = \frac{25 - 30}{30} = -16.67\%
    \]
\end{itemize}

\paragraph{2. Adjusting Divisor for Split}
\[
(10 + 20 + 45) / d = 40 \Rightarrow d = 1.875
\]
Maintains index continuity after Stock C split.

\paragraph{3. Price vs. Market-Cap Weight Example}
\begin{itemize}
    \item Three firms:

\begin{tabular}{|c|c|c|}
\hline
\textbf{Firm} & \textbf{Price (\$)} & \textbf{Shares} \\
\hline
A & 100 & 100{,}000 \\
B & 10 & 1{,}000{,}000 \\
C & 1 & 20{,}000{,}000 \\
\hline
\end{tabular}

\item \textbf{Base market capitalization:}
\[
(100\times100{,}000)+(10\times1{,}000{,}000)+(1\times20{,}000{,}000)=\$40\,000\,000
\]
\item \textbf{If A doubles to 200:}  
Price-Weighted Index $\uparrow$ 33 points; MCW Index $\uparrow$ less.
\item \textbf{If C doubles to 2:}  
Price-Weighted Index barely moves (0.33 pt); MCW Index moves strongly.
\end{itemize}

\begin{table}[h!]
\centering
\caption*{Exhibit 7: Price- vs. Market-Cap Weight Sensitivity}
\begin{tabular}{|l|l|l|}
\hline
\textbf{Stock Change} & \textbf{Effect on Price-Weighted Index} & \textbf{Effect on MC-Weighted Index} \\
\hline
High-price stock doubles & Large impact & Small impact if small-cap \\
\hline
Low-price stock doubles & Minimal impact & Large impact if large-cap \\
\hline
\end{tabular}
\end{table}

\paragraph{4. Equal-Weighted Index Example}
\begin{itemize}
    \item Three stocks initially equal-weighted; initial index = 131.
    \item Returns = 5\%, −2\%, 8\% → Average = (5 − 2 + 8)/3 = 3.67\%.
    \[
    I_1 = 131 \times (1.0367) = 135.8
    \]
    \item Rebalancing ensures equal dollar weight each period.
\end{itemize}

\begin{table}[h!]
\centering
\caption*{Exhibit 8: Index Return Calculation Summary}
\begin{tabular}{|l|l|l|}
\hline
\textbf{Method} & \textbf{Computation} & \textbf{Driver of Change} \\
\hline
Price-Weighted & Avg. of prices / adjusted divisor & Price of high-priced stocks \\
\hline
Equal-Weighted & Average of individual returns & Average of all stocks equally \\
\hline
Market-Cap-Weighted & Total market value / base value & Market value of large firms \\
\hline
\end{tabular}
\end{table}

\bigskip
\textbf{Summary Insight:}
\begin{itemize}
    \item \textbf{Price-weighted:} Impact $\propto$ stock price.
    \item \textbf{Equal-weighted:} Impact $\propto$ average return.
    \item \textbf{Market-cap-weighted:} Impact $\propto$ firm size.
    \item \textbf{Fundamental-weighted:} Impact $\propto$ accounting fundamentals.
\end{itemize}


\section*{Module 42.2: Uses and Types of Indexes}

\subsection*{LOS 42.f: Rebalancing and Reconstitution of an Index}

\paragraph{1. Rebalancing}
\begin{itemize}
    \item \textbf{Definition:} Process of adjusting the \textbf{weights of index constituents} to target weights after price movements.
    \item \textbf{Purpose:} Maintain index consistency with its weighting methodology.
    \item \textbf{Frequency:} Typically quarterly or semiannual.
    \item \textbf{Affects:} Especially relevant for \textbf{equal-weighted indexes}, where weights drift as prices change.
    \item \textbf{Mechanics:}
    \begin{itemize}
        \item Price-weighted and market-cap-weighted indexes adjust automatically via price changes.
        \item Equal-weighted indexes require \textbf{manual rebalancing}.
    \end{itemize}
    \item \textbf{Example:}
    \begin{itemize}
        \item If one stock doubles in price while others remain constant, it gains excess weight in the index.
        \item Rebalancing restores equal weight among constituents.
    \end{itemize}
\end{itemize}

\begin{table}[h!]
\centering
\caption*{Exhibit 1: Rebalancing Overview}
\begin{tabular}{|l|l|l|}
\hline
\textbf{Index Type} & \textbf{Rebalancing Need} & \textbf{Reason} \\
\hline
Price-weighted & Rare & Prices adjust weights automatically \\
\hline
Market-cap-weighted & Rare & Market value adjusts automatically \\
\hline
Equal-weighted & Frequent & Price changes distort equal weights \\
\hline
Fundamental-weighted & Periodic & Rebased on updated fundamentals \\
\hline
\end{tabular}
\end{table}

\paragraph{2. Reconstitution}
\begin{itemize}
    \item \textbf{Definition:} Process of \textbf{adding or deleting} securities in the index.
    \item \textbf{Purpose:} Ensure index reflects the intended market segment.
    \item \textbf{Triggers:}
    \begin{itemize}
        \item Corporate actions: mergers, bankruptcies, delistings.
        \item Firms no longer meeting index criteria.
    \end{itemize}
    \item \textbf{Effect on Prices:}
    \begin{itemize}
        \item \textbf{Additions:} Price often rises (index funds buy to match benchmark).
        \item \textbf{Deletions:} Price often falls (funds sell).
    \end{itemize}
\end{itemize}

\begin{table}[h!]
\centering
\caption*{Exhibit 2: Rebalancing vs. Reconstitution}
\begin{tabular}{|l|l|l|}
\hline
\textbf{Aspect} & \textbf{Rebalancing} & \textbf{Reconstitution} \\
\hline
Definition & Adjust weights & Add/delete constituents \\
\hline
Goal & Maintain target weights & Maintain representative universe \\
\hline
Frequency & Quarterly / periodic & Annual or as needed \\
\hline
Example & Equal-weighted index reset & Replace delisted stock \\
\hline
Impact & Transaction costs & Price impacts on added/removed securities \\
\hline
\end{tabular}
\end{table}

\bigskip

\subsection*{LOS 42.g: Uses of Security Market Indexes}

\paragraph{1. Reflection of Market Sentiment}
\begin{itemize}
    \item Index return represents \textbf{aggregate investor confidence}.
    \item Example: DJIA (30 U.S. stocks) tracks large-cap sentiment but not the entire market.
\end{itemize}

\paragraph{2. Benchmark for Performance Evaluation}
\begin{itemize}
    \item Active managers are compared to consistent benchmarks.
    \item \textbf{Style match required:}
    \begin{itemize}
        \item Value managers $\rightarrow$ value index.
        \item Growth managers $\rightarrow$ growth index.
    \end{itemize}
\end{itemize}

\paragraph{3. Measure of Market Return and Risk}
\begin{itemize}
    \item Historical index returns used to estimate:
    \[
    E(R_i), \ \sigma_i, \ \rho_{ij}
    \]
    \item Used for \textbf{asset allocation} decisions among asset classes.
\end{itemize}

\paragraph{4. Measure of Beta and Risk-Adjusted Return}
\begin{itemize}
    \item In CAPM:
    \[
        E(R_i) = R_f + \beta_i [E(R_m) - R_f]
    \]
    \item Index serves as proxy for market portfolio ($R_m$).
    \item Used for estimating beta, expected return, and Jensen’s alpha.
\end{itemize}

\paragraph{5. Model Portfolio for Passive Investing}
\begin{itemize}
    \item \textbf{Index funds:} Replicate benchmark index performance.
    \item \textbf{Forms:} Mutual funds, ETFs, and separately managed accounts.
    \item \textbf{Advantages:} Low cost, transparency, and diversification.
\end{itemize}

\begin{table}[h!]
\centering
\caption*{Exhibit 3: Summary – Uses of Market Indexes}
\begin{tabular}{|l|l|}
\hline
\textbf{Use} & \textbf{Description / Example} \\
\hline
Market Sentiment & Gauge investor confidence (e.g., DJIA, S\&P 500) \\
\hline
Benchmark & Evaluate manager skill vs. comparable index \\
\hline
Asset Allocation & Estimate expected return and risk of asset class \\
\hline
CAPM Input & Proxy for market portfolio for $\beta$ estimation \\
\hline
Passive Investment & Replication via index funds or ETFs \\
\hline
\end{tabular}
\end{table}

\bigskip

\subsection*{LOS 42.h: Types of Equity Indexes}

\paragraph{1. Broad Market Index}
\begin{itemize}
    \item Represents overall market performance.
    \item Covers large proportion (usually $>90\%$) of total market capitalization.
    \item \textbf{Example:} Wilshire 5000 Index (over 6,000 U.S. stocks).
\end{itemize}

\paragraph{2. Multi-Market Index}
\begin{itemize}
    \item Combines indexes from several countries.
    \item Used for regional or global comparisons.
    \item \textbf{Examples:}
    \begin{itemize}
        \item MSCI World Index (developed markets).
        \item MSCI Emerging Markets Index.
        \item Latin America Index.
    \end{itemize}
\end{itemize}

\paragraph{3. Multi-Market Index with Fundamental Weighting}
\begin{itemize}
    \item Country weights based on \textbf{economic fundamentals} (e.g., GDP) instead of market cap.
    \item \textbf{Purpose:} Avoid overweighting countries with previously inflated equity markets.
\end{itemize}

\paragraph{4. Sector Index}
\begin{itemize}
    \item Measures performance of industry sectors (health care, financials, technology).
    \item Used for \textbf{cyclical analysis} and sector rotation strategies.
    \item Can be national or global (e.g., MSCI Global Financials Index).
\end{itemize}

\paragraph{5. Style Index}
\begin{itemize}
    \item Captures market \textbf{capitalization segment} and \textbf{investment style} (value/growth).
    \item \textbf{Common classifications:}
    \begin{itemize}
        \item Large-cap, mid-cap, small-cap.
        \item Value (high book-to-market, high dividend yield).
        \item Growth (high P/E, low dividend yield).
    \end{itemize}
    \item \textbf{Example:} Russell 1000 Value Index, Russell 2000 Growth Index.
    \item \textbf{Note:} Higher turnover due to migration between categories.
\end{itemize}

\begin{table}[h!]
\centering
\caption*{Exhibit 4: Equity Index Classification Summary}
\begin{tabular}{|l|l|l|}
\hline
\textbf{Index Type} & \textbf{Basis} & \textbf{Example} \\
\hline
Broad Market & Total market coverage & Wilshire 5000, Russell 3000 \\
\hline
Multi-Market & Several national indexes & MSCI World, FTSE All-World \\
\hline
Fundamentally Weighted & Economic factor (GDP) & GDP-weighted Global Index \\
\hline
Sector & Industry/sector & MSCI Healthcare, S\&P Financials \\
\hline
Style & Market cap + style & Russell 2000 Growth/Value \\
\hline
\end{tabular}
\end{table}

\bigskip

\subsection*{LOS 42.i: Compare Types of Security Market Indexes}

\begin{itemize}
    \item \textbf{Equity Indexes:} Usually market-cap or float-adjusted.
    \item \textbf{Fixed-Income Indexes:} Segment by maturity, credit, issuer, or geography.
    \item \textbf{Commodity Indexes:} Based on futures prices, not spot.
    \item \textbf{Real Estate Indexes:} Based on property values or REITs.
    \item \textbf{Hedge Fund Indexes:} Equal-weighted fund performance averages.
\end{itemize}

\begin{table}[h!]
\centering
\caption*{Exhibit 5: Comparative Summary – Global Index Types}
\begin{tabular}{|l|l|l|}
\hline
\textbf{Category} & \textbf{Weighting Scheme} & \textbf{Typical Adjustment} \\
\hline
Equity & Market-cap or float-adjusted & Adjust for free float \\
\hline
Fixed-Income & Market value of bonds & Exclude illiquid issues \\
\hline
Commodity & Futures-based & Adjust for roll yield and weights \\
\hline
Real Estate & Market value / REITs & Use appraisal or transaction prices \\
\hline
Hedge Funds & Equal-weighted returns & Self-reporting biases \\
\hline
\end{tabular}
\end{table}

\bigskip

\subsection*{LOS 42.j: Types of Fixed-Income Indexes}

\paragraph{1. Characteristics}
\begin{itemize}
    \item Huge variety: government, corporate, municipal, mortgage-backed, high-yield, etc.
    \item \textbf{Index Segmentation:}
    \begin{itemize}
        \item By \textbf{sector:} Government, corporate, sovereign.
        \item By \textbf{maturity:} Short-, intermediate-, long-term.
        \item By \textbf{rating:} Investment-grade vs. high-yield.
        \item By \textbf{region:} Domestic vs. global bonds.
    \end{itemize}
\end{itemize}

\paragraph{2. Challenges in Construction}
\begin{itemize}
    \item \textbf{Large universe:} Thousands of issues, frequent maturities and redemptions.
    \item \textbf{Illiquidity:} Many bonds trade infrequently, requiring dealer quotes or estimation.
    \item \textbf{High turnover:} Bonds mature; constant rebalancing needed.
    \item \textbf{Replication difficulty:} High transaction costs and data intensity.
\end{itemize}

\paragraph{3. Examples}
\begin{itemize}
    \item Bloomberg Barclays Global Aggregate Bond Index.
    \item ICE BofA U.S. Corporate Bond Index.
    \item JP Morgan Emerging Market Bond Index (EMBI).
\end{itemize}

\begin{table}[h!]
\centering
\caption*{Exhibit 6: Fixed-Income Index Construction Issues}
\begin{tabular}{|l|l|}
\hline
\textbf{Issue} & \textbf{Explanation} \\
\hline
Vast universe & Many issuers and instruments \\
\hline
Infrequent trading & Prices estimated from dealer quotes \\
\hline
High turnover & Maturity and replacement frequent \\
\hline
Replication cost & Expensive to track due to illiquidity \\
\hline
\end{tabular}
\end{table}

\bigskip

\subsection*{LOS 42.k: Indexes Representing Alternative Investments}

\paragraph{1. Commodity Indexes}
\begin{itemize}
    \item \textbf{Based on:} Prices of futures contracts (not spot commodities).
    \item \textbf{Major examples:}
    \begin{itemize}
        \item S\&P GSCI (formerly Goldman Sachs Commodity Index).
        \item Thomson Reuters/CoreCommodity CRB Index.
    \end{itemize}
    \item \textbf{Weighting Methods:}
    \begin{itemize}
        \item Equal-weighted, production-weighted, or fixed-weighted.
        \item Index exposures differ significantly across sectors.
    \end{itemize}
    \item \textbf{Return components:}
    \[
    R_{futures} = R_{rf} + R_{price} + R_{roll}
    \]
\end{itemize}

\paragraph{2. Real Estate Indexes}
\begin{itemize}
    \item \textbf{Based on:}  
    \begin{itemize}
        \item Property appraisals (subjective).  
        \item Repeat sales (historical transactions).  
        \item REIT performance (market-based, liquid).  
    \end{itemize}
    \item \textbf{Example:} FTSE NAREIT Index (REIT-based real estate performance).
    \item \textbf{Advantage:} REIT indexes offer daily pricing and liquidity.
\end{itemize}

\paragraph{3. Hedge Fund Indexes}
\begin{itemize}
    \item \textbf{Construction:} Equal-weighted average of self-reported hedge fund returns.
    \item \textbf{Issues:}
    \begin{itemize}
        \item \textbf{Selection bias:} Only successful funds report results.
        \item \textbf{Survivorship bias:} Poor performers drop out, inflating returns.
        \item \textbf{Lack of standardization:} Funds report to different providers.
    \end{itemize}
\end{itemize}

\begin{table}[h!]
\centering
\caption*{Exhibit 7: Comparison – Alternative Investment Indexes}
\begin{tabular}{|l|l|l|}
\hline
\textbf{Asset Class} & \textbf{Index Type} & \textbf{Key Considerations} \\
\hline
Commodities & Futures-based (e.g., S\&P GSCI) & Weighting scheme, roll yield \\
\hline
Real Estate & REIT or appraisal-based & Liquidity, valuation method \\
\hline
Hedge Funds & Equal-weighted fund returns & Reporting bias, survivorship bias \\
\hline
\end{tabular}
\end{table}

\bigskip
\textbf{Summary Insight:}
\begin{itemize}
    \item Indexes represent various asset classes for benchmarking, analysis, and investment.
    \item Each index type faces unique construction and replication challenges.
    \item Alternative investment indexes improve diversification but have data biases and methodological inconsistencies.
\end{itemize}


\section*{Module 43.1: Market Efficiency}

\subsection*{LOS 43.a: Market Efficiency and Related Concepts}

\paragraph{Definition:}
\begin{itemize}
    \item A market is \textbf{informationally efficient} if prices \textbf{fully, quickly, and rationally reflect all available information.}
    \item Prices are \textbf{unbiased estimates} of intrinsic value—deviations are random.
    \item Intuitive restatement: “You can’t consistently beat the market.”
\end{itemize}

\paragraph{Implications:}
\begin{itemize}
    \item In an efficient market:
    \begin{itemize}
        \item Expected return = equilibrium risk-adjusted return.
        \item Active strategies cannot systematically outperform passive index strategies.
    \end{itemize}
    \item Passive strategy (index investing) preferred due to lower cost and minimal transaction friction.
\end{itemize}

\paragraph{Measurement of Efficiency:}
\begin{itemize}
    \item \textbf{Adjustment speed:} Time for prices to reflect new information.
    \item Highly efficient markets (e.g., FX markets): lag ≈ seconds.
    \item Inefficient markets: price adjustment lag permits abnormal profits.
\end{itemize}

\paragraph{Price Reactions:}
\begin{itemize}
    \item Only \textbf{unexpected} (new) information affects price.
    \item Expected information has no price effect.
    \item Example: If earnings +45\% vs. expected +20\% → price increases; if +45\% vs. expected +70\% → price falls.
\end{itemize}

\begin{table}[h!]
\centering
\caption*{Exhibit 1: Key Concept—Information and Price Response}
\begin{tabular}{|l|l|}
\hline
\textbf{Type of Information} & \textbf{Price Impact} \\
\hline
Expected (well-anticipated) & No effect on price \\
\hline
Unexpected positive & Price rises \\
\hline
Unexpected negative & Price falls \\
\hline
\end{tabular}
\end{table}

\bigskip

\subsection*{LOS 43.b: Market Value vs. Intrinsic Value}

\paragraph{Definitions:}
\begin{itemize}
    \item \textbf{Market value:} Current market price.
    \item \textbf{Intrinsic (fundamental) value:} Rational investor’s estimated true value, based on full knowledge of asset’s characteristics (cash flows, risk, maturity, liquidity).
\end{itemize}

\paragraph{Relationship:}
\begin{itemize}
    \item In efficient markets $\Rightarrow$ Market value ≈ Intrinsic value.
    \item In inefficient markets $\Rightarrow$ Mispricing occurs, allowing \textbf{active managers} to exploit price-value gaps.
\end{itemize}

\paragraph{Notes:}
\begin{itemize}
    \item Intrinsic value estimates vary among investors due to model differences.
    \item More complex assets → harder to estimate intrinsic value.
    \item Intrinsic values constantly change as new information arrives.
\end{itemize}

\begin{table}[h!]
\centering
\caption*{Exhibit 2: Market vs. Intrinsic Value}
\begin{tabular}{|l|l|}
\hline
\textbf{Concept} & \textbf{Description} \\
\hline
Market Value & Current observable trading price \\
\hline
Intrinsic Value & Estimated true worth based on fundamentals \\
\hline
Efficient Market & Market value ≈ Intrinsic value \\
\hline
Inefficient Market & Temporary divergence → opportunity \\
\hline
\end{tabular}
\end{table}

\bigskip

\subsection*{LOS 43.c: Factors Affecting Market Efficiency}

\paragraph{1. Number of Market Participants}
\begin{itemize}
    \item More participants (analysts, traders, investors) $\Rightarrow$ greater efficiency.
    \item Restricted foreign access or low participation $\Rightarrow$ inefficiency.
\end{itemize}

\paragraph{2. Availability of Information}
\begin{itemize}
    \item More publicly available, low-cost information $\Rightarrow$ higher efficiency.
    \item Developed markets (e.g., NYSE) more efficient than emerging markets.
    \item Regulations (e.g., SEC Reg FD) promote equal information access.
\end{itemize}

\paragraph{3. Impediments to Trading (Arbitrage)}
\begin{itemize}
    \item Arbitrage aligns prices across markets.
    \item \textbf{High transaction costs} or \textbf{restricted short-selling} reduce efficiency.
\end{itemize}

\paragraph{4. Transaction and Information Costs}
\begin{itemize}
    \item Market is efficient if, \textbf{after costs}, no abnormal risk-adjusted profits exist.
\end{itemize}

\begin{table}[h!]
\centering
\caption*{Exhibit 3: Determinants of Market Efficiency}
\begin{tabular}{|l|l|}
\hline
\textbf{Factor} & \textbf{Effect on Efficiency} \\
\hline
Number of Participants & ↑ Participants → ↑ Efficiency \\
\hline
Information Availability & ↑ Information → ↑ Efficiency \\
\hline
Trading Impediments & ↑ Impediments → ↓ Efficiency \\
\hline
Transaction Costs & ↑ Costs → ↓ Efficiency \\
\hline
Short-Selling Restrictions & ↓ Efficiency (overvaluation persists) \\
\hline
\end{tabular}
\end{table}

\bigskip

\subsection*{LOS 43.d: Forms of Market Efficiency}

\paragraph{Eugene Fama’s Three Forms of the EMH:}
\begin{enumerate}
    \item \textbf{Weak-Form Efficiency:}
    \begin{itemize}
        \item Prices reflect all past \textbf{market data} (price, volume).
        \item Technical analysis cannot consistently generate abnormal returns.
    \end{itemize}
    \item \textbf{Semi-Strong-Form Efficiency:}
    \begin{itemize}
        \item Prices reflect all \textbf{public information} (market + nonmarket).
        \item Fundamental analysis cannot consistently generate abnormal returns.
    \end{itemize}
    \item \textbf{Strong-Form Efficiency:}
    \begin{itemize}
        \item Prices reflect \textbf{all information}, public and private (insider).
        \item No investor (even insiders) can earn abnormal returns.
        \item Unrealistic due to insider trading laws.
    \end{itemize}
\end{enumerate}

\begin{table}[h!]
\centering
\caption*{Exhibit 4: Forms of the Efficient Market Hypothesis}
\begin{adjustbox}{max width=\textwidth}
\begin{tabular}{|l|l|l|l|}
\hline
\textbf{Form} & \textbf{Information Set} & \textbf{Strategy Impact} & \textbf{Efficiency Implication} \\
\hline
Weak & Past price and volume data & Technical analysis useless & Common in developed markets \\
\hline
Semi-Strong & All public information & Fundamental analysis useless & Generally supported \\
\hline
Strong & All public + private information & No one can beat market & Unrealistic in practice \\
\hline
\end{tabular}
\end{adjustbox}
\end{table}

\bigskip

\subsection*{LOS 43.e: Implications for Analysis and Portfolio Management}

\paragraph{Testing Efficiency:}
\begin{itemize}
    \item Abnormal return = Actual return – Expected return (from CAPM or multi-factor model).
    \[
    \text{Abnormal return} = R_i - [R_f + \beta_i (R_m - R_f)]
    \]
\end{itemize}

\paragraph{1. Technical Analysis}
\begin{itemize}
    \item Uses historical prices/volume.
    \item Ineffective under weak-form efficiency.
    \item Some success in emerging markets due to less efficiency.
\end{itemize}

\paragraph{2. Fundamental Analysis}
\begin{itemize}
    \item Uses financial statement and economic data.
    \item Ineffective under semi-strong efficiency (public info already priced in).
    \item Still useful:
    \begin{itemize}
        \item To improve efficiency through price discovery.
        \item For highly skilled analysts.
    \end{itemize}
\end{itemize}

\paragraph{3. Event Studies}
\begin{itemize}
    \item Tests price reaction to public events (e.g., earnings announcements).
    \item Efficient markets: No abnormal pre/post-announcement returns.
\end{itemize}

\paragraph{4. Active vs. Passive Management}
\begin{itemize}
    \item Semi-strong efficiency → Passive (index) investing preferred.
    \item Role of active manager:
    \begin{itemize}
        \item Define risk-return objectives.
        \item Optimize diversification, asset allocation, and tax strategy.
    \end{itemize}
\end{itemize}

\begin{table}[h!]
\centering
\caption*{Exhibit 5: Market Efficiency and Analytical Value}
\begin{tabular}{|l|l|l|}
\hline
\textbf{Form} & \textbf{Useful Strategy} & \textbf{Useless Strategy} \\
\hline
Weak & Fundamental / Active mgmt & Technical analysis \\
\hline
Semi-Strong & Index investing, asset allocation & Fundamental analysis \\
\hline
Strong & None (no abnormal returns possible) & All active management \\
\hline
\end{tabular}
\end{table}

\bigskip

\subsection*{LOS 43.f: Market Anomalies}

\paragraph{Definition:}
\begin{itemize}
    \item A \textbf{market anomaly} is a pattern in returns that contradicts EMH.
    \item May arise from \textbf{data mining, mispricing, or behavioral bias}.
\end{itemize}

\paragraph{Avoiding Data-Snooping Bias:}
\begin{itemize}
    \item Confirm anomalies with large, independent samples and across time.
    \item Check if relationship has valid \textbf{economic rationale}.
\end{itemize}

\subsubsection*{A. Time-Series Anomalies}
\begin{itemize}
    \item \textbf{Calendar Effects:}
    \begin{itemize}
        \item January effect: Small-cap stocks outperform early January (due to tax-loss selling and window dressing).
        \item Other effects (month-end, day-of-week, holiday) mostly disappeared after discovery.
    \end{itemize}
    \item \textbf{Overreaction / Momentum:}
    \begin{itemize}
        \item Loser stocks outperform winners in subsequent years (overreaction).
        \item Short-term momentum: high past returns continue briefly.
        \item Both contradict weak-form efficiency.
    \end{itemize}
\end{itemize}

\subsubsection*{B. Cross-Sectional Anomalies}
\begin{itemize}
    \item \textbf{Size Effect:} Small-cap stocks historically outperform large-caps (disappeared in later data).
    \item \textbf{Value Effect:} Value stocks (low P/E, low P/B, high dividend yield) outperform growth stocks.
        \begin{itemize}
            \item Violates semi-strong efficiency.
            \item May reflect higher (unmeasured) risk.
        \end{itemize}
\end{itemize}

\subsubsection*{C. Other Anomalies}
\begin{itemize}
    \item \textbf{Closed-End Fund Discounts:} Funds trade below NAV — partially explained by fees/taxes.
    \item \textbf{Earnings Announcements:} Post-announcement drift suggests delayed reaction to information.
    \item \textbf{IPOs:} Underpriced initially; underperform long-term due to investor overreaction.
    \item \textbf{Economic Fundamentals:} Returns correlated with dividend yield, volatility, interest rates—expected in efficient markets.
\end{itemize}

\paragraph{Investor Implications:}
\begin{itemize}
    \item Most anomalies disappear once discovered or are too small to profit after costs.
    \item Many are statistical artifacts or compensation for unobserved risk.
    \item \textbf{Conclusion:} Markets are largely efficient; anomaly-based strategies unreliable.
\end{itemize}

\begin{table}[h!]
\centering
\caption*{Exhibit 6: Common Market Anomalies and Possible Causes}
\begin{adjustbox}{max width=\textwidth}
\begin{tabular}{|l|l|l|}
\hline
\textbf{Anomaly} & \textbf{Observed Pattern} & \textbf{Possible Explanation} \\
\hline
January Effect & Small-cap returns spike in Jan & Tax-loss selling, window dressing \\
\hline
Overreaction & Losers outperform winners later & Investor overreaction to news \\
\hline
Momentum & Winners keep outperforming short term & Herding, delayed reaction \\
\hline
Size Effect & Small > large & Risk premium, vanished post-discovery \\
\hline
Value Effect & Value > growth & Risk premium or inefficiency \\
\hline
Earnings Drift & Post-announcement drift & Delayed info incorporation \\
\hline
Closed-End Discount & NAV > market price & Illiquidity, taxes, sentiment \\
\hline
IPO Effect & Short-term rise, long-term fall & Overreaction, mispricing \\
\hline
\end{tabular}
\end{adjustbox}
\end{table}

\bigskip

\subsection*{LOS 43.g: Behavioral Finance and Market Anomalies}

\paragraph{Definition:}
\begin{itemize}
    \item Behavioral finance studies how psychological biases affect investor decisions and market outcomes.
    \item Investors are not always rational, utility-maximizing agents.
\end{itemize}

\paragraph{Key Behavioral Biases:}
\begin{itemize}
    \item \textbf{Loss Aversion:} Investors dislike losses more than they like equivalent gains.
    \item \textbf{Overconfidence:} Investors overestimate their skill in interpreting data and identifying mispricings.
    \item \textbf{Herding:} Investors mimic others instead of relying on independent analysis.
    \item \textbf{Information Cascade:} Investors follow actions of perceived informed traders, potentially accelerating information incorporation.
\end{itemize}

\paragraph{Implications:}
\begin{itemize}
    \item Behavioral biases can cause temporary deviations from intrinsic value.
    \item However, markets can remain \textbf{efficient on average} if irrational actions offset.
    \item Semi-strong efficiency can still hold even if individuals act irrationally.
\end{itemize}

\begin{table}[h!]
\centering
\caption*{Exhibit 7: Behavioral Biases and Market Impact}
\begin{adjustbox}{max width=\textwidth}
\begin{tabular}{|l|l|l|}
\hline
\textbf{Bias} & \textbf{Description} & \textbf{Market Effect} \\
\hline
Loss Aversion & Dislike for losses stronger than for gains & Excess risk aversion, underreaction \\
\hline
Overconfidence & Overestimation of forecasting skill & Overtrading, mispricing \\
\hline
Herding & Following others' trades blindly & Asset bubbles, volatility \\
\hline
Information Cascade & Mimicking informed investors & Faster or distorted price reaction \\
\hline
\end{tabular}
\end{adjustbox}
\end{table}

\paragraph{Conclusion:}
\begin{itemize}
    \item Behavioral finance provides psychological explanations for anomalies.
    \item Even if individuals are irrational, overall market efficiency can still hold.
    \item Prices may deviate temporarily but generally revert to reflect fundamentals.
\end{itemize}


\section*{Module 44.1: Types of Equity Investments}

\subsection*{LOS 44.a: Characteristics of Types of Equity Securities}

\paragraph{1. Common Shares (Common Stock)}
\begin{itemize}
    \item Represent \textbf{ownership interest} in the firm and residual claim on assets.
    \item \textbf{Residual claim:} Paid after debtholders and preferred shareholders upon liquidation.
    \item \textbf{Control rights:}
    \begin{itemize}
        \item Vote for \textbf{board of directors}, mergers, auditor selection.
        \item Can vote in person or by \textbf{proxy} (appoint another person to vote).
    \end{itemize}
    \item \textbf{Dividend policy:} 
    \begin{itemize}
        \item Not contractually required; at discretion of firm.
        \item Dividends may vary or be omitted.
    \end{itemize}
\end{itemize}

\paragraph{2. Voting Systems}
\begin{itemize}
    \item \textbf{Statutory Voting:}
    \begin{itemize}
        \item Each share = one vote per director position.
        \item Example: 100 shares × 3 directors = 100 votes per election.
        \item Favours majority shareholders.
    \end{itemize}
    \item \textbf{Cumulative Voting:}
    \begin{itemize}
        \item Total votes = shares × number of directors.
        \item Votes can be allocated freely (e.g., all to one candidate).
        \item Enables \textbf{minority representation}.
        \item Example: 100 shares × 3 directors = 300 votes total; may allocate all 300 to one director.
        \item A 30\% shareholder can elect ≈3 of 10 directors.
    \end{itemize}
\end{itemize}

\begin{table}[h!]
\centering
\caption*{Exhibit 1: Comparison – Statutory vs. Cumulative Voting}
\begin{tabular}{|l|l|l|}
\hline
\textbf{Feature} & \textbf{Statutory Voting} & \textbf{Cumulative Voting} \\
\hline
Votes per share & 1 per director & Shares × directors \\
\hline
Allocation flexibility & Fixed per position & Flexible across candidates \\
\hline
Minority protection & Weak & Stronger \\
\hline
Favours & Majority holders & Minority holders \\
\hline
\end{tabular}
\end{table}

\bigskip

\paragraph{3. Preference (Preferred) Shares}
\begin{itemize}
    \item Hybrid security with features of both debt and equity.
    \item \textbf{Debt-like:} Fixed periodic payments (dividends), no voting rights.
    \item \textbf{Equity-like:} Perpetual maturity, dividends not a contractual obligation.
    \item Can be \textbf{callable}, \textbf{putable}, \textbf{convertible}, \textbf{participating}, or \textbf{non-participating}.
\end{itemize}

\paragraph{4. Types of Preference Shares}
\begin{itemize}
    \item \textbf{Cumulative Preferred:}
    \begin{itemize}
        \item Missed dividends accumulate and must be paid before common dividends.
    \end{itemize}
    \item \textbf{Non-Cumulative Preferred:}
    \begin{itemize}
        \item Missed dividends do not accumulate.
        \item Common dividends cannot be paid until current preferred dividend is paid.
    \end{itemize}
    \item \textbf{Participating Preferred:}
    \begin{itemize}
        \item Receive additional dividends if profits exceed a threshold.
        \item Receive more than par value in liquidation.
    \end{itemize}
    \item \textbf{Non-Participating Preferred:}
    \begin{itemize}
        \item Fixed dividend only; par value claim in liquidation.
    \end{itemize}
    \item \textbf{Convertible Preferred:}
    \begin{itemize}
        \item Can be converted into common shares at a fixed ratio.
        \item Offers fixed income + potential upside if stock price rises.
        \item Example: Convertible ratio = 1 preferred → 5 common shares.
    \end{itemize}
\end{itemize}

\paragraph{5. Example: Preferred Dividend Calculation}
\[
\text{Annual Dividend} = \text{Par Value} \times \text{Dividend Rate}
\]
Example: \$80 par, 10\% dividend rate $\Rightarrow$ \$8 annual dividend.

\begin{table}[h!]
\centering
\caption*{Exhibit 2: Comparison – Types of Preference Shares}
\begin{tabular}{|l|l|l|}
\hline
\textbf{Type} & \textbf{Dividend Treatment} & \textbf{Special Feature} \\
\hline
Cumulative & Arrears accumulate & Priority over common \\
\hline
Non-Cumulative & Missed dividends lost & Paid before common \\
\hline
Participating & Extra dividends if profits high & Share in upside \\
\hline
Non-Participating & Fixed dividend only & No participation in profits \\
\hline
Convertible & Fixed + conversion option & Equity upside potential \\
\hline
Callable & Firm can repurchase shares & Flexibility for issuer \\
\hline
Putable & Investor can sell back & Downside protection \\
\hline
\end{tabular}
\end{table}

\paragraph{6. Risk and Return Characteristics}
\begin{itemize}
    \item Preferred shares are \textbf{less risky} than common shares:
    \begin{itemize}
        \item Fixed dividend.
        \item Priority in dividends and liquidation.
    \end{itemize}
    \item Convertible preferred shares provide \textbf{upside participation}.
    \item Common shares provide \textbf{highest potential return} but also highest risk.
\end{itemize}

\paragraph{7. Application:}
\begin{itemize}
    \item Convertible and participating preferred shares are common in venture capital and private equity financing.
    \item Used to compensate investors for higher business risk.
\end{itemize}

\bigskip

\subsection*{LOS 44.b: Differences in Voting Rights and Ownership Characteristics Among Equity Classes}

\paragraph{1. Multiple Share Classes}
\begin{itemize}
    \item Firms may issue multiple classes of common stock (e.g., Class A, Class B).
    \item Differences include:
    \begin{itemize}
        \item \textbf{Voting power:} Some classes have superior voting rights.
        \item \textbf{Dividend rights:} Different payout ratios.
        \item \textbf{Liquidation priority:} One class may have senior claim.
    \end{itemize}
\end{itemize}

\paragraph{2. Example:}
\begin{itemize}
    \item Class A: 1 vote per share.
    \item Class B: 10 votes per share.
    \item Founders retain control with smaller ownership.
\end{itemize}

\begin{table}[h!]
\centering
\caption*{Exhibit 3: Dual-Class Share Structures}
\begin{tabular}{|l|l|l|}
\hline
\textbf{Class} & \textbf{Voting Rights} & \textbf{Notes} \\
\hline
Class A & 1 vote per share & Common retail class \\
\hline
Class B & 10 votes per share & Held by founders/insiders \\
\hline
Class C & No vote & Issued to raise capital without diluting control \\
\hline
\end{tabular}
\end{table}

\paragraph{3. Disclosure:}
\begin{itemize}
    \item Information on share classes and voting rights found in regulatory filings (e.g., SEC in the U.S.).
\end{itemize}

\bigskip

\subsection*{LOS 44.c: Comparison of Public and Private Equity Securities}

\paragraph{1. Public Equity Characteristics}
\begin{itemize}
    \item Traded on organized exchanges or OTC markets.
    \item \textbf{Advantages:}
    \begin{itemize}
        \item High liquidity and price transparency.
        \item Regulated disclosure requirements.
        \item Broad investor base.
    \end{itemize}
    \item \textbf{Disadvantages:}
    \begin{itemize}
        \item Public scrutiny and pressure for short-term results.
        \item Higher reporting costs and governance obligations.
    \end{itemize}
\end{itemize}

\paragraph{2. Private Equity Characteristics}
\begin{itemize}
    \item Issued through private placements to institutional or accredited investors.
    \item \textbf{Advantages:}
    \begin{itemize}
        \item Lower regulatory and reporting costs.
        \item Long-term focus without market pressure.
        \item Potentially higher returns upon exit (IPO or sale).
    \end{itemize}
    \item \textbf{Disadvantages:}
    \begin{itemize}
        \item Illiquid—no secondary trading market.
        \item Valuation negotiated, not market-determined.
        \item Limited disclosure; weaker corporate governance.
    \end{itemize}
\end{itemize}

\begin{table}[h!]
\centering
\caption*{Exhibit 4: Comparison – Public vs. Private Equity}
\begin{tabular}{|l|l|l|}
\hline
\textbf{Aspect} & \textbf{Public Equity} & \textbf{Private Equity} \\
\hline
Liquidity & High & Low (no public market) \\
\hline
Valuation & Market-determined & Negotiated \\
\hline
Disclosure & Extensive (regulatory) & Limited (private) \\
\hline
Reporting Costs & High & Low \\
\hline
Governance & Strong (public oversight) & Potentially weak \\
\hline
Time Horizon & Short- to medium-term & Long-term \\
\hline
Expected Return & Moderate & Potentially higher (with risk) \\
\hline
\end{tabular}
\end{table}

\bigskip

\paragraph{3. Main Types of Private Equity Investments}

\subsubsection*{A. Venture Capital (VC)}
\begin{itemize}
    \item Funding early-stage or growth-stage firms.
    \item \textbf{Stages:}
    \begin{itemize}
        \item \textbf{Seed/start-up:} Idea or prototype funding.
        \item \textbf{Early-stage:} Product development, initial revenue.
        \item \textbf{Mezzanine:} Expansion financing prior to IPO.
    \end{itemize}
    \item \textbf{Investors:} Angel investors, VC funds, wealthy individuals.
    \item \textbf{Features:}
    \begin{itemize}
        \item High risk, illiquid, long-term horizon (3–10 years).
        \item Returns realized via IPO or acquisition (exit event).
    \end{itemize}
\end{itemize}

\subsubsection*{B. Leveraged Buyout (LBO)}
\begin{itemize}
    \item Purchase of an entire firm using \textbf{debt financing}.
    \item \textbf{Types:}
    \begin{itemize}
        \item \textbf{Management Buyout (MBO):} Existing management acquires the firm.
        \item \textbf{Management Buy-in (MBI):} External management takes control.
    \end{itemize}
    \item \textbf{Typical Targets:}
    \begin{itemize}
        \item Firms with stable cash flows (to service debt).
        \item Firms with undervalued or sellable assets.
    \end{itemize}
\end{itemize}

\subsubsection*{C. Private Investment in Public Equity (PIPE)}
\begin{itemize}
    \item Public firm sells shares privately to institutional investors for quick capital.
    \item Usually at a \textbf{discount to market price}.
    \item \textbf{Used by:}
    \begin{itemize}
        \item Firms in financial distress.
        \item Firms seeking strategic investors or bridge financing.
    \end{itemize}
\end{itemize}

\begin{table}[h!]
\centering
\caption*{Exhibit 5: Types of Private Equity Investments}
\begin{adjustbox}{max width=\textwidth}
\begin{tabular}{|l|l|l|}
\hline
\textbf{Type} & \textbf{Purpose} & \textbf{Key Characteristics} \\
\hline
Venture Capital & Fund early growth firms & High risk, illiquid, staged financing \\
\hline
LBO / MBO & Acquire firm using leverage & Stable cash flow targets, debt-financed \\
\hline
PIPE & Inject capital into public firm privately & Discounted share issue, rapid capital access \\
\hline
\end{tabular}
\end{adjustbox}
\end{table}

\paragraph{4. Example: Venture Capital Stages and Cash Flow}
\[
\text{Timeline: Seed (0–1 yr)} \rightarrow \text{Early Stage (1–3 yrs)} \rightarrow \text{Mezzanine (3–5 yrs)} \rightarrow \text{Exit (IPO or Sale, 5–10 yrs)}
\]

\begin{itemize}
    \item \textbf{Cash Flow Profile:} Negative at start, large positive return at exit.
    \item \textbf{Exit Multiples:} 5×–10× initial investment if successful.
\end{itemize}

\bigskip

\paragraph{Summary Insight:}
\begin{itemize}
    \item Equity securities differ by claim priority, risk, return, and control.
    \item Preferred shares offer fixed income-like features; common shares offer ownership and voting rights.
    \item Private equity emphasizes long-term value creation and liquidity trade-offs.
\end{itemize}


\section*{Module 44.2: Foreign Equities and Equity Risk}

\subsection*{LOS 44.d: Methods for Investing in Non-Domestic Equity Securities}

\paragraph{1. Market Integration and Capital Flow}
\begin{itemize}
    \item When capital flows freely across borders, markets are said to be \textbf{integrated}.
    \item Global integration has increased due to technology and communication advances.
    \item Some countries still restrict foreign ownership to:
    \begin{itemize}
        \item Prevent foreign control of domestic firms.
        \item Reduce volatility of capital inflows and outflows.
    \end{itemize}
    \item Removal of capital barriers $\Rightarrow$ improved market performance and efficiency.
\end{itemize}

\paragraph{2. Firm Perspective on Foreign Listing}
\begin{itemize}
    \item \textbf{Benefits:}
    \begin{itemize}
        \item Increased visibility and liquidity of shares.
        \item Enhanced transparency due to stricter disclosure.
        \item Access to a broader investor base and lower cost of capital.
    \end{itemize}
\end{itemize}

\paragraph{3. Direct Investing}
\begin{itemize}
    \item Purchasing foreign company shares directly on foreign exchanges.
    \item \textbf{Challenges:}
    \begin{itemize}
        \item Currency risk (investment denominated in foreign currency).
        \item Market illiquidity.
        \item Less stringent reporting standards.
        \item Regulatory and procedural differences.
    \end{itemize}
\end{itemize}

\paragraph{4. Indirect Investing via Depository Receipts (DRs)}
\begin{itemize}
    \item \textbf{Depository Receipts (DRs):} Certificates representing ownership of foreign shares.
    \item Issued by a bank which holds the underlying foreign shares.
    \item Traded in local currency on foreign exchanges.
    \item Value influenced by:
    \begin{itemize}
        \item Exchange rate fluctuations.
        \item Firm fundamentals and market factors.
    \end{itemize}
\end{itemize}

\paragraph{5. Types of Depository Receipts}
\begin{itemize}
    \item \textbf{Sponsored DR:} 
    \begin{itemize}
        \item Firm involved in issuance.
        \item Greater disclosure requirements.
        \item Investors retain voting rights.
    \end{itemize}
    \item \textbf{Unsponsored DR:}
    \begin{itemize}
        \item Issued independently by bank.
        \item Voting rights retained by bank.
        \item Lower disclosure requirements.
    \end{itemize}
\end{itemize}

\begin{table}[h!]
\centering
\caption*{Exhibit 1: Comparison – Sponsored vs. Unsponsored DRs}
\begin{tabular}{|l|l|l|}
\hline
\textbf{Feature} & \textbf{Sponsored DR} & \textbf{Unsponsored DR} \\
\hline
Issuer involvement & Yes & No \\
\hline
Voting rights & Investor & Depository bank \\
\hline
Disclosure standards & High (regulatory) & Low \\
\hline
Investor protection & Stronger & Weaker \\
\hline
\end{tabular}
\end{table}

\paragraph{6. Global Depository Receipts (GDRs)}
\begin{itemize}
    \item Issued outside both the home country and the U.S.
    \item Typically listed in London or Luxembourg exchanges.
    \item Denominated in U.S. dollars; accessible to institutional investors.
    \item \textbf{Benefits:}
    \begin{itemize}
        \item Avoids national capital restrictions.
        \item Improves access to foreign capital markets.
    \end{itemize}
\end{itemize}

\paragraph{7. American Depository Receipts (ADRs)}
\begin{itemize}
    \item Denominated in U.S. dollars and traded on U.S. exchanges.
    \item The underlying security traded in the home market is the \textbf{American Depository Share (ADS)}.
    \item May be used to raise capital or acquire U.S. firms.
    \item Must comply with SEC regulations.
\end{itemize}

\paragraph{8. Types of ADRs}
\begin{table}[h!]
\centering
\caption*{Exhibit 2: Types of ADRs and Key Characteristics}
\begin{adjustbox}{max width=\textwidth}
\begin{tabular}{|l|l|l|l|}
\hline
\textbf{Type} & \textbf{Listed/Unlisted} & \textbf{Capital Raising?} & \textbf{Disclosure Level} \\
\hline
Level I & OTC (unlisted) & No & Minimal (not SEC-registered) \\
\hline
Level II & U.S. exchange & No & Full SEC reporting \\
\hline
Level III & U.S. exchange & Yes (new shares) & Full SEC registration and prospectus \\
\hline
Rule 144A / Reg S & Private placement & Yes (institutional only) & Exempt from public filing \\
\hline
\end{tabular}
\end{adjustbox}
\end{table}

\paragraph{9. Other Structures}
\begin{itemize}
    \item \textbf{Global Registered Shares (GRS):} 
    \begin{itemize}
        \item Same share traded in multiple currencies across exchanges.
        \item Equal rights for all holders globally.
    \end{itemize}
    \item \textbf{Basket of Listed Depository Receipts (BLDR):}
    \begin{itemize}
        \item Exchange-traded fund (ETF) composed of several DRs.
        \item Trades like a common stock.
    \end{itemize}
\end{itemize}

\bigskip

\subsection*{LOS 44.e: Risk and Return Characteristics of Equity Securities}

\paragraph{1. Components of Return}
\[
R_{eq} = \text{Price Change} + \text{Dividends} + \text{Currency Gain/Loss}
\]
\begin{itemize}
    \item For foreign equity, exchange rate movements affect returns.
    \item Example: A Japanese investor in euro stocks benefits if EUR appreciates vs. JPY.
\end{itemize}

\paragraph{2. Long-Term Return Illustration}
\begin{table}[h!]
\centering
\caption*{Exhibit 3: Real Growth of \$1 (1900–2016, U.S.)}
\begin{tabular}{|l|l|}
\hline
\textbf{Investment Type} & \textbf{Terminal Wealth (\$)} \\
\hline
Equities (with reinvested dividends) & 1,402 \\
\hline
Equities (price-only) & 11.9 \\
\hline
Bonds & 9.8 \\
\hline
Bills & 2.6 \\
\hline
\end{tabular}
\end{table}

\paragraph{3. Risk Hierarchy}
\begin{itemize}
    \item \textbf{Preferred stock} less risky than common stock due to:
    \begin{itemize}
        \item Fixed dividends.
        \item Priority in liquidation.
    \end{itemize}
    \item \textbf{Common stock} riskier due to variable dividends and residual claim.
\end{itemize}

\paragraph{4. Option Features and Risk}
\begin{table}[h!]
\centering
\caption*{Exhibit 4: Option Features and Relative Risk}
\begin{tabular}{|l|l|l|}
\hline
\textbf{Feature} & \textbf{Effect on Risk} & \textbf{Dividend Yield} \\
\hline
Putable shares & ↓ Risk (downside protection) & Lower yield \\
\hline
Callable shares & ↑ Risk (upside capped) & Higher yield \\
\hline
Cumulative preferred & ↓ Risk (dividend arrears paid) & Lower yield \\
\hline
Non-cumulative preferred & ↑ Risk & Higher yield \\
\hline
\end{tabular}
\end{table}

\bigskip

\subsection*{LOS 44.f: Role of Equity Securities in Company Financing}

\paragraph{Equity Financing Uses:}
\begin{itemize}
    \item Funding for long-term assets, expansion, and R\&D.
    \item Provides a non-repayable capital base (unlike debt).
    \item Enables use of shares as:
    \begin{itemize}
        \item Acquisition currency.
        \item Employee incentive compensation (stock options, RSUs).
    \end{itemize}
    \item Publicly traded equity enhances:
    \begin{itemize}
        \item Liquidity.
        \item Access to capital markets.
        \item Compliance with regulatory capital adequacy ratios.
    \end{itemize}
\end{itemize}

\bigskip

\subsection*{LOS 44.g: Market Value vs. Book Value of Equity}

\paragraph{1. Book Value of Equity (Accounting Value)}
\[
\text{Book Value} = \text{Total Assets} - \text{Total Liabilities}
\]
\begin{itemize}
    \item Represents cumulative historical capital invested and retained.
    \item Increases with positive net income and retained earnings.
\end{itemize}

\paragraph{2. Market Value of Equity}
\[
\text{Market Value} = \text{Share Price} \times \text{Number of Outstanding Shares}
\]
\begin{itemize}
    \item Reflects investor expectations about firm’s future performance, risk, and growth.
    \item Generally $\neq$ book value (market anticipates future cash flows).
\end{itemize}

\paragraph{3. Relationship}
\begin{itemize}
    \item Management aims to increase book value to support higher market value.
    \item However, book value ignores investor expectations, intangibles, and growth potential.
\end{itemize}

\paragraph{4. Key Ratio: Price-to-Book (P/B) or Market-to-Book}
\[
P/B = \frac{\text{Market Value of Equity}}{\text{Book Value of Equity}}
\]
\begin{itemize}
    \item High $P/B$ → growth stock (optimistic expectations).
    \item Low $P/B$ → value stock (potential undervaluation).
\end{itemize}

\begin{table}[h!]
\centering
\caption*{Exhibit 5: Market Value vs. Book Value Comparison}
\begin{adjustbox}{max width=\textwidth}
\begin{tabular}{|l|l|l|}
\hline
\textbf{Concept} & \textbf{Definition} & \textbf{Key Driver} \\
\hline
Book Value & Historical equity on balance sheet & Retained earnings \\
\hline
Market Value & Current share market capitalization & Expected future performance \\
\hline
Price-to-Book Ratio & Market / Book & Investor sentiment and growth outlook \\
\hline
\end{tabular}
\end{adjustbox}
\end{table}

\bigskip

\subsection*{LOS 44.h: Cost of Equity, Return on Equity, and Required Return}

\paragraph{1. Accounting Return on Equity (ROE)}
\[
ROE = \frac{\text{Net Income available to Common}}{\text{Average Book Value of Common Equity}}
\]
Alternative (single-period):
\[
ROE = \frac{\text{Net Income}}{\text{Beginning Book Value of Equity}}
\]

\paragraph{2. Interpretation}
\begin{itemize}
    \item Higher ROE indicates efficient use of shareholders’ capital.
    \item Must evaluate cause:
    \begin{itemize}
        \item Rising ROE due to lower equity base (share buybacks) is not necessarily positive.
        \item Excessive leverage can inflate ROE but raise financial risk.
    \end{itemize}
\end{itemize}

\paragraph{3. Example: ROE and Value Metrics}
\[
\text{Market Value} = P \times \text{Shares Outstanding}
\]
\[
\text{Book Value per Share} = \frac{\text{Book Value of Equity}}{\text{Shares Outstanding}}
\]
\[
\text{Price-to-Book Ratio} = \frac{P}{\text{Book Value per Share}}
\]
\begin{itemize}
    \item Example: Share price = \$16.80, shares = 3,710,000 $\Rightarrow$ Market Value = \$62,328,000.
\end{itemize}

\paragraph{4. Cost of Equity (Investor Perspective)}
\begin{itemize}
    \item The firm’s cost of equity = investor’s required rate of return.
    \item Estimated using:
    \[
    k_e = R_f + \beta (R_m - R_f) \quad \text{(CAPM)}
    \]
    or via Dividend Discount Model:
    \[
    k_e = \frac{D_1}{P_0} + g
    \]
    \item Higher cost of equity $\Rightarrow$ higher required return; lowers intrinsic value.
\end{itemize}

\paragraph{5. Relationship Summary}
\begin{table}[h!]
\centering
\caption*{Exhibit 6: Relationship Between ROE, Cost of Equity, and Required Return}
\begin{adjustbox}{max width=\textwidth}
\begin{tabular}{|l|l|l|}
\hline
\textbf{Measure} & \textbf{Definition} & \textbf{Interpretation} \\
\hline
ROE (Accounting) & Net Income / Avg. Book Equity & Measures profitability based on accounting data \\
\hline
Cost of Equity & Required return by investors & Opportunity cost of capital \\
\hline
Required Return & Minimum return acceptable to investors & Determines share valuation \\
\hline
\end{tabular}
\end{adjustbox}
\end{table}

\paragraph{6. Analytical Insight}
\begin{itemize}
    \item If $ROE > k_e$: firm adds value; management creating shareholder wealth.
    \item If $ROE < k_e$: firm destroying value; poor capital allocation.
    \item $ROE$ volatility indicates riskiness of earnings stream.
\end{itemize}

\paragraph{7. Connection with DuPont Decomposition}
\[
ROE = \text{Net Profit Margin} \times \text{Asset Turnover} \times \text{Equity Multiplier}
\]
\begin{itemize}
    \item Helps identify drivers of performance: profitability, efficiency, or leverage.
\end{itemize}

\bigskip

\paragraph{Summary Insight:}
\begin{itemize}
    \item Foreign equity exposure adds diversification and currency risk.
    \item Equity securities vary by control rights, dividend priority, and option features.
    \item Market value reflects expectations; book value reflects history.
    \item ROE vs. cost of equity comparison indicates value creation.
\end{itemize}

\section*{Module 45.1: Company Research Reports}

\subsection*{LOS 45.a: Elements of a Thorough Company Research Report}

\paragraph{Definition:}
\begin{itemize}
    \item A \textbf{company research report} presents an analyst’s \textbf{valuation} and \textbf{investment recommendation} (Buy/Hold/Sell) based on:
    \begin{itemize}
        \item Projected earnings, cash flows, and financial condition.
        \item Market position, competitive dynamics, and risk assessment.
    \end{itemize}
\end{itemize}

\paragraph{Purpose:}
\begin{itemize}
    \item To provide investors with an informed opinion on the intrinsic value and investment potential of a firm.
    \item Reports can be:
    \begin{itemize}
        \item \textbf{Initial (Initiating Coverage):} Comprehensive, in-depth analysis.
        \item \textbf{Follow-up Reports:} Updates or revisions to prior analyses.
        \item \textbf{Internal Reports:} Often less detailed, sometimes verbal.
    \end{itemize}
\end{itemize}

\paragraph{1. Structure of an Initial Research Report}

\begin{table}[h!]
\centering
\caption*{Exhibit 1: Components of an Initial (Full) Research Report}
\begin{tabular}{|p{5cm}|p{10cm}|}
\hline
\textbf{Section} & \textbf{Description / Contents} \\
\hline
\textbf{Front Matter} & Basic metadata: company name, ticker, analyst name, recommendation (Buy/Hold/Sell), target price range, report date, disclaimers, and legal disclosures. \\
\hline
\textbf{Investment Thesis \& Rationale} & Summary of valuation conclusion and justification — why the stock is attractive or unattractive; include catalysts and valuation drivers. \\
\hline
\textbf{Company Description} & Business model overview, strategy, revenue sources, geographic segments, management team, and ownership structure. \\
\hline
\textbf{Industry Overview \& Competitive Positioning} & Market size, growth drivers, profitability trends, peer comparison, Porter’s Five Forces, and SWOT analysis. \\
\hline
\textbf{Financial Analysis \& Modeling} & Historical and projected income statement, balance sheet, and cash flows; key performance ratios (margin, ROE, ROIC, leverage, liquidity); scenario/sensitivity analysis. \\
\hline
\textbf{Valuation Section} & Intrinsic (DCF) and/or relative valuation (multiples such as P/E, EV/EBITDA, P/B); compare to target price and peer medians. \\
\hline
\textbf{ESG Factors} & Environmental, social, and governance risk/opportunity assessment; sustainability metrics and governance quality. \\
\hline
\textbf{Risks (Upside \& Downside)} & Key catalysts that could alter valuation: regulatory, operational, macroeconomic, or competitive factors. Quantify sensitivity to key assumptions. \\
\hline
\end{tabular}
\end{table}

\paragraph{Example: Summary Table in a Professional Equity Report}
\[
\begin{array}{|l|l|}
\hline
\textbf{Company:} & ABC Corporation \\
\textbf{Sector:} & Consumer Staples \\
\textbf{Recommendation:} & \textbf{BUY} (Target Price: \$125, Upside: +20\%) \\
\textbf{Investment Horizon:} & 12 months \\
\hline
\textbf{Key Drivers:} & Volume growth in Asia, product premiumization, efficiency gains \\
\textbf{Key Risks:} & FX volatility, raw material inflation, competitive pricing \\
\hline
\end{array}
\]

\paragraph{2. Structure of a Follow-Up (Update) Report}

\begin{table}[h!]
\centering
\caption*{Exhibit 2: Components of a Subsequent / Update Report}
\begin{tabular}{|p{5cm}|p{10cm}|}
\hline
\textbf{Section} & \textbf{Description / Focus} \\
\hline
\textbf{Front Matter} & Updated recommendation, price targets, and disclosure updates. \\
\hline
\textbf{Revised Recommendation \& Rationale} & Reasons for any change (e.g., from Buy → Hold due to valuation or earnings). \\
\hline
\textbf{New Information Analysis} & Variance analysis: expected vs. actual results (revenues, margins, earnings). Discuss earnings call insights. \\
\hline
\textbf{Updated Valuation} & Incorporate revised forecasts and adjusted assumptions (e.g., WACC, growth rate). \\
\hline
\textbf{Revised Risk Assessment} & New risk events since last report: regulatory updates, macroeconomic shifts, management changes. \\
\hline
\end{tabular}
\end{table}

\paragraph{3. Legal and Compliance Requirements}
\begin{itemize}
    \item Disclosure of:
    \begin{itemize}
        \item Conflicts of interest (analyst holdings, firm banking relationships).
        \item Data sources and methodology.
        \item Regulatory statements per jurisdiction (e.g., FINRA, FCA, ESMA).
    \end{itemize}
\end{itemize}

\paragraph{4. Example Summary: Report Timeline}
\begin{itemize}
    \item \textbf{Initiating coverage:} Comprehensive—includes valuation model, full industry analysis.
    \item \textbf{Quarterly updates:} Focused—summarizes performance vs. expectations.
    \item \textbf{Event-driven updates:} Triggered by M\&A, regulatory changes, or earnings announcements.
\end{itemize}

\bigskip

\subsection*{LOS 45.b: Determining a Company’s Business Model}

\paragraph{1. Definition and Purpose}
\begin{itemize}
    \item A company’s \textbf{business model} describes how it creates value — i.e., how it earns revenue, generates profit, and manages costs.
    \item Understanding the business model is fundamental for:
    \begin{itemize}
        \item Projecting financial performance.
        \item Estimating intrinsic value.
        \item Identifying competitive advantages and risks.
    \end{itemize}
\end{itemize}

\paragraph{2. Core Components of a Business Model}
\begin{table}[h!]
\centering
\caption*{Exhibit 3: Key Components of a Business Model}
\begin{tabular}{|l|p{10cm}|}
\hline
\textbf{Component} & \textbf{Description / Analytical Focus} \\
\hline
Products \& Services & What the firm sells; key revenue sources, diversification level. \\
\hline
Customers & Who buys the product; customer concentration, bargaining power, switching costs. \\
\hline
Sales Channels & How the firm reaches customers (e.g., retail, online, distributors). \\
\hline
Pricing \& Payment Terms & Revenue model (subscription, volume-based, freemium); pricing power and credit terms. \\
\hline
Suppliers \& Key Relationships & Supply chain strength, input specialization, supplier concentration and power. \\
\hline
\end{tabular}
\end{table}

\paragraph{3. Analyst Approach}
\begin{itemize}
    \item Identify deviations from industry-standard business models.
    \item Evaluate:
    \begin{itemize}
        \item Barriers to entry and switching costs.
        \item Economies of scale and network effects.
        \item Integration (vertical/horizontal).
    \end{itemize}
\end{itemize}

\paragraph{4. Sources of Information}
\begin{itemize}
    \item Analysts use four major information categories:
    \begin{enumerate}
        \item \textbf{Company-Provided Information:} Annual/quarterly filings, investor presentations, press releases, IR communication, website.
        \item \textbf{Public Third-Party Sources:} Industry reports, government data, media coverage, social media insights.
        \item \textbf{Proprietary Third-Party Data:} Subscription databases (e.g., Bloomberg, Refinitiv, FactSet).
        \item \textbf{Proprietary Primary Research:} Surveys, expert interviews, channel checks, market studies.
    \end{enumerate}
\end{itemize}

\begin{table}[h!]
\centering
\caption*{Exhibit 4: Information Source Matrix for Business Model Analysis}
\begin{tabular}{|p{5.5cm}|p{5.5cm}|p{5.5cm}|}
\hline
\textbf{Source Type} & \textbf{Examples} & \textbf{Purpose} \\
\hline
Company Data & 10-K, Investor Decks, Calls & Quantitative details; management intent \\
\hline
Public Third-Party & Government, Media, Social Platforms & Industry positioning, sentiment, regulatory context \\
\hline
Proprietary Third-Party & Bloomberg, FactSet, IBES & Financial metrics, comparables, forecasts \\
\hline
Primary Research & Surveys, Channel Checks, Expert Panels & Validation of sales dynamics, demand patterns \\
\hline
\end{tabular}
\end{table}

\paragraph{5. Example Analysis: Business Model Comparison}
\begin{table}[h!]
\centering
\caption*{Exhibit 5: Illustrative Business Model Examples}
\begin{tabular}{|p{5.5cm}|p{5.5cm}|p{5.5cm}|}
\hline
\textbf{Company Type} & \textbf{Revenue Model} & \textbf{Key Analytical Focus} \\
\hline
Software-as-a-Service (SaaS) & Subscription-based recurring revenue & Churn rates, customer lifetime value, scalability \\
\hline
Retail / E-commerce & Product sales (margin-based) & Inventory turnover, price competition, logistics cost \\
\hline
Manufacturing & Unit sales, contract production & Input costs, capacity utilization, supplier risk \\
\hline
Financial Services & Interest income, fees, spreads & Asset quality, leverage, regulatory compliance \\
\hline
\end{tabular}
\end{table}

\paragraph{6. Analytical Considerations}
\begin{itemize}
    \item Assess \textbf{customer and supplier bargaining power} (Porter’s Five Forces).
    \item Examine whether firm’s business model deviates strategically from peers.
    \item Link business model structure directly to revenue and cost drivers used in valuation.
\end{itemize}

\bigskip

\paragraph{Summary Insight:}
\begin{itemize}
    \item A thorough research report integrates qualitative (business model, industry analysis) and quantitative (valuation, financial modeling) perspectives.
    \item Understanding the business model allows accurate forecasting of earnings, margins, and capital needs.
    \item The best reports clearly link business strategy → financial forecasts → valuation recommendation.
\end{itemize}

\section*{Module 45.1: Company Research Reports}

\subsection*{LOS 45.a: Elements of a Thorough Company Research Report}

\paragraph{Definition:}
\begin{itemize}
    \item A \textbf{company research report} presents an analyst’s \textbf{valuation} and \textbf{investment recommendation} (Buy/Hold/Sell) based on:
    \begin{itemize}
        \item Projected earnings, cash flows, and financial condition.
        \item Market position, competitive dynamics, and risk assessment.
    \end{itemize}
\end{itemize}

\paragraph{Purpose:}
\begin{itemize}
    \item To provide investors with an informed opinion on the intrinsic value and investment potential of a firm.
    \item Reports can be:
    \begin{itemize}
        \item \textbf{Initial (Initiating Coverage):} Comprehensive, in-depth analysis.
        \item \textbf{Follow-up Reports:} Updates or revisions to prior analyses.
        \item \textbf{Internal Reports:} Often less detailed, sometimes verbal.
    \end{itemize}
\end{itemize}

\paragraph{1. Structure of an Initial Research Report}

\begin{table}[h!]
\centering
\caption*{Exhibit 1: Components of an Initial (Full) Research Report}
\begin{tabular}{|p{5cm}|p{10cm}|}
\hline
\textbf{Section} & \textbf{Description / Contents} \\
\hline
\textbf{Front Matter} & Basic metadata: company name, ticker, analyst name, recommendation (Buy/Hold/Sell), target price range, report date, disclaimers, and legal disclosures. \\
\hline
\textbf{Investment Thesis \& Rationale} & Summary of valuation conclusion and justification — why the stock is attractive or unattractive; include catalysts and valuation drivers. \\
\hline
\textbf{Company Description} & Business model overview, strategy, revenue sources, geographic segments, management team, and ownership structure. \\
\hline
\textbf{Industry Overview \& Competitive Positioning} & Market size, growth drivers, profitability trends, peer comparison, Porter’s Five Forces, and SWOT analysis. \\
\hline
\textbf{Financial Analysis \& Modeling} & Historical and projected income statement, balance sheet, and cash flows; key performance ratios (margin, ROE, ROIC, leverage, liquidity); scenario/sensitivity analysis. \\
\hline
\textbf{Valuation Section} & Intrinsic (DCF) and/or relative valuation (multiples such as P/E, EV/EBITDA, P/B); compare to target price and peer medians. \\
\hline
\textbf{ESG Factors} & Environmental, social, and governance risk/opportunity assessment; sustainability metrics and governance quality. \\
\hline
\textbf{Risks (Upside \& Downside)} & Key catalysts that could alter valuation: regulatory, operational, macroeconomic, or competitive factors. Quantify sensitivity to key assumptions. \\
\hline
\end{tabular}
\end{table}

\begin{table}[h!]
\centering
\caption*{Example: Summary Table in a Professional Equity Report}
\small
\begin{tabularx}{0.8\textwidth}{@{} l X @{}}
\toprule
\textbf{Company:} & ABC Corporation \\
\textbf{Sector:} & Consumer Staples \\
\textbf{Recommendation:} & \textbf{BUY} (Target Price: \$125; Upside: +20\%) \\
\textbf{Investment Horizon:} & 12 months \\
\midrule
\textbf{Key Drivers:} & Volume growth in Asia; product premiumization; efficiency gains \\
\textbf{Key Risks:} & FX volatility; raw material inflation; competitive pricing \\
\bottomrule
\end{tabularx}
\end{table}

\paragraph{2. Structure of a Follow-Up (Update) Report}

\begin{table}[h!]
\centering
\caption*{Exhibit 2: Components of a Subsequent / Update Report}
\begin{tabular}{|p{5cm}|p{10cm}|}
\hline
\textbf{Section} & \textbf{Description / Focus} \\
\hline
\textbf{Front Matter} & Updated recommendation, price targets, and disclosure updates. \\
\hline
\textbf{Revised Recommendation \& Rationale} & Reasons for any change (e.g., from Buy → Hold due to valuation or earnings). \\
\hline
\textbf{New Information Analysis} & Variance analysis: expected vs. actual results (revenues, margins, earnings). Discuss earnings call insights. \\
\hline
\textbf{Updated Valuation} & Incorporate revised forecasts and adjusted assumptions (e.g., WACC, growth rate). \\
\hline
\textbf{Revised Risk Assessment} & New risk events since last report: regulatory updates, macroeconomic shifts, management changes. \\
\hline
\end{tabular}
\end{table}

\paragraph{3. Legal and Compliance Requirements}
\begin{itemize}
    \item Disclosure of:
    \begin{itemize}
        \item Conflicts of interest (analyst holdings, firm banking relationships).
        \item Data sources and methodology.
        \item Regulatory statements per jurisdiction (e.g., FINRA, FCA, ESMA).
    \end{itemize}
\end{itemize}

\paragraph{4. Example Summary: Report Timeline}
\begin{itemize}
    \item \textbf{Initiating coverage:} Comprehensive—includes valuation model, full industry analysis.
    \item \textbf{Quarterly updates:} Focused—summarizes performance vs. expectations.
    \item \textbf{Event-driven updates:} Triggered by M\&A, regulatory changes, or earnings announcements.
\end{itemize}

\bigskip

\subsection*{LOS 45.b: Determining a Company’s Business Model}

\paragraph{1. Definition and Purpose}
\begin{itemize}
    \item A company’s \textbf{business model} describes how it creates value — i.e., how it earns revenue, generates profit, and manages costs.
    \item Understanding the business model is fundamental for:
    \begin{itemize}
        \item Projecting financial performance.
        \item Estimating intrinsic value.
        \item Identifying competitive advantages and risks.
    \end{itemize}
\end{itemize}

\paragraph{2. Core Components of a Business Model}
\begin{table}[h!]
\centering
\caption*{Exhibit 3: Key Components of a Business Model}
\begin{tabular}{|l|p{10cm}|}
\hline
\textbf{Component} & \textbf{Description / Analytical Focus} \\
\hline
Products \& Services & What the firm sells; key revenue sources, diversification level. \\
\hline
Customers & Who buys the product; customer concentration, bargaining power, switching costs. \\
\hline
Sales Channels & How the firm reaches customers (e.g., retail, online, distributors). \\
\hline
Pricing \& Payment Terms & Revenue model (subscription, volume-based, freemium); pricing power and credit terms. \\
\hline
Suppliers \& Key Relationships & Supply chain strength, input specialization, supplier concentration and power. \\
\hline
\end{tabular}
\end{table}

\paragraph{3. Analyst Approach}
\begin{itemize}
    \item Identify deviations from industry-standard business models.
    \item Evaluate:
    \begin{itemize}
        \item Barriers to entry and switching costs.
        \item Economies of scale and network effects.
        \item Integration (vertical/horizontal).
    \end{itemize}
\end{itemize}

\paragraph{4. Sources of Information}
\begin{itemize}
    \item Analysts use four major information categories:
    \begin{enumerate}
        \item \textbf{Company-Provided Information:} Annual/quarterly filings, investor presentations, press releases, IR communication, website.
        \item \textbf{Public Third-Party Sources:} Industry reports, government data, media coverage, social media insights.
        \item \textbf{Proprietary Third-Party Data:} Subscription databases (e.g., Bloomberg, Refinitiv, FactSet).
        \item \textbf{Proprietary Primary Research:} Surveys, expert interviews, channel checks, market studies.
    \end{enumerate}
\end{itemize}

\begin{table}[h!]
\centering
\caption*{Exhibit 4: Information Source Matrix for Business Model Analysis}
\begin{tabular}{|p{5cm}|p{5cm}|p{5cm}|}
\hline
\textbf{Source Type} & \textbf{Examples} & \textbf{Purpose} \\
\hline
Company Data & 10-K, Investor Decks, Calls & Quantitative details; management intent \\
\hline
Public Third-Party & Government, Media, Social Platforms & Industry positioning, sentiment, regulatory context \\
\hline
Proprietary Third-Party & Bloomberg, FactSet, IBES & Financial metrics, comparables, forecasts \\
\hline
Primary Research & Surveys, Channel Checks, Expert Panels & Validation of sales dynamics, demand patterns \\
\hline
\end{tabular}
\end{table}

\paragraph{5. Example Analysis: Business Model Comparison}
\begin{table}[h!]
\centering
\caption*{Exhibit 5: Illustrative Business Model Examples}
\begin{tabular}{|p{5cm}|p{5cm}|p{5cm}|}
\hline
\textbf{Company Type} & \textbf{Revenue Model} & \textbf{Key Analytical Focus} \\
\hline
Software-as-a-Service (SaaS) & Subscription-based recurring revenue & Churn rates, customer lifetime value, scalability \\
\hline
Retail / E-commerce & Product sales (margin-based) & Inventory turnover, price competition, logistics cost \\
\hline
Manufacturing & Unit sales, contract production & Input costs, capacity utilization, supplier risk \\
\hline
Financial Services & Interest income, fees, spreads & Asset quality, leverage, regulatory compliance \\
\hline
\end{tabular}
\end{table}

\paragraph{6. Analytical Considerations}
\begin{itemize}
    \item Assess \textbf{customer and supplier bargaining power} (Porter’s Five Forces).
    \item Examine whether firm’s business model deviates strategically from peers.
    \item Link business model structure directly to revenue and cost drivers used in valuation.
\end{itemize}

\bigskip

\paragraph{Summary Insight:}
\begin{itemize}
    \item A thorough research report integrates qualitative (business model, industry analysis) and quantitative (valuation, financial modeling) perspectives.
    \item Understanding the business model allows accurate forecasting of earnings, margins, and capital needs.
    \item The best reports clearly link business strategy → financial forecasts → valuation recommendation.
\end{itemize}

\section*{Module 46.1: Industry Analysis}

\subsection*{LOS 46.a: Purpose and Steps of Industry and Competitive Analysis}

\paragraph{1. Definition and Objective}
\begin{itemize}
    \item \textbf{Industry and competitive analysis} is a macro-level evaluation of what drives:
    \begin{itemize}
        \item Industry size, profitability, and market share.
        \item Competitive dynamics and company positioning within the industry.
    \end{itemize}
    \item Objective: Identify the \textbf{industry’s base rate of profitability} and assess how structural and competitive forces affect long-term returns.
\end{itemize}

\paragraph{2. Importance}
\begin{itemize}
    \item Industry factors determine long-run profitability; firm-specific factors explain relative out/underperformance.
    \item Helps improve financial forecasts and identify attractive or overlooked investments.
    \item Useful for:
    \begin{itemize}
        \item Sector allocation in portfolios.
        \item Reducing company-specific risk via diversified industry exposure.
    \end{itemize}
\end{itemize}

\paragraph{3. Steps in Industry and Competitive Analysis}
\begin{enumerate}
    \item \textbf{Define the Industry:}
    \begin{itemize}
        \item Identify key characteristics — product type, geographic scope, or production process.
        \item Use standard classification systems (GICS, ICB, TRBC).
        \item Handle diversified firms carefully.
    \end{itemize}
    \item \textbf{Survey the Industry:}
    \begin{itemize}
        \item Assess size, growth rate, profitability trends, and market share evolution.
    \end{itemize}
    \item \textbf{Analyze Industry Structure (Porter’s Five Forces):}
    \begin{itemize}
        \item Rivalry among existing competitors.
        \item Threat of new entrants.
        \item Threat of substitutes.
        \item Bargaining power of buyers.
        \item Bargaining power of suppliers.
    \end{itemize}
    \item \textbf{Examine External Factors (PESTLE Framework):}
    \begin{itemize}
        \item Political, Economic, Social, Technological, Legal, Environmental factors.
    \end{itemize}
    \item \textbf{Evaluate Company Strategies:}
    \begin{itemize}
        \item Assess competitive advantages and business models relative to peers.
        \item Identify cost leadership, differentiation, or niche positioning.
    \end{itemize}
\end{enumerate}

\begin{table}[h!]
\centering
\caption*{Exhibit 1: Frameworks for Industry and Competitive Analysis}
\begin{tabular}{|p{5cm}|p{5cm}|p{5cm}|}
\hline
\textbf{Framework} & \textbf{Purpose} & \textbf{Example Insight} \\
\hline
Porter’s Five Forces & Structure and competition intensity & High entry barriers $\Rightarrow$ sustainable profits \\
\hline
PESTLE & Macro external environment & Regulation or technology shaping demand \\
\hline
SWOT & Company-level advantage & Leverage internal strengths against industry trends \\
\hline
\end{tabular}
\end{table}

\bigskip

\subsection*{LOS 46.b: Industry Classification Methods and Comparison}

\paragraph{1. Overview}
\begin{itemize}
    \item Industry classification helps compare companies, assess trends, and value sectors.
    \item Traditional systems (government-based) were production-oriented and local.
    \item Modern commercial systems (GICS, ICB, TRBC) are \textbf{global and market-oriented.}
\end{itemize}

\paragraph{2. Commercial Classification Systems}
\begin{table}[h!]
\centering
\caption*{Exhibit 2: Major Global Classification Systems}
\begin{tabular}{|p{2cm}|p{3cm}|p{7cm}|p{3cm}|}
\hline
\textbf{System} & \textbf{Developer} & \textbf{Hierarchy (Top → Bottom)} & \textbf{Coverage} \\
\hline
GICS & S\&P / MSCI & Sector → Industry Group → Industry → Subindustry & Public firms \\
\hline
ICB & FTSE Russell & Industry → Supersector → Sector → Subsector & Public firms \\
\hline
TRBC & Refinitiv & Economic Sector → Business Sector → Industry Group → Industry → Activity & Public \& private entities \\
\hline
\end{tabular}
\end{table}

\paragraph{3. Common Top-Level Sectors (Across GICS, ICB, TRBC)}
\begin{multicols}{2}
\begin{itemize}
    \item Energy
    \item Financials
    \item Basic Materials
    \item Information Technology
    \item Industrials
    \item Telecommunications
    \item Consumer Discretionary (Cyclicals)
    \item Consumer Staples (Noncyclicals)
    \item Utilities
    \item Real Estate
    \item Health Care
\end{itemize}
\end{multicols}

\paragraph{4. Classification Rules}
\begin{itemize}
    \item A firm with one business line → classified in that line.
    \item Multi-segment firms → classify by business line contributing:
    \begin{itemize}
        \item More than 60\% of total revenue.
        \item If none exceeds 60\%, use the one above 50\% of revenue, profit, or assets.
        \item If none applies → judgmental classification or “conglomerate.”
    \end{itemize}
\end{itemize}

\paragraph{5. Limitations of Classification Systems}
\begin{itemize}
    \item \textbf{Inappropriate groupings:} Categories too broad (e.g., “software” covers diverse firms).
    \item \textbf{Diversified firms:} One classification may misrepresent operations (e.g., Amazon in “retail” despite large cloud business).
    \item \textbf{Geographical variations:} Some sectors (e.g., healthcare) are regionally restricted.
    \item \textbf{Grouping changes over time:} Updates may distort comparability and create survivorship bias.
\end{itemize}

\paragraph{6. Alternative Grouping Approaches}
\begin{enumerate}
    \item \textbf{By Geography:} Classify by headquarter location or primary exchange (e.g., Toyota = Japanese firm despite U.S. revenues).
    \item \textbf{By Business Cycle Sensitivity:}
    \begin{itemize}
        \item \textbf{Defensive:} Demand stable across cycles (utilities, consumer staples, healthcare).
        \item \textbf{Cyclical:} Earnings tied to economic activity (industrials, energy, discretionary).
    \end{itemize}
    \item \textbf{By Financial Characteristics:} 
    \begin{itemize}
        \item Size (market cap), valuation (P/E), profitability (ROE), or growth rate.
        \item Example: “Large-cap growth” vs. “small-cap value.”
    \end{itemize}
    \item \textbf{By Statistical Correlation:} 
    \begin{itemize}
        \item Cluster analysis groups firms with historically correlated returns.
        \item Resulting clusters show lower correlation across groups.
    \end{itemize}
    \item \textbf{By ESG Attributes:} 
    \begin{itemize}
        \item Group by sustainability or governance quality (e.g., carbon footprint, diversity).
    \end{itemize}
\end{enumerate}

\begin{table}[h!]
\centering
\caption*{Exhibit 3: Comparison – Cyclical vs. Defensive Industries}
\begin{tabular}{|l|l|l|}
\hline
\textbf{Feature} & \textbf{Cyclical Industry} & \textbf{Defensive Industry} \\
\hline
Demand & Sensitive to economic cycles & Stable across cycles \\
\hline
Earnings Volatility & High & Low \\
\hline
Operating Leverage & High & Low \\
\hline
Examples & Energy, Autos, Industrials & Utilities, Healthcare, Staples \\
\hline
Investor Appeal & Growth during expansion & Stability during recession \\
\hline
\end{tabular}
\end{table}

\bigskip

\subsection*{LOS 46.c: Determining Industry Size, Growth, Profitability, and Market Share}

\paragraph{1. Industry Size}
\begin{itemize}
    \item \textbf{Definition:} Total annual sales of the industry’s products.
    \item May differ from total company revenues if firms operate across multiple industries.
    \item Estimation issues:
    \begin{itemize}
        \item Private companies and unlisted entities may lack data.
        \item Use alternative sources (government, trade associations, market studies).
    \end{itemize}
\end{itemize}

\paragraph{2. Industry Growth Characteristics}
\begin{itemize}
    \item Measured via annual growth rates or compound annual growth rate (CAGR).
    \item \textbf{Types:}
    \begin{itemize}
        \item \textbf{Growth Industry:} High expansion potential, often driven by new technology. Example: renewable energy, AI.
        \item \textbf{Mature Industry:} Stable demand, aligned with GDP growth. Example: utilities, construction.
        \item \textbf{Declining Industry:} Falling demand due to substitution or saturation. Example: print media.
    \end{itemize}
    \item Analysts should evaluate:
    \begin{itemize}
        \item Persistence of growth.
        \item Technological disruption.
        \item Business cycle sensitivity.
    \end{itemize}
\end{itemize}

\paragraph{3. Style Box Framework}
\begin{table}[h!]
\centering
\caption*{Exhibit 4: Industry Style Box — Growth vs. Cyclicality}
\begin{tabular}{|l|l|l|}
\hline
\textbf{Type} & \textbf{Business Cycle Sensitivity} & \textbf{Example} \\
\hline
Defensive Growth & Low sensitivity + high growth & Biotechnology \\
\hline
Defensive Mature & Low sensitivity + low growth & Utilities \\
\hline
Cyclical Growth & High sensitivity + high growth & Digital Advertising, Tech Hardware \\
\hline
Cyclical Mature & High sensitivity + low growth & Oil Production, Steel Manufacturing \\
\hline
\end{tabular}
\end{table}

\paragraph{4. Industry Profitability}
\begin{itemize}
    \item Profitability measured ideally via \textbf{Return on Invested Capital (ROIC)}:
    \[
    ROIC = \frac{\text{Net Operating Profit After Tax (NOPAT)}}{\text{Invested Capital}}
    \]
    \item Indicates whether firms earn above cost of capital.
    \item If unavailable (e.g., private firms), use proxy metrics like ROE, EBIT margin, or government/market data.
    \item Analyze trends: increasing, stable, or declining profitability.
\end{itemize}

\paragraph{5. Market Share and Concentration}
\begin{itemize}
    \item \textbf{Market Share:}
    \[
    \text{Market Share} = \frac{\text{Company Revenue}}{\text{Industry Size}}
    \]
    \item Market share trends reflect product acceptance and brand competitiveness.
    \item Acquisitions may distort organic share changes.
\end{itemize}

\paragraph{6. Industry Concentration (Herfindahl–Hirschman Index, HHI)}
\[
HHI = \sum_{i=1}^{n} s_i^2
\]
where $s_i$ = market share (\%) of each firm.

\begin{table}[h!]
\centering
\caption*{Exhibit 5: Interpretation of HHI Values}
\begin{tabular}{|l|l|}
\hline
\textbf{HHI Range} & \textbf{Interpretation} \\
\hline
$<$ 1{,}500 & Low concentration (competitive market) \\
\hline
1{,}500–2{,}500 & Moderate concentration \\
\hline
$>$ 2{,}500 & High concentration (oligopolistic market) \\
\hline
\end{tabular}
\end{table}

\paragraph{Example: HHI Calculation}
\[
\text{Market Shares: } 35\%, 25\%, 20\%, 10\%, 10\% \Rightarrow HHI = 35^2 + 25^2 + 20^2 + 10^2 + 10^2 = 2{,}450
\]
\begin{itemize}
    \item Indicates \textbf{moderate concentration}.
    \item Higher HHI $\Rightarrow$ lower competition, greater pricing power, higher margins.
\end{itemize}

\paragraph{7. Interpretation Summary}
\begin{table}[h!]
\centering
\caption*{Exhibit 6: Industry Metrics Summary}
\begin{tabular}{|l|l|l|}
\hline
\textbf{Dimension} & \textbf{High Value Interpretation} & \textbf{Low Value Interpretation} \\
\hline
Industry Growth & Expanding demand, innovation & Saturation, declining relevance \\
\hline
ROIC & Value creation above WACC & Value destruction, inefficiency \\
\hline
Market Share Trend & Gaining customer acceptance & Losing competitiveness \\
\hline
HHI & Concentrated, higher pricing power & Competitive, low pricing power \\
\hline
\end{tabular}
\end{table}

\bigskip

\paragraph{Summary Insight:}
\begin{itemize}
    \item Industry analysis links macro factors to firm performance.
    \item Analysts should integrate:
    \begin{itemize}
        \item Structural forces (Five Forces),
        \item External environment (PESTLE),
        \item Financial measures (growth, profitability, concentration).
    \end{itemize}
    \item The goal: Identify industries with sustainable economic profit potential and companies best positioned within them.
\end{itemize}

\section*{Module 46.2: Industry Structure and Competitive Positioning}

\subsection*{LOS 46.d: Analyzing Industry Structure and External Influences (Porter’s Five Forces \& PESTLE Frameworks)}

\paragraph{1. Purpose of Structural Analysis}
\begin{itemize}
    \item Industry structure analysis explains how competitive forces determine long-run profitability.
    \item Developed by \textbf{Michael Porter}, the framework identifies \textbf{five forces} that shape competition.
    \item Strong forces $\Rightarrow$ lower profitability and potential zero economic profit.
\end{itemize}

\paragraph{2. Porter’s Five Forces Overview}
\begin{table}[h!]
\centering
\caption*{Exhibit 1: Porter’s Five Forces Framework}
\begin{tabular}{|p{4cm}|p{5cm}|p{5cm}|}
\hline
\textbf{Force} & \textbf{Key Factors Increasing Strength} & \textbf{Impact on Industry Profitability} \\
\hline
1. Rivalry Among Existing Competitors & Many equally sized firms, slow growth, high fixed costs, undifferentiated products, costly exit & Intense price competition, reduced margins \\
\hline
2. Threat of New Entrants & Low barriers to entry, low capital needs, easy access to distribution, weak brand loyalty & New competition erodes prices and profits \\
\hline
3. Threat of Substitutes & Many close substitutes, low switching cost, high price elasticity of demand & Price ceiling imposed, reduced pricing power \\
\hline
4. Bargaining Power of Buyers & Few large buyers, high price sensitivity, standardized products, easy comparison shopping & Buyers can force price cuts or better quality \\
\hline
5. Bargaining Power of Suppliers & Few suppliers, scarce inputs, lack of substitutes, supplier integration risk & Higher input costs, margin pressure \\
\hline
\end{tabular}
\end{table}

\paragraph{3. Factors Affecting Competition}
\begin{itemize}
    \item \textbf{Barriers to Entry:} Higher barriers reduce competition (e.g., oil refining).
    \item \textbf{Market Concentration:} Few large players $\Rightarrow$ less rivalry; fragmentation $\Rightarrow$ high rivalry.
    \item \textbf{Capacity Utilization:} Unused capacity $\Rightarrow$ price wars (e.g., auto industry).
    \item \textbf{Market Stability:} Customer loyalty stabilizes profits.
    \item \textbf{Customer Price Sensitivity:} High sensitivity $\Rightarrow$ greater competition.
    \item \textbf{Industry Maturity:} Mature industries = slowing growth, stronger rivalry.
\end{itemize}

\paragraph{4. Example – Porter’s Five Forces Analysis: U.S. Retail Sector}
\begin{table}[h!]
\centering
\caption*{Exhibit 2: Example—Porter’s Analysis for U.S. Retail Industry}
\begin{tabular}{|l|p{5cm}|p{5cm}|}
\hline
\textbf{Force} & \textbf{Assessment} & \textbf{Explanation / Example} \\
\hline
Threat of New Entrants & Very High & Easy e-commerce entry, low regulatory barriers, 40,000+ new retailers formed monthly \\
\hline
Threat of Substitutes & Low & Limited substitutes for physical goods; services (travel, dining) not direct replacements \\
\hline
Power of Buyers & Moderate & Fragmented consumers, but easy price comparison online $\Rightarrow$ high price sensitivity \\
\hline
Power of Suppliers & Low to Moderate & Many suppliers, but brand owners (e.g., luxury goods) hold leverage \\
\hline
Rivalry Among Competitors & High & Thousands of similar retailers; price promotions common \\
\hline
\end{tabular}
\end{table}

\paragraph{5. PESTLE Framework for External Influences}
\begin{itemize}
    \item Complements Porter’s model by analyzing macro factors outside the industry’s direct control.
    \item PESTLE = \textbf{Political, Economic, Social, Technological, Legal, Environmental}.
\end{itemize}

\begin{table}[h!]
\centering
\caption*{Exhibit 3: PESTLE Analysis – Key Elements and Illustrations}
\begin{tabular}{|l|p{4cm}|p{6cm}|}
\hline
\textbf{Factor} & \textbf{Description} & \textbf{Example / Industry Impact} \\
\hline
Political & Government actions, taxation, regulation, subsidies & Energy (fuel taxes, carbon regulation), healthcare (public funding), defense (military budgets) \\
\hline
Economic & GDP growth, inflation, interest rates, credit, productivity & Cyclical sectors sensitive to GDP and rates; credit constraints limit consumer spending \\
\hline
Social & Demographics, lifestyle, consumer behavior, ethics & Demand for sustainable goods, social media trends, labor attitudes \\
\hline
Technological & Innovation pace, automation, disruptive technology & Digitalization, e-commerce, renewable energy, AI, 3D printing \\
\hline
Legal & Legislation, labor laws, antitrust, intellectual property & Tobacco restrictions, cannabis legalization, ESG disclosure laws \\
\hline
Environmental & Climate policy, resource scarcity, emissions standards & Transition to green energy, carbon taxes, sustainable packaging \\
\hline
\end{tabular}
\end{table}

\paragraph{6. Example – Sectoral Political Sensitivity}
\begin{itemize}
    \item \textbf{Energy:} Conflicts between low energy prices and emission-reduction goals; OPEC price coordination.
    \item \textbf{Healthcare:} Government = largest purchaser; may impose price caps or rationing.
    \item \textbf{Defense:} Governments as sole buyers; budgets depend on geopolitical risk and fiscal priorities.
\end{itemize}

\paragraph{7. Example – Technological Disruption}
\begin{itemize}
    \item \textbf{Sustaining Innovation:} Incremental improvements (e.g., higher-efficiency engines).
    \item \textbf{Disruptive Innovation:} Fundamental change creating new markets (e.g., film → digital photography).
    \item Firms must adapt or risk obsolescence.
\end{itemize}

\bigskip

\subsection*{LOS 46.e: Evaluating Competitive Strategy and Position}

\paragraph{1. Overview}
\begin{itemize}
    \item Every firm follows a \textbf{competitive strategy}, intentional or unintentional.
    \item Effective strategies produce sustained \textbf{economic profits (ROIC $>$ WACC)}.
    \item Strategies are evaluated based on:
    \begin{itemize}
        \item Responsiveness to competitive forces.
        \item Adaptation to external (PESTLE) factors.
        \item Execution capability.
    \end{itemize}
\end{itemize}

\paragraph{2. Porter’s Three Generic Competitive Strategies}
\begin{table}[h!]
\centering
\caption*{Exhibit 4: Porter’s Generic Strategies}
\begin{tabular}{|l|p{5cm}|p{5cm}|}
\hline
\textbf{Strategy} & \textbf{Characteristics} & \textbf{Requirements / Examples} \\
\hline
\textbf{Cost Leadership} & 
Lowest production cost, lowest price; large-scale operations. Used to protect or gain market share. &
Efficient operations, economies of scale, strict cost control (e.g., Walmart, Ryanair). \\
\hline
\textbf{Differentiation} & 
Unique product features, quality, or brand image; price premium sustainable if justified by value. &
Strong marketing, innovation, brand loyalty (e.g., Apple, BMW, LVMH). \\
\hline
\textbf{Focus (Niche)} & 
Target specific segment or geography, combining cost or differentiation advantage within niche. &
Specialization, deep market understanding (e.g., Ferrari, Rolex, boutique consultancies). \\
\hline
\end{tabular}
\end{table}

\paragraph{3. Strategy Effectiveness Criteria}
\begin{itemize}
    \item Internal consistency and alignment with company resources.
    \item Responsiveness to Porter’s five forces and PESTLE influences.
    \item Execution capability: management quality, incentives, and innovation.
\end{itemize}

\paragraph{4. Intentional vs. Unintentional Strategies}
\begin{table}[h!]
\centering
\caption*{Exhibit 5: Comparison of Strategic Approaches}
\begin{tabular}{|p{3cm}|p{6cm}|p{6cm}|}
\hline
\textbf{Type} & \textbf{Description} & \textbf{Examples / Implications} \\
\hline
Intentional Strategy & Deliberately planned, iterative (plan → execute → refine). & Most large firms (e.g., Toyota’s Kaizen process). \\
\hline
Unintentional Strategy & Emergent, unstructured, reactive. Sometimes yields innovation. & Pharma startups discovering drugs serendipitously. \\
\hline
\end{tabular}
\end{table}

\paragraph{5. Strategic Risks}
\begin{itemize}
    \item Being “stuck in the middle” — failure to commit to one primary strategy (cost vs. differentiation).
    \item Inability to sustain cost advantage or premium perception.
    \item External shocks (technological, legal) invalidating strategy.
\end{itemize}

\paragraph{6. Analytical Application – Evaluating a Firm’s Competitive Position}
\begin{itemize}
    \item Combine:
    \begin{itemize}
        \item \textbf{Internal metrics:} ROIC, margins, cost efficiency.
        \item \textbf{Industry metrics:} Five Forces assessment.
        \item \textbf{External metrics:} PESTLE exposure.
    \end{itemize}
    \item Determine if firm’s strategy is defensive (protect profits) or offensive (expand market share).
\end{itemize}

\bigskip

\subsection*{Key Takeaways Summary}

\begin{table}[h!]
\centering
\caption*{Exhibit 6: LOS 46.d–46.e Summary Overview}
\begin{tabular}{|l|p{11cm}|}
\hline
\textbf{Topic} & \textbf{Core Insights} \\
\hline
\textbf{Porter’s Five Forces} & Threat of entrants, substitutes, buyers, suppliers, and competitive rivalry determine long-term industry profitability. \\
\hline
\textbf{PESTLE Analysis} & Macro factors (Political, Economic, Social, Technological, Legal, Environmental) influence industry trends and risk. \\
\hline
\textbf{Cost Leadership Strategy} & Achieve lowest production cost; success depends on scale, efficiency, and cost discipline. \\
\hline
\textbf{Differentiation Strategy} & Provide unique products with sustainable premium pricing justified by perceived value. \\
\hline
\textbf{Focus Strategy} & Target a specific niche market—combine elements of cost leadership or differentiation. \\
\hline
\textbf{Strategic Evaluation} & Effective strategy = responsive to Five Forces, resilient to PESTLE, well-executed internally. \\
\hline
\end{tabular}
\end{table}

\paragraph{Overall Insight:}
\begin{itemize}
    \item Structural analysis (Porter) identifies internal industry pressures.
    \item Environmental analysis (PESTLE) captures external macro pressures.
    \item Strategic positioning (cost, differentiation, focus) determines firm performance relative to peers.
\end{itemize}


\section*{Module 47.1: Forecasting in Company Analysis}

\subsection*{LOS 47.a: Principles and Approaches to Forecasting Financial Results}

\paragraph{1. Purpose of Forecasting}
\begin{itemize}
    \item Financial forecasts are central to \textbf{valuation} and \textbf{investment recommendations}.
    \item External forecasts focus on key metrics (revenue, EPS).
    \item Internal forecasts are more detailed and long-term.
\end{itemize}

\paragraph{2. Forecast Objects}
\begin{table}[h!]
\centering
\caption*{Exhibit 1: Key Forecast Objects}
\begin{tabular}{|p{3cm}|p{5.5cm}|p{5.5cm}|}
\hline
\textbf{Category} & \textbf{Description} & \textbf{Example / Notes} \\
\hline
Financial items with clear drivers & Forecasted via measurable variables & Retail revenue = number of stores × sales/store \\
\hline
Items without clear drivers & Forecast directly using management inputs or prior data & “Other expenses” forecasted using past growth trend \\
\hline
Summary measures & Aggregated outcomes (EPS, FCF) & Fast but less transparent \\
\hline
Ad hoc items & Unusual events, contingencies & Lawsuits, regulatory changes, one-time windfalls \\
\hline
\end{tabular}
\end{table}

\paragraph{3. Forecasting Principles}
\begin{itemize}
    \item Use information that is \textbf{recurring, frequent, and reliable}.
    \item Avoid unnecessary complexity — more detailed models do not always improve accuracy.
    \item Focus on material line items.
\end{itemize}

\paragraph{4. Forecasting Approaches}
\begin{table}[h!]
\centering
\caption*{Exhibit 2: Forecasting Approaches Overview}
\begin{tabular}{|l|p{5cm}|p{6cm}|}
\hline
\textbf{Approach} & \textbf{Concept} & \textbf{Best Used For / Limitations} \\
\hline
Historical Results & Extrapolate from past trends & Mature, noncyclical firms; less useful for cyclical or transforming firms \\
\hline
Base Rate Convergence & Forecast converges to long-run average (industry or GDP growth) & Established industries with stable structure; not suitable for cyclical or disruptive sectors \\
\hline
Management Guidance & Use company-provided forecasts or ranges & Useful if management has strong forecasting record; must check for bias \\
\hline
Analyst Discretion & Model-driven or judgmental projections & Suitable for unique, cyclical, or transitioning firms \\
\hline
\end{tabular}
\end{table}

\paragraph{5. Forecast Horizon}
\begin{itemize}
    \item Depends on:
    \begin{itemize}
        \item Investor time horizon.
        \item Industry cyclicality.
        \item Duration of strategic changes.
    \end{itemize}
    \item For cyclical firms: include at least one full business cycle.
\end{itemize}

\bigskip

\subsection*{LOS 47.b: Forecasting Company Revenues}

\paragraph{1. Top-Down Approach}
\begin{itemize}
    \item Begins with macroeconomic variables (GDP growth, inflation, market size).
    \item Often relates \textbf{company sales to nominal GDP}.
    \[
    \text{Revenue Growth} = g_{\text{GDP}} \times (1 + \text{Premium/Discount})
    \]
    \item Example: GDP growth = 5\%, firm grows 20\% faster $\Rightarrow$ 5\% × 1.20 = 6\% forecasted sales growth.
    \item Alternative: Revenue = Market Share × Industry Sales
\end{itemize}

\paragraph{Example (Market Share Forecast):}
\[
\begin{aligned}
\text{Current Industry Sales} &= 100 \text{m GBP}, \quad \text{Firm Share} = 12\% \Rightarrow 12\text{m GBP} \\
\text{Next Year Industry Sales} &= 104 \text{m GBP}, \quad \text{Expected Share} = 13\% \\
\Rightarrow \text{Forecast Revenue} &= 0.13 \times 104 = 13.52 \text{m GBP (↑12.7\%)}
\end{aligned}
\]

\paragraph{2. Bottom-Up Approach}
\begin{itemize}
    \item Start from company or segment level.
    \item Key revenue drivers:
    \begin{enumerate}
        \item \textbf{Price × Quantity (P×Q)} forecasts.
        \item \textbf{Product line / segment analysis.}
        \item \textbf{Capacity-based:} forecast sales per store/factory + new capacity.
        \item \textbf{Yield-based:} e.g., banks → revenue = interest spread × balance sheet exposure.
    \end{enumerate}
\end{itemize}

\paragraph{3. Combined Approach}
\begin{itemize}
    \item Integrate top-down and bottom-up for cross-validation.
    \item Example: GDP-based forecast vs. company capacity forecast — identify inconsistencies.
\end{itemize}

\paragraph{4. Recurring vs. Nonrecurring Revenue Items}
\begin{itemize}
    \item Exclude one-time or unusual items from recurring forecasts.
    \item \textbf{Nonrecurring items disclosed by management:} e.g., large one-off orders.
    \item \textbf{Undisclosed nonrecurring items:} Analyst identifies via context (e.g., temporary COVID-19 spike in e-commerce).
\end{itemize}

\paragraph{5. Forecasting Risks and Scenarios}
\begin{itemize}
    \item Key risks: competition, inflation, technology, business cycle.
    \item Use \textbf{scenario analysis} to test forecast sensitivity.
\end{itemize}

\bigskip

\subsection*{LOS 47.c: Forecasting Operating Expenses and Working Capital}

\paragraph{1. Cost of Sales (COGS) and Gross Margins}
\[
\text{Forecast COGS} = (1 - \text{Gross Margin}) \times \text{Forecast Revenue}
\]
\begin{itemize}
    \item Examine trends in \textbf{input prices} and \textbf{output prices}.
    \item Analyze gross margin vs. competitors for reasonableness.
    \item Consider hedging policies on input costs (e.g., oil hedges for airlines).
\end{itemize}

\paragraph{Example: Price and Cost Effect on Gross Margin}
\[
\text{Initial: } \frac{\text{COGS}}{\text{Sales}} = 25\%, \quad \text{Input Costs Double}, \quad \text{Price ↑ 25\%}
\]
\[
\Rightarrow \text{New COGS/Sales} = 40\%, \ \text{Gross Margin ↓ from 75\% to 60\%}
\]
Even if gross profit in absolute terms remains constant, margin compresses.

\paragraph{2. SG\&A Expenses}
\begin{itemize}
    \item Include fixed (R\&D, HQ costs) and variable (sales commissions, logistics) components.
    \item Model as:
    \[
    \text{SG\&A}_{t+1} = \text{SG\&A}_t \times (1 + g_{\text{inflation}} + g_{\text{volume}})
    \]
    \item Segment reporting often limited — analysts use average operating margin by segment.
\end{itemize}

\paragraph{3. Working Capital Forecasting}
\begin{itemize}
    \item Working capital = A/R + Inventory – A/P.
    \item Use activity ratios to project components:
\end{itemize}

\begin{table}[h!]
\centering
\caption*{Exhibit 3: Working Capital Forecasting Ratios}
\begin{tabular}{|l|l|l|}
\hline
\textbf{Item} & \textbf{Ratio} & \textbf{Forecast Formula} \\
\hline
Accounts Receivable & DSO = 365 / Receivables Turnover & A/R = DSO × (Revenue / 365) \\
\hline
Inventory & DOH = 365 / Inventory Turnover & Inventory = DOH × (COGS / 365) \\
\hline
Accounts Payable & DPO = 365 / Payables Turnover & A/P = DPO × (COGS / 365) \\
\hline
\end{tabular}
\end{table}

\paragraph{4. Analytical Insight}
\begin{itemize}
    \item A longer cash conversion cycle $\Rightarrow$ more external financing required.
    \item Changes in DSO, DOH, and DPO indicate operational efficiency trends.
\end{itemize}

\bigskip

\subsection*{LOS 47.d: Forecasting Capital Investments and Capital Structure}

\paragraph{1. Capital Investments}
\begin{itemize}
    \item Derived from:
    \begin{itemize}
        \item \textbf{Cash Flow Statement:} CapEx acquisitions/disposals.
        \item \textbf{Income Statement:} Depreciation \& Amortization (D\&A).
    \end{itemize}
    \item Separate into:
    \begin{itemize}
        \item \textbf{Maintenance CapEx:} Sustain operations — linked to depreciation, inflation-adjusted.
        \item \textbf{Growth CapEx:} Supports expansion; based on strategic plans or capacity goals.
    \end{itemize}
    \item \textbf{Depreciation Forecast:}
    \[
    \text{Depreciation} = \frac{\text{Net PPE}}{\text{Useful Life of Assets}}
    \]
\end{itemize}

\paragraph{2. Capital Structure Forecast}
\begin{itemize}
    \item Assess based on \textbf{leverage ratios:}
    \[
    \text{Debt-to-Assets}, \quad \text{Debt-to-Equity}, \quad \text{Interest Coverage}
    \]
    \item Incorporate:
    \begin{itemize}
        \item Debt required for CapEx plans.
        \item Management’s target leverage or debt covenant limits.
    \end{itemize}
\end{itemize}

\begin{table}[h!]
\centering
\caption*{Exhibit 4: Capital Forecasting Summary}
\begin{tabular}{|l|l|l|}
\hline
\textbf{Item} & \textbf{Primary Source} & \textbf{Forecast Driver} \\
\hline
Maintenance CapEx & Historical D\&A, inflation rate & Replacement cost of assets \\
\hline
Growth CapEx & Strategic plan, capacity expansion & Revenue growth assumptions \\
\hline
Capital Structure & Balance sheet, debt covenants & Target leverage, financing needs \\
\hline
\end{tabular}
\end{table}

\bigskip

\subsection*{LOS 47.e: Scenario Analysis in Forecasting}

\paragraph{1. Definition and Purpose}
\begin{itemize}
    \item Scenario analysis tests sensitivity of forecasts to alternative assumptions.
    \item Goal: Estimate a \textbf{range of potential outcomes} instead of a single point estimate.
\end{itemize}

\paragraph{2. Types of Scenarios}
\begin{itemize}
    \item \textbf{Base Case:} Expected macro and company assumptions.
    \item \textbf{Upside Case:} Strong economy, rising margins, favorable competition.
    \item \textbf{Downside Case:} Recession, cost increases, product failure.
\end{itemize}

\paragraph{3. Example – EPS Sensitivity Table}
\begin{table}[h!]
\centering
\caption*{Exhibit 5: Scenario Analysis Example}
\begin{tabular}{|l|c|c|c|}
\hline
\textbf{Scenario} & \textbf{Revenue Growth} & \textbf{Operating Margin} & \textbf{Forecast EPS} \\
\hline
Base Case & +5\% & 15\% & €2.50 \\
\hline
Upside Case & +8\% & 17\% & €3.10 \\
\hline
Downside Case & 0\% & 12\% & €1.85 \\
\hline
\end{tabular}
\end{table}

\paragraph{4. Application}
\begin{itemize}
    \item Helps evaluate risk-adjusted valuation.
    \item Identifies key value drivers (revenues, margins, leverage).
    \item Often combined with sensitivity analysis and Monte Carlo simulations.
\end{itemize}

\bigskip

\subsection*{Key Takeaways Summary}

\begin{table}[h!]
\centering
\caption*{Exhibit 6: Module 47.1 Summary Overview}
\begin{tabular}{|l|p{11cm}|}
\hline
\textbf{Learning Objective} & \textbf{Key Insights} \\
\hline
LOS 47.a & Forecasting requires balancing simplicity, data quality, and key drivers. Use historical, base rate, management, or discretionary methods. \\
\hline
LOS 47.b & Revenue forecasts may be top-down (macro-based) or bottom-up (micro-based). Combine both for consistency checks. Exclude nonrecurring items. \\
\hline
LOS 47.c & Forecast COGS, SG\&A, and working capital using ratios and driver relationships. Gross margin and DSO/DOH/DPO are central. \\
\hline
LOS 47.d & CapEx divided into maintenance vs. growth. Capital structure forecasts rely on leverage ratios and target policies. \\
\hline
LOS 47.e & Scenario analysis produces a range of forecasts, revealing sensitivity to macro or firm-specific assumptions. \\
\hline
\end{tabular}
\end{table}

\paragraph{Overall Insight:}
\begin{itemize}
    \item Sound forecasting integrates macro context, firm-level drivers, and management guidance.
    \item Analysts should maintain transparency, consistency, and robustness across assumptions.
    \item Scenario analysis ensures realistic valuation under multiple economic conditions.
\end{itemize}


\section*{Module 48.1: Dividends, Splits, and Repurchases}

\subsection*{LOS 48.a: Determining Overvaluation or Undervaluation}

\paragraph{1. Concept of Intrinsic Value}
\begin{itemize}
    \item \textbf{Intrinsic (Fundamental) Value:} Rational value based on all known characteristics of a security.
    \item \textbf{Market Price:} Observed trading price reflecting current supply and demand.
    \item Comparison of both determines if a stock is:
    \[
    \text{Market Price} \gtrless \text{Intrinsic Value} \Rightarrow \text{Overvalued / Undervalued / Fairly Valued.}
    \]
\end{itemize}

\paragraph{2. Decision Criteria}
\begin{itemize}
    \item The larger the \textbf{percentage deviation} between price and intrinsic value → greater conviction to act.
    \item The higher the analyst’s \textbf{confidence in model and inputs} → greater willingness to trade.
    \item The \textbf{sensitivity of the model} to inputs must be checked (e.g., small growth change may change valuation outcome).
    \item Market prices are generally reliable; mispricing requires a plausible rationale.
    \item The investor must believe \textbf{price convergence} toward intrinsic value will occur within investment horizon.
\end{itemize}

\paragraph{3. Example:}
\[
\begin{aligned}
\text{Intrinsic Value (DCF)} &= \$60 \\
\text{Market Price} &= \$48 \\
\Rightarrow\quad \text{Undervalued by } \frac{60-48}{48} &= 25\%
\end{aligned}
\]
If analyst confidence is high and a catalyst is expected, \textbf{a buy recommendation} is justified.

\bigskip

\subsection*{LOS 48.b: Major Categories of Equity Valuation Models}

\begin{table}[h!]
\centering
\caption*{Exhibit 1: Categories of Equity Valuation Models}
\small
\begin{tabularx}{\textwidth}{@{} p{2.6cm} p{3.6cm} >{\RaggedRight\arraybackslash}X @{}}
\toprule
\textbf{Model Type} & \textbf{Concept} & \textbf{Example / Key Inputs} \\
\midrule
\textbf{Discounted Cash Flow (DCF)} &
Intrinsic value = PV of expected future cash flows to equity holders &
\makecell[tl]{Dividend Discount Model (DDM):\\[4pt] \(V_0 = \dfrac{D_1}{r - g}\)\\[4pt]
Free Cash Flow to Equity (FCFE): PV of cash available 
\\ after reinvestment and debt service} \\
\addlinespace
\textbf{Multiplier / Market Multiple Models} &
Compare firm valuation ratios to peers or market benchmarks &
\makecell[tl]{\textbf{Type 1:} Price multiples: P/E, P/S, P/B, P/CF\\
\textbf{Type 2:} Enterprise multiples: EV/EBITDA, EV/Sales} \\
\addlinespace
\textbf{Asset-Based Models} &
Intrinsic value = Fair Value of Assets $-$ Liabilities $-$ Preferred Equity &
Useful for asset-heavy firms; adjust book values to market (e.g., real estate, financials) \\
\bottomrule
\end{tabularx}
\end{table}

\paragraph{2. Key Notes}
\begin{itemize}
    \item Analysts often use \textbf{multiple models} to derive a valuation range.
    \item DCF models emphasize long-term fundamentals.
    \item Multiple models rely on \textbf{market comparables}.
    \item Asset-based models less relevant for firms with large intangibles (tech, pharma).
\end{itemize}

\bigskip

\subsection*{LOS 48.c: Types of Dividends, Splits, and Repurchases}

\paragraph{1. Cash Dividends}
\begin{itemize}
    \item \textbf{Regular Dividends:} Paid periodically (e.g., quarterly). Signal stability.
    \item \textbf{Special (Extra) Dividends:} One-time distributions due to exceptional profits.
    \item Common in cyclical industries (e.g., autos, mining).
\end{itemize}

\paragraph{Example:}
If firm earns unusually high profit from commodity boom, may pay special dividend while maintaining normal payout ratio in future.

\paragraph{2. Stock Dividends}
\begin{itemize}
    \item Paid in the form of additional shares rather than cash.
    \item Increases number of shares outstanding, reduces per-share price proportionally.
    \item Total equity and shareholder wealth remain unchanged.
    \[
    \text{Example: } 20\% \text{ stock dividend} \Rightarrow 100 \text{ shares} \to 120 \text{ shares.}
    \]
\end{itemize}

\paragraph{3. Stock Splits}
\begin{itemize}
    \item Each share is split into multiple shares.
    \item Price adjusts downward to maintain total value.
    \[
    \text{Example: } 3{:}1 \text{ split, } P_{\text{before}} = \$90 \Rightarrow P_{\text{after}} = \$30.
    \]
    \item Used to improve trading liquidity or make shares more affordable.
\end{itemize}

\paragraph{4. Reverse Stock Splits}
\begin{itemize}
    \item Reduces number of shares; price increases proportionally.
    \[
    \text{Example: } 1{:}5 \text{ reverse split, } 100 \text{ shares at } \$2 \Rightarrow 20 \text{ shares at } \$10.
    \]
    \item Often done to avoid delisting or improve stock image.
\end{itemize}

\paragraph{5. Share Repurchases (Buybacks)}
\begin{itemize}
    \item Company repurchases its own shares from the market.
    \item \textbf{Motives:}
    \begin{itemize}
        \item Alternative to dividends (tax-efficient if capital gains taxed less than dividends).
        \item Support share price or offset dilution (e.g., from employee stock options).
        \item Signal undervaluation.
    \end{itemize}
    \item Reduces shares outstanding $\Rightarrow$ increases EPS and ROE mechanically.
\end{itemize}

\paragraph{6. Comparison Summary}
\begin{table}[h!]
\centering
\caption*{Exhibit 2: Summary of Dividend and Repurchase Types}
\begin{tabular}{|l|p{4cm}|p{4cm}|p{4cm}|}
\hline
\textbf{Type} & \textbf{Mechanics} & \textbf{Effect on Share Price / Equity} & \textbf{Example / Note} \\
\hline
Regular Dividend & Periodic cash payment & Stock price drops by approx. dividend amount on ex-date & Quarterly payout (e.g., Coca-Cola) \\
\hline
Special Dividend & One-time cash payout & Temporary price increase; no ongoing commitment & Cyclical firms in good years \\
\hline
Stock Dividend & Additional shares & Price per share decreases, total equity unchanged & 20\% stock dividend \\
\hline
Stock Split & More shares at lower price & No wealth change; improves liquidity & 3-for-1 split (Apple, 2020) \\
\hline
Reverse Split & Fewer shares at higher price & No wealth change; often cosmetic & 1-for-5 to avoid delisting \\
\hline
Share Repurchase & Company buys back shares & EPS and ROE increase; equity reduced & Often substitutes for dividend \\
\hline
\end{tabular}
\end{table}

\bigskip

\subsection*{LOS 48.d: Dividend Payment Chronology}

\paragraph{1. Key Dates and Process}
\begin{table}[h!]
\centering
\caption*{Exhibit 3: Dividend Payment Timeline}
\begin{tabular}{|l|p{4cm}|p{7cm}|}
\hline
\textbf{Date} & \textbf{Definition} & \textbf{Notes / Example} \\
\hline
\textbf{Declaration Date} & Board approves dividend, announces amount and schedule & Creates legal obligation to pay; e.g., “\$1.00 per share declared on June 1.” \\
\hline
\textbf{Ex-Dividend Date} & First day new buyers \textit{do not} receive the dividend & Usually 1–2 business days before record date; stock price drops by ≈ dividend amount. \\
\hline
\textbf{Record Date} & Company determines which shareholders receive dividend & Only holders of record as of this date receive payment. \\
\hline
\textbf{Payment Date} & Dividend is actually paid (cash or transfer) & Typically weeks after record date. \\
\hline
\end{tabular}
\end{table}

\paragraph{2. Example: Dividend Timeline}
\[
\begin{aligned}
\text{Declaration Date} &= \text{June 1, 2025} \\
\text{Ex-Dividend Date} &= \text{June 14, 2025} \\
\text{Record Date} &= \text{June 16, 2025} \\
\text{Payment Date} &= \text{July 1, 2025}
\end{aligned}
\]

\paragraph{3. Ex-Dividend Price Adjustment Example}
\[
\text{Stock Price Before Ex-Date} = \$25, \quad \text{Dividend} = \$1.00
\]
\[
\Rightarrow \text{Price After Ex-Date} \approx 25 - 1 = \$24
\]
\begin{itemize}
    \item Buyer before ex-date receives share + dividend.
    \item Buyer on/after ex-date receives only share.
\end{itemize}

\bigskip

\subsection*{Key Takeaways Summary}

\begin{table}[h!]
\centering
\caption*{Exhibit 4: Module 48.1 Summary Overview}
\begin{tabular}{|l|p{11cm}|}
\hline
\textbf{LOS} & \textbf{Core Insights} \\
\hline
LOS 48.a & Compare intrinsic vs. market value to determine over/undervaluation. Confidence in model and convergence expectation essential. \\
\hline
LOS 48.b & Major valuation models: DCF (DDM, FCFE), Multiplier (P/E, EV/EBITDA), and Asset-based. Each has unique use cases. \\
\hline
LOS 48.c & Firms distribute profits via dividends (cash/stock), splits, or repurchases. Economic wealth unaffected by splits/dividends, but signaling and tax effects matter. \\
\hline
LOS 48.d & Dividend chronology: Declaration → Ex-Date → Record → Payment. Ex-date price drop ≈ dividend amount. \\
\hline
\end{tabular}
\end{table}

\paragraph{Overall Insight:}
\begin{itemize}
    \item \textbf{Valuation} aligns intrinsic and market prices to identify mispricing.
    \item \textbf{Dividends and buybacks} are both methods of cash distribution, differing in taxation, flexibility, and signaling.
    \item Understanding \textbf{timing and type of distributions} is essential for forecasting shareholder returns and price behavior.
\end{itemize}

\section*{Module 48.2: Dividend Discount Models}

\subsection*{LOS 48.e: Rationale for Present Value Models (DDM \& FCFE)}

\paragraph{1. Conceptual Basis}
\begin{itemize}
    \item The \textbf{intrinsic value} of a stock equals the present value (PV) of future cash flows to shareholders.
    \item PV models assume investors are rational and value equity as the discounted stream of dividends or free cash flows to equity (FCFE).
\end{itemize}

\paragraph{2. Dividend Discount Model (DDM)}
\begin{itemize}
    \item The DDM expresses stock value as:
    \[
    V_0 = \sum_{t=1}^{\infty} \frac{D_t}{(1+k_e)^t}
    \]
    where \( D_t \) = dividend in period \( t \), and \( k_e \) = required return on equity.
    \item Applicable to firms that pay regular, predictable dividends.
\end{itemize}

\paragraph{3. Free Cash Flow to Equity (FCFE) Model}
\begin{itemize}
    \item FCFE represents the cash available to shareholders after operating, investing, and financing needs:
    \[
    \text{FCFE} = \text{NI} + \text{Depreciation} - \Delta WC - FCInv - \text{Debt Repayments} + \text{New Debt Issuance}
    \]
    or
    \[
    \text{FCFE} = \text{CFO} - FCInv + \text{Net Borrowing}
    \]
    \item FCFE models are particularly useful for firms:
    \begin{itemize}
        \item That do not pay dividends.
        \item Where dividends are not reflective of earning capacity.
    \end{itemize}
\end{itemize}

\paragraph{4. Required Return Estimation (CAPM)}
\[
k_e = R_f + \beta_i [E(R_m) - R_f]
\]
\begin{itemize}
    \item \( R_f \): risk-free rate (e.g., government bond).
    \item \( \beta_i \): stock’s systematic risk.
    \item \( E(R_m) - R_f \): market risk premium.
\end{itemize}

\paragraph{Alternative Approach:}
\begin{itemize}
    \item If firm debt is publicly traded: \( k_e = \text{Bond Yield} + \text{Equity Risk Premium} \)
    \item If not: add a larger premium to government yield.
\end{itemize}

\bigskip

\subsection*{LOS 48.g: Valuation of Preferred Stock}

\paragraph{1. Characteristics}
\begin{itemize}
    \item Fixed dividend, perpetual duration, non-callable and non-convertible (in this context).
    \item Essentially a perpetuity with fixed cash flow.
\end{itemize}

\paragraph{2. Formula}
\[
V_p = \frac{D_p}{k_p}
\]

\paragraph{Example: Preferred Stock Valuation}
\[
D_p = \$5, \quad k_p = 8\%
\]
\[
V_p = \frac{5}{0.08} = \$62.50
\]

\paragraph{Interpretation:}
\begin{itemize}
    \item Value equals the present value of perpetual dividends discounted at required return.
    \item Equivalent to DDM with zero growth (\( g = 0 \)).
\end{itemize}

\bigskip

\subsection*{LOS 48.h: Constant and Multistage Growth Dividend Discount Models}

\paragraph{1. Constant Growth (Gordon Growth) Model}
\begin{itemize}
    \item Assumes dividends grow perpetually at constant rate \( g_c \).
    \item Formula:
    \[
    V_0 = \frac{D_1}{k_e - g_c}
    \]
    \item Assumptions:
    \begin{enumerate}
        \item Dividends are relevant measure of shareholder wealth.
        \item Growth rate \( g_c \) and required return \( k_e \) remain constant.
        \item \( k_e > g_c \).
    \end{enumerate}
\end{itemize}

\paragraph{Example: Constant Growth Valuation}
\[
D_0 = 1.50, \quad g_c = 8\%, \quad k_e = 12\%
\]
\[
D_1 = D_0 (1 + g_c) = 1.62
\]
\[
V_0 = \frac{1.62}{0.12 - 0.08} = \$40.50
\]

\paragraph{2. Decomposition of Value due to Growth}
\[
V_{\text{no growth}} = \frac{D}{k_e} = \frac{1.50}{0.12} = 12.50
\]
\[
V_{\text{growth}} = 40.50 - 12.50 = 28.00
\]
→ Growth accounts for \(\approx 69\%\) of intrinsic value.

\paragraph{3. Sensitivity Insight}
\begin{itemize}
    \item As \((k_e - g_c)\) narrows, \(V_0\) increases sharply.
    \item Small changes in \(g_c\) or \(k_e\) cause large valuation shifts.
\end{itemize}

\paragraph{4. Estimating Dividend Growth Rate}
\begin{enumerate}
    \item Historical average dividend growth.
    \item Median industry dividend growth.
    \item \textbf{Sustainable Growth Rate:}
    \[
    g = (1 - \text{Payout Ratio}) \times ROE
    \]
    or
    \[
    g = \text{Retention Rate} \times ROE
    \]
\end{enumerate}

\paragraph{Example: Sustainable Growth}
\[
\text{Payout} = 25\%, \quad ROE = 21\%
\Rightarrow g = (1 - 0.25)(0.21) = 15.75\%
\]
\textbf{Note:} Long-term growth usually limited to single digits; check realism.

\bigskip

\subsection*{LOS 48.h (continued): Firms with No Current Dividends}

\paragraph{1. Approach}
\begin{itemize}
    \item Estimate when dividends will start.
    \item Use constant growth model to value dividends starting from that year, then discount back.
\end{itemize}

\paragraph{Example: No Current Dividend}
\[
E_4 = 1.64, \quad \text{Payout} = 50\%, \quad g = 5\%, \quad k_e = 10\%
\]
\[
D_4 = 0.5 \times 1.64 = 0.82
\]
\[
V_3 = \frac{D_4}{k_e - g} = \frac{0.82}{0.10 - 0.05} = 16.40
\]
\[
V_0 = \frac{16.40}{(1.10)^3} = 12.32
\]

\paragraph{Interpretation:}
Stock value today equals PV of future dividends beginning in Year 4.

\bigskip

\subsection*{LOS 48.h (continued): Multistage Dividend Discount Model}

\paragraph{1. Rationale}
\begin{itemize}
    \item Firms often experience a temporary high-growth phase before stabilizing.
    \item The multistage model incorporates both phases:
    \[
    V_0 = \sum_{t=1}^{n} \frac{D_t}{(1+k_e)^t} + \frac{P_n}{(1+k_e)^n}
    \]
    where
    \[
    P_n = \frac{D_{n+1}}{k_e - g_c}
    \]
\end{itemize}

\paragraph{2. Example: Two-Stage Growth}
\[
D_0 = 1.00, \quad g^* = 15\% \text{ for 2 years}, \quad g_c = 5\%, \quad k_e = 11\%
\]

\[
D_1 = 1.00(1.15) = 1.15, \quad D_2 = 1.15(1.15) = 1.32
\]
\[
P_1 = \frac{D_2}{k_e - g_c} = \frac{1.32}{0.11 - 0.05} = 22.00
\]
\[
V_0 = \frac{D_1 + P_1}{(1.11)} = \frac{1.15 + 22.00}{1.11} = 20.86
\]
\[
V_0 = 20.86 \text{ (approx.)}
\]

\paragraph{3. Key Steps for Multistage Model}
\begin{enumerate}
    \item Determine discount rate \( k_e \).
    \item Estimate dividend growth rate during high-growth period \( g^* \).
    \item Project dividends during high-growth phase.
    \item Estimate stable growth rate \( g_c \) after transition.
    \item Compute terminal value \( P_n = D_{n+1}/(k_e - g_c) \).
    \item Discount all cash flows to present value.
\end{enumerate}

\paragraph{4. Common Mistakes}
\begin{itemize}
    \item Forgetting to discount the terminal value \( P_n \) correctly.
    \item Using wrong time index (e.g., discounting over too many periods).
\end{itemize}

\bigskip

\subsection*{LOS 48.i: Applicability of Constant vs. Multistage Models}

\paragraph{1. Constant Growth (Gordon) Model Appropriate for:}
\begin{itemize}
    \item Mature, stable, non-cyclical firms with predictable dividend policies.
    \item Examples: Utilities, consumer staples.
\end{itemize}

\paragraph{2. Multistage Growth Model Appropriate for:}
\begin{itemize}
    \item Firms with temporary high or low growth transitioning to stability.
    \item Firms in cyclical or rapidly evolving industries.
    \item Can include 2- or 3-stage variants:
    \begin{itemize}
        \item \textbf{Stage 1:} High growth.
        \item \textbf{Stage 2:} Transition (moderating growth).
        \item \textbf{Stage 3:} Constant, stable growth.
    \end{itemize}
\end{itemize}

\paragraph{3. When to Use FCFE Models Instead}
\begin{itemize}
    \item When dividends are non-existent or unpredictable.
    \item When payout policy does not reflect earnings capacity.
    \item FCFE model applies if earnings and reinvestment assumptions are available.
\end{itemize}

\bigskip

\subsection*{Key Takeaways Summary}

\begin{table}[h!]
\centering
\caption*{Exhibit 1: Module 48.2 Summary Overview}
\begin{tabular}{|l|p{11cm}|}
\hline
\textbf{LOS} & \textbf{Core Insights} \\
\hline
48.e & PV models (DDM, FCFE) value equity by discounting future cash flows to shareholders. FCFE reflects potential dividends. \\
\hline
48.g & Preferred stock = fixed dividend perpetuity: \(V_p = D_p / k_p\). \\
\hline
48.h & Gordon Growth Model: \(V_0 = D_1 / (k_e - g_c)\). Use for stable, dividend-paying firms. Growth and discount rate differences highly sensitive. \\
\hline
48.h (2-stage) & Multistage DDM combines finite high-growth and infinite stable-growth phases. Requires accurate growth transition assumptions. \\
\hline
48.i & Constant-growth model fits mature firms; multistage fits firms in transition; FCFE models suit non-dividend-paying firms. \\
\hline
\end{tabular}
\end{table}

\paragraph{Overall Insight:}
\begin{itemize}
    \item Dividend and FCFE models both rest on the time value of money principle.
    \item Choice of model depends on dividend policy, growth stage, and data availability.
    \item Key valuation drivers: \(k_e\), \(g\), payout ratio, and ROE.
\end{itemize}

\section*{Module 48.3: Relative Valuation Measures}

\subsection*{LOS 48.j: Rationale for Using Price Multiples}

\paragraph{1. Motivation for Price Multiples}
\begin{itemize}
    \item Because DDMs are highly sensitive to input assumptions, analysts often use \textbf{price multiples} as simpler, market-based valuation tools.
    \item A \textbf{price multiple} compares a company’s market price to a financial metric (earnings, sales, book value, cash flow).
\end{itemize}

\paragraph{2. Common Multiples}
\begin{itemize}
    \item Price-to-Earnings (P/E)
    \item Price-to-Sales (P/S)
    \item Price-to-Book Value (P/B)
    \item Price-to-Cash Flow (P/CF)
\end{itemize}

\paragraph{3. Advantages}
\begin{itemize}
    \item Easily calculated and widely published.
    \item Useful for \textbf{time-series (historical)} and \textbf{cross-sectional (peer)} comparison.
    \item Empirical evidence: low multiples often predict higher future returns.
\end{itemize}

\paragraph{4. Drawbacks}
\begin{itemize}
    \item Historical (“trailing”) data may not reflect future performance.
    \item Forecast (“forward”) data may vary by analyst and cause inconsistency.
    \item Requires consistency across firms in metric definitions.
\end{itemize}

\paragraph{5. Types of Multiple-Based Valuation}
\begin{itemize}
    \item \textbf{Multiples Based on Comparables (Market-Based):}
    \begin{itemize}
        \item Compare firm’s multiples to peers or industry average.
        \item Reflects “law of one price”: similar assets should trade at similar multiples.
    \end{itemize}
    \item \textbf{Multiples Based on Fundamentals:}
    \begin{itemize}
        \item Derived from valuation models (e.g., DDM, FCFE) → \textbf{justified} multiples.
        \item Reflect what multiples \textit{should be} based on fundamentals ($D_1$, $E_1$, $k$, $g$).
    \end{itemize}
\end{itemize}

\bigskip

\subsection*{LOS 48.k: Calculation and Interpretation of Price Multiples}

\paragraph{1. Common Price Multiples and Formulas}
\begin{table}[h!]
\centering
\caption*{Exhibit 1: Key Price Multiples}
\begin{tabular}{|l|l|p{7cm}|}
\hline
\textbf{Multiple} & \textbf{Formula} & \textbf{Interpretation} \\
\hline
P/E & $P / \text{EPS}$ & Indicates how much investors pay per \$1 of earnings. \\
\hline
P/S & $P / (\text{Sales per share})$ & Useful when earnings are negative; less affected by accounting policies. \\
\hline
P/B & $P / (\text{Book value per share})$ & Indicates market valuation relative to accounting book value; $>1$ implies future profitability. \\
\hline
P/CF & $P / (\text{Cash flow per share})$ & Cash-based measure less prone to earnings manipulation. \\
\hline
\end{tabular}
\end{table}

\paragraph{2. Fundamental (Justified) P/E from Gordon Growth Model}
\[
\begin{aligned}
V_0 &= \frac{D_1}{k - g} \\
\Rightarrow \frac{P_0}{E_1} &= \frac{D_1 / E_1}{k - g} = \frac{\text{Payout Ratio}}{k - g}
\]
\]

\paragraph{Example:}
\[
\text{Payout Ratio} = 30\%, \quad k = 13\%, \quad g = 6\%
\]
\[
\text{Justified P/E} = \frac{0.3}{0.13 - 0.06} = 4.3
\]
\textbf{Interpretation:}
\begin{itemize}
    \item If actual P/E = 8 → Overvalued (market > justified value).
    \item If actual P/E = 2 → Undervalued (market < justified value).
\end{itemize}

\paragraph{3. P/E Sensitivity:}
\begin{itemize}
    \item Increases with higher payout ratio or higher growth.
    \item Decreases with higher required return $k$.
    \item \textbf{Dividend Displacement of Earnings:} Higher payout $\Rightarrow$ lower retention $\Rightarrow$ lower sustainable growth.
\end{itemize}

\paragraph{Example: Holt Industries}
\begin{itemize}
    \item Higher payout and higher growth → higher P/E justified.
    \item Higher leverage → higher risk → higher $k$ → lower P/E justified.
\end{itemize}

\bigskip

\paragraph{4. Multiples Based on Comparables (Comps)}
\begin{itemize}
    \item \textbf{Law of One Price:} Similar assets should trade at similar multiples.
    \item \textbf{Benchmarks:}
    \begin{itemize}
        \item Historical average (time-series).
        \item Industry or peer group (cross-sectional).
    \end{itemize}
    \item Must ensure comparability:
    \begin{itemize}
        \item Similar size, leverage, growth, accounting policies.
        \item Cyclicality: use P/S instead of P/E when earnings fluctuate.
    \end{itemize}
\end{itemize}

\paragraph{Example: Renee’s Bakery}
\begin{itemize}
    \item P/E higher, but P/S, P/B, P/CF lower than peers.
    \item Suggests potential undervaluation.
    \item Downward trend in multiples → possible undervaluation vs. own history.
\end{itemize}

\paragraph{Example Table:}
\begin{table}[h!]
\centering
\caption*{Exhibit 2: Example Multiples Comparison (Renee’s Bakery vs. Industry)}
\begin{tabular}{|l|c|c|}
\hline
\textbf{Ratio} & \textbf{Renee’s Bakery} & \textbf{Industry Avg.} \\
\hline
P/E & 15 & 12 \\
P/CF & 5 & 7 \\
P/S & 1.0 & 1.5 \\
P/B & 1.2 & 1.6 \\
\hline
\end{tabular}
\end{table}
\textbf{Interpretation:}  
Lower multiples except P/E → potentially undervalued; check if earnings depressed.

\bigskip

\subsection*{LOS 48.l: Enterprise Value (EV) Multiples}

\paragraph{1. Concept}
\[
EV = \text{Market Value of Equity} + \text{Market Value of Debt} + \text{Preferred Equity} - \text{Cash}
\]
\begin{itemize}
    \item Represents total firm value, not just equity value.
    \item Adjusts for differences in capital structure.
    \item Used to compare firms regardless of leverage.
\end{itemize}

\paragraph{2. Common Ratios}
\begin{table}[h!]
\centering
\caption*{Exhibit 3: Common EV Multiples}
\begin{tabular}{|l|l|p{7cm}|}
\hline
\textbf{Multiple} & \textbf{Formula} & \textbf{Use / Advantage} \\
\hline
EV/EBITDA & $\frac{EV}{EBITDA}$ & Most common; ignores capital structure, tax, and non-cash items. \\
\hline
EV/EBIT & $\frac{EV}{EBIT}$ & Compares total firm value to operating earnings. \\
\hline
EV/Sales & $\frac{EV}{Sales}$ & Useful for early-stage or negative-earnings firms. \\
\hline
\end{tabular}
\end{table}

\paragraph{Example: EV/EBITDA}
\[
\text{Equity Value} = 40 \times 200,000 = 8,000,000
\]
\[
\text{Debt} = 600,000 + 1,200,000 = 1,800,000, \quad \text{Cash} = 250,000
\]
\[
EV = 8,000,000 + 1,800,000 - 250,000 = 9,550,000
\]
\[
\text{EBITDA} = 1,000,000 \Rightarrow \text{EV/EBITDA} = 9.55
\]
\textbf{Interpretation:}
\begin{itemize}
    \item If industry average = 11 → undervalued.
    \item If industry average = 8 → overvalued.
\end{itemize}

\paragraph{3. Notes:}
\begin{itemize}
    \item Market value of debt may need estimation (use bond market data or book values).
    \item Use EV multiples when:
    \begin{itemize}
        \item Comparing firms with different leverage.
        \item Earnings are negative (P/E not meaningful).
    \end{itemize}
\end{itemize}

\bigskip

\subsection*{LOS 48.m: Asset-Based Valuation Models}

\paragraph{1. Concept}
\[
V_{Equity} = V_{Assets} - V_{Liabilities}
\]
\begin{itemize}
    \item Based on market (fair) value of assets and liabilities.
    \item Reflects “floor” or liquidation value.
\end{itemize}

\paragraph{2. When Appropriate}
\begin{itemize}
    \item Firms with primarily tangible, short-term, or marketable assets.
    \item Financial institutions, real estate, or natural resource firms.
    \item Liquidation or distressed valuations.
\end{itemize}

\paragraph{3. Example: Williams Optical}
\[
\text{Assets (adjusted)} = 10{,}000 + 20{,}000 + 50{,}000 + 120{,}000(1.20) = 224{,}000
\]
\[
\text{Liabilities} = 5{,}000 + 30{,}000 + 45{,}000 = 80{,}000
\]
\[
\text{Equity Value} = 224{,}000 - 80{,}000 = 144{,}000
\]
\[
\text{Shares} = 2{,}000 \Rightarrow \text{Per Share Value} = 72
\]

\paragraph{4. Limitations}
\begin{itemize}
    \item Market values of assets/liabilities may be hard to obtain.
    \item Intangibles (brand, customer base) not captured.
    \item Inflation and accounting conventions distort book values.
\end{itemize}

\bigskip

\subsection*{LOS 48.f: Advantages and Disadvantages of Valuation Models}

\begin{table}[h!]
\centering
\caption*{Exhibit 4: Comparative Overview of Valuation Models}
\begin{tabular}{|p{3cm}|p{6cm}|p{6cm}|}
\hline
\textbf{Model Type} & \textbf{Advantages} & \textbf{Disadvantages} \\
\hline
\textbf{Discounted Cash Flow (DCF)} &
- Based on fundamental PV theory. \newline
- Widely accepted and theoretically sound. &
- Requires subjective input estimates. \newline
- Highly sensitive to small input changes. \\
\hline
\textbf{Price Multiples (Comparables)} &
- Empirical support for return prediction. \newline
- Easy to compute, available, intuitive. \newline
- Allows time-series and peer comparison. &
- May mix firms of different growth/risk. \newline
- Affected by accounting differences. \newline
- Cyclical distortions possible. \newline
- Negative denominators invalid (e.g., negative EPS). \\
\hline
\textbf{Price Multiples (Fundamentals)} &
- Grounded in valuation models. \newline
- Highlights relation to payout, growth, and return. &
- Extremely sensitive to $(k - g)$ inputs. \\
\hline
\textbf{EV Multiples} &
- Independent of capital structure. \newline
- Useful when earnings are negative. &
- Requires market value of debt (often estimated). \newline
- EBITDA may include non-cash items. \\
\hline
\textbf{Asset-Based Models} &
- Provide floor values. \newline
- Reliable for tangible-asset firms or liquidation. &
- Hard to estimate market values. \newline
- Poor for intangible-heavy firms. \newline
- Misleading under inflation. \\
\hline
\end{tabular}
\end{table}

\bigskip

\subsection*{Key Takeaways Summary}

\begin{itemize}
    \item \textbf{Price Multiples} provide quick, market-based assessments of relative valuation.
    \item \textbf{Fundamental Multiples} (e.g., justified P/E) link valuation directly to growth, payout, and required return.
    \item \textbf{Enterprise Value Multiples} adjust for leverage and are robust when earnings are negative.
    \item \textbf{Asset-Based Models} give conservative, floor-level estimates.
    \item Always apply multiple methods and test sensitivity across scenarios.
\end{itemize}


\end{document}
