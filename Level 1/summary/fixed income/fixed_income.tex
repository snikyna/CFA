\documentclass[12pt]{article}
\usepackage{amsmath}
\usepackage{geometry}
\usepackage{graphicx} % for including images and figures
\usepackage{booktabs}
\usepackage{caption}
\usepackage{titlesec}
\usepackage{float}
\usepackage{makecell}
\usepackage{tabularx}
\usepackage{enumitem}
\usepackage[utf8]{inputenc}
\usepackage{textcomp}
\usepackage{adjustbox}  % put in preamble


\geometry{margin=1in}

\title{Equity Investment}
\author{}
\date{}

\begin{document}
\maketitle

\section*{Module 49.1: Fixed-Income Instrument Features}

\subsection*{LOS 49.a: Features of a Fixed-Income Security}

\paragraph{1. Definition}
\begin{itemize}
    \item Fixed-income instruments represent \textbf{debt investments} in which investors lend capital to an issuer in exchange for promised interest (coupon) and repayment of principal (par value).
    \item Two major forms:
    \begin{enumerate}
        \item \textbf{Loans:} Private, non-tradable debt agreements.
        \item \textbf{Bonds:} Standardized, tradable securities.
    \end{enumerate}
\end{itemize}

\paragraph{2. Key Bond Features}
\begin{table}[h!]
\centering
\caption*{Exhibit 1: Core Bond Features and Examples}
\begin{tabular}{|p{3cm}|p{5cm}|p{6cm}|}
\hline
\textbf{Feature} & \textbf{Definition} & \textbf{Example / Notes} \\
\hline
\textbf{Issuer} & Entity issuing the bond & Governments (sovereign, local), corporations, supranationals (e.g., IMF, World Bank), or special purpose entities (SPEs). \\
\hline
\textbf{Maturity / Tenor} & Date or period until final payment. Bonds $\leq 1$ year = \textbf{Money Market}; $> 1$ year = \textbf{Capital Market}; perpetual bonds = no maturity. & A 10-year corporate bond matures in 2035. A perpetual bond (e.g., UK Consol) never matures. \\
\hline
\textbf{Principal (Par)} & Amount repaid at maturity; base for coupon calculation. & Typically \$1,000 or €1,000. Mortgage loans may amortize principal over time. \\
\hline
\textbf{Coupon Rate \& Frequency} & Annual interest rate applied to par; defines periodic payments. & 5\% annual coupon on \$1,000 = \$50/year. Semiannual = \$25 every 6 months. \\
\hline
\textbf{Seniority} & Ranking of debt repayment priority in liquidation. & Senior debt repaid before subordinated (junior) debt. \\
\hline
\textbf{Contingency Provisions} & Embedded options giving rights to issuer or bondholder. & Callable, putable, or convertible bonds. \\
\hline
\end{tabular}
\end{table}

\paragraph{3. Coupon Structures}
\begin{itemize}
    \item \textbf{Fixed-rate bond:} Constant coupon based on stated rate.
    \item \textbf{Floating-rate note (FRN):} Coupon = Market Reference Rate (MRR) + fixed margin (spread).
    \[
    \text{Coupon Payment} = (\text{MRR} + \text{Margin}) \times \text{Par Value}
    \]
    Example: If MRR = 4\%, Margin = 50 bps = 0.5\%, Coupon = 4.5\%.
    \item \textbf{Zero-coupon bond (pure discount):} No coupon; sold below par, pays par at maturity.
    \[
    \text{Price} = \frac{\text{Par}}{(1 + y)^n}
    \]
    Example: 10-year \$1,000 bond at 7\% $\Rightarrow P \approx \$508$.
\end{itemize}

\paragraph{4. Yield and Price Relationship}
\[
P \downarrow \Rightarrow Y \uparrow, \quad P \uparrow \Rightarrow Y \downarrow
\]
\begin{itemize}
    \item Fixed cash flows $\Rightarrow$ inverse price-yield relationship.
    \item Yield reflects expected return based on current price and future cash flows.
\end{itemize}

\paragraph{5. Yield Curves}
\begin{itemize}
    \item Graph plotting yield (\%) vs. maturity.
    \item \textbf{Normal curve:} Upward sloping (long-term yields > short-term yields).
    \item \textbf{Inverted curve:} Downward sloping (long-term yields < short-term yields).
    \item \textbf{Benchmark:} Government yield curves used as reference for credit spreads.
\end{itemize}

\paragraph{Example: Credit Spread}
\[
\text{Corporate Yield} - \text{Gov’t Yield} = 6\% - 5\% = 1\% \text{ spread}
\]
A higher spread indicates higher credit or liquidity risk.

\bigskip

\subsection*{LOS 49.b: Bond Indenture and Covenants}

\paragraph{1. Definition}
\begin{itemize}
    \item The \textbf{bond indenture} (or trust deed) is the legal contract between the bond issuer and bondholders.
    \item Specifies all rights, obligations, and restrictions:
    \begin{itemize}
        \item Bond features (coupon, maturity, par).
        \item Sources of repayment.
        \item Collateral and guarantees.
        \item Covenants (affirmative and negative).
    \end{itemize}
\end{itemize}

\paragraph{2. Sources of Repayment}
\begin{table}[h!]
\centering
\caption*{Exhibit 2: Sources of Bond Repayment by Issuer Type}
\begin{tabular}{|p{3cm}|p{5cm}|p{6cm}|}
\hline
\textbf{Issuer Type} & \textbf{Repayment Source} & \textbf{Example} \\
\hline
\textbf{Sovereign Government} & Tax revenues and monetary powers (printing money). & U.S. Treasury or German Bunds. \\
\hline
\textbf{Local Government} & Local taxes or project revenues (e.g., tolls). & Municipal bonds repaid from property taxes. \\
\hline
\textbf{Corporations} & Operating cash flows; possibly backed by collateral. & Secured (mortgage-backed) vs. unsecured (debenture). \\
\hline
\textbf{Special Purpose Entity (SPE)} & Cash flows from underlying assets (ABS/MBS). & Mortgage-backed securities. \\
\hline
\end{tabular}
\end{table}

\paragraph{3. Secured vs. Unsecured Debt}
\begin{itemize}
    \item \textbf{Secured Bonds:} Have legal claim (lien/pledge) on specific assets (collateral).
    \item \textbf{Unsecured Bonds (Debentures):} Backed only by issuer’s overall creditworthiness.
\end{itemize}

\paragraph{4. Types of Covenants}
\begin{table}[h!]
\centering
\caption*{Exhibit 3: Bond Covenant Comparison}
\begin{tabular}{|p{4cm}|p{4cm}|p{7cm}|}
\hline
\textbf{Covenant Type} & \textbf{Definition / Purpose} & \textbf{Examples} \\
\hline
\textbf{Affirmative (Positive)} & Specify actions issuer \textbf{must perform}. & 
\begin{itemize}
\item Provide audited financial statements.
\item Maintain insurance and compliance with laws.
\item Use proceeds for declared purpose.
\item \textbf{Cross-default clause:} Default on any other debt = default here.
\item \textbf{Pari passu clause:} Equal ranking with other senior debt.
\end{itemize} \\
\hline
\textbf{Negative (Restrictive)} & Restrict actions that could increase credit risk. & 
\begin{itemize}
\item Limitations on additional debt.
\item Restrictions on asset sales, leasebacks, and collateral pledges.
\item \textbf{Negative pledge:} Cannot issue more senior secured debt.
\item Dividend or share buyback limits (based on leverage or coverage ratios).
\end{itemize} \\
\hline
\end{tabular}
\end{table}

\paragraph{5. Incurrence Test Example}
\[
\text{Restriction: New debt or dividend allowed only if } \frac{\text{Debt}}{\text{EBITDA}} < 3.5
\]
Protects bondholders by limiting financial risk.

\paragraph{6. Covenant Balance}
\begin{itemize}
    \item \textbf{Goal:} Protect bondholders without overly restricting issuer flexibility.
    \item Too lenient → increased default risk.
    \item Too restrictive → limits operational agility.
\end{itemize}

\bigskip

\subsection*{LOS Integration: Yield Curve \& Indenture Context}

\paragraph{1. Yield Curve Insights}
\begin{itemize}
    \item Provides benchmark yields across maturities.
    \item Corporate spreads over government curve reflect perceived credit risk.
\end{itemize}

\paragraph{2. Credit Structure Summary}
\begin{table}[h!]
\centering
\caption*{Exhibit 4: Bond Hierarchy and Claim Priority}
\begin{tabular}{|l|p{9cm}|}
\hline
\textbf{Type} & \textbf{Priority / Risk Level} \\
\hline
Senior Secured Debt & First claim on specific assets (lowest credit risk). \\
\hline
Senior Unsecured Debt & Claim on issuer’s general assets. \\
\hline
Subordinated (Junior) Debt & Paid after senior claims; higher yield required. \\
\hline
Equity (Common / Preferred) & Residual claim only; highest risk, highest expected return. \\
\hline
\end{tabular}
\end{table}

\bigskip

\subsection*{Example Summary Calculations}

\paragraph{1. Fixed Coupon Example}
\[
\text{Coupon Payment} = 0.05 \times \$1{,}000 = \$50 \text{ per year}
\]
Semiannual = \$25 every 6 months.

\paragraph{2. Zero-Coupon Example}
\[
P_0 = \frac{1{,}000}{(1.07)^{10}} = \$508
\]
Investor earns entire yield from price appreciation at maturity.

\paragraph{3. Floating-Rate Example}
\[
\text{Coupon Rate} = \text{MRR} + 1.25\%
\]
If MRR = 3.75\% → Coupon = 5.00\% → Payment = 0.05 × \$1,000 = \$50.

\bigskip

\subsection*{Key Takeaways Summary}

\begin{table}[h!]
\centering
\caption*{Exhibit 5: Module 49.1 Summary Overview}
\begin{tabular}{|l|p{11cm}|}
\hline
\textbf{LOS} & \textbf{Core Insights} \\
\hline
49.a & Fixed-income securities define issuer, maturity, par, coupon, seniority, and options. Prices and yields move inversely. Yield curves plot yield vs. maturity. \\
\hline
49.b & The bond indenture defines repayment sources, collateral, and covenants. Affirmative covenants require issuer actions; negative covenants restrict risk-increasing actions. Proper covenant balance protects bondholders. \\
\hline
\end{tabular}
\end{table}

\paragraph{Conceptual Summary:}
\begin{itemize}
    \item Bonds = legally binding debt with predictable cash flows.
    \item Issuer’s credit quality, maturity, and covenants define risk-return tradeoff.
    \item Yield curves and spreads reflect market perception of credit risk.
    \item The bond indenture legally protects investors through covenants and collateral provisions.
\end{itemize}

\section*{Module 50.1: Fixed-Income Cash Flows and Types}

\subsection*{LOS 50.a: Common Cash Flow Structures and Contingency Provisions}

\paragraph{1. Overview}
\begin{itemize}
    \item Fixed-income instruments differ by \textbf{cash flow timing, composition, and embedded options}.
    \item Main structures:
    \begin{enumerate}
        \item Bullet (single repayment at maturity)
        \item Amortizing (principal repaid over time)
        \item Partially amortizing (balloon payment)
        \item Sinking fund or structured (ABS/MBS waterfalls)
    \end{enumerate}
\end{itemize}

\paragraph{2. Bullet Structure}
\begin{itemize}
    \item Interest (\textbf{coupon}) paid periodically; principal repaid fully at maturity.
    \item Example: \$1{,}000 par, 5\% annual coupon, 5-year maturity.
\end{itemize}

\[
\text{Coupon Payment} = 0.05 \times 1{,}000 = \$50 \text{ annually}
\]

\begin{table}[h!]
\centering
\caption*{Exhibit 1: Bullet Bond Cash Flow Schedule}
\begin{tabular}{|c|c|c|}
\hline
\textbf{Year} & \textbf{Coupon (\$)} & \textbf{Principal (\$)} \\
\hline
1–4 & 50 & 0 \\
\hline
5 & 50 & 1,000 \\
\hline
\end{tabular}
\end{table}

\paragraph{3. Fully Amortizing Loan}
\begin{itemize}
    \item Each payment includes both interest and principal.
    \item Final payment reduces principal to zero.
    \item Example parameters: $N = 5, I/Y = 5\%, PV = 1,000, FV = 0$
    \[
    \text{PMT} = 230.97
    \]
\end{itemize}

\begin{table}[h!]
\centering
\caption*{Exhibit 2: Fully Amortizing Bond Example (\$1,000, 5\%, 5 years)}
\begin{tabular}{|c|c|c|c|}
\hline
\textbf{Year} & \textbf{Payment} & \textbf{Interest} & \textbf{Principal Repay.} \\
\hline
1 & 230.97 & 50.00 & 180.97 \\
2 & 230.97 & 40.95 & 190.02 \\
3 & 230.97 & 31.45 & 199.52 \\
4 & 230.97 & 21.48 & 209.49 \\
5 & 230.97 & 10.47 & 220.50 \\
\hline
\end{tabular}
\end{table}

\paragraph{4. Partially Amortizing (Balloon Payment)}
\begin{itemize}
    \item Partial principal repayment with final “balloon” at maturity.
    \item Example parameters: $N=5$, $I/Y=5\%$, $PV=1,000$, $FV=-200$
    \[
    \text{PMT} = 194.78
    \]
\end{itemize}

\paragraph{5. Sinking Fund and Waterfall Structures}
\begin{itemize}
    \item \textbf{Sinking Fund:} Issuer repays fixed portions of principal over time.
    \[
    \text{Example: } \$300\,\text{m issue, redeem \$20m annually from Year 6–20.}
    \]
    \begin{itemize}
        \item \textbf{Advantage:} Lower credit risk (gradual repayment).
        \item \textbf{Disadvantage:} Reinvestment risk if rates fall (cash returned early).
    \end{itemize}
    \item \textbf{Waterfall (ABS/MBS):} Cash flows distributed hierarchically by tranche.
    \begin{itemize}
        \item Senior tranches repaid first.
        \item Junior tranches absorb default losses.
    \end{itemize}
\end{itemize}

\paragraph{6. Coupon Structures Overview}
\begin{table}[h!]
\centering
\caption*{Exhibit 3: Coupon Structures}
\begin{tabular}{|l|p{5cm}|p{6cm}|}
\hline
\textbf{Type} & \textbf{Definition} & \textbf{Example / Note} \\
\hline
\textbf{Fixed-Rate} & Coupon fixed at issue date. & Most common. Predictable cash flows. \\
\hline
\textbf{Floating-Rate (FRN)} & Coupon = MRR + Margin (spread). & Quarterly coupon: $(2.3\% + 0.75\%)/4 = 0.7625\%$ of par. \\
\hline
\textbf{Step-Up} & Coupon increases by schedule. & Protects investors if rates rise. \\
\hline
\textbf{Credit-Linked} & Coupon adjusts with credit rating. & MRR + 2.5\% $\to$ MRR + 3\% if leverage > 3×. \\
\hline
\textbf{PIK (Payment-in-Kind)} & Interest paid by issuing more debt instead of cash. & Used by highly leveraged issuers; high yield. \\
\hline
\textbf{Green Bonds} & Coupon rises if environmental targets not met. & Example: CO$_2$ reduction failure $\Rightarrow$ coupon +25bps. \\
\hline
\textbf{Inflation-Linked} & Coupon or principal linked to CPI or similar. & Protects real returns; e.g., U.S. TIPS. \\
\hline
\end{tabular}
\end{table}

\paragraph{7. Inflation-Linked Bonds}
\begin{itemize}
    \item \textbf{Interest-Indexed:} Coupon adjusts for inflation; principal fixed.
    \item \textbf{Capital-Indexed (TIPS):} Coupon fixed; principal adjusts by inflation rate.
    \item Example: 3\% coupon, semiannual, inflation +1\%:
    \[
    \text{New Principal} = 1{,}000(1.01) = 1{,}010, \quad \text{Semiannual Coupon} = 1.5\%\times1{,}010 = 15.15
    \]
    \item Protects real purchasing power: real rate = 3\%.
\end{itemize}

\paragraph{8. Deferred and Zero-Coupon Bonds}
\begin{itemize}
    \item \textbf{Zero-Coupon:} Single payment at maturity; trades below par.
    \item \textbf{Deferred Coupon:} No interest initially; coupons begin later.
    \item Used by issuers expecting delayed revenue (e.g., project financing).
\end{itemize}

\paragraph{9. Embedded Contingency Provisions (Embedded Options)}
\begin{itemize}
    \item \textbf{Callable Bonds:} Option to issuer.
    \item \textbf{Putable Bonds:} Option to investor.
    \item \textbf{Convertible Bonds:} Option to investor to convert to shares.
    \item Bonds without options = \textbf{straight bonds}.
\end{itemize}

\subsubsection*{Callable Bonds}
\begin{itemize}
    \item Issuer may redeem early at preset price (call price).
    \item \textbf{Example:}
    \begin{itemize}
        \item Callable after 5 years at 102\% of par.
        \item Price cap in market $\Rightarrow$ higher yield, lower price.
    \end{itemize}
    \item \textbf{Call Risk:} Reinvestment risk for investors when rates fall.
\end{itemize}

\[
\text{If yield falls from 6\% to 4\%, issuer can refinance at lower cost.}
\]

\subsubsection*{Putable Bonds}
\begin{itemize}
    \item Bondholder can sell bond back at par before maturity.
    \item Protects investor when yields rise or credit worsens.
    \item \textbf{Put Option Value} $\Rightarrow$ higher price, lower yield vs. straight bond.
\end{itemize}

\subsubsection*{Convertible Bonds}
\begin{itemize}
    \item Bondholder may convert bond into equity.
    \item \textbf{Formulas:}
    \[
    \text{Conversion Ratio} = \frac{\text{Par Value}}{\text{Conversion Price}}
    \]
    \[
    \text{Conversion Value} = (\text{Share Price}) \times (\text{Conversion Ratio})
    \]
    \item \textbf{Example:} \$1{,}000 bond, conversion price = \$40:
    \[
    \text{Ratio} = 25 \text{ shares}, \quad \text{Share Price} = 50 \Rightarrow V_c = 25 \times 50 = 1{,}250
    \]
\end{itemize}

\subsubsection*{Warrants}
\begin{itemize}
    \item Detachable right to buy stock at a fixed price.
    \item Example: Warrant strike \$40 → value if share price $>40$.
\end{itemize}

\subsubsection*{Contingent Convertible Bonds (CoCos)}
\begin{itemize}
    \item Convert to equity automatically if trigger (e.g., bank’s capital ratio) is breached.
    \item Used by banks to meet regulatory capital requirements.
    \item Benefit issuer: automatic recapitalization without new issue.
\end{itemize}

\bigskip

\subsection*{LOS 50.b: Legal, Regulatory, and Tax Considerations}

\paragraph{1. Bond Market Classifications}
\begin{table}[h!]
\centering
\caption*{Exhibit 4: Bond Issuance Categories}
\begin{tabular}{|l|p{5cm}|p{6cm}|}
\hline
\textbf{Type} & \textbf{Definition} & \textbf{Example} \\
\hline
\textbf{Domestic Bond} & Issued and traded in issuer’s home market. & Deutsche Bank issuing euro bonds in Germany. \\
\hline
\textbf{Foreign Bond} & Issued by foreign entity in domestic market. & Toyota issues USD bonds in the U.S. (Yankee bond). \\
\hline
\textbf{Eurobond} & Issued outside jurisdiction of any single country, any currency. & Chinese firm issues yen-denominated Euroyen bonds in London. \\
\hline
\textbf{Global Bond} & Trades in both domestic and Eurobond markets. & Apple global USD bond traded in U.S. and London. \\
\hline
\end{tabular}
\end{table}

\paragraph{2. Eurobond Features}
\begin{itemize}
    \item Typically less regulated; may be bearer or registered form.
    \item \textbf{Naming convention:} by currency (Eurodollar, Euroyen, etc.).
    \item Currency denomination primarily determines yield differences due to domestic interest rates.
\end{itemize}

\paragraph{3. Islamic (Sukuk) Bonds}
\begin{itemize}
    \item Must comply with Sharia (no interest).
    \item Investors earn \textbf{rent income} from underlying tangible assets.
    \item Funds must be used for permissible (halal) purposes.
\end{itemize}

\paragraph{4. Taxation Considerations}
\begin{itemize}
    \item \textbf{Interest Income:} Taxed as ordinary income.
    \item \textbf{Municipal Bonds (U.S.):} Usually exempt from federal and in-state taxes.
    \item \textbf{Capital Gains:} Taxed upon sale; lower rate for long-term holdings.
    \item \textbf{Original Issue Discount (OID):}
    \begin{itemize}
        \item Zero-coupon or deep-discount bonds.
        \item Accreted interest taxed annually even with no cash payment.
    \end{itemize}
    \item \textbf{Premium Bonds:} Amortized premium can reduce taxable coupon income.
\end{itemize}

\bigskip

\subsection*{Key Takeaways Summary}

\begin{table}[h!]
\centering
\caption*{Exhibit 5: Module 50.1 Summary Overview}
\begin{tabular}{|l|p{11cm}|}
\hline
\textbf{LOS} & \textbf{Core Insights} \\
\hline
50.a & Bonds differ by cash flow structure (bullet, amortizing, sinking fund) and coupon design (fixed, floating, step-up, PIK, inflation-linked). Embedded options (call, put, convertibility) redistribute value between issuer and investor. \\
\hline
50.b & Legal/regulatory frameworks define issuance categories (domestic, foreign, Eurobond, global). Tax regimes affect after-tax returns. Sukuk and OID bonds have specialized treatment. \\
\hline
\end{tabular}
\end{table}

\paragraph{Conceptual Summary:}
\begin{itemize}
    \item \textbf{Structure defines timing of cash flows.}
    \item \textbf{Coupon design} affects sensitivity to rates, inflation, and credit events.
    \item \textbf{Embedded options} shift risk–reward between issuer and holder.
    \item \textbf{Legal jurisdiction} impacts regulation, taxation, and yield levels.
\end{itemize}

\section*{Module 51.1: Fixed-Income Issuance and Trading}

\subsection*{LOS 51.a: Fixed-Income Market Segments and Participants}

\paragraph{1. Global Bond Market Segmentation}
Bond markets can be segmented by several criteria:
\begin{itemize}
    \item \textbf{Type of Issuer:} Governments, Corporates, and Special Purpose Entities (SPEs) issuing Asset-Backed Securities (ABSs).
    \item \textbf{Credit Quality:} Investment-grade vs. High-yield (speculative/junk).
    \item \textbf{Maturity:} Short-term (≤ 1 year), Intermediate (1–10 years), Long-term (> 10 years).
    \item \textbf{Other Classifications:} Currency, Geography, and ESG features.
\end{itemize}

\paragraph{2. Issuer Categories}
\begin{table}[h!]
\centering
\caption*{Exhibit 1: Issuer Types and Features}
\begin{tabular}{|p{3cm}|p{4.5cm}|p{7cm}|}
\hline
\textbf{Issuer Type} & \textbf{Description} & \textbf{Example / Notes} \\
\hline
\textbf{Sovereign Governments} & National governments issuing Treasuries or sovereign bonds. & U.S. Treasury, German Bunds; lowest credit risk. \\
\hline
\textbf{Non-Sovereign Governments} & Regional or municipal authorities. & U.S. municipal bonds, German Länder bonds. \\
\hline
\textbf{Corporates} & Private sector issuers for funding operations, acquisitions, or investment. & Investment-grade: Apple Inc. / High-yield: Ford Motor Co. \\
\hline
\textbf{Special Purpose Entities (SPEs)} & Issue Asset-Backed or Mortgage-Backed Securities (ABS/MBS). & Cash flows come from underlying financial assets (e.g., auto loans, mortgages). \\
\hline
\end{tabular}
\end{table}

\paragraph{3. Credit Quality Classifications}
\begin{table}[h!]
\centering
\caption*{Exhibit 2: Credit Rating Tiers}
\begin{tabular}{|p{3cm}|p{3cm}|p{3cm}|p{3cm}|}
\hline
\textbf{Category} & \textbf{S\&P} & \textbf{Moody's} & \textbf{Comment} \\
\hline
Investment Grade & AAA–BBB– & Aaa–Baa3 & Suitable for institutional portfolios. \\
\hline
High Yield / Speculative & BB+ and below & Ba1 and below & Also called “junk bonds.” \\
\hline
Fallen Angels & Downgraded from IG to HY & — & Example: BBB → BB downgrade. \\
\hline
\end{tabular}
\end{table}

\paragraph{4. Maturity Segments}
\begin{itemize}
    \item \textbf{Money Market:} ≤ 1 year (e.g., Treasury bills, commercial paper, repos).
    \item \textbf{Intermediate-Term:} 1–10 years (e.g., notes, mid-term corporate debt).
    \item \textbf{Long-Term:} > 10 years (e.g., Treasury bonds, corporate debentures, mortgages).
\end{itemize}

\paragraph{5. Example: Corporate Issuance Across Maturities}
\begin{itemize}
    \item Short-term → \textbf{Commercial Paper} for seasonal working capital.
    \item Medium-term → \textbf{Bank Syndicated Loans} for equipment or expansion.
    \item Long-term → \textbf{Bond Issuance} for capital investment projects.
\end{itemize}

\paragraph{6. Investor Positioning Along the Credit/Maturity Spectrum}
\begin{table}[h!]
\centering
\caption*{Exhibit 3: Investor Positioning by Risk and Maturity}
\begin{tabular}{|p{4cm}|p{5cm}|p{6cm}|}
\hline
\textbf{Investor Type} & \textbf{Preferred Segment} & \textbf{Rationale / Objective} \\
\hline
\textbf{Pension Funds / Insurers} & Long-term, investment grade & Match long-term liabilities (pensions, insurance claims). \\
\hline
\textbf{Corporations} & Short-term, high credit quality & Manage working capital, excess liquidity (e.g., CP, repos). \\
\hline
\textbf{Central Banks} & Intermediate-term sovereign bonds & Use for monetary policy and liquidity control. \\
\hline
\textbf{Bond Funds / ETFs} & Intermediate, investment grade (ex-Treasuries) & Diversified fixed-income portfolios for stable returns. \\
\hline
\textbf{Asset Managers / Hedge Funds} & Intermediate, high-yield or distressed & Seek higher yield via credit risk. \\
\hline
\textbf{Banks / Financial Intermediaries} & Treasuries across maturity spectrum & Manage interest rate and liquidity risk. \\
\hline
\end{tabular}
\end{table}

\bigskip
\subsection*{LOS 51.b: Types of Fixed-Income Indexes}

\paragraph{1. Key Distinctions from Equity Indexes}
\begin{enumerate}
    \item \textbf{More Constituents:} A single issuer can have hundreds of bond issues.
    \item \textbf{Higher Turnover:} Due to frequent issuance, maturities, and redemptions.
    \item \textbf{Sector Weights Shift:} Government and corporate issuance patterns affect index weights over time.
\end{enumerate}

\paragraph{2. Aggregate and Sector Indexes}
\begin{table}[h!]
\centering
\caption*{Exhibit 4: Major Fixed-Income Index Types}
\begin{tabular}{|l|p{5cm}|p{6cm}|}
\hline
\textbf{Index Type} & \textbf{Description} & \textbf{Examples} \\
\hline
\textbf{Aggregate Index} & Broad-based; includes multiple sectors, maturities, and currencies. & Bloomberg Barclays Aggregate Index (excludes high-yield/unrated). \\
\hline
\textbf{Geographic Index} & Bonds from specific countries or regions. & ICE BofA U.S. Corporate Index, Euro Aggregate Index. \\
\hline
\textbf{Credit Quality Index} & Classified by rating (IG vs. HY). & Bloomberg U.S. High Yield Index. \\
\hline
\textbf{Sector Index} & Focused on specific issuers (corporate, sovereign, ABS). & JPMorgan Emerging Market Bond Index Plus. \\
\hline
\textbf{ESG / Sustainable Index} & Exclude certain industries; minimum ESG score. & Bloomberg Barclays MSCI Euro Corporate Sustainable SRI Index. \\
\hline
\end{tabular}
\end{table}

\paragraph{3. ESG-Integrated Index Example}
\begin{itemize}
    \item Excludes firms in non-sustainable sectors (e.g., alcohol, coal).
    \item Requires \textbf{min credit rating = BBB– (Baa3)} and \textbf{min ESG rating = BBB}.
\end{itemize}

\paragraph{4. Index as Benchmark}
\begin{itemize}
    \item Benchmark must align with portfolio exposure:
    \begin{itemize}
        \item Sector, maturity, currency, and credit quality.
        \item Example: Emerging-market fund → JPM EMBI+ benchmark.
    \end{itemize}
\end{itemize}

\bigskip
\subsection*{LOS 51.c: Primary vs. Secondary Markets}

\paragraph{1. Primary Market}
\begin{itemize}
    \item \textbf{Definition:} Issuance of new bonds → issuer receives new capital.
    \item \textbf{Two Forms:}
    \begin{enumerate}
        \item \textbf{Public Offering:} Registered with regulator; open to all investors.
        \item \textbf{Private Placement:} Sold to select investors (e.g., institutions).
    \end{enumerate}
\end{itemize}

\paragraph{2. Issuance Methods}
\begin{table}[h!]
\centering
\caption*{Exhibit 5: Primary Market Issuance Types}
\begin{tabular}{|l|p{5cm}|p{6cm}|}
\hline
\textbf{Method} & \textbf{Description} & \textbf{Example / Implication} \\
\hline
\textbf{Underwritten Offering} & Investment bank guarantees sale price to issuer. & Used by large investment-grade corporates; lower risk for issuer. \\
\hline
\textbf{Best-Efforts Offering} & Bank sells bonds on commission basis (no guarantee). & Used for smaller or high-yield issues; higher issuance risk. \\
\hline
\textbf{Shelf Registration} & Bonds pre-approved by regulator, issued over time. & Flexible funding for frequent issuers. \\
\hline
\textbf{Auction (Government)} & Bonds sold to bidders in competitive/noncompetitive auctions. & Common for Treasuries and sovereign debt. \\
\hline
\end{tabular}
\end{table}

\paragraph{3. Example: Debut vs. Repeat Issuer}
\begin{itemize}
    \item \textbf{Debut Issue:} Requires marketing roadshows; time-intensive.
    \item \textbf{Repeat Issue:} For frequent issuers with shelf registrations, issuance may complete in hours.
\end{itemize}

\paragraph{4. Secondary Market}
\begin{itemize}
    \item \textbf{Definition:} Trading of existing bonds among investors.
    \item \textbf{Structure:} Primarily \textbf{dealer (OTC) markets}, not exchanges.
    \item \textbf{Dealer Quote:} Bid (buy) and Ask (sell) prices → spread = dealer’s margin.
\end{itemize}

\paragraph{5. Bid–Ask Spread by Liquidity}
\begin{table}[h!]
\centering
\caption*{Exhibit 6: Liquidity and Spreads}
\begin{tabular}{|l|c|l|}
\hline
\textbf{Bond Type} & \textbf{Typical Spread (bps)} & \textbf{Liquidity Note} \\
\hline
On-the-run Sovereigns & $< 1$ & Very liquid; high trading volume. \\
\hline
Investment-Grade Corporate & 2–5 & Actively traded, lower risk. \\
\hline
High-Yield / Seasoned Bonds & 10–20+ & Less liquid; wider spreads. \\
\hline
Distressed Debt & 50–200+ & Illiquid; speculative trading. \\
\hline
\end{tabular}
\end{table}

\paragraph{6. Distressed Debt Market}
\begin{itemize}
    \item Bonds of issuers near or in bankruptcy.
    \item Bought by specialized investors (hedge or distress funds).
    \item Value from potential recovery or restructuring.
    \item Temporary surge in trading volume when distress emerges.
\end{itemize}

\bigskip
\subsection*{Key Takeaways Summary}

\begin{table}[h!]
\centering
\caption*{Exhibit 7: Summary of Module 51.1 LOS Concepts}
\begin{tabular}{|l|p{11cm}|}
\hline
\textbf{LOS} & \textbf{Core Insights} \\
\hline
51.a & Bond markets segmented by issuer, credit quality, and maturity. Issuers: sovereigns, corporates, SPEs. Investors allocate based on desired risk, return, and liability matching. \\
\hline
51.b & Fixed-income indexes differ from equities: more constituents, higher turnover, and sector shifts. Aggregate and ESG indexes cover global, regional, or sector exposures. \\
\hline
51.c & Primary market = issuance (underwritten, best-efforts, shelf, auction). Secondary market = OTC trading with bid–ask spreads by liquidity. Distressed debt involves high-risk, high-return repositioning. \\
\hline
\end{tabular}
\end{table}

\paragraph{Conceptual Summary:}
\begin{itemize}
    \item \textbf{Issuers:} Governments, Corporates, SPEs.
    \item \textbf{Investors:} Position along maturity and risk curve based on objectives.
    \item \textbf{Indexes:} Track market segments; used for benchmarking.
    \item \textbf{Primary Market:} Capital-raising via offerings or auctions.
    \item \textbf{Secondary Market:} Liquidity via dealer networks.
    \item \textbf{Liquidity Indicator:} Bid–ask spread $\uparrow$ → liquidity $\downarrow$.
\end{itemize}

\section*{Module 52.1: Fixed-Income Markets for Corporate Issuers}

\subsection*{LOS 52.a: Short-Term Funding Alternatives}

\paragraph{1. Overview}
Corporations and financial institutions use short-term financing to meet \textbf{working capital} needs, manage liquidity, or bridge to long-term funding.  
Funding can be classified as:
\begin{itemize}
    \item \textbf{Loan-based financing (bank credit lines)}
    \item \textbf{Security-based financing (commercial paper, repos, ABCP)}
\end{itemize}

\subsubsection*{A. Short-Term Funding for Nonfinancial Corporations}

\paragraph{2. External Loan Financing}
\begin{table}[h!]
\centering
\caption*{Exhibit 1: Bank Lines of Credit}
\begin{tabular}{|p{3cm}|p{5cm}|p{6cm}|}
\hline
\textbf{Type} & \textbf{Definition / Features} & \textbf{Example / Notes} \\
\hline
\textbf{Uncommitted Line of Credit} & Bank offers short-term funds at MRR + spread but may refuse to lend. Flexible but unreliable. & Suitable for stable firms maintaining deposits with bank. No commitment fees. \\
\hline
\textbf{Committed Line of Credit} & Bank commits funds for a defined period (e.g., 1 year). Borrower pays a commitment fee (\(\approx 50 \text{ bps}\)). & Reliable; subject to renewal risk. Syndicated among banks to spread risk. \\
\hline
\textbf{Revolving Line of Credit (Revolver)} & Multi-year facility allowing repeated borrowing and repayment. & More reliable; includes restrictive covenants (e.g., leverage ratios). \\
\hline
\end{tabular}
\end{table}

\paragraph{3. Secured Loans and Factoring}
\begin{itemize}
    \item \textbf{Secured Loan:} Borrower pledges collateral (e.g., receivables, inventory, fixed assets).
    \item \textbf{Factoring:} Sale of receivables to a “factor” at a discount → immediate cash inflow.
    \[
    \text{Discount} = \text{Face Value} - \text{Proceeds}
    \]
    \item Discount reflects customer credit quality + collection costs.
\end{itemize}

\paragraph{4. Security-Based Financing (Commercial Paper)}
\begin{itemize}
    \item \textbf{Commercial Paper (CP):} Unsecured, short-term promissory notes issued by highly rated firms.
    \item \textbf{Maturity:} Usually \(< 270\) days (often 30–90 days).
    \item \textbf{Cost:} Lower than bank loans (no intermediation margin).
    \item \textbf{Rollover Risk:} Risk that CP cannot be refinanced at maturity.
    \item \textbf{Mitigation:} Backup credit lines (\textit{liquidity enhancement lines}) with banks.
\end{itemize}

\paragraph{Example: CP Issuance}
\[
\text{Face Value} = \$10{,}000{,}000, \quad \text{Term} = 90 \text{ days}, \quad y = 5\%
\]
\[
P = 10{,}000{,}000 \times \left(1 - 0.05 \times \frac{90}{360}\right) = 9{,}875{,}000
\]
Issuer receives \$9.875m today, repays \$10m at maturity.

\paragraph{5. Eurocommercial Paper (ECP)}
\begin{itemize}
    \item Issued in international markets outside home jurisdiction.
    \item Typically denominated in major currencies (USD, EUR, GBP, JPY).
\end{itemize}

\subsubsection*{B. Short-Term Funding for Financial Institutions}

\paragraph{6. Deposit Funding}
\begin{table}[h!]
\centering
\caption*{Exhibit 2: Bank Deposit Types}
\begin{tabular}{|l|p{5cm}|p{6cm}|}
\hline
\textbf{Deposit Type} & \textbf{Description} & \textbf{Notes} \\
\hline
\textbf{Demand Deposits} & Checking accounts; withdrawable anytime; usually non-interest-bearing. & Provide transaction services and liquidity. \\
\hline
\textbf{Operational Deposits} & Large clients’ deposits used for clearing/custody. & Integral to daily operations. \\
\hline
\textbf{Savings Deposits / CDs} & Pay interest; fixed maturity. & \textbf{Nonnegotiable CD:} cannot be sold before maturity. \textbf{Negotiable CD:} tradable before maturity. \\
\hline
\end{tabular}
\end{table}

\paragraph{7. Interbank and Central Bank Funding}
\begin{itemize}
    \item \textbf{Interbank Funds:} Loans between banks (1 day–1 year) at interbank MRR (e.g., SOFR, €STR).
    \item \textbf{Central Bank Funds Market:} Banks lend/borrow reserves held at central bank.
    \item \textbf{Discount Window:} Lender of last resort facility; higher rate and regulatory scrutiny.
\end{itemize}

\paragraph{8. Asset-Backed Commercial Paper (ABCP)}
\begin{enumerate}
    \item Financial institution sells loans to a Special Purpose Entity (SPE).
    \item SPE issues short-term securities backed by these loans.
    \item Investors buy ABCP; SPE repays them with loan cash flows.
    \item Sponsor bank provides liquidity line or credit enhancement.
\end{enumerate}

\bigskip
\subsection*{LOS 52.b: Repurchase Agreements (Repos)}

\paragraph{1. Definition}
\begin{itemize}
    \item A \textbf{repo} = sale of securities with an agreement to repurchase later at higher price.
    \item Economically = short-term \textbf{collateralized loan}.
\end{itemize}

\paragraph{2. Key Formulas}
\[
\text{Repo Rate} = \frac{P_{\text{repurchase}} - P_{\text{initial}}}{P_{\text{initial}}} \times \frac{360}{\text{Days}}
\]
\[
\text{Initial Margin} = \frac{\text{Collateral Value}}{\text{Loan Amount}} \times 100\%
\]
\[
\text{Haircut} = 1 - \frac{1}{\text{Initial Margin}}
\]

\paragraph{Example: Repo Calculation}
\begin{itemize}
    \item Collateral Value = \$1{,}000{,}000
    \item Initial Margin = 103\%
    \item Repo Rate = 2\%, Term = 90 days
\end{itemize}
\[
\text{Purchase Price} = \frac{1{,}000{,}000}{1.03} = 970{,}874
\]
\[
\text{Repurchase Price} = 970{,}874[1 + 0.02(90/360)] = 975{,}728
\]
\[
\text{Haircut} = 1 - \frac{970{,}874}{1{,}000{,}000} = 2.91\%
\]

\paragraph{3. Margining}
\begin{itemize}
    \item If collateral value ↓ → lender requests \textbf{variation margin}.
    \item If collateral ↑ → borrower may request release of excess collateral.
\end{itemize}

\paragraph{4. Repo Types}
\begin{table}[h!]
\centering
\caption*{Exhibit 3: Types of Repo Agreements}
\begin{tabular}{|l|p{5cm}|p{6cm}|}
\hline
\textbf{Type} & \textbf{Description} & \textbf{Comment} \\
\hline
\textbf{Overnight Repo} & Maturity = 1 day. & Used for daily liquidity management. \\
\hline
\textbf{Term Repo} & Maturity > 1 day (up to 1 year). & Common durations: 1 week, 1 month, 3 months. \\
\hline
\textbf{General Collateral Repo} & Collateral = any eligible security. & Common for liquidity operations. \\
\hline
\textbf{Special Repo} & Specific hard-to-borrow security as collateral. & Repo rate may be very low or even negative. \\
\hline
\textbf{Tri-Party Repo} & Third-party custodian manages collateral. & Reduces settlement & margining risk. \\
\hline
\textbf{Bilateral Repo} & Two-party direct agreement. & Greater counterparty risk. \\
\hline
\end{tabular}
\end{table}

\paragraph{5. Repo Applications}
\begin{itemize}
    \item \textbf{Borrowers:} Finance trading positions.
    \item \textbf{Lenders:} Earn repo rate on excess liquidity.
    \item \textbf{Central Banks:} Monetary policy tool.
    \item \textbf{Short Sellers:} Borrow securities for short sale via reverse repo.
\end{itemize}

\paragraph{6. Repo Rate Drivers}
\begin{itemize}
    \item \textbf{Higher} when: term longer, collateral quality lower, undercollateralized.
    \item \textbf{Lower} when: collateral in high demand, overcollateralized, short term.
\end{itemize}

\paragraph{7. Risks}
\begin{itemize}
    \item Default risk (counterparty fails to repurchase).
    \item Collateral risk (decline in collateral value).
    \item Margining risk (incorrect/late margin).
    \item Legal risk (unenforceable contract).
    \item Netting/settlement risk (cross-contract payment failures).
\end{itemize}

\bigskip
\subsection*{LOS 52.c: Long-Term Funding—Investment Grade vs. High Yield}

\paragraph{1. General Observations}
\begin{itemize}
    \item \textbf{Normal Yield Curve:} Longer maturity → higher yield.
    \item Spread between short vs. long maturity is \textbf{wider for high-yield} issuers.
\end{itemize}

\paragraph{2. Comparison Table}
\begin{table}[h!]
\centering
\caption*{Exhibit 4: Investment Grade vs. High Yield Debt Characteristics}
\begin{tabular}{|l|p{5cm}|p{6cm}|}
\hline
\textbf{Feature} & \textbf{Investment Grade} & \textbf{High Yield} \\
\hline
\textbf{Default Risk} & Low; focus on downgrade risk. & High; focus on default and recovery rate. \\
\hline
\textbf{Credit Spread Component} & Small portion of total yield. & Dominant component of total yield. \\
\hline
\textbf{Covenants} & Limited, mainly restricting liens and sale/leaseback. & Extensive; include leverage and dividend restrictions. \\
\hline
\textbf{Collateral Requirement} & Typically unsecured. & Often secured or asset-backed. \\
\hline
\textbf{Maturity} & Broad range, often >10 years. & Shorter, typically ≤10 years. \\
\hline
\textbf{Standardization} & Highly standardized, benchmark issues. & Customized; less standardized, higher legal complexity. \\
\hline
\textbf{Flexibility} & Easy to refinance when rates fall. & Limited; higher refinancing cost. \\
\hline
\textbf{Optionality} & Few embedded options. & Often callable or prepayable. \\
\hline
\textbf{Return Profile} & Bond-like, predictable. & Equity-like; higher volatility. \\
\hline
\end{tabular}
\end{table}

\paragraph{3. Example: Yield Composition}
\[
\text{Yield}_{IG} = \text{Risk-free} + \text{Spread}_{IG} = 3\% + 1\% = 4\%
\]
\[
\text{Yield}_{HY} = \text{Risk-free} + \text{Spread}_{HY} = 3\% + 5\% = 8\%
\]

\paragraph{4. Callable High-Yield Bonds}
\begin{itemize}
    \item Allow issuer to redeem early when credit improves.
    \item Benefit to issuer: refinancing flexibility.
    \item Cost to investor: reinvestment and call risk → higher required yield.
\end{itemize}

\paragraph{5. Summary of Funding Strategy}
\begin{itemize}
    \item \textbf{Investment Grade:} 
        \begin{itemize}
            \item Long-term maturities.
            \item Broad investor base.
            \item Benchmark curve issuance.
        \end{itemize}
    \item \textbf{High Yield:}
        \begin{itemize}
            \item Shorter-term maturities.
            \item Heavier covenant and collateral structures.
            \item Issuance driven by market timing and refinancing needs.
        \end{itemize}
\end{itemize}

\bigskip
\subsection*{Key Takeaways Summary}

\begin{table}[h!]
\centering
\caption*{Exhibit 5: Module 52.1 Summary Overview}
\begin{tabular}{|l|p{11cm}|}
\hline
\textbf{LOS} & \textbf{Core Insights} \\
\hline
52.a & Nonfinancial firms: credit lines, secured loans, factoring, commercial paper. Financial institutions: deposits, CDs, interbank funds, central bank borrowing, and ABCP. \\
\hline
52.b & Repos are collateralized loans; repo rate determined by collateral quality, term, and market liquidity. Tri-party repos mitigate margining and settlement risk. \\
\hline
52.c & Investment-grade debt = longer maturities, fewer covenants, lower spreads. High-yield debt = shorter maturities, higher spreads, more covenants, often callable. \\
\hline
\end{tabular}
\end{table}

\paragraph{Conceptual Summary:}
\begin{itemize}
    \item \textbf{Short-term corporate funding} = flexibility vs. reliability trade-off.
    \item \textbf{Repos} provide efficient, low-cost, collateralized funding.
    \item \textbf{Credit quality} drives structure, maturity, and investor protections in long-term debt markets.
\end{itemize}

\section*{Module 53.1: Fixed-Income Markets for Government Issuers}

\subsection*{LOS 53.a: Funding Choices by Sovereign, Nonsovereign, Quasi-Government, and Supranational Entities}

\paragraph{1. Overview of Sovereign Debt}
\begin{itemize}
    \item \textbf{Definition:} Bonds issued by national governments to finance expenditures on public goods, infrastructure, and fiscal deficits.
    \item \textbf{Backing:} Full faith and credit of the government, supported by \textbf{tax-raising power}.
    \item \textbf{Issuer Profile:} Largest and typically highest credit quality issuer within domestic markets.
\end{itemize}

\paragraph{2. Accounting and Economic Perspective}
\begin{itemize}
    \item Government financial reports use \textbf{cash-based accounting} rather than accruals.
    \item Analysts should view governments through an \textbf{“economic balance sheet”}:
    \begin{itemize}
        \item \textbf{Implied Assets:} Future tax revenues.
        \item \textbf{Implied Liabilities:} Future social expenditures and obligations.
    \end{itemize}
\end{itemize}

\paragraph{3. Developed vs. Emerging Market Sovereigns}
\begin{table}[h!]
\centering
\caption*{Exhibit 1: Comparison of Developed and Emerging Market Sovereign Issuers}
\begin{tabular}{|l|p{6cm}|p{6cm}|}
\hline
\textbf{Aspect} & \textbf{Developed Market Sovereign} & \textbf{Emerging Market Sovereign} \\
\hline
Economic Base & Diversified, stable, transparent fiscal policy. & Concentrated, volatile, often commodity-dependent. \\
\hline
Currency & Reserve currency (e.g., USD, EUR, RMB). & Domestic or foreign reserve currency. \\
\hline
Debt Type & Domestic debt dominates. & Mix of domestic and external debt. \\
\hline
Investor Base & Predominantly domestic. & Often foreign due to limited domestic capacity. \\
\hline
Fiscal Stability & Predictable, credible. & Sensitive to commodity cycles and exchange rates. \\
\hline
Credit Rating & High (AAA to A). & Variable; may face downgrade or restructuring risk. \\
\hline
\end{tabular}
\end{table}

\paragraph{4. Types of Sovereign Debt}
\begin{itemize}
    \item \textbf{Domestic Debt:} Issued in local currency to domestic investors.
    \begin{itemize}
        \item May face \textit{liquidity} and \textit{convertibility} constraints.
    \end{itemize}
    \item \textbf{External Debt:} Issued to foreign investors, denominated in domestic or foreign (reserve) currency.
    \begin{itemize}
        \item Investors face \textbf{indirect currency risk} — repayment depends on the issuer’s ability to generate foreign currency.
    \end{itemize}
\end{itemize}

\paragraph{5. Fiscal Policy and Debt Issuance}
\begin{itemize}
    \item Government issues debt to finance fiscal deficits.
    \item Fiscal expansion (↑ spending or ↓ taxes) → higher debt issuance.
    \item Debt management policy targets stable maturity distribution between short- and long-term debt.
\end{itemize}

\paragraph{6. Ricardian Equivalence (Theoretical Reference)}
\begin{itemize}
    \item Suggests taxpayers are indifferent between:
    \[
    \text{Higher taxes now vs. issuing debt now (taxes later).}
    \]
    \item Holds only if:
    \begin{itemize}
        \item Rational expectations and no capital market frictions.
        \item Taxpayers can perfectly smooth consumption and intergenerational transfer.
    \end{itemize}
    \item In practice, these assumptions fail → governments must \textbf{manage maturity mix}.
\end{itemize}

\paragraph{7. Benefits of Debt Across Maturities}
\begin{itemize}
    \item \textbf{Short-term debt:} Safe, liquid, alternative to bank deposits. But entails \textbf{rollover risk}.
    \item \textbf{Long-term debt:} Provides benchmark yields, supports monetary policy, and hedges interest rate risk.
\end{itemize}

\paragraph{8. Use of Government Bonds in Financial System}
\begin{itemize}
    \item Used as \textbf{collateral} in repo markets.
    \item Define \textbf{risk-free benchmark yield curve}.
    \item \textbf{Central banks} use sovereign bonds for open-market operations.
\end{itemize}

\bigskip
\subsubsection*{B. Nonsovereign and Quasi-Government Issuers}

\paragraph{1. Nonsovereign Governments}
\begin{itemize}
    \item Include states, provinces, municipalities, and local authorities.
    \item Issue debt to fund public works and infrastructure.
\end{itemize}

\begin{table}[h!]
\centering
\caption*{Exhibit 2: Types of Nonsovereign Bonds}
\begin{tabular}{|p{3cm}|p{5.5cm}|p{5.5cm}|}
\hline
\textbf{Type} & \textbf{Definition} & \textbf{Repayment Source} \\
\hline
\textbf{General Obligation (GO) Bonds} & Backed by full taxing power of the issuer. & Local/regional tax revenues. \\
\hline
\textbf{Revenue Bonds} & Finances specific projects (e.g., toll roads, hospitals). & Fees or user charges from funded project. \\
\hline
\end{tabular}
\end{table}

\paragraph{2. Quasi-Government (Agency) Bonds}
\begin{itemize}
    \item Issued by \textbf{government-sponsored entities (GSEs)} to fulfill specific mandates:
    \begin{itemize}
        \item Examples: Ginnie Mae (U.S.), KfW (Germany), Japan Finance Corp.
    \end{itemize}
    \item Often enjoy \textbf{implicit or explicit sovereign guarantee}.
    \item \textbf{Yield} typically close to sovereign benchmark.
\end{itemize}

\paragraph{3. Supranational Agencies}
\begin{itemize}
    \item Formed by multiple governments to promote economic development.
    \item \textbf{Examples:} World Bank (IBRD), IMF, Asian Development Bank, European Investment Bank.
    \item \textbf{Credit Quality:} Extremely high (AA–AAA).
    \item \textbf{Purpose:} Infrastructure, poverty reduction, financial stability.
\end{itemize}

\bigskip
\subsection*{LOS 53.b: Issuance and Trading of Government vs. Corporate Debt}

\paragraph{1. Issuance Method: Auctions vs. Underwriting}
\begin{itemize}
    \item \textbf{Sovereign Debt:} Issued via \textbf{public auctions}.
    \item \textbf{Corporate Debt:} Issued via \textbf{underwritten offerings} or \textbf{private placements}.
\end{itemize}

\paragraph{2. Government Bond Auctions}
\begin{itemize}
    \item \textbf{Bid Types:}
    \begin{itemize}
        \item \textbf{Competitive Bid:} Investor specifies price/yield.
        \item \textbf{Noncompetitive Bid:} Investor accepts auction yield and is guaranteed allocation.
    \end{itemize}
    \item \textbf{Allocation Process:}
    \begin{enumerate}
        \item Noncompetitive bids filled first at auction yield.
        \item Competitive bids ranked from highest price (lowest yield) downward.
        \item Allocation continues until issue amount met.
    \end{enumerate}
\end{itemize}

\paragraph{3. Auction Pricing Types}
\begin{table}[h!]
\centering
\caption*{Exhibit 3: Auction Mechanisms}
\begin{tabular}{|p{3cm}|p{6cm}|p{6cm}|}
\hline
\textbf{Auction Type} & \textbf{Mechanism} & \textbf{Implication} \\
\hline
\textbf{Single-Price (Uniform)} & All successful bidders pay the same price — corresponding to the \textbf{cut-off yield}. & Reduces yield volatility; encourages broader participation. \\
\hline
\textbf{Multiple-Price (Discriminatory)} & Each bidder pays the price they actually bid. & Yields closer together; encourages large, informed bids. \\
\hline
\end{tabular}
\end{table}

\paragraph{4. Primary Dealers}
\begin{itemize}
    \item Licensed institutions required to:
    \begin{itemize}
        \item Submit competitive bids in auctions.
        \item Act as counterparties for central bank open-market operations.
        \item Distribute government bonds to the broader market.
    \end{itemize}
\end{itemize}

\paragraph{5. Secondary Market Trading}
\begin{itemize}
    \item \textbf{Market Type:} OTC quote-driven dealer markets.
    \item \textbf{On-the-Run Bonds:} Most recent issue of a given maturity; most liquid and used as \textbf{benchmark yields}.
    \item \textbf{Off-the-Run Bonds:} Older issues; less liquid, trade at higher yields.
\end{itemize}

\paragraph{6. Investor Types and Objectives}
\begin{itemize}
    \item \textbf{Economic Investors:} Seek yield and total return (e.g., mutual funds, pension funds).
    \item \textbf{Non-Economic Investors:}
    \begin{itemize}
        \item Central banks (for monetary policy operations).
        \item Foreign governments (reserve holdings).
        \item Financial institutions (regulatory liquidity requirements).
    \end{itemize}
\end{itemize}

\paragraph{7. Yield Effects}
\begin{itemize}
    \item Non-economic demand (central banks, regulators) pushes yields below equilibrium.
    \item Government debt thus serves as \textbf{low-risk benchmark} for corporate credit spreads.
\end{itemize}

\bigskip
\subsection*{Key Formula and Example: Cut-Off Yield Determination}

\[
\text{Cut-off Yield} = \text{Yield of the lowest-price (highest-yield) successful bid.}
\]

\textbf{Example:}  
A Treasury auction issues \$100 million of 10-year bonds.

\begin{table}[h!]
\centering
\caption*{Exhibit 4: Sample Auction Bids}
\begin{tabular}{|c|c|c|}
\hline
\textbf{Bidder} & \textbf{Price (\% of Par)} & \textbf{Bid Amount (\$m)} \\
\hline
A & 101.00 & 40 \\
B & 100.75 & 30 \\
C & 100.50 & 50 \\
D & 100.25 & 40 \\
\hline
\end{tabular}
\end{table}

\begin{itemize}
    \item Allocation proceeds from highest price down.
    \item Total \$100m issued → allocations to A (\$40m) + B (\$30m) + C (\$30m).
    \item \textbf{Cut-off yield} corresponds to price = 100.50.
\end{itemize}

\begin{itemize}
    \item Under a \textbf{single-price auction}, all pay 100.50.
    \item Under a \textbf{multiple-price auction}, each pays their bid price.
\end{itemize}

\bigskip
\subsection*{Key Summary Table}

\begin{table}[h!]
\centering
\caption*{Exhibit 5: Summary of Government Debt Characteristics}
\begin{tabular}{|l|p{11cm}|}
\hline
\textbf{Issuer Type} & \textbf{Key Features} \\
\hline
\textbf{Sovereign Government} & Backed by taxation power; issues across maturity spectrum; benchmark yield curve; issued via public auctions. \\
\hline
\textbf{Nonsovereign Government} & States, provinces, municipalities; issue general obligation or revenue bonds; rely on local taxes or project fees. \\
\hline
\textbf{Agency / Quasi-Government} & Created to serve specific policy goals (e.g., housing, infrastructure); often sovereign-backed. \\
\hline
\textbf{Supranational Agency} & Multinational entities (World Bank, IMF); high credit quality; foster economic development. \\
\hline
\end{tabular}
\end{table}

\paragraph{Conceptual Takeaways}
\begin{itemize}
    \item Sovereign and supranational issuers anchor global fixed-income markets.
    \item Governments issue across maturities to balance cost and rollover risk.
    \item Auctions ensure transparency and wide investor participation.
    \item On-the-run bonds define the \textbf{risk-free yield curve} for valuation.
    \item Non-economic investors contribute to lower sovereign yields relative to corporate bonds.
\end{itemize}

\section*{Module 54.1: Fixed-Income Bond Valuation — Prices and Yields}

\subsection*{LOS 54.a: Calculating a Bond’s Price Given a Yield-to-Maturity (YTM)}

\paragraph{1. Bond Pricing Fundamentals}
\begin{itemize}
    \item \textbf{Definition:} The price of a bond equals the present value (PV) of its future cash flows, discounted at the bond’s yield to maturity (YTM).
    \item \textbf{YTM:} The internal rate of return (IRR) assuming:
    \begin{itemize}
        \item The bond is held to maturity.
        \item All coupons are paid in full and reinvested at the same YTM.
    \end{itemize}
\end{itemize}

\paragraph{2. Formula: Annual Coupon Bond}
\[
P = \sum_{t=1}^{N} \frac{C}{(1+r)^t} + \frac{F}{(1+r)^N}
\]
where:  
\begin{tabular}{ll}
\( P \) & = price of bond (per \$100 par) \\
\( C \) & = annual coupon payment (\(C = \text{coupon rate} \times F\)) \\
\( F \) & = face value (par) \\
\( r \) & = yield per period (YTM) \\
\( N \) & = number of periods \\
\end{tabular}

\paragraph{Example 1: Annual-Pay Bond (Par, Premium, Discount)}
\begin{table}[h!]
\centering
\caption*{Exhibit 1: 5-Year, 10\% Annual Coupon, \$100 Par Bond}
\begin{tabular}{|c|c|c|}
\hline
\textbf{YTM} & \textbf{Price (PV)} & \textbf{Comment} \\
\hline
10\% & 100.00 & Par bond (coupon = yield). \\
\hline
8\% & 107.99 & Premium bond (coupon > yield). \\
\hline
12\% & 92.79 & Discount bond (coupon < yield). \\
\hline
\end{tabular}
\end{table}

\textbf{Interpretation:}
\begin{itemize}
    \item ↓ YTM ⇒ ↑ Bond Price (Inverse relationship).
    \item Price increase for 2\% ↓ yield (from 10\% to 8\%) = +7.99.
    \item Price decrease for 2\% ↑ yield (from 10\% to 12\%) = −7.21.
    \item ⇒ \textbf{Asymmetry shows convexity.}
\end{itemize}

\paragraph{3. Semiannual Coupon Bond Pricing}
\[
P = \sum_{t=1}^{2N} \frac{C/2}{(1 + r/2)^t} + \frac{F}{(1 + r/2)^{2N}}
\]
\textbf{Example:}  
5-year, 10\% coupon, semiannual payments, YTM = 8\%.  
\[
N = 10; \quad PMT = 5; \quad FV = 100; \quad I/Y = 4\%.
\]
\[
PV = -108.11.
\]
\textbf{Result:} Price = 108.11 → Premium bond.

\paragraph{4. Calculating YTM Given Price}
\[
N = 10; \quad PMT = 5; \quad FV = 100; \quad PV = -105.
\]
\[
\text{Solve: } I/Y = 4.37\% \quad \Rightarrow \text{YTM} = 4.37\% \times 2 = 8.74\%.
\]
\textbf{Interpretation:}  
If price > par → YTM < coupon rate.

\paragraph{5. Accrued Interest, Flat Price, and Full Price}

\textbf{Concepts:}
\begin{itemize}
    \item \textbf{Accrued Interest (AI)} = interest earned between coupon payments.  
    \[
    \text{AI} = \text{Coupon Payment} \times \frac{\text{Days Since Last Coupon}}{\text{Days in Coupon Period}}
    \]
    \item \textbf{Flat (Clean) Price:} Quoted price excluding accrued interest.
    \item \textbf{Full (Dirty) Price:} Invoice price including accrued interest.
    \[
    \text{Full Price} = \text{Flat Price} + \text{Accrued Interest}
    \]
\end{itemize}

\paragraph{6. Day-Count Conventions}
\begin{itemize}
    \item \textbf{Actual/Actual:} Use actual number of days.
    \item \textbf{30/360:} Assume 30 days per month, 360 days per year.
\end{itemize}

\paragraph{Example 2: Accrued Interest Calculation}
\textbf{Bond:} 4\% annual coupon, coupon date May 15, settlement Aug 10.  
\[
\text{Coupon} = 4\% \times 100 = \$4.
\]
\textbf{30/360 method:}  
Days = 15 + 30 + 30 + 10 = 85 days  
\[
AI = 4 \times \frac{85}{360} = 0.944.
\]
\textbf{Actual/Actual method:}  
Days = 16 + 30 + 31 + 10 = 87 days  
\[
AI = 4 \times \frac{87}{365} = 0.954.
\]

\paragraph{Example 3: Full vs. Flat Price}
\begin{table}[h!]
\centering
\caption*{Exhibit 2: Semiannual 5\% Bond, YTM = 4\%, Settlement Aug 21}
\begin{tabular}{|l|l|}
\hline
\textbf{Step} & \textbf{Calculation} \\
\hline
1. PV on last coupon date & \( N = 4; \, PMT = 2.5; \, FV = 100; \, I/Y = 2\% \Rightarrow PV = 101.904 \). \\
\hline
2. Days since last coupon & 67 days; total coupon period = 183 days. \\
\hline
3. Full price & \( 101.904 \times (1.02)^{67/183} = 102.645 \). \\
\hline
4. Accrued interest & \( 2.5 \times \frac{67}{183} = 0.915 \). \\
\hline
5. Flat price & \( 102.645 - 0.915 = 101.73 \). \\
\hline
\end{tabular}
\end{table}

\paragraph{Interpretation:} Flat price < value at last coupon date due to partial period discounting.



\subsection*{LOS 54.b: Relationships Among Price, Coupon, Maturity, and Yield-to-Maturity}

\paragraph{1. Fundamental Relationships}
\begin{itemize}
    \item Price and YTM are \textbf{inversely related}.
    \item Lower coupon rate → \textbf{higher price sensitivity}.
    \item Longer maturity → \textbf{higher duration and price sensitivity}.
    \item Price-Yield relationship is \textbf{convex}.
\end{itemize}

\paragraph{2. Convexity Illustration}
\begin{itemize}
    \item For equal yield changes:
    \[
    |\Delta P_{\text{down}}| > |\Delta P_{\text{up}}|
    \]
    \item Example: 2\% ↓ yield increases price more than 2\% ↑ yield decreases it.
\end{itemize}

\paragraph{3. Pull-to-Par (Constant-Yield Price Trajectory)}
\begin{itemize}
    \item Regardless of coupon rate or YTM, bond price → par as maturity approaches.
    \item This convergence is called the \textbf{pull to par}.
\end{itemize}

\begin{table}[h!]
\centering
\caption*{Exhibit 3: Bond Price vs. Yield Relationship}
\begin{tabular}{|l|p{10cm}|}
\hline
\textbf{Scenario} & \textbf{Outcome} \\
\hline
Coupon = YTM & Price = Par (\(P = 100\)). \\
\hline
Coupon > YTM & Price > Par (Premium Bond). \\
\hline
Coupon < YTM & Price < Par (Discount Bond). \\
\hline
\end{tabular}
\end{table}

\paragraph{4. Graphical Summary}
\begin{itemize}
    \item \textbf{X-axis:} Yield to Maturity.
    \item \textbf{Y-axis:} Bond Price.
    \item The curve slopes downward and is convex.
\end{itemize}

\[
\text{Convexity: } \frac{d^2P}{dr^2} > 0
\]

\paragraph{5. Example: Pull-to-Par Illustration}
\textbf{3-year 6\% semiannual bond:}
\begin{table}[h!]
\centering
\caption*{Exhibit 4: Convergence Toward Par Over Time}
\begin{tabular}{|c|c|c|c|}
\hline
\textbf{Time to Maturity} & \textbf{YTM = 3\%} & \textbf{YTM = 6\%} & \textbf{YTM = 12\%} \\
\hline
3 years & 107.7 & 100.0 & 85.8 \\
\hline
2 years & 105.1 & 100.0 & 90.6 \\
\hline
1 year & 102.4 & 100.0 & 95.3 \\
\hline
At maturity & 100.0 & 100.0 & 100.0 \\
\hline
\end{tabular}
\end{table}

\textbf{Interpretation:} Prices converge to par even though yields differ.



\subsection*{LOS 54.c: Matrix Pricing}

\paragraph{1. Concept}
\begin{itemize}
    \item Used when bonds are \textbf{illiquid or not traded}.
    \item Estimates YTM or price using \textbf{comparable bonds} (same credit quality and maturity).
    \item Involves \textbf{linear interpolation} between yields of similar traded issues.
\end{itemize}

\paragraph{2. Step-by-Step Process}
\begin{enumerate}
    \item Identify traded bonds with same rating and similar maturity.
    \item Average yields if multiple bonds exist per maturity.
    \item Interpolate YTM for the desired maturity.
    \item Use interpolated YTM to calculate bond price.
\end{enumerate}

\paragraph{Example 1: Pricing an Illiquid Bond}
\begin{table}[h!]
\centering
\caption*{Exhibit 5: Interpolation for A+ 3-Year Bond}
\begin{tabular}{|c|c|}
\hline
\textbf{Traded Bonds} & \textbf{YTM} \\
\hline
2-year A+ bond & 4.3\% \\
5-year A+ bonds (avg) & 5.2\% \\
\hline
\end{tabular}
\end{table}

\[
\text{Interpolated 3-year YTM} = 4.3 + (5.2 - 4.3) \times \frac{3 - 2}{5 - 2} = 4.6\%.
\]
\[
N = 3; \; PMT = 4; \; FV = 100; \; I/Y = 4.6 \Rightarrow PV = 98.354.
\]
\textbf{Estimated Value:} \$98.35 per \$100 par.

\paragraph{3. Example 2: Spread-Based Matrix Pricing for New Issues}
\textbf{Given:}
\[
\begin{aligned}
4\text{-year Treasury YTM} &= 1.48\% \\
6\text{-year Treasury YTM} &= 2.15\% \\
5\text{-year A-rated corporate YTM} &= 2.64\%
\end{aligned}
\]

\textbf{Interpolated 5-year Treasury YTM:}
\[
1.48 + (2.15 - 1.48) \times \frac{5 - 4}{6 - 4} = 1.815\%.
\]
\textbf{Spread:}
\[
2.64 - 1.815 = 0.825\%.
\]
\textbf{Required YTM for 6-year A-rated:}
\[
2.15 + 0.825 = 2.975\%.
\]

\paragraph{4. Practical Uses}
\begin{itemize}
    \item Valuing infrequently traded corporate or municipal bonds.
    \item Estimating new issue pricing spreads.
    \item Calibrating bond curves for valuation models.
\end{itemize}



\subsection*{Key Takeaways Summary}

\begin{table}[h!]
\centering
\caption*{Exhibit 6: LOS Summary Table}
\begin{tabular}{|l|p{11cm}|}
\hline
\textbf{LOS} & \textbf{Core Insights} \\
\hline
54.a & Price = PV of all future CFs discounted at YTM. Flat price excludes accrued interest; full price includes it. Accrued interest = coupon × fraction of period elapsed. \\
\hline
54.b & Price–yield relationship is inverse and convex. Lower coupon and longer maturity → higher sensitivity. All bond prices converge to par as maturity nears (pull to par). \\
\hline
54.c & Matrix pricing estimates yield or price for illiquid issues by interpolating between yields of traded bonds with similar rating and maturity. \\
\hline
\end{tabular}
\end{table}

\paragraph{Conceptual Summary:}
\begin{itemize}
    \item Bonds = discounted CF models.
    \item Prices rise when yields fall, but asymmetrically (convexity).
    \item Quoted price (flat) + accrued interest = full (dirty) price.
    \item Matrix pricing bridges illiquid and benchmark bond valuations.
\end{itemize}

\section*{Module 55.1: Yield and Yield Spread Measures for Fixed-Rate Bonds}

\subsection*{LOS 55.a: Calculate Annual Yield on a Bond for Varying Compounding Periods}

\paragraph{1. Definition: Yield to Maturity (YTM)}
\begin{itemize}
    \item The \textbf{YTM} is the discount rate that equates the present value of a bond’s cash flows to its current market price:
    \[
    P = \sum_{t=1}^{N} \frac{C}{(1 + r)^t} + \frac{F}{(1 + r)^N}
    \]
    \item It is the internal rate of return (IRR) assuming:
    \begin{itemize}
        \item The bond is held to maturity.
        \item All coupon payments are made and reinvested at the same YTM.
    \end{itemize}
\end{itemize}

\paragraph{2. Example: Annual vs. Semiannual Yield}
\textbf{Given:} 5-year, 7\% annual-pay bond, price = 102.078.  
\[
N = 5, \; PMT = 7, \; FV = 100, \; PV = -102.078 \Rightarrow I/Y = 6.5\%.
\]
\textbf{Interpretation:} Bond trades at a premium (price > par, YTM < coupon).

\smallskip
\textbf{Semiannual version:}
\[
\text{YTM (semiannual)} = 3.253\% \times 2 = 6.506\%.
\]

\paragraph{3. Effective Annual Yield (EAY)}
\[
\boxed{\text{EAY} = \left( 1 + \frac{\text{YTM}}{n} \right)^n - 1}
\]
where \(n\) = number of coupon periods per year (periodicity).

\begin{table}[h!]
\centering
\caption*{Exhibit 1: Effective Annual Yield Comparison}
\begin{tabular}{|c|c|c|}
\hline
\textbf{Periodicity (n)} & \textbf{Quoted YTM} & \textbf{Effective Annual Yield (EAY)} \\
\hline
1 (Annual) & 10.00\% & 10.00\% \\
\hline
2 (Semiannual) & 10.00\% & 10.25\% \\
\hline
4 (Quarterly) & 10.00\% & 10.38\% \\
\hline
\end{tabular}
\end{table}

\paragraph{4. Adjusting Yields for Periodicity}
\textbf{Example:} Atlas Corp. bond quoted at 4\% (semiannual).  
\[
\text{Semiannual rate} = 2\% \text{ per 6 months}.
\]
\textbf{Equivalent Annual (EAY):}
\[
(1.02)^2 - 1 = 4.04\%.
\]
\textbf{Equivalent Quarterly Quoted Yield:}
\[
(1.02)^{1/2} - 1 = 0.995\% \Rightarrow 4 \times 0.995 = 3.98\%.
\]

\textbf{Conclusion:}  
\begin{itemize}
    \item Annual-pay bond at 4.04\% EAY.
    \item Semiannual-pay bond at 4.00\% nominal.
    \item Quarterly-pay bond at 3.98\% nominal.
\end{itemize}

\paragraph{5. Street vs. True Yield}
\begin{itemize}
    \item \textbf{Street convention:} Uses stated coupon dates.
    \item \textbf{True yield:} Adjusts for actual payment dates (weekends/holidays).  
    $\Rightarrow$ True yield is slightly lower.
\end{itemize}

\paragraph{6. Current Yield (CY)}
\[
\boxed{\text{Current Yield} = \frac{\text{Annual Coupon Payment}}{\text{Flat Price}}}
\]
\textbf{Example:} 20-year, 6\% semiannual bond, price = 802.07.
\[
CY = \frac{60}{802.07} = 7.48\%.
\]

\paragraph{7. Simple Yield}
\[
\boxed{\text{Simple Yield} = \frac{\text{Annual Coupon} \pm \text{Amortized Discount/Premium}}{\text{Price}}}
\]
\textbf{Example:} 3-year, 8\% semiannual bond, price = 90.165.
\[
\text{Discount} = 100 - 90.165 = 9.835; \quad \text{Amortization per year} = \frac{9.835}{3} = 3.278.
\]
\[
\text{Simple Yield} = \frac{8 + 3.278}{90.165} = 12.47\%.
\]

\paragraph{8. Yield to Call (YTC) and Yield to Worst (YTW)}
\begin{itemize}
    \item \textbf{YTC:} Yield assuming bond is called at a specific date and price.
    \item \textbf{YTW:} The lowest of YTM and all YTC values.
\end{itemize}

\textbf{Example:}  
6\% bond, price = 102, callable:
\begin{itemize}
    \item At 102 in 3 years.
    \item At 101 in 4 years.
\end{itemize}
\[
\text{YTM} = 5.54\%, \quad \text{YTC}_1 = 5.59\%, \quad \text{YTC}_2 = 5.56\%.
\]
\textbf{YTW} = 5.54\% (lowest value).

\paragraph{9. Option-Adjusted Yield (OAY)}
\[
\text{Callable bond value} = \text{Straight bond value} - \text{Call option value.}
\]
\[
\Rightarrow \text{Option-adjusted price} = \text{Callable bond price} + \text{Option value.}
\]
\[
\text{Option-adjusted yield (OAY)} < \text{Callable bond YTM.}
\]
\textbf{Interpretation:} OAY “removes” option effects — yield as if option-free.



\subsection*{LOS 55.b: Compare, Calculate, and Interpret Yield and Yield Spread Measures}

\paragraph{1. Definition: Yield Spread}
\[
\boxed{\text{Yield Spread} = \text{Bond Yield} - \text{Benchmark Yield}}
\]
\begin{itemize}
    \item Measured in \textbf{basis points (bps)}.
    \item Reflects compensation for credit, liquidity, and optionality risk.
\end{itemize}

\paragraph{2. G-Spread (Government Spread)}
\[
\text{G-Spread} = \text{YTM}_{\text{bond}} - \text{YTM}_{\text{govt (same maturity)}}.
\]
\textbf{Example:}
\begin{itemize}
    \item Bond YTM = 6.82\%.
    \item Interpolated Treasury yield (3-year) = 4.33\%.
\end{itemize}
\[
G\text{-spread} = 6.82 - 4.33 = 2.49\% = 249 \text{ bps.}
\]

\paragraph{3. I-Spread (Interpolated Swap Spread)}
\[
\text{I-Spread} = \text{YTM}_{\text{bond}} - \text{Swap rate}_{\text{same tenor}}.
\]
\textbf{Interpretation:} Excess yield over interbank reference rate (e.g., EURIBOR or SOFR).

\paragraph{4. Interpreting Yield Spreads}
\begin{itemize}
    \item \textbf{Macro factors:} Affect benchmark yield (risk-free rate, inflation).
    \item \textbf{Micro factors:} Affect credit/liquidity → spread widens.
\end{itemize}
\begin{table}[h!]
\centering
\caption*{Exhibit 2: Interpreting Spread Movements}
\begin{tabular}{|l|l|}
\hline
\textbf{Observation} & \textbf{Interpretation} \\
\hline
Bond yield ↑, spread unchanged & Benchmark yield ↑ → macro factors. \\
\hline
Bond yield ↑, spread ↑ & Bond-specific risk ↑ → micro factors. \\
\hline
Bond yield ↓, spread ↓ & Improved credit/liquidity. \\
\hline
\end{tabular}
\end{table}

\paragraph{5. Z-Spread (Zero-Volatility Spread)}
\[
\boxed{\text{Z-spread} = \text{Constant spread added to benchmark spot curve to match bond price.}}
\]
\[
P = \sum_{t=1}^{N} \frac{CF_t}{(1 + s_t + ZS)^t}
\]
where \(s_t\) = spot rate at time \(t\).

\textbf{Example:}
\begin{itemize}
    \item 3-year, 9\% coupon, price = 89.464.
    \item Treasury spot rates = 4\%, 8.167\%, 12.377\%.
\end{itemize}
\[
\text{G-spread} = 13.5\% - 12.0\% = 1.5\%.
\]
Find \(Z\)-spread by trial-and-error such that PV = 89.464.

\paragraph{6. OAS (Option-Adjusted Spread)}
\[
\boxed{\text{OAS} = \text{Z-spread} - \text{Option Value (bps)}}
\]
\begin{itemize}
    \item For \textbf{callable bonds:} \( \text{OAS} < \text{Z-spread} \).
    \item For \textbf{putable bonds:} \( \text{OAS} > \text{Z-spread} \).
\end{itemize}

\textbf{Example:}
\[
\text{Z-spread} = 180\text{ bps}, \; \text{Option value} = 60\text{ bps}.
\Rightarrow \text{OAS} = 120\text{ bps.}
\]

\textbf{Interpretation:}
\begin{itemize}
    \item Z-spread = total yield premium (credit + liquidity + optionality).
    \item OAS = yield premium excluding optionality.
\end{itemize}

\begin{table}[h!]
\centering
\caption*{Exhibit 3: Types of Yield Spreads}
\begin{tabular}{|l|l|l|}
\hline
\textbf{Type} & \textbf{Benchmark} & \textbf{Interpretation} \\
\hline
G-spread & Government yield & Simple yield difference vs. Treasury. \\
\hline
I-spread & Swap rate & Spread vs. interbank curve. \\
\hline
Z-spread & Spot yield curve & Spread constant over maturities. \\
\hline
OAS & Spot yield curve (option-free) & Spread net of embedded option. \\
\hline
\end{tabular}
\end{table}



\subsection*{Key Relationships and Formulas}

\paragraph{Effective Yield Conversions:}
\[
\text{EAY} = (1 + \frac{r_{nom}}{n})^n - 1
\]
\paragraph{Current Yield:}
\[
CY = \frac{\text{Annual Coupon}}{\text{Price}}
\]
\paragraph{Simple Yield:}
\[
SY = \frac{\text{Annual Coupon} \pm \frac{(100 - P)}{T}}{P}
\]
\paragraph{Yield to Worst:}
\[
YTW = \min(\text{YTM}, \text{YTC}_1, \text{YTC}_2, \ldots)
\]
\paragraph{Z-Spread vs. OAS Relationship:}
\[
Z\text{-spread} = OAS + \text{Option Value}
\]



\subsection*{Summary Table}

\begin{table}[h!]
\centering
\caption*{Exhibit 4: LOS 55 Summary}
\begin{tabular}{|l|p{11cm}|}
\hline
\textbf{LOS} & \textbf{Summary} \\
\hline
55.a & YTM depends on compounding frequency. Effective annual yield rises with periodicity. Street yields assume nominal dates; true yields adjust for holidays. Current yield = coupon / price. Simple yield includes amortized discount/premium. Yield to worst = lowest YTM or YTC. Option-adjusted yield removes embedded option effect. \\
\hline
55.b & Yield spreads measure risk premium vs. benchmark (G, I, Z, or OAS). Z-spread adjusts for the term structure of spot rates. OAS removes option impact, isolating credit + liquidity effects. G-spread and I-spread use YTMs; Z-spread and OAS use spot curves. \\
\hline
\end{tabular}
\end{table}

\paragraph{Key Insights:}
\begin{itemize}
    \item \textbf{Higher compounding frequency =$>$ higher effective yield.}
    \item \textbf{Callable bonds:} YTW = minimum of YTM and YTCs.
    \item \textbf{Z-spread vs. OAS:} OAS = Z-spread minus option value.
    \item \textbf{Macro vs. micro factors:} Benchmark vs. spread changes.
\end{itemize}

\section*{Module 56.1: Yield and Yield Spread Measures for Floating-Rate Instruments}

\subsection*{LOS 56.a: Calculate and Interpret Yield Spread Measures for Floating-Rate Notes (FRNs)}

\paragraph{1. Overview of Floating-Rate Notes (FRNs)}
\begin{itemize}
    \item \textbf{Definition:} A bond whose coupon payments adjust periodically according to a variable market reference rate (MRR) plus a fixed margin.
    \item \textbf{General Formula:}
    \[
    \text{Coupon Rate}_t = \text{MRR}_t + \text{Quoted Margin (QM)}
    \]
    \item \textbf{Coupon Reset:} Based on current market reference rate at the beginning of each coupon period; interest is paid \textbf{in arrears}.
\end{itemize}

\paragraph{2. Key Terminology}
\begin{table}[h!]
\centering
\caption*{Exhibit 1: Floating-Rate Note Margin Definitions}
\begin{tabular}{|p{5cm}|p{10cm}|}
\hline
\textbf{Term} & \textbf{Definition / Interpretation} \\
\hline
\textbf{Market Reference Rate (MRR)} & Short-term market rate (e.g., LIBOR, EURIBOR, SOFR, etc.). \\
\hline
\textbf{Quoted Margin (QM)} & Fixed spread over MRR specified at issuance. \\
\hline
\textbf{Discount Margin (DM)} & Required margin over MRR that equates market price to par (reflects current credit risk). \\
\hline
\textbf{Required Margin} & Same as discount margin — the yield premium investors currently demand. \\
\hline
\end{tabular}
\end{table}

\paragraph{3. Par, Premium, and Discount Pricing Relationships}
\begin{table}[h!]
\centering
\caption*{Exhibit 2: FRN Valuation Logic}
\begin{tabular}{|l|l|l|}
\hline
\textbf{Condition} & \textbf{Relationship Between QM and DM} & \textbf{Price Behavior} \\
\hline
At issuance & QM = DM & FRN trades at par. \\
\hline
Credit quality worsens & QM < DM & FRN trades at a discount. \\
\hline
Credit quality improves & QM > DM & FRN trades at a premium. \\
\hline
\end{tabular}
\end{table}

\paragraph{4. Analogy to Fixed-Rate Bonds}
\begin{itemize}
    \item Fixed-rate bond analogy:
    \[
    \text{Coupon vs. YTM} \Longleftrightarrow \text{Quoted Margin vs. Discount Margin}
    \]
    \item If yield demanded $>$ coupon → bond trades below par.  
      Similarly, if DM $>$ QM → FRN trades below par.
\end{itemize}

\paragraph{5. Simplified Valuation on Reset Date}
\[
P = \sum_{t=1}^{N} \frac{CF_t}{(1 + \text{MRR} + \text{DM})^t}
\]
where \( CF_t = (\text{MRR} + \text{QM}) \times \text{Face Value} \).

\paragraph{Example 1: FRN Valuation}
\textbf{Given:}
\begin{itemize}
    \item Par value = \$100,000.
    \item Semiannual coupon = MRR + 120 bps.
    \item MRR = 3.0\% (annualized), DM = 1.5\% (annualized).
    \item Remaining maturity = 5 years → 10 semiannual periods.
\end{itemize}

\textbf{Step 1: Compute coupon per period}
\[
\text{Coupon rate} = (3.0 + 1.2)\% = 4.2\% \Rightarrow \text{semiannual coupon} = 2.1\%.
\]

\textbf{Step 2: Compute discount rate per period}
\[
\text{Discount rate} = (3.0 + 1.5)\% / 2 = 2.25\%.
\]

\textbf{Step 3: Calculate PV}
\[
N = 10; \; PMT = 2.1; \; FV = 100; \; I/Y = 2.25; \Rightarrow PV = 98.67.
\]

\textbf{Result:} FRN Value = \$98,670 → trades below par since DM > QM.

\paragraph{6. Interpretation Summary}
\begin{itemize}
    \item \textbf{If DM > QM:} Investors demand higher yield → price $<$ 100 (discount).
    \item \textbf{If DM < QM:} Investors accept lower yield → price $>$ 100 (premium).
    \item \textbf{If DM = QM:} Price = 100 (at par on reset dates).
\end{itemize}



\subsection*{LOS 56.b: Calculate and Interpret Yield Measures for Money Market Instruments}

\paragraph{1. Overview}
\begin{itemize}
    \item \textbf{Money Market Instruments:} Short-term debt (maturity ≤ 1 year).
    \item Examples: Treasury bills, commercial paper, CDs, repos.
    \item Yields vary by:
    \begin{itemize}
        \item Quotation basis (add-on vs. discount).
        \item Day-count convention (360-day vs. 365-day year).
    \end{itemize}
\end{itemize}

\paragraph{2. Add-On Yield (AOY) — Interest on Principal}
\[
\boxed{\text{Quoted Add-On Yield} = \text{HPY} \times \frac{\text{Year Days}}{\text{Days to Maturity}}}
\]
where:
\[
\text{HPY} = \frac{\text{Interest Earned}}{\text{Price Paid}}.
\]

\textbf{Example:} 100-day CD, AOY = 1.5\% (365-day year)
\[
\text{HPY} = 1.5\% \times \frac{100}{365} = 0.41\%.
\]
\[
\text{Future Value} = 1000(1.0041) = 1004.10.
\]

\paragraph{3. Discount Yield (DY) — Discount from Face Value}
\[
\boxed{\text{Quoted Discount Yield} = \text{Discount} \times \frac{360}{\text{Days to Maturity}}}
\]
\[
\text{Discount} = 1 - \frac{P}{F}.
\]

\textbf{Example:} 180-day T-bill quoted at 2.2\% on a 360-day basis.
\[
\text{Discount} = 0.022 \times \frac{180}{360} = 0.011.
\]
\[
\text{Price} = 1000(1 - 0.011) = 989.
\]
\[
\text{HPY} = \frac{1000 - 989}{989} = 1.11\%.
\]

\paragraph{4. Converting Between Yield Conventions}
\begin{itemize}
    \item Required to compare securities quoted on different bases.
\end{itemize}

\begin{table}[h!]
\centering
\caption*{Exhibit 3: Money Market Yield Conversions}
\begin{tabular}{|l|l|}
\hline
\textbf{Conversion} & \textbf{Formula} \\
\hline
Discount yield $\rightarrow$ Price & $P = F \left(1 - \frac{d \times t}{360}\right)$ \\
\hline
Price $\rightarrow$ Add-on yield & $r = \frac{F - P}{P} \times \frac{365}{t}$ \\
\hline
Add-on yield (360) $\rightarrow$ Bond Equivalent Yield (BEY) & $r_{BEY} = r_{360} \times \frac{365}{360}$ \\
\hline
\end{tabular}
\end{table}

\paragraph{5. Example: Comparing Different Money Market Quotes}
\textbf{Case 1: 90-day T-bill}  
Discount = 1.2\%, 360-day basis.
\[
P = 1000(1 - 0.012 \times 90 / 360) = 997.
\]
\[
\text{Add-on yield (BEY)} = \frac{3}{997} \times \frac{365}{90} = 1.22\%.
\]

\textbf{Case 2: 120-day CD}  
Add-on yield = 1.4\% (365-day).
\[
\text{Interest} = 0.014 \times \frac{120}{365} = 0.004603.
\]
\[
\text{Maturity value} = 1,000,000(1.004603) = 1,004,603.
\]

\textbf{Case 3: 100-day deposit}  
Add-on yield = 1.5\% (360-day).
\[
\text{Bond equivalent yield (BEY)} = 1.5 \times \frac{365}{360} = 1.5208\%.
\]
\[
\text{HPY} = 1.5\% \times \frac{100}{360} = 0.4167\%.
\]
\[
\text{Effective annual yield} = (1.004167)^{3.65} - 1 = 1.5294\%.
\]
\[
\text{Semiannual bond yield} = 2 \times \left( (1.015294)^{1/2} - 1 \right) = 1.5236\%.
\]

\paragraph{6. Bond Equivalent Yield (BEY)}
\begin{itemize}
    \item \textbf{Definition:} Add-on yield annualized using 365 days — allows direct comparison with bond yields.
    \item \textbf{Formula:}
    \[
    \boxed{\text{BEY} = \frac{\text{HPY} \times 365}{\text{Days to Maturity}}}
    \]
\end{itemize}

\paragraph{7. Interpretation of Yield Basis}
\begin{table}[h!]
\centering
\caption*{Exhibit 4: Quotation Conventions in Money Markets}
\begin{tabular}{|l|l|l|}
\hline
\textbf{Instrument} & \textbf{Quotation Basis} & \textbf{Year Convention} \\
\hline
Treasury bills, commercial paper & Discount yield & 360-day \\
\hline
Bank CDs, repos, interbank deposits & Add-on yield & 365-day \\
\hline
Bond equivalent yields (BEY) & Add-on yield & 365-day \\
\hline
\end{tabular}
\end{table}



\subsection*{Key Formulas Summary}

\begin{table}[h!]
\centering
\caption*{Exhibit 5: Summary of Core Yield Formulas}
\begin{tabular}{|l|l|}
\hline
\textbf{Concept} & \textbf{Formula} \\
\hline
Floating-Rate Coupon & \( \text{Coupon} = (\text{MRR} + \text{QM}) \times F \) \\
\hline
FRN Discount Rate & \( r = \text{MRR} + \text{DM} \) \\
\hline
Add-On Yield & \( r_{AOY} = \frac{F - P}{P} \times \frac{\text{Year Days}}{\text{Days to Maturity}} \) \\
\hline
Discount Yield & \( r_{DY} = \frac{F - P}{F} \times \frac{360}{\text{Days to Maturity}} \) \\
\hline
Bond Equivalent Yield (BEY) & \( r_{BEY} = \text{HPY} \times \frac{365}{\text{Days to Maturity}} \) \\
\hline
FRN Price Approximation & \( P = \sum_{t=1}^{N} \frac{CF_t}{(1 + \text{MRR} + \text{DM})^t} \) \\
\hline
\end{tabular}
\end{table}



\subsection*{Summary Table: LOS 56.a–56.b}

\begin{table}[h!]
\centering
\caption*{Exhibit 6: Conceptual Summary}
\begin{tabular}{|l|p{11cm}|}
\hline
\textbf{LOS} & \textbf{Core Insights} \\
\hline
56.a & FRNs pay MRR + quoted margin (QM). Required margin (DM) reflects investor credit demands. FRN trades at par if QM = DM, at discount if QM < DM, and at premium if QM > DM. Value estimated using MRR + DM as discount rate. \\
\hline
56.b & Money market yields differ by quotation (add-on vs. discount) and year convention (360 or 365). Bond equivalent yield (BEY) = add-on yield using 365-day basis. Yield conversions allow cross-comparison of securities. \\
\hline
\end{tabular}
\end{table}

\paragraph{Key Takeaways:}
\begin{itemize}
    \item FRNs reset coupons periodically, making price less volatile than fixed-rate bonds.
    \item The relationship \( \text{QM vs. DM} \) drives FRN price deviations from par.
    \item Money market yields require standardization (BEY) for comparison across instruments.
    \item Discount instruments quote yields below face; add-on instruments quote above price paid.
\end{itemize}

\section*{Module 57.1: The Term Structure of Interest Rates — Spot, Par, and Forward Curves}

\subsection*{LOS 57.a: Define Spot Rates and the Spot Curve; Calculate the Price of a Bond Using Spot Rates}

\paragraph{1. Definition of Spot Rates}
\begin{itemize}
    \item A \textbf{spot rate} (\(S_t\)) is the market discount rate for a single future payment at time \(t\).
    \item Also called \textbf{zero-coupon rate} or \textbf{zero rate}.
    \item Represents yield on a zero-coupon bond maturing at \(t\).
\end{itemize}

\paragraph{2. Bond Valuation Using Spot Rates}
\[
\boxed{
P_0 = \sum_{t=1}^{N} \frac{CF_t}{(1 + S_t)^t}
}
\]
where \(CF_t\) = coupon or principal at time \(t\), discounted by the corresponding spot rate \(S_t\).

\paragraph{3. Example: Bond Valuation Using Spot Rates}
\textbf{Given:}
\[
S_1 = 3\%, \quad S_2 = 4\%, \quad S_3 = 5\%, \quad \text{Coupon rate} = 5\%, \quad FV = 100
\]
\[
P = \frac{5}{1.03} + \frac{5}{(1.04)^2} + \frac{105}{(1.05)^3} = 100.18
\]

\textbf{Interpretation:}
\begin{itemize}
    \item Bond value slightly $>$ par → YTM slightly $<$ coupon (4.93\%).
    \item Price derived from spot rates = \textbf{no-arbitrage price}.
\end{itemize}

\paragraph{4. Conceptual View}
\begin{itemize}
    \item Each cash flow discounted at its own appropriate zero-coupon yield.
    \item YTM is a weighted average of spot rates.
    \item Spot curve: graphical plot of spot rates (\(y\)-axis) vs. maturity (\(x\)-axis).
\end{itemize}

\begin{table}[h!]
\centering
\caption*{Exhibit 1: Spot Rates vs. Bond Pricing}
\begin{tabular}{|p{5cm}|p{5cm}|p{5cm}|}
\hline
\textbf{Scenario} & \textbf{Relative Spot/YTM Levels} & \textbf{Resulting Bond Price} \\
\hline
Spot rates increase with maturity (normal) & YTM slightly below long-term spot & Price $\approx$ par \\
\hline
Flat spot curve & All spot rates equal to YTM & Price = par \\
\hline
Inverted spot curve & YTM higher than early spot rates & Price $<$ par \\
\hline
\end{tabular}
\end{table}



\subsection*{LOS 57.b: Define Par and Forward Rates; Compute Par Rates, Forward Rates, and Spot Rates}

\paragraph{1. Par Yield (or Par Rate)}
\begin{itemize}
    \item The \textbf{par yield} is the coupon rate that causes a bond’s price to equal its par value.
    \item For an \(N\)-year bond:
    \[
    100 = \sum_{t=1}^{N} \frac{PMT}{(1 + S_t)^t} + \frac{100}{(1 + S_N)^N}
    \]
    Solve for \(PMT\) → the \textbf{par yield}.
\end{itemize}

\paragraph{Example:}
Given spot rates \(S_1=1\%, S_2=2\%, S_3=3\%\):
\[
\frac{PMT}{1.01} + \frac{PMT}{(1.02)^2} + \frac{100 + PMT}{(1.03)^3} = 100
\]
\[
\Rightarrow PMT = 2.96\%
\]
\textbf{Interpretation:} 3-year par bond has coupon = 2.96\% to trade at par.



\paragraph{2. Forward Rates (Notation and Concept)}
\begin{itemize}
    \item Forward rate \(a y b y\): rate for a \(b\)-year loan beginning \(a\) years from now.
    \[
    \text{Example: } 2y1y \text{ = 1-year rate beginning 2 years from today.}
    \]
    \item Forward rates are implied future rates ensuring no-arbitrage between investing now vs. investing later.
\end{itemize}

\paragraph{3. Relationship Between Spot and Forward Rates}
\[
\boxed{(1 + S_N)^N = \prod_{t=1}^{N} (1 + f_{t-1,t})}
\]
where \(f_{t-1,t}\) = forward rate for period \(t\).

\textbf{Example 1: Compute Spot from Forward Rates}
\[
S_1 = 2\%, \quad 1y1y = 3\%, \quad 2y1y = 4\%
\]
\[
(1 + S_3)^3 = (1.02)(1.03)(1.04) \Rightarrow S_3 = (1.02 \times 1.03 \times 1.04)^{1/3} - 1 = 2.997\%
\]

\textbf{Interpretation:} The 3-year spot rate is the geometric mean of one 1-year spot and two forward rates.



\paragraph{4. Computing Forward Rate from Spot Rates}
\[
\boxed{(1 + S_2)^2 = (1 + S_1)(1 + 1y1y)}
\Rightarrow 1y1y = \frac{(1 + S_2)^2}{(1 + S_1)} - 1
\]
\textbf{Example 2:}
\[
S_1 = 4\%, \quad S_2 = 8\% \Rightarrow 1y1y = \frac{1.08^2}{1.04} - 1 = 12.154\%.
\]
\textbf{Meaning:} Investors earn 4\% first year, then 12.15\% next year, equivalent to 8\% annualized for two years.



\paragraph{5. 3-Year Extension}
\[
(1 + S_3)^3 = (1 + S_2)^2 (1 + 2y1y)
\Rightarrow 2y1y = \frac{(1 + S_3)^3}{(1 + S_2)^2} - 1
\]
\textbf{Example 3:}
\[
S_1 = 4\%, \quad S_2 = 8\%, \quad S_3 = 12\%
\Rightarrow 2y1y = \frac{1.12^3}{1.08^2} - 1 = 20.45\%.
\]

\textbf{Quick Approximation:}
\[
2y1y \approx (3 \times 12) - (2 \times 8) = 20\%.
\]



\paragraph{6. Multi-Period Forward Rates}
\[
(1 + S_4)^4 = (1 + S_2)^2 (1 + 2y2y)^2
\Rightarrow 2y2y = \sqrt{\frac{(1 + S_4)^4}{(1 + S_2)^2}} - 1
\]
\textbf{Example 4:}
\[
S_2 = 6\%, \quad S_4 = 8\% \Rightarrow 2y2y = 10.04\%.
\]
\textbf{Approximation:} \((4 \times 8 - 2 \times 6)/2 = 10\%\).



\paragraph{7. Valuing a Bond Using Forward Rates}
\textbf{Given:}
\[
S_1 = 4\%, \; 1y1y = 5\%, \; 2y1y = 6\%; \quad 3\text{-year, 5\% coupon, } FV = 1000.
\]

\textbf{Discount Factors:}
\[
DF_1 = \frac{1}{1.04}, \quad DF_2 = \frac{1}{(1.04)(1.05)}, \quad DF_3 = \frac{1}{(1.04)(1.05)(1.06)}.
\]
\[
P = 50(DF_1 + DF_2 + DF_3) + 1000(DF_3) = 1000.7.
\]

\textbf{Interpretation:} Forward-based valuation = spot-based valuation → both yield \textbf{no-arbitrage price}.



\subsection*{LOS 57.c: Compare the Spot Curve, Par Curve, and Forward Curve}

\paragraph{1. Spot Curve (Zero Curve)}
\begin{itemize}
    \item Plots spot rates (\(S_t\)) vs. maturities.
    \item Derived from zero-coupon or stripped Treasury bonds.
    \item Represents pure time-value yields.
\end{itemize}

\paragraph{2. Par Curve}
\begin{itemize}
    \item Plots \textbf{par yields} — coupon rates for bonds trading exactly at par.
    \item Constructed using spot rates.
    \item Less affected by illiquidity and taxation distortions than direct yield curves.
\end{itemize}

\paragraph{3. Forward Curve}
\begin{itemize}
    \item Plots forward rates (\(f_t\)) for successive periods.
    \item Represents expected future short-term rates.
\end{itemize}

\paragraph{4. Relationships Among Curves}
\begin{table}[h!]
\centering
\caption*{Exhibit 2: Curve Relationship Summary}
\begin{tabular}{|l|p{10cm}|}
\hline
\textbf{Curve Type} & \textbf{Description} \\
\hline
Spot Curve & Set of yields on zero-coupon bonds (spot rates). \\
\hline
Par Curve & Hypothetical yields of par bonds for each maturity. \\
\hline
Forward Curve & Implied future short-term rates consistent with spot curve. \\
\hline
\end{tabular}
\end{table}

\paragraph{5. Comparative Dynamics}
\begin{itemize}
    \item \textbf{Upward-sloping (normal) yield curve:}
    \[
    \text{Forward rates} > \text{Spot rates} > \text{Par yields}.
    \]
    \item \textbf{Downward-sloping (inverted) yield curve:}
    \[
    \text{Forward rates} < \text{Spot rates} < \text{Par yields}.
    \]
    \item \textbf{Flat curve:}
    \[
    \text{Forward rates} = \text{Spot rates} = \text{Par yields}.
    \]
\end{itemize}

\paragraph{6. Example: Normal Yield Curve Intuition}
\[
S_1 = 1\%, \; 1y1y = 3\%.
\Rightarrow S_2 \approx (1 + 0.01)(1 + 0.03)^{1/2} - 1 \approx 2\%.
\]
\textbf{Interpretation:}
\begin{itemize}
    \item Forward curve steeper than spot curve.
    \item Par yield curve smoother (weighted average of spots).
\end{itemize}

\begin{table}[h!]
\centering
\caption*{Exhibit 3: Comparative Behavior of Yield Curves}
\begin{tabular}{|l|l|l|}
\hline
\textbf{Yield Curve Type} & \textbf{Forward Curve vs. Spot Curve} & \textbf{Par Curve vs. Spot Curve} \\
\hline
Upward-sloping & Above & Slightly below \\
\hline
Flat & Equal & Equal \\
\hline
Downward-sloping & Below & Slightly above \\
\hline
\end{tabular}
\end{table}



\subsection*{Summary Table: LOS 57.a–57.c}

\begin{table}[h!]
\centering
\caption*{Exhibit 4: Summary of Core Concepts}
\begin{tabular}{|l|p{11cm}|}
\hline
\textbf{LOS} & \textbf{Summary} \\
\hline
57.a & Spot rates are zero-coupon yields for specific maturities. Bond prices are found by discounting each cash flow at its respective spot rate (no-arbitrage price). \\
\hline
57.b & Par yields are coupon rates that make bonds price at par. Forward rates are implied future short-term rates consistent with the spot curve. Relations: $(1+S_2)^2=(1+S_1)(1+1y1y)$ and $(1+S_3)^3=(1+S_2)^2(1+2y1y)$. \\
\hline
57.c & Spot, par, and forward curves are interrelated: forward rates drive spot rates; spot rates drive par yields. In a normal curve, forward $>$ spot $>$ par. In an inverted curve, forward $<$ spot $<$ par. \\
\hline
\end{tabular}
\end{table}

\paragraph{Key Takeaways:}
\begin{itemize}
    \item Spot rates = pure time-value yields from zero-coupon bonds.
    \item Forward rates = implied future short-term borrowing/lending rates.
    \item Par yields = hypothetical coupon rates that price bonds at par.
    \item Forward $\rightarrow$ Spot $\rightarrow$ Par sequence defines yield curve dynamics.
\end{itemize}

\section*{Module 58.1: Interest Rate Risk and Return}

\subsection*{LOS 58.a: Sources of Return from Fixed-Rate Bonds}

\paragraph{1. Overview}
A fixed-rate bond provides three primary sources of return:

\begin{enumerate}
    \item \textbf{Coupon and Principal Payments:} Fixed periodic cash inflows promised by the issuer.
    \item \textbf{Reinvestment Income:} Interest earned on reinvested coupon payments.
    \item \textbf{Capital Gain or Loss:} Difference between sale price and purchase price if sold before maturity.
\end{enumerate}

\paragraph{2. Key Assumptions}
\begin{itemize}
    \item The bond makes all payments as promised (no default).
    \item Coupons are reinvested at the bond’s original YTM.
    \item YTM remains constant unless otherwise stated.
\end{itemize}

\paragraph{3. Five Key Results}
\begin{itemize}
    \item[(1)] Holding to maturity with constant YTM → realized return = YTM.
    \item[(2)] Selling before maturity with unchanged YTM → realized return = YTM.
    \item[(3)] YTM rises → realized return $>$ original YTM (if held to maturity).
    \item[(4)] YTM rises → realized return $<$ original YTM (if short horizon).
    \item[(5)] YTM falls → realized return $<$ original YTM (if long horizon).
\end{itemize}



\subsection*{Unchanged YTM — Bond Held to Maturity}

\paragraph{Example 1:}
\[
\text{6\% annual-pay, 3-year bond, } YTM = 7\%
\]
\[
N=3; I/Y=7; PMT=6; FV=100; \Rightarrow PV = -97.376
\]

\textbf{Coupon and Reinvestment Income:}
\[
FV_{\text{coupons}} = 6(1.07)^2 + 6(1.07) + 6 = 19.289
\]
Reinvestment income $= 19.289 - 18 = 1.289$

\textbf{Total Future Value:}
\[
119.289 = 97.376 (1.07)^3
\]
\textbf{Conclusion:} Realized annual return = YTM = 7\%.



\subsection*{Unchanged YTM — Bond Sold Before Maturity}

\paragraph{Concept:}
If YTM remains constant, bond value converges to par as maturity approaches.  
Intermediate values = \textbf{carrying values} (on the constant-yield trajectory).

\paragraph{Example 2: Capital Gain/Loss}
\[
\text{20-year bond, } 5\% \text{ semiannual coupon, } YTM = 6\%.
\]
Sold after 5 years (15 years remaining).

\textbf{Carrying value:}
\[
N=30; I/Y=3; PMT=2.5; FV=100; \Rightarrow PV=-90.20
\]
Sold at 91.40 → Capital gain = 91.40 - 90.20 = 1.20.

\paragraph{Conceptual Note:}
\begin{itemize}
    \item Bonds held or sold at constant YTM have no true capital gain/loss.
    \item Price converges toward par through amortization of premium/discount.
\end{itemize}



\subsection*{Changed YTM — Bond Held to Maturity}

\paragraph{Example 3: YTM Change After Purchase}
\[
\text{3-year 6\% bond, } YTM = 7\%, \text{Price = 97.376.}
\]

\textbf{Case 1: YTM rises to 8\%}
\[
FV_{\text{coupons}} = 6(1.08)^2 + 6(1.08) + 6 = 19.478
\Rightarrow \text{Realized return } > 7\%.
\]

\textbf{Case 2: YTM falls to 6\%}
\[
FV_{\text{coupons}} = 6(1.06)^2 + 6(1.06) + 6 = 19.102
\Rightarrow \text{Realized return } < 7\%.
\]

\paragraph{Interpretation:}
\begin{itemize}
    \item Reinvestment income dominates for long horizons.
    \item Higher YTM $\Rightarrow$ more reinvestment income $\Rightarrow$ higher return.
\end{itemize}



\subsection*{Changed YTM — Bond Sold Before Maturity}

\paragraph{Example 4: One-Year Horizon}
\[
\text{3-year 6\% bond, purchased at } 97.376, \text{ horizon = 1 year.}
\]

\textbf{Case 1: YTM increases to 8\%}
\[
N=2; I/Y=8; PMT=6; FV=100; PV=-96.433.
\]
Return $<$ 7\% (price loss dominates).

\textbf{Case 2: YTM decreases to 6\%}
\[
N=2; I/Y=6; PMT=6; FV=100; PV=-100.00.
\]
Return $>$ 7\% (price gain dominates).

\paragraph{Key Idea: Price Risk vs. Reinvestment Risk}
\begin{table}[h!]
\centering
\caption*{Exhibit 1: Price vs. Reinvestment Risk Tradeoff}
\begin{tabular}{|l|l|l|}
\hline
\textbf{Investment Horizon} & \textbf{Dominant Risk} & \textbf{Effect of YTM Change} \\
\hline
Short-term & Price risk & YTM↑ → Return↓ (price loss) \\
\hline
Long-term & Reinvestment risk & YTM↑ → Return↑ (more coupon reinvestment income) \\
\hline
At balance (duration) & Equal & Return unaffected by small YTM shifts \\
\hline
\end{tabular}
\end{table}



\subsection*{LOS 58.b: Relationship Among Holding Period Return, Macaulay Duration, and Investment Horizon}

\paragraph{Conceptual Link:}
\begin{itemize}
    \item \textbf{Holding Period Return (HPR):} Realized yield over investor’s horizon.
    \item \textbf{Macaulay Duration:} Time-weighted average of cash flows; balance point of price vs. reinvestment risk.
    \item \textbf{Investment Horizon:} Time investor plans to hold bond.
\end{itemize}

\paragraph{Relationship Summary:}
\[
\boxed{\text{Duration Gap} = D_{Mac} - \text{Investment Horizon}}
\]

\begin{itemize}
    \item Positive duration gap → price risk dominates (YTM↑ → loss).
    \item Negative duration gap → reinvestment risk dominates (YTM↓ → loss).
    \item Zero duration gap → risks offset → horizon yield = original YTM.
\end{itemize}



\subsection*{LOS 58.c: Macaulay Duration — Definition, Calculation, and Interpretation}

\paragraph{Definition:}
\[
\boxed{
D_{Mac} = \sum_{t=1}^{N} \left( \frac{t \times PV(CF_t)}{P_0} \right)
}
\]
where \(PV(CF_t) = \dfrac{CF_t}{(1+y)^t}\), and \(P_0 = \sum PV(CF_t)\).

\paragraph{Interpretation:}
Average time (in years) it takes to receive bond’s cash flows (weighted by PVs).



\paragraph{Example 5: Macaulay Duration Calculation}

\textbf{Bond:} 5-year, 11\% annual coupon, Price = 86.59, YTM = 15\%.

\begin{table}[h!]
\centering
\caption*{Exhibit 2: Present Values and Weights for Macaulay Duration}
\begin{tabular}{|c|c|c|c|c|}
\hline
\textbf{Year} & \textbf{Cash Flow (\$)} & \textbf{PV(CF)} & \textbf{Weight} & \textbf{t × Weight} \\
\hline
1 & 11 & 9.57 & 0.1105 & 0.1105 \\
\hline
2 & 11 & 8.32 & 0.0961 & 0.1922 \\
\hline
3 & 11 & 7.23 & 0.0835 & 0.2505 \\
\hline
4 & 11 & 6.29 & 0.0726 & 0.2904 \\
\hline
5 & 111 & 55.18 & 0.6373 & 3.1865 \\
\hline
\multicolumn{4}{|r|}{\textbf{Macaulay Duration:}} & \textbf{4.03 years} \\
\hline
\end{tabular}
\end{table}

\paragraph{Interpretation:}
\begin{itemize}
    \item Investment horizon = 4 years → horizon yield = original YTM (15\%) regardless of rate change.
    \item Confirms duration as point of risk neutrality.
\end{itemize}



\subsection*{Example: Horizon Yield Stability (Duration Balance)}

\textbf{Bond:} 5-year, 11\% coupon, Price = 86.59, YTM = 15\%, Horizon = 4 years.

\begin{table}[h!]
\centering
\caption*{Exhibit 3: Horizon Yield Outcomes}
\begin{tabular}{|l|l|l|}
\hline
\textbf{YTM Change} & \textbf{Sale Price (PV)} & \textbf{Horizon Yield} \\
\hline
Falls to 14\% & 97.368 & 15.0\% \\
\hline
Rises to 16\% & 95.690 & 15.0\% \\
\hline
\end{tabular}
\end{table}

\textbf{Conclusion:}
\[
\boxed{\text{When Horizon} = D_{Mac}, \; \text{Price Risk} = \text{Reinvestment Risk}.}
\]



\subsection*{Summary Tables}

\begin{table}[h!]
\centering
\caption*{Exhibit 4: LOS Summary}
\begin{tabular}{|l|p{11cm}|}
\hline
\textbf{LOS} & \textbf{Summary} \\
\hline
58.a & Three return components: coupon/principal, reinvested coupons, capital gain/loss. Holding to maturity at constant YTM → realized return = YTM. \\
\hline
58.b & Price risk dominates short horizons; reinvestment risk dominates long horizons. Horizon yield equals YTM when investment horizon = Macaulay duration. \\
\hline
58.c & Macaulay duration = weighted average time to receipt of CFs. Duration gap = $D_{Mac} -$ investment horizon. Positive gap → price risk; negative gap → reinvestment risk. \\
\hline
\end{tabular}
\end{table}

\begin{table}[h!]
\centering
\caption*{Exhibit 5: Risk and Horizon Relationship}
\begin{tabular}{|l|l|l|}
\hline
\textbf{Case} & \textbf{Dominant Risk} & \textbf{Effect on Realized Return} \\
\hline
Short Horizon & Price risk & YTM↑ → Return↓, YTM↓ → Return↑ \\
\hline
Long Horizon & Reinvestment risk & YTM↑ → Return↑, YTM↓ → Return↓ \\
\hline
Horizon = Duration & Balanced & Return = Original YTM (risk offset) \\
\hline
\end{tabular}
\end{table}



\subsection*{Key Takeaways}
\begin{itemize}
    \item \textbf{Sources of Return:} Coupon, reinvestment, capital gain/loss.
    \item \textbf{Risks:} 
        \begin{itemize}
            \item Price risk dominates short horizons.
            \item Reinvestment risk dominates long horizons.
        \end{itemize}
    \item \textbf{Duration Neutral Point:} When horizon = Macaulay duration, realized return = YTM regardless of rate change.
    \item \textbf{Duration Gap:} 
    \[
    D_{Mac} - \text{Investment Horizon} = 0 \Rightarrow \text{Perfect Immunization.}
    \]
\end{itemize}

\section*{Module 59.1: Yield-Based Bond Duration Measures and Properties}

\subsection*{LOS 59.a: Define, Calculate, and Interpret Modified Duration, Money Duration, and PVBP}

\paragraph{1. Concept Overview}
\begin{itemize}
    \item \textbf{Macaulay Duration (MacDur):} Weighted average time to receive the bond’s cash flows.
    \item \textbf{Modified Duration (ModDur):} Sensitivity measure of the bond’s price to a change in YTM.
    \item \textbf{Money Duration (Dollar Duration):} Currency change in bond value per 1\% change in YTM.
    \item \textbf{Price Value of a Basis Point (PVBP):} Currency change in bond value per 1 basis point (0.01\%) change in YTM.
\end{itemize}



\paragraph{2. Modified Duration (ModDur)}
\[
\boxed{
\text{ModDur} = \frac{\text{MacDur}}{1 + \dfrac{\text{YTM}}{m}}
}
\]
where \(m\) = number of coupon payments per year.

\begin{itemize}
    \item For annual-pay bonds: \(\text{ModDur} = \dfrac{\text{MacDur}}{1 + \text{YTM}}\)
    \item For semiannual-pay bonds: \(\text{ModDur} = \dfrac{\text{MacDur}}{1 + \text{YTM}/2}\)
\end{itemize}

\paragraph{3. Price–Yield Approximation Using ModDur}
\[
\boxed{
\frac{\Delta P}{P} \approx -\text{ModDur} \times \Delta \text{YTM}
}
\]
\begin{itemize}
    \item Estimates percentage price change for a 1\% change in YTM.
    \item Negative sign reflects inverse relationship between price and yield.
\end{itemize}



\paragraph{Example 1: Modified Duration Calculation}

\textbf{Given:}  
5-year, 11\% annual coupon bond, Price = 86.59, YTM = 15\%, MacDur = 4.03.

\[
\text{ModDur} = \frac{4.03}{1.15} = 3.50
\]

\textbf{Interpretation:}
\[
\text{Approx.\% price change for 0.5\% yield increase} = -3.50 \times 0.005 = -1.75\%.
\]
\[
\text{New Price} = 86.59 \times (1 - 0.0175) = 85.09 \text{ (approx.)}
\]

\textbf{Check:}
\[
\text{Exact Price at 15.5\% YTM} = 85.092 \Rightarrow \text{Estimation accuracy confirmed.}
\]



\paragraph{4. Approximate Modified Duration (Finite Difference Method)}
\[
\boxed{
\text{Approx.\ ModDur} = \frac{V_{-} - V_{+}}{2 \times V_0 \times \Delta \text{YTM}}
}
\]
where:
\begin{itemize}
    \item \(V_{-}\): Price if YTM decreases by $\Delta$YTM.
    \item \(V_{+}\): Price if YTM increases by $\Delta$YTM.
    \item \(V_0\): Current bond price.
    \item $\Delta$YTM is entered as a decimal (e.g., 0.005 for 50 bps).
\end{itemize}

\paragraph{Example 2: Approximate ModDur}
\textbf{Given:} 5-year, 11\% bond, Price = 86.59, $\Delta$YTM = 0.005.

\[
V_{+} = 85.092, \quad V_{-} = 88.127.
\]
\[
\text{Approx.\ ModDur} = \frac{88.127 - 85.092}{2 \times 86.59 \times 0.005} = 3.505.
\]

\textbf{Result:} Approximation very close to theoretical ModDur = 3.50.



\paragraph{5. Interpretation of ModDur}
\begin{itemize}
    \item ModDur measures \textbf{first-order price sensitivity}.
    \item Linear approximation of convex (curved) price–yield relationship.
    \item Accurate for small yield changes (≤ 100 bps).
    \item Underestimates price increase and overestimates price decrease for larger yield moves due to convexity.
\end{itemize}

\begin{figure}[h!]
\centering
\fbox{\parbox{0.85\linewidth}{
\textbf{Figure Concept: Price–Yield Curve}  
\\
Curved relationship → convex shape.  
\\
ModDur is tangent slope — linear approximation around current YTM.
}}
\end{figure}



\paragraph{6. Money Duration (Dollar Duration)}
\[
\boxed{
\text{Money Duration} = \text{ModDur} \times \text{Full Price of Bond Position}
}
\]
\begin{itemize}
    \item Expressed in currency terms.
    \item Represents dollar change for a 1\% change in YTM.
\end{itemize}

\paragraph{Example 3: Money Duration}
\textbf{Given:}  
ModDur = 7.42, Full Price = 101.32, Par = \$2,000,000.

\[
\text{Money Duration} = 7.42 \times 101.32\% \times 2,000,000 = 15,035,888.
\]

\textbf{Per \$100 of Par:}
\[
7.42 \times 101.32 = 751.79.
\]

\textbf{Impact of 25 bp Yield Increase:}
\[
15,035,888 \times 0.0025 = 37,589.72 \Rightarrow \text{Loss of \$37,589.72.}
\]



\paragraph{7. Price Value of a Basis Point (PVBP)}
\[
\boxed{
\text{PVBP} = \frac{V_{-} - V_{+}}{2}
}
\]

\begin{itemize}
    \item Change in bond’s full price for a 1 bp (0.01\%) change in YTM.
    \item Expressed in currency units per \$100 par.
\end{itemize}

\paragraph{Example 4: PVBP Calculation}
\textbf{Given:} 20-year, 6\% annual bond, Price = 101.39, Par = \$1,000,000.

\[
YTM = 5.88\%, \quad V_{+} = 101.273, \quad V_{-} = 101.507.
\]
\[
\text{PVBP per \$100 par} = \frac{101.507 - 101.273}{2} = 0.117.
\]
\[
\text{PVBP total} = 0.117 \times \$1,000,000 \times 0.01 = \$1,170.
\]

\textbf{Alternative:}
\[
\text{PVBP} = \text{Money Duration} \times 0.0001.
\]



\begin{table}[h!]
\centering
\caption*{Exhibit 1: Summary of Duration and PVBP Relationships}
\begin{tabular}{|l|l|}
\hline
\textbf{Measure} & \textbf{Formula / Interpretation} \\
\hline
Macaulay Duration & Weighted average time to receive CFs \\
\hline
Modified Duration & \( \dfrac{\text{MacDur}}{1 + \text{YTM}/m} \) — \% price change per 1\% $\Delta$YTM \\
\hline
Money Duration & \( \text{ModDur} \times \text{Full Price} \) — \$ change per 1\% $\Delta$YTM \\
\hline
PVBP & \( \dfrac{V_{-} - V_{+}}{2} \) — \$ change per 1 bp $\Delta$YTM \\
\hline
Relationship & PVBP = Money Duration × 0.0001 \\
\hline
\end{tabular}
\end{table}



\subsection*{LOS 59.b: Factors Affecting Interest Rate Risk}

\paragraph{1. Maturity Effect}
\begin{itemize}
    \item Longer maturity $\Rightarrow$ greater interest rate risk (usually).
    \item Distant CFs more sensitive to discount rate changes.
    \item Exception: for deep-discount bonds, duration increases then decreases with maturity.
\end{itemize}

\paragraph{2. Coupon Rate Effect}
\begin{itemize}
    \item Higher coupon $\Rightarrow$ lower duration $\Rightarrow$ less interest rate risk.
    \item More CFs received sooner, reducing sensitivity to yield changes.
    \item Zero-coupon bond has highest duration for given maturity.
\end{itemize}

\paragraph{3. Yield Level Effect}
\begin{itemize}
    \item Higher YTM $\Rightarrow$ lower duration $\Rightarrow$ lower price sensitivity.
    \item Due to flatter (less steep) slope of price–yield curve at high yields.
\end{itemize}

\paragraph{4. Coupon Reset (for FRNs)}
\begin{itemize}
    \item Duration $\approx$ time to next reset date.
    \item Frequent resets $\Rightarrow$ low price sensitivity.
\end{itemize}

\paragraph{5. Time Passage Effect (Rolling Down the Curve)}
\begin{itemize}
    \item Between coupon payments, duration decreases smoothly as time passes.
    \item After coupon payment, duration jumps upward slightly as time to next payment resets.
\end{itemize}



\begin{table}[h!]
\centering
\caption*{Exhibit 2: Factors Influencing Duration and Interest Rate Sensitivity}
\begin{tabular}{|p{4cm}|p{4cm}|p{8cm}|}
\hline
\textbf{Factor} & \textbf{Effect on Duration} & \textbf{Interpretation} \\
\hline
Maturity ↑ & Duration ↑ & Long-term bonds = higher price sensitivity. \\
\hline
Coupon Rate ↑ & Duration ↓ & High-coupon bonds = lower interest rate risk. \\
\hline
YTM ↑ & Duration ↓ & Higher yields = flatter curve, less sensitivity. \\
\hline
FRN Reset Frequency ↑ & Duration ↓ & More frequent resets = lower risk. \\
\hline
Time (between coupons) & Duration ↓ & Duration declines steadily until coupon date, then resets. \\
\hline
\end{tabular}
\end{table}



\subsection*{Key Takeaways Summary}

\begin{itemize}
    \item \textbf{Modified Duration:} Measures \% change in price for 1\% $\Delta$YTM.
    \item \textbf{Approx.\ ModDur:} Computed using $V_{-}$ and $V_{+}$ around current yield.
    \item \textbf{Money Duration:} Currency change for 1\% $\Delta$YTM = ModDur × Price.
    \item \textbf{PVBP:} Currency change for 1 bp = Money Duration × 0.0001.
    \item \textbf{Interest Rate Risk Drivers:}
        \begin{itemize}
            \item Longer maturity → higher risk.
            \item Lower coupon → higher risk.
            \item Lower YTM → higher risk.
            \item FRNs and high-coupon bonds → lower risk.
        \end{itemize}
\end{itemize}



\begin{table}[h!]
\centering
\caption*{Exhibit 3: Formula Summary}
\begin{tabular}{|l|l|}
\hline
\textbf{Concept} & \textbf{Formula} \\
\hline
Modified Duration & \( \text{ModDur} = \dfrac{\text{MacDur}}{1 + \text{YTM}/m} \) \\
\hline
Approx.\ ModDur & \( \dfrac{V_{-} - V_{+}}{2V_0 \Delta \text{YTM}} \) \\
\hline
Money Duration & \( \text{ModDur} \times \text{Full Price} \) \\
\hline
PVBP & \( \dfrac{V_{-} - V_{+}}{2} \) or \( \text{Money Duration} \times 0.0001 \) \\
\hline
Price Change Estimate & \( \dfrac{\Delta P}{P} \approx -\text{ModDur} \times \Delta \text{YTM} \) \\
\hline
Interest Rate Sensitivity & ↑ Maturity, ↓ Coupon, ↓ YTM → ↑ Duration \\
\hline
\end{tabular}
\end{table}

\section*{Module 60.1: Yield-Based Bond Convexity and Portfolio Properties}

\subsection*{LOS 60.a: Calculate and Interpret Convexity and Describe the Convexity Adjustment}

\paragraph{1. Concept Overview}
\begin{itemize}
    \item Duration gives a \textbf{linear} (first-order) estimate of price sensitivity to yield changes.
    \item The actual price–yield curve is \textbf{convex}; thus, duration alone underestimates price increases and overestimates price decreases.
    \item \textbf{Convexity} measures the curvature (second-order sensitivity) of the price–yield relationship.
    \item Adding convexity to duration improves price-change estimates, especially for larger $\Delta$YTM.
\end{itemize}



\paragraph{2. Formula: Convexity of a Single Cash Flow}
\[
\boxed{
\text{Convexity of CF}_t = \frac{t(t+1)}{(1+r)^{t+2}}
}
\]
where $r$ is the periodic yield ($\text{YTM}/m$).

\paragraph{Bond Convexity (Weighted Average):}
\[
\boxed{
\text{Convexity} = \frac{1}{P_0} \sum_{t=1}^{N} \frac{CF_t \times t(t+1)}{(1+r)^{t+2}}
}
\]
\begin{itemize}
    \item \( CF_t \): Cash flow at period \( t \).
    \item \( P_0 \): Full price (present value) of the bond.
    \item For semiannual bonds, divide by $m^2$ (typically $m = 2$) to annualize.
\end{itemize}



\paragraph{3. Example 1: Convexity by Cash-Flow Weighting}

\textbf{Bond:} 5-year, 11\% annual coupon, price = 86.59, YTM = 15\%, MacDur = 4.03.  

\begin{table}[h!]
\centering
\caption*{Exhibit 1: Convexity of Each Cash Flow}
\begin{tabular}{|c|c|c|c|c|}
\hline
\textbf{Year} & \textbf{Cash Flow (\$)} & \textbf{Weight} & \textbf{CF Convexity Term} & \textbf{Weighted Contribution} \\
\hline
1 & 11 & 0.1105 & $(1\times2)/1.15^2=1.512$ & $0.1669$ \\
2 & 11 & 0.0961 & $(2\times3)/1.15^4=4.537$ & $0.4359$ \\
3 & 11 & 0.0835 & $(3\times4)/1.15^6=9.074$ & $0.7583$ \\
4 & 11 & 0.0726 & $(4\times5)/1.15^8=15.123$ & $1.0979$ \\
5 & 111 & 0.6373 & $(5\times6)/1.15^{10}=22.684$ & $14.4560$ \\
\hline
\multicolumn{4}{|r|}{\textbf{Bond Convexity:}} & \textbf{16.915} \\
\hline
\end{tabular}
\end{table}

\[
\boxed{\text{Convexity} = 16.915}
\]



\paragraph{4. Approximate Convexity (Finite-Difference Formula)}
\[
\boxed{
\text{Approx.\ Convexity} =
\frac{V_{-} + V_{+} - 2V_0}{V_0(\Delta YTM)^2}
}
\]
where
\begin{itemize}
    \item $V_{-}$ = bond price if yield decreases by $\Delta$YTM,  
    \item $V_{+}$ = bond price if yield increases by $\Delta$YTM,  
    \item $V_0$ = current bond price,  
    \item $\Delta$YTM in decimal form (e.g., 0.005 for 50 bps).
\end{itemize}

\paragraph{Example 2: Approximate Convexity}
\[
V_0=86.59138, \; V_{+}=85.09217, \; V_{-}=88.12721, \; \Delta YTM=0.005
\]
\[
\text{Approx.\ Convexity} = \frac{88.12721 + 85.09217 - 2(86.59138)}{86.59138(0.005)^2} = 16.92
\]

\textbf{Interpretation:}  
Approximation matches true convexity ≈ 16.9.



\paragraph{5. Characteristics Affecting Convexity}
\begin{table}[h!]
\centering
\caption*{Exhibit 2: Determinants of Convexity}
\begin{tabular}{|l|l|}
\hline
\textbf{Characteristic} & \textbf{Effect on Convexity} \\
\hline
Longer maturity & Increases convexity \\
\hline
Lower coupon rate & Increases convexity \\
\hline
Lower YTM & Increases convexity \\
\hline
Greater CF dispersion & Higher convexity (for equal duration) \\
\hline
\end{tabular}
\end{table}



\subsection*{LOS 60.b: Estimate Price Change Using Duration and Convexity}

\paragraph{1. Combined Duration–Convexity Approximation}
\[
\boxed{
\frac{\Delta P}{P}
\approx -(\text{ModDur})\Delta y + \tfrac{1}{2}(\text{Convexity})(\Delta y)^2
}
\]
where $\Delta y$ = change in yield (decimal).



\paragraph{Example 3: Price Estimate with Duration + Convexity}
\textbf{Given:} 5-year 11\% bond, $P_0=86.59138$, $\text{ModDur}=3.50$, $\text{Convexity}=16.9$, $\Delta y=-0.005$.

\begin{align*}
\text{Duration Effect} &= -3.50(-0.005)=+0.0175=1.75\%\\
\text{Convexity Effect} &= 0.5(16.9)(-0.005)^2 = +0.000211 = 0.0211\%\\
\text{Total \% Change} &= 1.75\% + 0.0211\% = 1.7711\%\\
\text{Estimated New Price} &= 86.59138(1.017711)=\mathbf{88.125}
\end{align*}

\textbf{Interpretation:} Convexity adds a small positive adjustment improving accuracy of duration-only estimates.



\paragraph{2. Money Duration and Money Convexity}
\[
\boxed{
\begin{aligned}
\text{Money Duration} &= \text{ModDur} \times \text{Full Price (Position)}\\
\text{Money Convexity} &= \text{Convexity} \times \text{Full Price (Position)}
\end{aligned}
}
\]

\paragraph{Example 4: Money Measures for a \$10 million Position}
\[
P_0 = 0.8659138 \times 10,000,000 = 8,659,138
\]
\[
\text{MoneyDur} = 3.50 \times 8,659,138 = 30,306,983
\]
\[
\text{MoneyConv} = 16.9 \times 8,659,138 = 146,339,432
\]

\textbf{Estimate Change for } $\Delta y = -0.005$:
\[
\text{Duration Effect} = - (30,306,983)(-0.005) = +151,534.9
\]
\[
\text{Convexity Effect} = 0.5(146,339,432)(-0.005)^2 = +1,829.3
\]
\[
\text{Total Change} = 151,534.9 + 1,829.3 = 153,364.2
\]
\[
\text{New Value} = 8,659,138 + 153,364.2 = \mathbf{8,812,502}
\]

\textbf{Consistency Check:} Matches price estimate ≈ 88.125 from previous example.



\paragraph{3. Convexity Adjustment Interpretation}
\begin{itemize}
    \item The convexity adjustment is always \textbf{positive} for a bond with positive convexity.
    \item \textbf{When yields fall:} Duration-only estimate underestimates price increase → convexity correction adds to price.
    \item \textbf{When yields rise:} Duration-only estimate overestimates price drop → convexity correction reduces loss.
\end{itemize}

\begin{figure}[h!]
\centering
\fbox{\parbox{0.85\linewidth}{
\textbf{Figure Concept: Duration vs. Actual Price Curve} \\
-- Duration line: tangent, linear. \\
-- True curve: convex upward. \\
-- Convexity term adds curvature, improving both sides of estimate.
}}
\end{figure}



\subsection*{LOS 60.c: Portfolio Duration and Convexity}

\paragraph{1. Two Approaches}

\begin{enumerate}
    \item \textbf{Aggregate Cash-Flow Approach (Theoretical):}  
    Compute a single duration and convexity based on the portfolio’s total cash-flow stream.
    \[
    \text{Use aggregate CFs to find } P_0, \; D, \; \text{and Convexity.}
    \]
    \item \textbf{Weighted-Average Approach (Practical):}
    \[
    \boxed{
    \begin{aligned}
    D_{p} &= \sum_{i=1}^{n} w_i D_i \\
    C_{p} &= \sum_{i=1}^{n} w_i C_i
    \end{aligned}
    }
    \]
    where $w_i =$ market-value weight of bond $i$ in portfolio.
\end{enumerate}



\paragraph{2. Limitation of Weighted Approach}
\begin{itemize}
    \item Assumes a \textbf{parallel shift} in the yield curve — all maturities change by the same $\Delta$YTM.
    \item In practice, yield-curve changes often involve twists or steepening/flattening.
    \item For nonparallel shifts, the weighted-average method will misestimate portfolio price change.
\end{itemize}

\begin{table}[h!]
\centering
\caption*{Exhibit 3: Portfolio Duration and Convexity Properties}
\begin{tabular}{|l|p{10cm}|}
\hline
\textbf{Approach} & \textbf{Description / Limitation} \\
\hline
Aggregate CF & Exact but computationally intensive. Requires full cash-flow projection. \\
\hline
Weighted Average & Practical; uses individual bond durations/convexities and market-value weights. Assumes parallel yield shifts. \\
\hline
\end{tabular}
\end{table}



\subsection*{Summary Tables and Key Formulas}

\begin{table}[h!]
\centering
\caption*{Exhibit 4: Duration–Convexity–Price Relationship}
\begin{tabular}{|l|l|}
\hline
\textbf{Measure} & \textbf{Formula / Interpretation} \\
\hline
Convexity (cash-flow basis) & $\displaystyle \frac{1}{P_0}\sum \frac{CF_t t(t+1)}{(1+r)^{t+2}}$ \\
\hline
Approx.\ Convexity & $\displaystyle \frac{V_{-}+V_{+}-2V_0}{V_0(\Delta y)^2}$ \\
\hline
Price Change (\%) & $\displaystyle -\text{ModDur}\Delta y + 0.5(\text{Convexity})(\Delta y)^2$ \\
\hline
Money Duration & $\text{ModDur}\times \text{Full Price}$ \\
\hline
Money Convexity & $\text{Convexity}\times \text{Full Price}$ \\
\hline
Portfolio Duration & $\sum w_i D_i$ (approx., assumes parallel shift) \\
\hline
Portfolio Convexity & $\sum w_i C_i$ (approx., assumes parallel shift) \\
\hline
\end{tabular}
\end{table}



\subsection*{Key Takeaways}

\begin{itemize}
    \item \textbf{Convexity} captures the curvature of the bond’s price–yield relationship and refines duration estimates.
    \item \textbf{Positive convexity} implies that duration-only estimates understate price gains and overstate price losses.
    \item \textbf{Approx.\ Price Change:}
        \[
        \frac{\Delta P}{P} \approx -D_{mod}\Delta y + \tfrac{1}{2}C(\Delta y)^2
        \]
    \item \textbf{Money Measures:}
        \[
        \begin{aligned}
        \text{MoneyDur} &= D_{mod} \times P_0 \\
        \text{MoneyConv} &= C \times P_0
        \end{aligned}
        \]
    \item \textbf{Portfolio Measures:}
        \[
        D_p = \sum w_i D_i, \quad C_p = \sum w_i C_i
        \]
        Valid only under parallel yield-curve shifts.
    \item \textbf{Drivers of Convexity:}
        \begin{itemize}
            \item ↑ Maturity → ↑ Convexity  
            \item ↓ Coupon → ↑ Convexity  
            \item ↓ YTM → ↑ Convexity
        \end{itemize}
\end{itemize}

\section*{Module 61.1: Curve-Based and Empirical Fixed-Income Risk Measures}

\subsection*{LOS 61.a: Why Effective Duration and Effective Convexity Are Appropriate for Bonds with Embedded Options}

\paragraph{1. Key Concept: Uncertain Cash Flows}
\begin{itemize}
    \item For \textbf{option-free bonds}, cash flows and maturities are known $\Rightarrow$ YTM-based duration is applicable.
    \item For \textbf{bonds with embedded options} (e.g., callable, putable, MBS):
    \begin{itemize}
        \item Future cash flows and redemption dates are \textbf{uncertain}.
        \item Exercise of the embedded option changes timing and amount of CFs.
        \item These bonds have \textbf{no single well-defined YTM}.
    \end{itemize}
\end{itemize}

\paragraph{2. Option Equivalence}
\begin{itemize}
    \item Callable Bond = Straight Bond $-$ Call Option (issuer owns call right).
    \item Putable Bond = Straight Bond $+$ Put Option (investor owns put right).
    \item MBS = Straight Bond $-$ Prepayment Option (borrowers can prepay).
\end{itemize}



\paragraph{3. Effective Duration (EffDur)}
\[
\boxed{
\text{EffDur} = \frac{V_{-} - V_{+}}{2V_0 \Delta \text{Curve}}
}
\]
where:
\begin{itemize}
    \item $V_{-}$ = bond value when benchmark curve decreases by $\Delta$Curve.
    \item $V_{+}$ = bond value when benchmark curve increases by $\Delta$Curve.
    \item $V_0$ = current bond value.
    \item $\Delta$Curve = change in benchmark yield (decimal, e.g., 0.01 for 1\%).
\end{itemize}

\paragraph{Interpretation:}
\begin{itemize}
    \item Measures price sensitivity to \textbf{benchmark yield curve shifts}.
    \item Reflects the option’s effect on future cash flows.
    \item Assumes spreads for credit and liquidity risk remain constant.
\end{itemize}



\paragraph{4. Effective Convexity (EffCon)}
\[
\boxed{
\text{EffCon} = \frac{V_{-} + V_{+} - 2V_0}{V_0 (\Delta \text{Curve})^2}
}
\]

\paragraph{Interpretation:}
\begin{itemize}
    \item Captures curvature of price–yield relation for bonds with embedded options.
    \item Uses changes in \textbf{benchmark curve} rather than YTM.
\end{itemize}



\paragraph{5. Convexity Profiles by Bond Type}
\begin{table}[h!]
\centering
\caption*{Exhibit 1: Convexity Behavior by Bond Type}
\begin{tabular}{|p{4cm}|p{4cm}|p{6cm}|}
\hline
\textbf{Bond Type} & \textbf{Convexity Type} & \textbf{Reason} \\
\hline
Option-Free & Positive & Price–yield relation always convex. \\
\hline
Callable & Can be Negative (at low yields) & Call price caps price increase; issuer likely to call when rates fall. \\
\hline
Putable & Always Positive & Put option limits price drop at high yields. \\
\hline
MBS & Negative (similar to callable) & Borrowers prepay when rates fall. \\
\hline
\end{tabular}
\end{table}

\paragraph{6. Price–Yield Relationships}
\begin{itemize}
    \item Callable bonds flatten upward (negative convexity) when rates fall.
    \item Putable bonds flatten downward (positive convexity reinforced) when rates rise.
\end{itemize}

\begin{figure}[h!]
\centering
\fbox{\parbox{0.9\linewidth}{
\textbf{Figure Concept:} Callable vs. Option-Free vs. Putable Price–Yield Curves \\
- Callable bond: price capped at call price (negative convexity region). \\
- Putable bond: price floor near put price (always positive convexity). \\
}}
\end{figure}



\paragraph{7. Modified vs. Effective Duration (Option-Free Bonds)}
\begin{itemize}
    \item For option-free bonds, ModDur ≈ EffDur when yield curve is flat.
    \item In non-flat curves, $\Delta$ in par curve $\Rightarrow$ nonparallel shift in spot curve.
    \item Hence, ModDur and EffDur differ slightly due to:
    \begin{itemize}
        \item Nonparallel shifts of government spot rates.
        \item Constant credit spread assumption.
    \end{itemize}
\end{itemize}



\subsection*{LOS 61.b: Estimate Price Change Using Effective Duration and Convexity}

\paragraph{1. Formula for Price Change (with Respect to Benchmark Curve)}
\[
\boxed{
\frac{\Delta P}{P_0} \approx -(\text{EffDur}) \times \Delta \text{Curve} + \frac{1}{2}(\text{EffCon}) \times (\Delta \text{Curve})^2
}
\]

\paragraph{2. Example 1: Price Change Using Effective Measures}
\textbf{Given:}  
EffDur = 10.5, EffCon = 97.3, $\Delta \text{Curve} = -0.02$ (−200 bps).

\begin{align*}
\text{Duration Effect} &= -10.5(-0.02) = +0.21 = 21.0\% \\
\text{Convexity Effect} &= 0.5(97.3)(-0.02)^2 = 0.0195 = 1.95\% \\
\text{Total Change} &= 21.0 + 1.95 = \mathbf{22.95\%}
\end{align*}

\textbf{Interpretation:} Bond price expected to rise 22.95\% for a 200 bp curve drop.



\paragraph{3. Effective vs. Modified Duration Summary}
\begin{table}[h!]
\centering
\caption*{Exhibit 2: Effective vs. Modified Duration}
\begin{tabular}{|p{3cm}|p{4cm}|p{6cm}|}
\hline
\textbf{Aspect} & \textbf{Modified Duration} & \textbf{Effective Duration} \\
\hline
Applicable to & Option-free bonds & Bonds with embedded options \\
\hline
Yield Input & Bond’s YTM & Benchmark yield curve shift ($\Delta$Curve) \\
\hline
CF Certainty & Fixed & Uncertain (depends on option exercise) \\
\hline
Risk Captured & Total yield change (benchmark + spread) & Benchmark-only change \\
\hline
Use of Models & Analytical & Requires bond-pricing model \\
\hline
\end{tabular}
\end{table}



\subsection*{LOS 61.c: Key Rate Duration (Partial Duration)}

\paragraph{1. Concept}
\begin{itemize}
    \item Effective duration assumes \textbf{parallel yield curve shifts}.
    \item Key Rate Duration (KRD) measures price sensitivity to changes in the benchmark yield for a \textbf{specific maturity}, with other yields held constant.
    \item The sum of all key rate durations equals the effective duration.
\end{itemize}

\[
\boxed{
D_{\text{Eff}} = \sum_{i=1}^{n} D_{\text{KeyRate},i}
}
\]

\paragraph{2. Application: Measuring Shaping Risk}
\begin{itemize}
    \item Shaping risk = risk from \textbf{nonparallel} yield curve shifts (steepening, flattening, twists).
    \item KRDs allow assessment of sensitivity at individual maturities (e.g., 2y, 5y, 10y, 30y).
\end{itemize}



\paragraph{Example 2: Portfolio Key Rate Duration}

\textbf{Portfolio:}
\begin{itemize}
    \item 50\% in 5-year zero-coupon bond ($y=5\%$).
    \item 50\% in 10-year bond ($y=6\%$).
\end{itemize}

\textbf{Step 1. Compute Modified Durations:}
\[
D_{5} = \frac{5}{1.05} = 4.762, \quad D_{10} = \frac{10}{1.06} = 9.434
\]

\textbf{Step 2. Compute Key Rate Durations:}
\[
\text{KRD}_{5} = 4.762 \times 0.5 = 2.381, \quad
\text{KRD}_{10} = 9.434 \times 0.5 = 4.717
\]

\textbf{Step 3. Apply Yield Changes:}
\[
\Delta y_5 = +0.005, \quad \Delta y_{10} = -0.0025
\]

\[
\text{Impact (5-year)} = -2.381(0.005) = -0.0119 = -1.19\%
\]
\[
\text{Impact (10-year)} = -4.717(-0.0025) = +0.0118 = +1.18\%
\]
\[
\text{Total Change} = -1.19\% + 1.18\% = \mathbf{-0.01\%} \text{ (negligible change).}
\]

\textbf{Interpretation:}  
Portfolio is well balanced for this nonparallel yield movement.



\begin{table}[h!]
\centering
\caption*{Exhibit 3: Key Rate Duration Characteristics}
\begin{tabular}{|l|l|}
\hline
\textbf{Feature} & \textbf{Explanation} \\
\hline
Definition & Sensitivity to yield change at specific maturity \\
\hline
Sum Property & $\sum \text{KRD}_i = \text{EffDur}$ \\
\hline
Captures & Nonparallel yield curve shifts (shape changes) \\
\hline
Uses & Portfolio immunization, yield-curve twist risk control \\
\hline
\end{tabular}
\end{table}



\subsection*{LOS 61.d: Empirical Duration vs. Analytical Duration}

\paragraph{1. Analytical Duration}
\begin{itemize}
    \item Based on mathematical models and present value relationships.
    \item Examples: Macaulay, Modified, and Effective Duration.
    \item Assume:
        \begin{itemize}
            \item Parallel yield curve shifts.
            \item Constant credit spread.
        \end{itemize}
\end{itemize}

\paragraph{2. Empirical Duration}
\begin{itemize}
    \item Derived from \textbf{observed historical data}:
    \[
    \text{Empirical Duration} = \frac{\Delta P / P}{\Delta \text{Benchmark Yield}}
    \]
    \item Captures the \textbf{actual co-movement} of bond prices and benchmark yields.
\end{itemize}

\paragraph{3. When Empirical Duration is Preferred}
\begin{itemize}
    \item When changes in benchmark yields and spreads are \textbf{correlated}.
    \item Especially for \textbf{credit-risky bonds} (e.g., corporates).
    \item In “flight-to-quality” scenarios:
        \begin{itemize}
            \item Gov’t yields ↓ (prices ↑).
            \item Corporate spreads ↑ (prices ↓ or stable).
            \item Analytical duration overstates sensitivity; empirical duration is smaller.
        \end{itemize}
\end{itemize}



\begin{table}[h!]
\centering
\caption*{Exhibit 4: Analytical vs. Empirical Duration}
\begin{tabular}{|p{3cm}|p{4cm}|p{6cm}|}
\hline
\textbf{Aspect} & \textbf{Analytical Duration} & \textbf{Empirical Duration} \\
\hline
Basis & Model / PV relationship & Historical regression data \\
\hline
Assumes & Constant spreads, parallel curve shifts & Real-world correlations (spreads + yields) \\
\hline
Best for & Government or low-credit-risk bonds & Corporate or spread-sensitive bonds \\
\hline
Example Scenario & Flat curve shift & Flight-to-quality (credit spreads widen) \\
\hline
Value of Duration & Typically higher & Typically lower \\
\hline
\end{tabular}
\end{table}



\subsection*{Key Takeaways Summary}

\begin{itemize}
    \item \textbf{EffDur / EffCon:}  
    Best for option-embedded bonds (uncertain CFs).  
    Measured w.r.t. benchmark yield curve, not YTM.
    \[
    \frac{\Delta P}{P} \approx -(\text{EffDur})\Delta \text{Curve} + 0.5(\text{EffCon})(\Delta \text{Curve})^2
    \]
    \item \textbf{Callable Bonds:} May exhibit negative convexity at low yields.
    \item \textbf{Putable Bonds:} Always have positive convexity.
    \item \textbf{Key Rate Duration:}  
    Measures partial sensitivity to specific maturities — captures nonparallel shifts.
    \item \textbf{Empirical Duration:}  
    Based on observed data; preferred for corporate bonds when credit spreads move with yields.
\end{itemize}



\begin{table}[h!]
\centering
\caption*{Exhibit 5: Summary Formulas}
\begin{tabular}{|l|l|}
\hline
\textbf{Concept} & \textbf{Formula} \\
\hline
Effective Duration & $\displaystyle \frac{V_{-} - V_{+}}{2V_0 \Delta \text{Curve}}$ \\
\hline
Effective Convexity & $\displaystyle \frac{V_{-} + V_{+} - 2V_0}{V_0 (\Delta \text{Curve})^2}$ \\
\hline
Price Change (\%) & $-\text{EffDur}(\Delta \text{Curve}) + 0.5(\text{EffCon})(\Delta \text{Curve})^2$ \\
\hline
Key Rate Duration (KRD) & Sensitivity to yield at one maturity (others constant) \\
\hline
Portfolio Effective Duration & $\sum w_i D_i$ (parallel shift assumption) \\
\hline
Empirical Duration & Observed $\frac{\Delta P / P}{\Delta \text{Benchmark Yield}}$ \\
\hline
\end{tabular}
\end{table}

\section*{Module 62.1: Credit Risk}

\subsection*{LOS 62.a: Describe Credit Risk and Its Components}

\paragraph{1. Definition and Nature of Credit Risk}
\begin{itemize}
    \item \textbf{Credit risk:} The risk of loss to a fixed-income investor due to a borrower's failure to pay interest or principal (i.e., failure to service its debt).
    \item A borrower in default fails to meet contractual debt obligations.
    \item Credit risk can be analyzed from:
    \begin{itemize}
        \item \textbf{Bottom-up (issuer-specific factors).}
        \item \textbf{Top-down (macroeconomic and systemic factors).}
    \end{itemize}
\end{itemize}



\paragraph{2. Bottom-Up Credit Factors (The 5 Cs of Credit Analysis)}
\begin{table}[h!]
\centering
\caption*{Exhibit 1: Bottom-Up Credit Analysis – The 5 Cs}
\begin{tabular}{|l|l|}
\hline
\textbf{Factor} & \textbf{Explanation} \\
\hline
Capacity & Ability to make timely debt payments (e.g., interest coverage, cash flows). \\
\hline
Capital & Net worth and other funding sources that reduce reliance on debt. \\
\hline
Collateral & Assets pledged as security for debt repayment. \\
\hline
Covenants & Legal restrictions protecting lenders (e.g., leverage limits, payout limits). \\
\hline
Character & Integrity and reliability of management or borrower. \\
\hline
\end{tabular}
\end{table}



\paragraph{3. Top-Down Credit Factors (The 3 Cs)}
\begin{table}[h!]
\centering
\caption*{Exhibit 2: Top-Down Factors Affecting Creditworthiness}
\begin{tabular}{|l|l|}
\hline
\textbf{Factor} & \textbf{Explanation} \\
\hline
Conditions & General macroeconomic environment affecting debt servicing ability. \\
\hline
Country & Political/legal stability and fiscal institutions of the borrower’s country. \\
\hline
Currency & FX fluctuations affecting foreign-denominated debt repayment. \\
\hline
\end{tabular}
\end{table}



\paragraph{4. Key Sources of Repayment by Borrower Type}
\begin{itemize}
    \item \textbf{Corporate Debt:}
    \begin{itemize}
        \item Secured debt $\rightarrow$ backed by operations + collateral cash flows.
        \item Unsecured debt $\rightarrow$ backed only by operating CFs and assets.
    \end{itemize}
    \item \textbf{Sovereign Debt:}
    \begin{itemize}
        \item Backed by tax revenues, tariffs, and fees.
        \item Secondary sources: privatizations, additional debt issuance.
    \end{itemize}
\end{itemize}



\paragraph{5. Liquidity vs. Insolvency}
\begin{itemize}
    \item \textbf{Illiquidity:} Inability to raise cash (temporary shortfall).
    \item \textbf{Insolvency:} Assets $<$ Liabilities (structural deficiency).
\end{itemize}



\paragraph{6. Bond Indenture Clauses}
\begin{itemize}
    \item \textbf{Cross-Default:} Default on one issue $\Rightarrow$ default on all issues.
    \item \textbf{Pari Passu:} All bonds of same type rank equally in default process.
    \item For secured debt, creditors have claim to pledged assets + general assets.
\end{itemize}



\paragraph{7. Measuring Credit Risk}
\[
\boxed{
\text{Expected Loss (EL)} = \text{Probability of Default (POD)} \times \text{Loss Given Default (LGD)}
}
\]

\begin{itemize}
    \item \textbf{Probability of Default (POD):} Annualized probability borrower will fail to make payments.
    \item \textbf{Loss Given Default (LGD):} Monetary or percentage loss incurred if default occurs.
    \item \textbf{Recovery Rate:} Fraction of the claim recovered in default.
    \[
    \text{LGD\%} = \text{Expected Exposure} \times (1 - \text{Recovery Rate})
    \]
\end{itemize}



\paragraph{8. Credit Spread as Compensation for Expected Loss}
\[
\boxed{
\text{Credit Spread} \approx \text{POD} \times \text{LGD\%}
}
\]
\begin{itemize}
    \item If actual credit spread $>$ estimated spread → investor overcompensated.
    \item If actual credit spread $<$ estimated spread → investor undercompensated.
\end{itemize}



\paragraph{Example 1: Expected Loss and Credit Spread}
\begin{align*}
\text{POD} &= 3\% = 0.03, \quad \text{Recovery Rate} = 75\% \\
\text{LGD\%} &= 1 - 0.75 = 0.25 \\
\text{Estimated Spread} &= 0.03 \times 0.25 = 0.0075 = 0.75\%
\end{align*}
\textbf{Actual Spread:} $4\% - 2.5\% = 1.5\%$ \\
$\Rightarrow$ Bond fairly overcompensates for credit risk (1.5\% > 0.75\%).



\paragraph{9. Relationship Between Credit Quality and Risk Metrics}
\begin{itemize}
    \item High credit quality $\Rightarrow$ low POD, strong fundamentals:
    \[
    \text{High EBIT margin, high interest coverage, low debt/EBITDA, high CF/net debt.}
    \]
    \item LGD depends on:
    \begin{itemize}
        \item Seniority of issue (secured vs. unsecured).
        \item Collateral value.
    \end{itemize}
\end{itemize}



\begin{table}[h!]
\centering
\caption*{Exhibit 3: Example Comparison of Credit Risk Characteristics}
\begin{tabular}{|l|c|c|c|}
\hline
\textbf{Issuer Type} & \textbf{Seniority} & \textbf{POD} & \textbf{LGD\%} \\
\hline
Investment Grade – Unsecured & Senior Unsecured & Low & Moderate \\
\hline
High Yield – Secured & Collateralized & Higher POD & Lower LGD (collateral recovery) \\
\hline
\end{tabular}
\end{table}



\subsection*{LOS 62.b: Credit Ratings – Uses and Limitations}

\paragraph{1. Definition and Purpose}
\begin{itemize}
    \item Ratings reflect forward-looking \textbf{creditworthiness} of:
        \begin{itemize}
            \item Issuer (corporate or sovereign).
            \item Specific debt issue.
        \end{itemize}
    \item Ratings indicate \textbf{expected loss} (POD × LGD).
\end{itemize}



\paragraph{2. Common Uses of Ratings}
\begin{itemize}
    \item Compare credit risk across issuers, sectors, maturities.
    \item Assess \textbf{credit migration risk} (downgrade risk).
    \item Meet \textbf{regulatory, statutory, or contractual} investment requirements.
\end{itemize}



\paragraph{3. Major Credit Rating Agencies and Scales}
\begin{table}[h!]
\centering
\caption*{Exhibit 4: Rating Scale Comparison (S\&P, Moody’s, Fitch)}
\begin{tabular}{|l|l|l|}
\hline
\textbf{Category} & \textbf{S\&P/Fitch} & \textbf{Moody’s} \\
\hline
Highest Quality & AAA & Aaa \\
\hline
High Grade & AA+, AA, AA- & Aa1, Aa2, Aa3 \\
\hline
Upper Medium Grade & A+, A, A- & A1, A2, A3 \\
\hline
Lower Medium Grade (Investment Grade) & BBB+, BBB, BBB- & Baa1, Baa2, Baa3 \\
\hline
Non-Investment Grade (High Yield) & BB+, BB, BB- & Ba1, Ba2, Ba3 \\
\hline
Highly Speculative & B+, B, B- & B1, B2, B3 \\
\hline
Substantial Risk / Default & CCC–C / D & Caa–C \\
\hline
\end{tabular}
\end{table}

\begin{itemize}
    \item \textbf{Investment Grade:} BBB– / Baa3 or higher.
    \item \textbf{High Yield (Junk):} BB+ / Ba1 or lower.
\end{itemize}



\paragraph{4. Limitations of Ratings}
\begin{enumerate}
    \item \textbf{Lag Effect:} Ratings adjust slower than market prices/spreads.
    \item \textbf{Inherent Uncertainty:} Ratings cannot capture all risks (e.g., litigation, disasters, fraud).
    \item \textbf{Agency Errors:} Historical overrating (e.g., subprime 2008 crisis).
    \item \textbf{Split Ratings:} Different agencies may assign different ratings.
\end{enumerate}

\textbf{Investor Implication:}  
Rely on independent analysis — not only on ratings.



\subsection*{LOS 62.c: Factors Influencing Yield Spreads}

\paragraph{1. Definition}
\begin{itemize}
    \item \textbf{Credit spread risk:} The risk that yield spreads widen due to worsening credit conditions.
    \item Affects both price level and volatility of credit-risky bonds.
\end{itemize}



\paragraph{2. Macroeconomic Factors}
\begin{itemize}
    \item Credit spreads narrow during economic expansions (low POD).
    \item Credit spreads widen during recessions (high POD).
    \item \textbf{Investment Grade:} Smaller, less volatile spreads.
    \item \textbf{High Yield:} Larger, more volatile spreads (strong cyclical sensitivity).
\end{itemize}

\paragraph{Typical Spread Curve Behavior:}
\begin{itemize}
    \item Expansions → spreads fall and steepen.
    \item Recessions → spreads rise and flatten (even invert for HY).
\end{itemize}



\paragraph{3. Market Factors}
\begin{itemize}
    \item \textbf{Liquidity Risk:} Cost of trading bonds (bid–offer spread width).
    \item \textbf{Broker/Dealer Regulation:} Limits capital available for market making.
    \item \textbf{Funding Stress:} Increases risk aversion, widens spreads.
    \item \textbf{Supply–Demand Imbalances:} Heavy issuance vs. weak demand widens spreads.
\end{itemize}



\paragraph{4. Issuer-Specific Factors}
\begin{itemize}
    \item Issuer profitability, leverage, coverage ratios.
    \item Worsening financial condition $\Rightarrow$ spread widens.
    \item Analysts compare issuer’s spread to peer averages.
\end{itemize}



\paragraph{5. Example: Decomposing Yield Spread into Credit and Liquidity Components}
\textbf{Given:} 10-year bond, 5\% coupon, Bid/Ask = 99.5/100.5, Benchmark yield = 3\%.

\begin{align*}
\text{Midprice} &= \frac{99.5 + 100.5}{2} = 100 \\
\text{Bond Yield (mid)} &= 5\% \quad \Rightarrow \text{Spread} = 5 - 3 = 2\% \\
\text{Yield}_{bid} &= 5.065\%, \quad \text{Yield}_{ask} = 4.935\% \\
\text{Liquidity Spread} &= 5.065 - 4.935 = 0.13\% \\
\text{Credit Spread} &= 2 - 0.13 = 1.87\%
\end{align*}

\[
\boxed{
\text{Total Yield Spread} = \text{Credit Spread} + \text{Liquidity Spread}
}
\]



\begin{table}[h!]
\centering
\caption*{Exhibit 5: Summary of Spread Determinants}
\begin{tabular}{|l|l|}
\hline
\textbf{Factor Type} & \textbf{Effect on Yield Spread} \\
\hline
Macroeconomic & Widen in recessions, narrow in expansions \\
\hline
Market & Widen with low liquidity or funding stress \\
\hline
Issuer-Specific & Widen with weak credit metrics \\
\hline
Maturity & Longer maturities → higher spreads (greater POD) \\
\hline
Credit Quality & Lower quality → wider spreads, higher volatility \\
\hline
\end{tabular}
\end{table}



\paragraph{6. Estimating Price Change from Spread Change}
\[
\boxed{
\frac{\Delta P}{P} \approx -(\text{Duration}) \times \Delta \text{Spread} + \tfrac{1}{2}(\text{Convexity})(\Delta \text{Spread})^2
}
\]
\textit{Replace $\Delta$YTM with $\Delta$Spread to measure credit spread risk.}



\subsection*{Key Takeaways Summary}

\begin{itemize}
    \item \textbf{Credit Risk =} Probability of Default (POD) × Loss Given Default (LGD).
    \item \textbf{Expected Loss:} 
    \[
    \text{EL} = \text{POD} \times (1 - \text{Recovery Rate})
    \]
    \item \textbf{Credit Spread Compensation:}
    \[
    \text{Credit Spread} \approx \text{POD} \times \text{LGD\%}
    \]
    \item \textbf{Ratings:} Useful benchmarks but lag market information.
    \item \textbf{Spread Drivers:}
    \begin{itemize}
        \item Economic cycle (macro factor)
        \item Issuer fundamentals (micro factor)
        \item Market liquidity and funding (market factor)
    \end{itemize}
    \item \textbf{Decomposition:}
    \[
    \text{Yield Spread} = \text{Credit Spread} + \text{Liquidity Spread}
    \]
\end{itemize}



\begin{table}[h!]
\centering
\caption*{Exhibit 6: Formula Summary}
\begin{tabular}{|l|l|}
\hline
\textbf{Concept} & \textbf{Formula / Relationship} \\
\hline
Expected Loss (EL) & $\text{POD} \times \text{LGD}$ \\
\hline
Loss Given Default (\%) & $\text{Exposure} \times (1 - \text{Recovery Rate})$ \\
\hline
Credit Spread & $\text{POD} \times \text{LGD\%}$ \\
\hline
Total Yield Spread & Credit Spread + Liquidity Spread \\
\hline
Spread-Based Price Change & $-\text{Dur}(\Delta \text{Spread}) + 0.5(\text{Convexity})(\Delta \text{Spread})^2$ \\
\hline
Investment Grade Threshold & BBB– / Baa3 or higher \\
\hline
High Yield Threshold & BB+ / Ba1 or lower \\
\hline
\end{tabular}
\end{table}

\section*{Module 63.1: Credit Analysis for Government Issuers}

\subsection*{LOS 63.a: Explain Special Considerations in Evaluating Credit of Sovereign and Non-Sovereign Government Debt Issuers}

\paragraph{1. Overview}
\begin{itemize}
    \item \textbf{Government (public sector) credit analysis} assesses the ability and willingness of a national or sub-national authority to service its debt obligations.
    \item Two categories:
    \begin{itemize}
        \item \textbf{Sovereign Issuers:} National governments (countries).
        \item \textbf{Non-Sovereign Issuers:} Regional, municipal, agency, or supranational bodies.
    \end{itemize}
    \item \textbf{Key distinction:} Sovereigns can control monetary policy (can print money); non-sovereigns cannot.
\end{itemize}



\subsection*{Sovereign Government Debt Analysis}

\paragraph{2. Primary Source of Repayment}
\begin{itemize}
    \item Derived from the government’s ability to \textbf{tax economic activity}.
    \item Assessment focuses on the long-term stability and credibility of fiscal and monetary frameworks.
\end{itemize}


\paragraph{3. Qualitative Factors (Five Pillars of Sovereign Creditworthiness)}

\begin{table}[h!]
\centering
\caption*{Exhibit 1: Qualitative Factors in Sovereign Credit Analysis}
\begin{tabular}{|p{4cm}|p{12cm}|}
\hline
\textbf{Factor} & \textbf{Description and Analytical Focus} \\
\hline
\textbf{1. Institutions \& Policy Framework} & 
Examines governance quality, political stability, rule of law, property rights, data transparency, and policy credibility.  
-- Economically: enforcement of contracts, culture of debt repayment, data reliability.  
-- Politically: stable regime, peaceful foreign relations.  
-- Willingness to pay is crucial (sovereign immunity = limited legal recourse). \\
\hline
\textbf{2. Fiscal Flexibility} & 
Assesses the ability to raise taxes or reduce expenditures to service debt.  
-- Flexibility decreases with rigid spending (e.g., entitlement programs). \\
\hline
\textbf{3. Monetary Effectiveness} & 
Measures the independence and credibility of the central bank.  
-- Independent banks can control inflation without political interference.  
-- Limited monetary independence $\Rightarrow$ higher inflation and currency risk. \\
\hline
\textbf{4. Economic Flexibility} & 
Evaluates the structure and diversity of the economy.  
-- High GDP per capita, diversified exports, and stable growth indicate stronger creditworthiness. \\
\hline
\textbf{5. External Status} & 
Assesses position in global financial system and foreign reserve adequacy.  
-- Reserve-currency nations can borrow externally in their own currency and sustain higher debt.  
-- Geopolitical risk also affects external standing. \\
\hline
\end{tabular}
\end{table}



\paragraph{4. Quantitative Factors (Three Core Metrics)}

\begin{table}[h!]
\centering
\caption*{Exhibit 2: Quantitative Indicators of Sovereign Credit Strength}
\begin{tabular}{|p{4cm}|p{12cm}|}
\hline
\textbf{Factor} & \textbf{Typical Metrics and Interpretation} \\
\hline
\textbf{1. Fiscal Strength} & 
-- Debt-to-GDP and Debt-to-Revenue ratios: measure debt burden.  
-- Interest-to-GDP or Interest-to-Revenue ratios: measure affordability.  
$\Downarrow$ Lower ratios $\Rightarrow$ stronger fiscal capacity. \\
\hline
\textbf{2. Economic Growth and Stability} & 
-- High real GDP growth, large and diversified economy, stable output.  
-- Low volatility of growth $\Rightarrow$ lower risk. \\
\hline
\textbf{3. External Stability} & 
-- FX reserves to GDP or external debt ratios $\uparrow$ → strong liquidity buffer.  
-- Low short-term external debt (due within 12 months) reduces rollover risk.  
-- Commodity-dependent exporters face concentration risk tied to commodity prices. \\
\hline
\end{tabular}
\end{table}



\paragraph{5. Special Analytical Considerations for Sovereigns}
\begin{itemize}
    \item Sovereigns can refinance via:
        \begin{itemize}
            \item Tax increases.
            \item Domestic debt issuance.
            \item Currency devaluation (via money creation).
        \end{itemize}
    \item Investors must assess \textbf{ability} and \textbf{willingness} to repay.
    \item Political stability and institutional quality are paramount.
\end{itemize}


\subsection*{Non-Sovereign Government Debt}

\paragraph{6. Types of Non-Sovereign Issuers}
\begin{table}[h!]
\centering
\caption*{Exhibit 3: Categories of Non-Sovereign Government Issuers}
\begin{tabular}{|p{4cm}|p{12cm}|}
\hline
\textbf{Issuer Type} & \textbf{Description and Credit Support} \\
\hline
\textbf{Agencies} & 
Quasi-governmental entities for specific functions (e.g., infrastructure, housing).  
Backed by statutory or implicit sovereign guarantees → ratings near sovereign level. \\
\hline
\textbf{Government-Sector Banks / Financing Institutions} & 
Publicly sponsored intermediaries for strategic projects (e.g., green bonds).  
Implied sovereign backing; similar ratings to sovereign. \\
\hline
\textbf{Supranational Organizations} & 
Entities formed by multiple sovereigns (e.g., World Bank, IMF, IDB).  
Credit quality depends on support of member nations and callable capital. \\
\hline
\textbf{Regional / Local Governments} & 
States, provinces, municipalities.  
Debt known as \textbf{municipal bonds} (in the U.S.).  
Typically issued as:
\begin{itemize}
    \item \textbf{General Obligation (GO) Bonds:} Backed by full faith, credit, and taxing power of issuer.  
    \item \textbf{Revenue Bonds:} Backed by specific project revenues (airports, toll roads, hospitals).  
\end{itemize} \\
\hline
\end{tabular}
\end{table}



\paragraph{7. Analytical Distinctions Between Sovereign and Non-Sovereign Debt}
\begin{itemize}
    \item \textbf{Monetary Policy:}  
    -- Sovereigns can create money; non-sovereigns cannot.
    \item \textbf{Budget Constraints:}  
    -- Regional governments usually must maintain balanced budgets.
    \item \textbf{Tax Base:}  
    -- Ability to service GO bonds depends on local economic strength and tax revenues.
    \item \textbf{Project-Specific Risk:}  
    -- Revenue bonds rely solely on project CFs (higher credit risk).
\end{itemize}



\paragraph{8. Revenue Bond Analysis (Corporate-Style Approach)}
\begin{itemize}
    \item Evaluate:
        \begin{itemize}
            \item Stability and predictability of project revenues.
            \item Operating efficiency and cost control.
            \item Debt-Service Coverage Ratio (DSCR):
            \[
            \boxed{
            \text{DSCR} = \frac{\text{Revenue after Operating Costs}}{\text{Interest + Principal Payments}}
            }
            \]
            \item DSCR $>$ 1.0 indicates adequate coverage.
        \end{itemize}
    \item Focus is similar to corporate credit analysis:
        \begin{itemize}
            \item Cash flow predictability.
            \item Leverage and liquidity ratios.
            \item Contingent support (e.g., state guarantees).
        \end{itemize}
\end{itemize}



\paragraph{9. Comparative Example: Sovereign vs. Municipal}
\begin{table}[h!]
\centering
\caption*{Exhibit 4: Comparison of Sovereign and Non-Sovereign (Municipal) Debt}
\begin{tabular}{|p{5cm}|p{5cm}|p{5cm}|}
\hline
\textbf{Feature} & \textbf{Sovereign Bonds} & \textbf{Municipal / Regional Bonds} \\
\hline
Primary Repayment Source & National tax revenues & Local tax revenues / project revenues \\
\hline
Monetary Control & Yes (can print money) & No \\
\hline
Budget Requirement & Flexible deficit financing & Often legally required to balance \\
\hline
Legal Enforcement & Limited (sovereign immunity) & Legal recourse possible within jurisdiction \\
\hline
Currency Risk & Often domestic currency; FX risk for foreign issues & Minimal if local currency \\
\hline
Common Bond Types & Treasury bills, notes, bonds & GO bonds, revenue bonds \\
\hline
Credit Support & Economic/fiscal/monetary policy credibility & Local tax base or project cash flows \\
\hline
Typical Credit Risk Level & Generally lower for reserve-currency sovereigns & Varies by local economic health \\
\hline
\end{tabular}
\end{table}



\paragraph{10. Example: Interpreting Quantitative Ratios for Sovereign Credit Quality}
\begin{itemize}
    \item \textbf{High Credit Quality Sovereign:}
    \begin{itemize}
        \item Low Debt-to-GDP ($<50\%$)
        \item High FX Reserves to External Debt ($>100\%$)
        \item Low Interest-to-Revenue ($<10\%$)
        \item Stable Real GDP Growth ($\sim$2–4\%)
    \end{itemize}
    \item \textbf{Low Credit Quality Sovereign:}
    \begin{itemize}
        \item High Debt-to-GDP ($>90\%$)
        \item Low FX Reserves to External Debt ($<30\%$)
        \item Rising Inflation and Weak Currency
        \item Volatile or negative real GDP growth
    \end{itemize}
\end{itemize}



\subsection*{Key Takeaways Summary}

\begin{itemize}
    \item \textbf{Sovereign Credit Risk Drivers:}
    \begin{enumerate}
        \item \textit{Qualitative:} Institutions, Fiscal Flexibility, Monetary Effectiveness, Economic Flexibility, External Status.
        \item \textit{Quantitative:} Fiscal Strength, Economic Growth/Stability, External Stability.
    \end{enumerate}
    \item \textbf{Non-Sovereign Issuers:}
        \begin{itemize}
            \item Agencies, government banks, supranationals, and regional governments.
            \item Regional governments issue:
                \begin{itemize}
                    \item \textbf{GO Bonds:} Backed by taxing power (lower risk).
                    \item \textbf{Revenue Bonds:} Backed by project CFs (higher risk).
                \end{itemize}
        \end{itemize}
    \item \textbf{Key Ratios:}
        \[
        \text{Debt Burden} = \frac{\text{Debt}}{\text{GDP}}, \quad
        \text{Debt Affordability} = \frac{\text{Interest}}{\text{Revenue}}, \quad
        \text{FX Reserve Ratio} = \frac{\text{FX Reserves}}{\text{External Debt}}
        \]
    \item Sovereign risk is influenced by both \textbf{ability} (economic/fiscal) and \textbf{willingness} (institutional/political) to repay.
\end{itemize}



\begin{table}[h!]
\centering
\caption*{Exhibit 5: Formula and Concept Summary}
\begin{tabular}{|l|l|}
\hline
\textbf{Concept} & \textbf{Formula / Definition} \\
\hline
Debt Burden Ratio & $\text{Debt}/\text{GDP}$ or $\text{Debt}/\text{Revenue}$ \\
\hline
Debt Affordability Ratio & $\text{Interest}/\text{GDP}$ or $\text{Interest}/\text{Revenue}$ \\
\hline
FX Reserve Ratio & $\text{FX Reserves}/\text{External Debt}$ \\
\hline
DSCR (Revenue Bonds) & $(\text{Revenue – Opex})/(\text{Interest + Principal})$ \\
\hline
GO Bond Support & Full faith and taxing power of issuer \\
\hline
Revenue Bond Support & Cash flows from specific project \\
\hline
\end{tabular}
\end{table}

\section*{Module 64.1: Credit Analysis for Corporate Issuers}

\subsection*{LOS 64.a: Qualitative and Quantitative Factors in Corporate Credit Analysis}

\paragraph{1. Overview}
\begin{itemize}
    \item Corporate credit analysis evaluates a firm’s \textbf{likelihood of default (POD)} and potential \textbf{loss given default (LGD)}.
    \item Uses both:
    \begin{itemize}
        \item \textbf{Qualitative factors:} business model, industry risk, governance.
        \item \textbf{Quantitative factors:} profitability, leverage, coverage, and liquidity ratios.
    \end{itemize}
\end{itemize}



\paragraph{2. Qualitative Factors}
\begin{table}[h!]
\centering
\caption*{Exhibit 1: Key Qualitative Factors in Corporate Credit Analysis}
\begin{tabular}{|l|p{12cm}|}
\hline
\textbf{Factor} & \textbf{Description and Implications for Credit Quality} \\
\hline
\textbf{Business Model} &
\begin{itemize}
    \item Predictable, recurring, and diversified cash flows → higher credit quality.  
    \item Exposure to cyclical industries or customer concentration → higher risk.  
    \item Long-term sustainability and adaptability of business model must be assessed.
\end{itemize} \\
\hline
\textbf{Industry Competition} &
\begin{itemize}
    \item Lower competition = stronger pricing power and margins.  
    \item Evaluate structural changes, barriers to entry, and future competitive threats.
\end{itemize} \\
\hline
\textbf{Business Risk} &
\begin{itemize}
    \item Low volatility in revenue and operating margins → higher creditworthiness.  
    \item Business risk sources: issuer-specific, industry-specific, or macroeconomic.
\end{itemize} \\
\hline
\textbf{Corporate Governance} &
\begin{itemize}
    \item Fair treatment of debtholders, transparency, compliance with regulations.  
    \item Focus areas:  
        – Covenant protection  
        – Accounting quality  
        – Historical management behavior toward creditors.
\end{itemize} \\
\hline
\end{tabular}
\end{table}



\paragraph{3. Governance Considerations}
\textbf{a) Covenants:}
\begin{itemize}
    \item \textbf{Investment-grade debt:} usually includes \textit{affirmative covenants} — maintain assets, pay taxes, comply with laws.
    \item \textbf{High-yield debt:} includes \textit{negative covenants} — restrict dividend payments, new debt, or asset sales.
    \item Analysts assess management history for debt-financed equity actions or covenant breaches.
\end{itemize}

\textbf{b) Accounting Policies:}
\begin{itemize}
    \item Aggressive accounting = red flag:
    \begin{itemize}
        \item Premature revenue recognition.
        \item Capitalizing expenses instead of expensing.
        \item Off-balance-sheet financing (e.g., operating leases).
        \item Frequent auditor or CFO changes.
    \end{itemize}
\end{itemize}



\paragraph{4. Quantitative Factors}
\begin{itemize}
    \item Quantitative analysis forecasts future financials to assess:
    \begin{itemize}
        \item Probability of Default (POD).
        \item Loss Given Default (LGD).
    \end{itemize}
    \item \textbf{Top-down analysis:} Macroeconomic and industry trends.  
    \textbf{Bottom-up analysis:} Company-specific drivers.  
    \textbf{Hybrid approach:} Both combined.
\end{itemize}

\paragraph{Indicators of Strong Credit Quality:}
\begin{itemize}
    \item Strong, recurring operating profits.
    \item Low leverage and debt reliance.
    \item High interest coverage.
    \item Ample liquidity (cash and undrawn lines of credit).
\end{itemize}



\subsection*{LOS 64.b: Financial Ratios Used in Corporate Credit Analysis}

\paragraph{1. Key Cash Flow Measures}
\begin{table}[h!]
\centering
\caption*{Exhibit 2: Common Cash Flow Metrics}
\begin{tabular}{|p{5cm}|p{12cm}|}
\hline
\textbf{Metric} & \textbf{Definition and Purpose} \\
\hline
\textbf{EBITDA} & 
Operating Income + Depreciation + Amortization.  
Proxy for operating cash flow but ignores capital expenditures and working capital needs. \\
\hline
\textbf{CFO (Cash Flow from Operations)} & 
Net cash from core operations.  
$= \text{Net Income} + \text{Non-Cash Charges} - \Delta \text{Working Capital}$.  
Better indicator of cash available for debt service. \\
\hline
\textbf{FFO (Funds From Operations)} & 
Net Income + Depreciation + Amortization + Deferred Taxes + Non-Cash Items.  
Similar to CFO but excludes changes in working capital. \\
\hline
\textbf{FCF (Free Cash Flow)} & 
CFO $-$ Capital Expenditures $+$ Net Interest Expense.  
Represents discretionary cash available to debtholders and shareholders. \\
\hline
\textbf{RCF (Retained Cash Flow)} & 
Operating Cash Flow $-$ Dividends.  
Used to assess retained liquidity after shareholder payouts. \\
\hline
\end{tabular}
\end{table}



\paragraph{2. Common Credit Ratios}
\begin{table}[h!]
\centering
\caption*{Exhibit 3: Key Financial Ratios in Corporate Credit Analysis}
\begin{tabular}{|p{5cm}|p{5cm}|p{5cm}|}
\hline
\textbf{Category} & \textbf{Ratio} & \textbf{Interpretation} \\
\hline
\textbf{Profitability} & EBIT Margin = EBIT / Revenue & Higher margin = stronger profitability and higher credit quality. \\
\hline
\textbf{Coverage} & Interest Coverage = EBIT / Interest Expense & Higher coverage = stronger ability to service debt. \\
\hline
\textbf{Leverage} & Debt / EBITDA & Lower ratio = lower leverage, stronger credit profile. \\
\hline
\textbf{Leverage (Cash-based)} & RCF / Net Debt & Higher ratio = better internal funding capacity, lower credit risk. \\
\hline
\end{tabular}
\end{table}



\paragraph{3. Example: Comparative Credit Assessment}
\textbf{York, Inc. vs. Zale, Inc.}

\begin{table}[h!]
\centering
\caption*{Exhibit 4: Input Data}
\begin{tabular}{|l|c|c|}
\hline
 & \textbf{York} & \textbf{Zale} \\
\hline
Revenue & \$2,200,000 & \$11,000,000 \\
EBIT & \$550,000 & \$2,250,000 \\
Interest Expense & \$40,000 & \$160,000 \\
Depreciation/Amortization & \$220,000 & \$900,000 \\
Debt & \$1,900,000 & \$2,700,000 \\
CFO & \$300,000 & \$850,000 \\
Dividends & \$30,000 & \$200,000 \\
Cash & \$500,000 & \$1,000,000 \\
\hline
\end{tabular}
\end{table}

\textbf{Step 1: Profitability}
\[
\text{EBIT Margin (York)} = \frac{550,000}{2,200,000} = 25\%, \quad
\text{EBIT Margin (Zale)} = \frac{2,250,000}{11,000,000} = 20.5\%
\]
$\Rightarrow$ York is more profitable.

\textbf{Step 2: Coverage}
\[
\text{Interest Coverage (York)} = \frac{550,000}{40,000} = 13.8\times, \quad
\text{Interest Coverage (Zale)} = \frac{2,250,000}{160,000} = 14.1\times
\]
$\Rightarrow$ Both strong; Zale slightly higher.

\textbf{Step 3: Leverage}
\[
\text{Debt/EBITDA (York)} = \frac{1,900,000}{770,000} = 2.5\times, \quad
\text{Debt/EBITDA (Zale)} = \frac{2,700,000}{3,150,000} = 0.9\times
\]
$\Rightarrow$ York more leveraged.

\textbf{Step 4: Cash Flow Leverage}
\[
\text{RCF/Net Debt (York)} = \frac{270,000}{1,400,000} = 19\%, \quad
\text{RCF/Net Debt (Zale)} = \frac{650,000}{1,700,000} = 38\%
\]
$\Rightarrow$ Zale generates more retained cash relative to its debt.

\textbf{Conclusion:}
\begin{itemize}
    \item York has higher profitability but also higher leverage → slightly lower credit quality.
    \item Zale stronger liquidity and lower leverage → stronger credit standing.
\end{itemize}



\subsection*{LOS 64.c: Debt Seniority, Security, and Priority of Claims}

\paragraph{1. Concept of Seniority}
\begin{itemize}
    \item \textbf{Seniority ranking:} Determines priority of repayment in default.  
    \item \textbf{Secured Debt:} Backed by collateral; has direct claim on pledged assets.  
    \item \textbf{Unsecured Debt:} General claim on issuer’s assets and CFs.
\end{itemize}



\paragraph{2. Debt Hierarchy in Default}
\begin{table}[h!]
\centering
\caption*{Exhibit 5: Priority of Claims}
\begin{tabular}{|l|l|}
\hline
\textbf{Rank} & \textbf{Debt Type / Example} \\
\hline
1 & First Lien / Mortgage (specific pledged asset) \\
2 & Senior Secured (second lien) \\
3 & Junior Secured \\
4 & Senior Unsecured \\
5 & Senior Subordinated \\
6 & Subordinated \\
7 & Junior Subordinated \\
\hline
\end{tabular}
\end{table}

\begin{itemize}
    \item All within same class rank \textbf{pari passu} (equal claim priority).
    \item Secured creditors have claim on collateral + general assets.
    \item If collateral value < obligation, deficiency ranks pari passu with senior unsecured.
\end{itemize}



\paragraph{3. Recovery and Credit Ratings}
\begin{itemize}
    \item \textbf{Recovery rates:} Decline as seniority decreases.  
    \item \textbf{Credit risk:} Increases with subordination.  
    \item Rating agencies assign:
    \begin{itemize}
        \item \textbf{Corporate Family Rating (CFR):} Issuer-level credit quality (usually senior unsecured).  
        \item \textbf{Corporate Credit Rating (CCR):} Issue-specific rating (depends on seniority, collateral, covenants).  
    \end{itemize}
\end{itemize}



\paragraph{4. Notching and Structural Subordination}
\begin{itemize}
    \item \textbf{Notching:} Adjusting an issue’s rating above/below issuer rating to reflect recovery potential.
    \item \textbf{Structural Subordination:} Occurs when subsidiary debt has claim on subsidiary cash flows before parent’s debt.
    \begin{itemize}
        \item Parent bonds effectively subordinated to subsidiary’s debt.  
        \item Common in holding company structures.
    \end{itemize}
    \item \textbf{Implication:}  
    – Highly rated issuers → minimal notching (differences in recovery less relevant).  
    – Lower-rated issuers → more notching (recovery differences matter more).
\end{itemize}



\paragraph{5. Practical Considerations in Bankruptcy}
\begin{itemize}
    \item \textbf{Absolute priority rule:} Senior creditors paid before junior ones.  
    \item \textbf{In practice:} May be violated if junior creditors or equity accept early settlement to expedite resolution.
    \item Bankruptcies are costly and slow — value of assets may deteriorate over time.
\end{itemize}



\subsection*{Key Takeaways Summary}

\begin{itemize}
    \item \textbf{Qualitative Factors:}  
    Stable business model, low competition, low business risk, strong governance.
    \item \textbf{Quantitative Factors:}  
    High margins, high coverage, low leverage, high liquidity.
    \item \textbf{Common Ratios:}
    \[
    \text{EBIT Margin} = \frac{\text{EBIT}}{\text{Revenue}}, \quad
    \text{Interest Coverage} = \frac{\text{EBIT}}{\text{Interest}}, \quad
    \text{Debt/EBITDA}, \quad
    \text{RCF/Net Debt}
    \]
    \item \textbf{Seniority Hierarchy:}
    \[
    \text{First Lien} > \text{Senior Secured} > \text{Junior Secured} > \text{Senior Unsecured} > \text{Subordinated}
    \]
    \item \textbf{Notching:}  
    Adjustment between issuer rating (CFR) and issue rating (CCR) based on collateral and seniority.
    \item \textbf{Structural Subordination:}  
    Parent debt is subordinate to subsidiary debt if upstream transfers are restricted.
\end{itemize}



\begin{table}[h!]
\centering
\caption*{Exhibit 6: Formula and Concept Summary}
\begin{tabular}{|p{4cm}|p{8cm}|}
\hline
\textbf{Concept} & \textbf{Formula / Definition} \\
\hline
EBIT Margin & EBIT / Revenue \\
\hline
Interest Coverage & EBIT / Interest Expense \\
\hline
Debt / EBITDA & Total Debt / (EBIT + Depreciation + Amortization) \\
\hline
RCF / Net Debt & (CFO - Dividends) / (Debt - Cash) \\
\hline
Free Cash Flow & CFO - CapEx + Net Interest Expense \\
\hline
DSCR (Project Finance) & (Revenue - Opex) / (Interest + Principal) \\
\hline
Seniority Order & 1st Lien → Senior Secured → Junior Secured → Senior Unsecured → Subordinated \\
\hline
Notching & Difference between Issuer (CFR) and Issue (CCR) ratings \\
\hline
Structural Subordination & Parent debt subordinated to subsidiary debt due to cash transfer restrictions \\
\hline
\end{tabular}
\end{table}

\section*{Module 65.1: Fixed-Income Securitization}

\subsection*{LOS 65.a: Benefits of Securitization}

\paragraph{1. Definition and Overview}
\begin{itemize}
    \item \textbf{Securitization:} The process of transforming illiquid financial assets (e.g., loans, receivables) into tradable securities known as \textbf{asset-backed securities (ABSs)}.
    \item Connects \textbf{capital providers (investors)} directly with \textbf{capital users (borrowers)}.
    \item Removes the originating institution (bank/corporation) from the intermediation process.
\end{itemize}

\paragraph{2. The Securitization Process}
\begin{table}[h!]
\centering
\caption*{Exhibit 1: Three-Step Securitization Process}
\begin{tabular}{|l|p{12cm}|}
\hline
\textbf{Step} & \textbf{Description} \\
\hline
\textbf{1. Origination} & A bank or corporation (the \textbf{originator}) creates a pool of loans or receivables (the \textbf{collateral}). Example: mortgages, auto loans, credit card receivables. \\
\hline
\textbf{2. Sale to SPE} & The pool of assets is sold to a \textbf{Special Purpose Entity (SPE)}—a separate legal vehicle that owns the assets and is bankruptcy remote. \\
\hline
\textbf{3. Issuance of ABS} & The SPE issues fixed-income securities (\textbf{asset-backed securities}) to investors. These are backed by cash flows from the underlying collateral. \\
\hline
\end{tabular}
\end{table}

\paragraph{3. Key Terms and Roles}
\begin{itemize}
    \item \textbf{Originator:} The bank/corporation that creates and sells the assets (e.g., mortgage lender, auto company).
    \item \textbf{SPE / SPV:} The entity that buys the assets and issues ABSs. It is \textbf{bankruptcy remote}.
    \item \textbf{Investors:} Buy ABSs and receive periodic cash flows from the underlying pool.
\end{itemize}



\paragraph{4. Benefits of Securitization}

\subsubsection*{(a) Benefits to Issuers (Originators)}
\begin{table}[h!]
\centering
\caption*{Exhibit 2: Advantages of Securitization to Issuers}
\begin{tabular}{|p{5cm}|p{12cm}|}
\hline
\textbf{Benefit} & \textbf{Explanation} \\
\hline
\textbf{Increased Lending Capacity} & Originators can re-lend proceeds from selling loans to the SPE, allowing greater loan creation beyond balance sheet limits. \\
\hline
\textbf{Improved Profitability} & Earn fees for loan origination and for structuring/selling collateral to the SPE. \\
\hline
\textbf{Lower Regulatory Capital} & Removal of loans from balance sheet lowers capital requirements for banks (Basel capital ratios). \\
\hline
\textbf{Improved Liquidity} & Converts illiquid assets (e.g., mortgages) into cash; improves liquidity management and risk-return efficiency. \\
\hline
\end{tabular}
\end{table}



\subsubsection*{(b) Benefits to Investors}
\begin{table}[h!]
\centering
\caption*{Exhibit 3: Advantages of Securitization to Investors}
\begin{tabular}{|p{3cm}|p{12cm}|}
\hline
\textbf{Benefit} & \textbf{Explanation} \\
\hline
\textbf{Tailored Risk–Return Profile} & ABS structures (tranches) allow investors to select exposure suited to their risk appetite (senior vs. junior tranches). \\
\hline
\textbf{Diversified Access} & Investors gain access to diversified loan pools (e.g., auto loans, mortgages) without having to originate loans themselves. \\
\hline
\textbf{Enhanced Liquidity} & ABSs are tradable securities — more liquid than the underlying loans. Investors can exit positions more easily. \\
\hline
\end{tabular}
\end{table}



\subsubsection*{(c) Benefits to Economies and Financial Markets}
\begin{table}[h!]
\centering
\caption*{Exhibit 4: Macroeconomic and Market Benefits}
\begin{tabular}{|p{4cm}|p{12cm}|}
\hline
\textbf{Benefit} & \textbf{Explanation} \\
\hline
\textbf{Reduced Liquidity Risk} & ABSs improve market liquidity relative to the underlying assets. \\
\hline
\textbf{Improved Market Efficiency} & Trading of ABSs allows for equilibrium pricing and better risk distribution. \\
\hline
\textbf{Lower Funding Costs} & Provides cheaper financing for originators than direct debt/equity issuance. \\
\hline
\textbf{Lower Leverage Ratios} & Growth via off-balance-sheet financing allows originators to expand without raising leverage. \\
\hline
\end{tabular}
\end{table}



\subsubsection*{(d) Risks to Investors}
\begin{itemize}
    \item \textbf{Cash flow uncertainty:} Timing and amount of cash flows vary due to prepayments or defaults (e.g., early mortgage repayments).
    \item \textbf{Credit risk:} Defaults or deterioration of collateral credit quality.
    \item \textbf{Systemic risk:} Excessive securitization can amplify systemic shocks (e.g., 2007–2009 financial crisis).
\end{itemize}



\paragraph{5. Conceptual Summary}
\[
\text{Securitization = Financial Intermediation by Market Mechanism}
\]
\[
\text{Capital Providers (Investors)} \longleftrightarrow \text{Borrowers (Collateral Pool)}
\]
\[
\text{Originator sells assets} \Rightarrow \text{SPE issues ABS} \Rightarrow \text{Investors fund the collateral.}
\]



\subsection*{LOS 65.b: Parties and Roles in Securitization}

\paragraph{1. Example: Fred Motor Company Securitization}
\begin{itemize}
    \item \textbf{Scenario:} Fred Motor Company originates \$1 billion in auto loans (50,000 contracts).
    \item \textbf{Goal:} Remove these loans from its balance sheet and free up cash for new lending.
\end{itemize}

\textbf{Structure:}
\[
\text{Fred (Originator/Seller)} \Rightarrow \text{Sells \$1bn auto loans} \Rightarrow \text{Auto Loan Trust (SPE)} \Rightarrow \text{Issues ABSs to Investors}
\]

\textbf{Roles:}
\begin{itemize}
    \item \textbf{Seller / Depositor:} Fred Motor Company — originates and sells loans to SPE.
    \item \textbf{SPE / Issuer (Auto Loan Trust):} Buys loans and issues ABS to investors.
    \item \textbf{Servicer:} Fred continues servicing loans (collects payments, handles repossessions).
    \item \textbf{Trustee:} Independent third party ensuring proper cash flow distribution and reporting.
\end{itemize}



\paragraph{2. Bankruptcy Remoteness}
\begin{itemize}
    \item The SPE is legally independent from the originator.  
    \item If the originator (Fred) becomes insolvent, ABS investors still have full claim on collateral cash flows.
    \item This legal separation ensures the ABS structure is \textbf{bankruptcy remote}.
\end{itemize}

\begin{table}[h!]
\centering
\caption*{Exhibit 5: Legal and Contractual Protections}
\begin{tabular}{|p{5cm}|p{12cm}|}
\hline
\textbf{Document} & \textbf{Purpose} \\
\hline
\textbf{Purchase Agreement} & Defines terms of sale of collateral from originator to SPE. \\
\hline
\textbf{Prospectus / Offering Memorandum} & Outlines terms of ABS issuance, fee structure, and waterfall (cash flow allocation). \\
\hline
\textbf{Trust Agreement} & Governs duties of trustee, reporting, and safeguarding of investor interests. \\
\hline
\end{tabular}
\end{table}



\paragraph{3. Cash Flow Waterfall}
\begin{itemize}
    \item Payments from borrowers → received by servicer → distributed by SPE:
    \begin{enumerate}
        \item Servicing and trustee fees.
        \item Interest payments to ABS investors.
        \item Principal repayments.
    \end{enumerate}
    \item Any losses are absorbed according to tranche hierarchy (senior → mezzanine → junior).
\end{itemize}



\paragraph{4. Distinction Between Key Entities}
\begin{table}[h!]
\centering
\caption*{Exhibit 6: Roles in a Typical Securitization}
\begin{tabular}{|l|l|p{9cm}|}
\hline
\textbf{Party} & \textbf{Also Known As} & \textbf{Primary Function} \\
\hline
Originator & Seller, Depositor & Creates and sells pool of loans to SPE. \\
\hline
SPE / SPV & Issuer, Trust & Purchases collateral, issues ABS to investors. \\
\hline
Servicer & Administrator & Collects payments, enforces contracts, manages delinquencies. \\
\hline
Trustee & Custodian & Ensures proper operation of trust and payment waterfall; provides reporting. \\
\hline
Investors & Bondholders & Provide funding; receive periodic principal and interest. \\
\hline
\end{tabular}
\end{table}



\subsection*{Key Takeaways Summary}

\begin{itemize}
    \item \textbf{Definition:} Securitization converts illiquid financial assets into tradable ABSs through an SPE.
    \item \textbf{Benefits to Issuers:}
    \begin{itemize}
        \item Free up balance sheet capital.
        \item Generate fee income.
        \item Reduce leverage and regulatory capital.
    \end{itemize}
    \item \textbf{Benefits to Investors:}
    \begin{itemize}
        \item Tailored risk/return profiles (via tranching).  
        \item Access to diverse collateral.  
        \item Greater liquidity.
    \end{itemize}
    \item \textbf{Benefits to Economy:}
    \begin{itemize}
        \item Improved liquidity and efficiency.  
        \item Broader credit availability and reduced funding costs.
    \end{itemize}
    \item \textbf{Risks:} Cash flow uncertainty and credit risk of collateral.
    \item \textbf{Key Entities:}
    \[
    \text{Originator (Seller)} \rightarrow \text{SPE (Issuer)} \rightarrow \text{Investors (ABS Holders)}
    \]
    \item \textbf{Bankruptcy Remote:} SPE’s assets and liabilities are isolated from those of the originator.
\end{itemize}



\begin{table}[h!]
\centering
\caption*{Exhibit 7: Formula and Concept Summary}
\begin{tabular}{|p{4cm}|p{8cm}|}
\hline
\textbf{Concept} & \textbf{Definition / Formula / Note} \\
\hline
ABS (Asset-Backed Security) & Security backed by loan or receivable cash flows. \\
\hline
SPE / SPV & Separate legal entity that issues ABS; bankruptcy remote. \\
\hline
Tranche & Division of ABS into senior, mezzanine, and junior risk classes. \\
\hline
Cash Flow Waterfall & Sequential payment: servicing fees → interest → principal → residuals. \\
\hline
Originator’s Benefit & Liquidity, reduced capital requirements, fee income. \\
\hline
Investor’s Benefit & Tailored risk exposure and improved liquidity. \\
\hline
Risks to Investor & Credit risk + prepayment (timing) risk. \\
\hline
Documents & Purchase Agreement, Prospectus, Trust Agreement. \\
\hline
Bankruptcy Remoteness & SPE assets unaffected by originator default. \\
\hline
\end{tabular}
\end{table}

\section*{Module 66.1: Asset-Backed Security (ABS) Instrument and Market Features}

\subsection*{LOS 66.a: Covered Bonds – Characteristics, Risks \& Differences from ABS}

\paragraph{1. Definition}
\begin{itemize}
  \item \textbf{Covered Bonds:} Senior debt obligations of banks, backed by a segregated \textbf{cover pool} of assets (usually mortgages) that remain \textbf{on-balance-sheet} of the issuer.  
  \item No SPE is created; the bondholders have \textbf{dual recourse}:  
    \begin{enumerate}
      \item To the \textbf{cover pool}, and  
      \item To the issuer’s \textbf{unencumbered assets}.
    \end{enumerate}
\end{itemize}

\paragraph{2. Structure and Credit Enhancements}

\begin{table}[h!]
\centering
\caption*{Exhibit 1: Covered Bond Structure vs. ABS}
\begin{tabular}{|l|p{12cm}|}
\hline
\textbf{Feature} & \textbf{Covered Bond Characteristics} \\
\hline
Legal Structure & Issued directly by a bank; cover pool remains on issuer’s balance sheet (no SPE). \\
\hline
Investor Recourse & Dual recourse → claims on cover pool + issuer’s other assets. \\
\hline
Credit Enhancements & 
\begin{itemize}
  \item \textbf{Overcollateralization:} Collateral value > bond value.  
  \item \textbf{Loan-to-Value (LTV) Limits:} Maximum LTV ratios on mortgages.  
  \item \textbf{Third-party Monitoring:} Independent supervision of cover pool quality.
\end{itemize} \\
\hline
Capital Treatment & Assets remain on balance sheet → no capital relief (unlike ABS). \\
\hline
Collateral Maintenance & Dynamic pool → nonperforming/prepaid loans must be replaced to preserve coverage. \\
\hline
Tranching & Usually none – covered bonds are single-class instruments. \\
\hline
\end{tabular}
\end{table}

\paragraph{3. Types of Covered Bonds}
\begin{itemize}
  \item \textbf{Hard-bullet:} Default occurs immediately upon missed payment → accelerated repayment.
  \item \textbf{Soft-bullet:} Maturity may be extended (up to 1 year) before default is triggered.
  \item \textbf{Conditional Pass-Through:} If payments still due at maturity, bond converts to pass-through mode — cash flows are passed through as recovered from collateral.
\end{itemize}

\paragraph{4. Comparison to Traditional ABS}

\begin{table}[h!]
\centering
\caption*{Exhibit 2: Covered Bonds vs. ABS}
\begin{tabular}{|p{4cm}|p{5cm}|p{7cm}|}
\hline
\textbf{Aspect} & \textbf{Covered Bond} & \textbf{Asset-Backed Security (ABS)} \\
\hline
Issuer & Bank or financial institution & Special Purpose Entity (SPE) \\
\hline
Collateral Ownership & Remains on issuer’s balance sheet & Sold to SPE (bankruptcy remote) \\
\hline
Recourse & Dual recourse (cover pool + issuer) & Limited recourse (to SPE assets only) \\
\hline
Capital Relief for Bank & No (assets stay on balance sheet) & Yes (loans removed from balance sheet) \\
\hline
Collateral Maintenance & Dynamic (replacement of non-performing loans) & Static (pool fixed at issuance) \\
\hline
Yield Level & Lower (due to dual recourse and overcollateralization) & Higher (riskier, no recourse to issuer) \\
\hline
Tranching & Rare & Common (subordination structure) \\
\hline
\end{tabular}
\end{table}


\subsection*{LOS 66.b: Credit Enhancement Structures in Securitization}

\paragraph{1. Purpose}
To mitigate investor exposure to credit losses in the collateral pool.  
Two broad categories:
\begin{itemize}
  \item \textbf{Internal Enhancements:} Built into the ABS structure.  
  \item \textbf{External Enhancements:} Provided by third-party guarantees (not covered in detail here).
\end{itemize}

\paragraph{2. Internal Credit Enhancement Methods}

\begin{table}[h!]
\centering
\caption*{Exhibit 3: Forms of Internal Credit Enhancement}
\begin{tabular}{|p{5cm}|p{10cm}|}
\hline
\textbf{Type} & \textbf{Mechanism and Example} \\
\hline
\textbf{Overcollateralization} &
Collateral value $>$ face value of ABS.  
Example: Collateral \$600 m vs. ABS \$500 m → \$100 m excess = 16.7\% buffer.  
Defaults smaller or equal 16.7\% absorbed without loss to ABS investors. \\
\hline
\textbf{Excess Spread} &
Interest income on collateral $>$ coupon promised to ABS investors.  
The difference builds up a reserve fund to cover losses. \\
\hline
\textbf{Subordination (Credit Tranching)} &
ABS divided into classes (senior, mezzanine, junior).  
Losses allocated from bottom to top → junior tranche absorbs first losses.  
Provides credit protection for senior tranches. \\
\hline
\end{tabular}
\end{table}

\paragraph{3. Example: Senior–Subordinated Structure}

\begin{table}[h!]
\centering
\caption*{Exhibit 4: Credit Tranching Illustration}
\begin{tabular}{|l|l|l|}
\hline
\textbf{Tranche} & \textbf{Principal (\$ millions)} & \textbf{Loss Absorption Order} \\
\hline
A (Senior) & 400 & Protected until losses > \$110 m \\
B (Subordinated) & 80 & Absorbs losses from \$30 m → \$110 m \\
C (Junior/Equity) & 30 & First to absorb losses up to \$30 m \\
\hline
\end{tabular}
\end{table}

\textbf{Investor Example:}
\[
\text{Tranche B Investor: Principal = \$80 m, Coupon = 4\% + 1.5\% = 5.5\%}
\]
\[
\text{If losses = \$50 m → Tranche C absorbs \$30 m, Tranche B absorbs \$20 m.}
\]
\[
\begin{aligned}
\text{New Principal} &= \$60\ \text{m},\\
\text{Coupon Payment} &= \$550{,}000 \times \frac{60}{80} \\
&= \$412{,}500.
\end{aligned}
\]

\paragraph{4. Key Insight}
\begin{itemize}
  \item Tranching redistributes credit risk similar to a corporate capital structure (senior → subordinated → equity).  
  \item Senior tranches may achieve credit ratings higher than the originator due to structural protection.
\end{itemize}


\subsection*{LOS 66.c: Types and Risks of Non-Mortgage ABSs}

\paragraph{1. Collateral Spectrum}
ABSs can be backed by any cash-flow-producing assets (e.g., credit cards, auto loans, solar loans, royalties, franchise fees).

\paragraph{2. Credit Card ABSs}
\begin{table}[h!]
\centering
\caption*{Exhibit 5: Credit Card ABS Features}
\begin{tabular}{|l|p{12cm}|}
\hline
\textbf{Feature} & \textbf{Explanation} \\
\hline
Collateral Type & Non-amortizing receivables (credit card balances). \\
\hline
Cash Flows & Finance charges, membership/late fees, principal repayments. \\
\hline
Interest Rate Type & Fixed or floating (often with a cap). \\
\hline
Lockout / Revolving Period & 
\begin{itemize}
  \item Investors receive only interest and fees.  
  \item Principal repayments used to buy new receivables.  
  \item Maintains pool size and avoids prepayment risk.
\end{itemize} \\
\hline
Amortization Period & After lockout ends, principal flows through to investors. \\
\hline
Early Amortization Trigger & Activated to preserve credit quality when defaults rise. \\
\hline
\end{tabular}
\end{table}


\paragraph{3. Solar ABSs (ESG Linked)}

\begin{table}[h!]
\centering
\caption*{Exhibit 6: Characteristics of Solar ABSs}
\begin{tabular}{|l|p{12cm}|}
\hline
\textbf{Feature} & \textbf{Explanation} \\
\hline
Collateral Type & Loans to homeowners for solar panel installation. \\
\hline
Security & Secured by solar equipment or as junior mortgage on property. \\
\hline
Investor Attraction & ESG appeal — supports renewable energy transition. \\
\hline
Credit Quality & Typically high (qualified borrowers with energy savings). \\
\hline
Credit Enhancements & Overcollateralization, excess spread, subordination. \\
\hline
Pre-Funding Period & Allows pool expansion post-issuance for diversification. \\
\hline
Key Risk & New asset class — performance not yet tested over full credit cycle. \\
\hline
\end{tabular}
\end{table}

\paragraph{4. Amortizing vs. Non-Amortizing Collateral}
\begin{itemize}
  \item \textbf{Amortizing:} Principal reduces over time (e.g., auto loans, mortgages).  
  \item \textbf{Non-amortizing:} Principal repayment unscheduled (e.g., credit cards).
\end{itemize}


\subsection*{LOS 66.d: Collateralized Debt Obligations (CDOs)}

\paragraph{1. Definition and Structure}
\begin{itemize}
  \item \textbf{CDO:} Structured security issued by an SPE backed by a pool of debt obligations.  
  \item Includes:  
    \begin{itemize}
      \item \textbf{CBO – Collateralized Bond Obligation:} Corporate / emerging-market bonds.  
      \item \textbf{CLO – Collateralized Loan Obligation:} Leveraged bank loans.
    \end{itemize}
  \item Managed by a \textbf{collateral manager} who actively buys / sells assets to meet return objectives.
\end{itemize}

\paragraph{2. CLO Types}
\begin{table}[h!]
\centering
\caption*{Exhibit 7: Types of CLO Structures}
\begin{tabular}{|l|p{12cm}|}
\hline
\textbf{Type} & \textbf{Description} \\
\hline
\textbf{Cash-Flow CLO} & Payments to investors from interest \& principal cash flows of underlying loans. \\
\hline
\textbf{Market-Value CLO} & Manager trades collateral to profit from price changes; payments depend on market values. \\
\hline
\textbf{Synthetic CLO} & Credit exposure via derivatives (e.g., credit default swaps); SPE does not own collateral directly. \\
\hline
\end{tabular}
\end{table}

\paragraph{3. Collateral Quality and Investor Protection}
\begin{itemize}
  \item \textbf{Coverage Tests:} Ensure cash flows suffice to meet debt service.  
  \item \textbf{Overcollateralization Tests:} If breached, cash flows diverted to repay senior tranches.  
  \item \textbf{Diversification Tests:} Limits exposure to single borrower / industry.  
  \item \textbf{Rating Constraints:} Caps on low-rated (CCC) assets within portfolio.
\end{itemize}

\paragraph{4. Comparison of ABS and CDO Cash Flow Mechanics}

\begin{table}[h!]
\centering
\caption*{Exhibit 8: ABS vs. CDO Structures}
\begin{tabular}{|p{3cm}|p{5cm}|p{5cm}|}
\hline
\textbf{Feature} & \textbf{ABS} & \textbf{CDO} \\
\hline
Collateral Type & Consumer loans (e.g., mortgages, auto, credit cards) & Corporate / leveraged loans / bonds \\
\hline
Collateral Manager & None (static pool) & Active management of portfolio \\
\hline
Cash Flow Source & Loan repayments from borrowers & Cash flows or market value changes from managed assets \\
\hline
Risk Profile & Linked to borrower defaults and prepayments & Linked to credit spreads \& manager performance \\
\hline
Complexity & Moderate & Higher (due to active management and leverage) \\
\hline
\end{tabular}
\end{table}


\subsection*{Key Concept Summary}

\begin{itemize}
  \item \textbf{Covered Bonds:} On-balance-sheet debt with dual recourse; dynamic cover pool; low yield.  
  \item \textbf{Credit Enhancement Methods:} Overcollateralization, Excess Spread, Subordination.  
  \item \textbf{Non-Mortgage ABSs:}
    \begin{itemize}
      \item \textit{Credit Card ABS:} Non-amortizing receivables with lockout period.  
      \item \textit{Solar ABS:} Amortizing green loans with ESG appeal and pre-funding provision.
    \end{itemize}
  \item \textbf{CDOs / CLOs:} SPE-issued securities backed by actively managed debt portfolios; include cash-flow, market-value, and synthetic types.  
  \item \textbf{Key Risks:} Credit risk, prepayment risk, structural complexity.
\end{itemize}


\begin{table}[h!]
\centering
\caption*{Exhibit 9: Formula and Concept Recap}
\begin{tabular}{|l|l|}
\hline
\textbf{Concept} & \textbf{Formula / Explanation} \\
\hline
Overcollateralization Ratio & $\frac{\text{Collateral Value – ABS Face Value}}{\text{Collateral Value}}$ \\
\hline
Excess Spread & Interest Income on Collateral – Interest Payments to Investors \\
\hline
Tranching Hierarchy & Senior $\rightarrow$ Mezzanine $\rightarrow$ Junior (Loss Absorption Order) \\
\hline
Dual Recourse & Bondholders’ claim on cover pool + unencumbered issuer assets \\
\hline
Lockout Period & Interest only; principal reinvested in new receivables \\
\hline
Amortization Trigger & Early principal repayment when credit quality deteriorates \\
\hline
CLO Tests & Coverage, Overcollateralization, Diversification, Rating Limits \\
\hline
\end{tabular}
\end{table}

\section*{Module 67.1: Mortgage-Backed Security (MBS) Instrument and Market Features}

\subsection*{LOS 67.a: Prepayment Risk and Time Tranching in Securitizations}

\paragraph{1. Definition of Prepayment Risk}
\begin{itemize}
  \item \textbf{Prepayments:} Principal repayments made by borrowers earlier than scheduled (via refinancing, selling, or prepayment).
  \item \textbf{Prepayment Risk:} Uncertainty regarding the timing of these principal repayments — causes cash flows to differ from expected.
\end{itemize}

\begin{table}[h!]
\centering
\caption*{Exhibit 1: Components of Prepayment Risk}
\begin{tabular}{|l|p{12cm}|}
\hline
\textbf{Risk Type} & \textbf{Description} \\
\hline
\textbf{Contraction Risk} & 
Risk that prepayments are faster than expected (usually when interest rates fall).  
\textit{Effects:} Cash received sooner $\Rightarrow$ lower reinvestment rates, smaller price appreciation (negative convexity). \\
\hline
\textbf{Extension Risk} & 
Risk that prepayments are slower than expected (usually when rates rise).  
\textit{Effects:} Longer duration, lower PV (cash flows discounted longer at higher rate). \\
\hline
\end{tabular}
\end{table}

\paragraph{2. Interest Rate Link and Price Behavior}
\begin{itemize}
  \item When rates $\downarrow$ → refinancing $\uparrow$ → faster prepayments → \textbf{contraction risk}.
  \item When rates $\uparrow$ → refinancing $\downarrow$ → slower prepayments → \textbf{extension risk}.
  \item \textbf{MBS exhibit negative convexity:} Price rises less when yields fall, but declines more when yields rise.
\end{itemize}

\paragraph{3. Time Tranching (Reallocation of Prepayment Risk)}
\begin{itemize}
  \item Technique to redistribute prepayment risk among multiple bond classes (tranches) with different maturities.
  \item \textbf{Purpose:} Align prepayment exposure with investor preference.
  \begin{itemize}
    \item Short tranches → protect against \textbf{extension risk}.
    \item Long tranches → protect against \textbf{contraction risk}.
  \end{itemize}
  \item Example structure: \textbf{Sequential Pay CMO (see LOS 67.c)} — Tranche 1 paid first, Tranche 2 second, etc.
\end{itemize}


\subsection*{LOS 67.b: Residential Mortgage Loan Characteristics}

\paragraph{1. Collateral and Legal Structure}
\begin{itemize}
  \item \textbf{Collateral:} Residential real estate (house, apartment).
  \item \textbf{First Lien:} Lender has legal claim to property if borrower defaults (foreclosure process).
\end{itemize}

\paragraph{2. Common Loan Features}
\begin{table}[h!]
\centering
\caption*{Exhibit 2: Common Features of Residential Mortgage Loans}
\begin{tabular}{|l|p{12cm}|}
\hline
\textbf{Feature} & \textbf{Description} \\
\hline
\textbf{Prepayment Penalty} & 
Fee for repaying principal early (common in Europe, rare in U.S.); mitigates prepayment risk for lenders. \\
\hline
\textbf{Recourse Loan} & 
Lender can claim both collateral property and other borrower assets (common in Europe). \\
\hline
\textbf{Nonrecourse Loan} & 
Lender limited to collateral claim; borrower can walk away if underwater (common in U.S.). \\
\hline
\end{tabular}
\end{table}

\paragraph{3. Risk Ratios}
\[
\text{LTV} = \frac{\text{Loan Amount}}{\text{Appraised Property Value}}, \quad
\text{DTI} = \frac{\text{Monthly Debt Payments}}{\text{Monthly Pretax Income}}
\]

\textbf{Example:}
\[
\text{Loan} = 300{,}000, \quad \text{Value} = 400{,}000 \Rightarrow LTV = 75\%
\]
\[
\text{PMT} = 1{,}932.9, \quad \text{Monthly Income} = 6{,}667 \Rightarrow DTI = 29\%
\]

\begin{itemize}
  \item Lower LTV $\Rightarrow$ higher borrower equity $\Rightarrow$ lower default probability.
  \item Lower DTI $\Rightarrow$ higher repayment capacity $\Rightarrow$ lower credit risk.
\end{itemize}

\paragraph{4. Loan Categories}
\begin{itemize}
  \item \textbf{Prime Loans:} Low LTV, low DTI, good credit score.
  \item \textbf{Subprime Loans:} High LTV, high DTI, lower credit quality, higher default probability.
\end{itemize}

\paragraph{5. Agency vs. Non-Agency RMBS}
\begin{table}[h!]
\centering
\caption*{Exhibit 3: Agency vs. Non-Agency RMBS}
\begin{tabular}{|p{3cm}|p{4cm}|p{4cm}|}
\hline
\textbf{Feature} & \textbf{Agency RMBS} & \textbf{Non-Agency RMBS} \\
\hline
Issuer & Gov’t or GSE (e.g., Fannie Mae, Freddie Mac) & Private banks or institutions \\
\hline
Guarantee & Gov’t or GSE guarantee & No gov’t guarantee \\
\hline
Collateral Type & Conforming (meets underwriting standards) & Often nonconforming or subprime \\
\hline
Credit Enhancement & Implicit/explicit guarantee & Structural (insurance, tranching, etc.) \\
\hline
Example Risk Event & None (low risk) & 2007–2009 subprime crisis losses \\
\hline
\end{tabular}
\end{table}


\subsection*{LOS 67.c: Residential MBS Types and Cash Flow Characteristics}

\paragraph{1. Mortgage Pass-Through Securities (MPTS)}
\begin{itemize}
  \item Represent claims on cash flows from a pool of mortgages (principal + interest, minus servicing/insurance fees).
  \item \textbf{Pass-Through Rate (Net Coupon):} Coupon on MBS = WAC – fees.
  \item Cash flows include scheduled repayments + unscheduled prepayments.
\end{itemize}

\paragraph{2. Key Metrics}
\[
\text{WAM} = \frac{\sum (w_i \times \text{Mortgage Maturity})}{\sum w_i}, \quad 
\text{WAC} = \frac{\sum (w_i \times \text{Mortgage Rate})}{\sum w_i}
\]
where \( w_i = \frac{\text{Outstanding Balance of Loan } i}{\text{Total Balance of Pool}} \).

\paragraph{3. Collateralized Mortgage Obligations (CMOs)}
\begin{itemize}
  \item Structured securities backed by RMBS or mortgage pools.
  \item Redistribute prepayment risk among tranches.
  \item Expand investor base by matching duration and risk preferences.
\end{itemize}

\paragraph{4. Example: Sequential Pay Tranches}
\[
\text{Tranche A} \rightarrow \text{Principal Paid First}; \quad
\text{Tranche B} \rightarrow \text{Next in Line}
\]
\begin{itemize}
  \item Tranche A: Lower extension risk.  
  \item Tranche B: Lower contraction risk.
\end{itemize}

\paragraph{5. Other CMO Tranche Types}

\begin{table}[h!]
\centering
\caption*{Exhibit 4: Common CMO Tranche Types and Features}
\begin{tabular}{|p{4cm}|p{10cm}|}
\hline
\textbf{Tranche Type} & \textbf{Description / Risk Exposure} \\
\hline
\textbf{Z-Tranche (Accrual)} & No interest paid initially; interest accrues to principal. Paid after accrual period ends. Lowest priority. \\
\hline
\textbf{Principal-Only (PO)} & Receives only principal. Benefits from faster prepayments (higher returns). \\
\hline
\textbf{Interest-Only (IO)} & Receives only interest. Harmed by faster prepayments (fewer coupon payments). \\
\hline
\textbf{Floating-Rate / Inverse Floaters} & Coupon linked to reference rate (LIBOR or SOFR). Inverse floaters move opposite to rate changes. \\
\hline
\textbf{Residual (Equity) Tranche} & Last in payment priority; absorbs residual risk and cash flows. \\
\hline
\textbf{Planned Amortization Class (PAC)} & Receives predictable payments within defined prepayment speed range; supported by support tranche. \\
\hline
\textbf{Support Tranche (Companion)} & Absorbs excess or shortfall in prepayments to stabilize PAC payments; higher prepayment risk. \\
\hline
\end{tabular}
\end{table}


\subsection*{LOS 67.d: Commercial Mortgage-Backed Securities (CMBS)}

\paragraph{1. Definition and Collateral}
\begin{itemize}
  \item CMBS backed by commercial mortgages on income-producing real estate:  
  apartments, offices, shopping centers, hotels, warehouses, etc.
  \item Collateral often less diversified (few or single large property loans).
\end{itemize}

\paragraph{2. Key Ratios for Credit Analysis}
\[
\text{DSCR} = \frac{\text{Net Operating Income (NOI)}}{\text{Debt Service Payment}}, \quad
\text{LTV} = \frac{\text{Loan Amount}}{\text{Property Value}}
\]
\begin{itemize}
  \item \textbf{Higher DSCR} → better ability to service debt.  
  \item \textbf{Lower LTV} → higher equity cushion, lower credit risk.
\end{itemize}

\paragraph{3. Loan-Level and Structural Call Protection}

\begin{table}[h!]
\centering
\caption*{Exhibit 5: CMBS Call Protection Mechanisms}
\begin{tabular}{|p{4cm}|p{10cm}|}
\hline
\textbf{Type} & \textbf{Description} \\
\hline
\textbf{Prepayment Lockout} & Prohibits prepayment for 2–5 years. \\
\hline
\textbf{Prepayment Penalty Points} & Fee (as \%) charged on prepaid principal (e.g., 2 points = 2\%). \\
\hline
\textbf{Defeasance} & Borrower replaces loan with government securities producing identical payments — allows sale of property but preserves lender cash flow. \\
\hline
\textbf{Structural Call Protection} & Sequential-pay tranching at CMBS level ensures priority-based repayment order. \\
\hline
\end{tabular}
\end{table}

\paragraph{4. Balloon Maturity and Balloon Risk}
\begin{itemize}
  \item \textbf{Partially amortizing loans:} Large principal balance due at maturity = \textbf{Balloon payment.}
  \item \textbf{Balloon risk:} Borrower may fail to refinance → term extended (\textbf{extension risk}).  
  \item Lender may enter “workout period” with modified terms.
\end{itemize}

\paragraph{5. CMBS vs. RMBS Comparison}
\begin{table}[h!]
\centering
\caption*{Exhibit 6: RMBS vs. CMBS Comparison}
\begin{tabular}{|p{3cm}|p{5cm}|p{5cm}|}
\hline
\textbf{Feature} & \textbf{RMBS} & \textbf{CMBS} \\
\hline
Collateral Type & Residential homes & Commercial income-producing property \\
\hline
Borrower Type & Individuals & Real estate investors / companies \\
\hline
Cash Flow Source & Mortgage payments from homeowners & Rental income (via tenants) \\
\hline
Loan Type & Fully amortizing & Often partially amortizing (balloon maturity) \\
\hline
Risk Focus & Borrower credit quality & Property NOI, LTV, DSCR \\
\hline
Prepayment Risk & Contraction and extension & Limited via call protection \\
\hline
Balloon Risk & Rare & Common (extension form) \\
\hline
\end{tabular}
\end{table}


\subsection*{Key Concept Summary}

\begin{itemize}
  \item \textbf{Prepayment Risk:} Timing uncertainty of principal repayments.  
    \[
    \text{Contraction Risk (faster prepayment)} \quad \text{vs.} \quad \text{Extension Risk (slower prepayment)}
    \]
  \item \textbf{Time Tranching:} Distributes prepayment risk across tranches (short vs. long).
  \item \textbf{Residential Loans:} Characterized by LTV, DTI, recourse type, prepayment terms.
  \item \textbf{Agency RMBS:} Gov’t or GSE guaranteed; conforming standards.
  \item \textbf{Non-Agency RMBS:} Private issue; structural credit enhancements.
  \item \textbf{MBS Structures:}
    \begin{itemize}
      \item \textbf{Pass-Throughs:} Direct flow of principal + interest.
      \item \textbf{CMOs:} Structured tranches redistributing prepayment risk.
      \item \textbf{PAC/Support Tranches:} Stabilize payments across prepayment speed ranges.
    \end{itemize}
  \item \textbf{CMBS:} Backed by commercial properties; key ratios = DSCR, LTV; risks = call protection, balloon risk.
\end{itemize}


\begin{table}[h!]
\centering
\caption*{Exhibit 7: Formula Recap}
\begin{tabular}{|l|l|}
\hline
\textbf{Concept} & \textbf{Formula / Definition} \\
\hline
Loan-to-Value (LTV) & $\displaystyle \frac{\text{Loan Amount}}{\text{Property Value}}$ \\
\hline
Debt-to-Income (DTI) & $\displaystyle \frac{\text{Monthly Debt Payments}}{\text{Monthly Gross Income}}$ \\
\hline
Debt Service Coverage Ratio (DSCR) & $\displaystyle \frac{\text{Net Operating Income}}{\text{Debt Service Payment}}$ \\
\hline
Weighted Average Maturity (WAM) & Weighted avg. of mortgage maturities by balance \\
\hline
Weighted Average Coupon (WAC) & Weighted avg. of mortgage rates by balance \\
\hline
Pass-Through Rate & WAC – servicing/insurance fees \\
\hline
\end{tabular}
\end{table}


\end{document}