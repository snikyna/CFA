% Corporate Issuers Study Notes - CFA Level I
% Detailed analysis of different organizational forms and corporate structures
% Covers governance principles, stakeholder management, and business entity types
% Essential for understanding how companies are organized and managed
% Source: CFA Level I Curriculum 2024 - Corporate Issuers Module

\documentclass[12pt]{article}
\usepackage{amsmath}
\usepackage{geometry}
\usepackage{graphicx}
\usepackage{booktabs}
\usepackage{caption}
\usepackage{titlesec}
\usepackage{graphicx} % make sure to include in preamble
\usepackage{float}
\usepackage{makecell}
\usepackage{tabularx}
\usepackage{enumitem}


\geometry{margin=1in}

\title{Corporate Issuers}
\date{}

\begin{document}
\maketitle
\section*{22 ORGANIZATIONAL FORMS, CORPORATE ISSUER FEATURES, AND OWNERSHIP}

\subsection*{Module 22.1: Features of Corporate Issuers}

\subsubsection*{LOS 22.a: Organizational Forms of Businesses}

\begin{itemize}
    \item \textbf{Key Features of Organizational Forms:}
    \begin{itemize}
        \item Separate legal identity or not.
        \item Owners vs. operators distinction.
        \item Liability: limited vs. unlimited.
        \item Tax treatment of profits/losses.
        \item Access to capital and risk distribution.
    \end{itemize}
\end{itemize}

\begin{table}[H]
\centering
\resizebox{\textwidth}{!}{%
\footnotesize
\begin{tabular}{|l|p{3.5cm}|p{3.5cm}|p{3.5cm}|p{3.5cm}|}
\hline
\textbf{Feature} & \textbf{Sole Proprietorship} & \textbf{General Partnership} & \textbf{Limited Partnership} & \textbf{Corporation (Ltd./Inc.)} \\
\hline
Legal Entity & Not separate from owner & Not separate from owners & Separate for limited partners only & Fully separate legal entity \\
\hline
Liability & Unlimited (personal assets at risk) & Unlimited for all partners & Unlimited (general partners), Limited (limited partners) & Limited to shareholder investment \\
\hline
Taxation & Profits taxed as personal income & Profits taxed as personal income of partners & Same as general partnership & Subject to corporate tax; dividends taxed again (double taxation possible) \\
\hline
Capital Access & Limited to owner’s resources & Limited to partners’ resources & Higher than general partnership & High (equity + debt financing) \\
\hline
Management & Owner-managed & Managed jointly by partners & Managed by general partners & Managed by board-appointed managers \\
\hline
Scale & Small & Small to medium & Medium to large & Large, common for multinational firms \\
\hline
\end{tabular}}
\caption{Comparison of Business Structures}
\end{table}

\textbf{Examples:}
\begin{itemize}
    \item \textit{Sole Proprietorship} - A freelance consultant earns €80,000. All profits taxed as personal income; unlimited liability.  
    \item \textit{Limited Partnership} - A real estate project with one general partner (developer) and five limited partners (€5M investment). Limited partners risk only their contributions.  
    \item \textit{Corporation} - Apple Inc. shareholders cannot lose more than the value of their shares.  
\end{itemize}

---

\subsubsection*{LOS 22.b: Key Features of Corporate Issuers}

\begin{itemize}
    \item Corporations = \textbf{separate legal entity}.
    \item Rights: hire, borrow/lend, contract, sue/be sued.
    \item Shares issued $\rightarrow$ raise large capital, easily transferable.  
    \item Shareholders elect \textbf{board of directors} $\rightarrow$ board hires management.  
    \item Profits reinvested or distributed as dividends.  
    \item Double taxation possible.  
\end{itemize}

\textbf{Example: Double Taxation of Dividends}
\begin{enumerate}[label=(\alph*)]
    \item \textbf{100\% payout:} Effective tax rate = 40\%.
    \item \textbf{40\% payout:} Effective tax rate = 31\%.
\end{enumerate}

---

\subsubsection*{LOS 22.c: Public vs. Private Corporate Issuers}

\begin{itemize}
    \item \textbf{Public Companies:}
    \begin{itemize}
        \item Listed on exchanges, transparent pricing.
        \item Heavy reporting requirements (quarterly/annual).
        \item Free float = shares available for public trading.
    \end{itemize}
    \item \textbf{Private Companies:}
    \begin{itemize}
        \item Shares not traded, illiquid, valuation difficult.
        \item Light regulation, less disclosure.
        \item Equity via private placements to accredited investors.
    \end{itemize}
\end{itemize}

\begin{table}[H]
\centering
\footnotesize
\begin{tabular}{|l|p{6cm}|p{6cm}|}
\hline
\textbf{Feature} & \textbf{Public Company} & \textbf{Private Company} \\
\hline
Ownership & Widely held, incl. retail + institutions & Few shareholders, often founders \\
\hline
Share Trading & Exchange-listed, liquid & Not exchange-listed, illiquid \\
\hline
Valuation & Market-determined & Negotiated privately \\
\hline
Regulation & Heavy compliance & Light compliance \\
\hline
Capital Raising & IPOs, secondary offerings, bonds & Private placements, VC, PE \\
\hline
Time Horizon & Short-term due to market pressure & Long-term focus possible \\
\hline
\end{tabular}
\caption{Public vs. Private Corporations}
\end{table}

\subsubsection*{Methods of Going Public / Private}

\begin{itemize}
    \item \textbf{IPO} – New shares issued, underwritten, raises fresh capital.
    \item \textbf{Direct Listing} – Existing shares listed, no new capital, quicker.
    \item \textbf{SPAC} – Blank-check company IPO $\rightarrow$ acquires target later.
    \item \textbf{Going Private} – Buyout + delisting to reduce compliance costs.  
\end{itemize}

\section*{23 INVESTORS AND OTHER STAKEHOLDERS}

\subsection*{Module 23.1: Stakeholders and ESG Factors}

\subsubsection*{LOS 23.a: Financial Claims and Motivations of Lenders vs. Shareholders}

\begin{itemize}
    \item \textbf{Debtholders (Lenders):}
    \begin{itemize}
        \item Legal, contractual claim on \textbf{interest + principal}.
        \item Higher claim priority than equity holders.
        \item Lower risk $\rightarrow$ lower required return.
        \item Limited liability: losses cannot exceed invested amount.
        \item No upside beyond contractual payments.
    \end{itemize}

    \item \textbf{Equity Holders (Shareholders):}
    \begin{itemize}
        \item Residual claim on company’s net assets after all liabilities.
        \item Unlimited upside potential if company grows.
        \item Limited liability (max loss = investment).
        \item Dilution risk if new equity is issued.
    \end{itemize}

    \item \textbf{Company Value Relation:}
    \[
    V_{\text{Company}} = V_{\text{Debt}} + V_{\text{Equity}}
    \]
    \begin{itemize}
        \item If $V_{\text{Company}} > V_{\text{Debt}}$ $\Rightarrow$ Equity increases with company value.
        \item If $V_{\text{Company}} < V_{\text{Debt}}$ $\Rightarrow$ Equity = 0; Debt value falls.
    \end{itemize}
\end{itemize}

\textbf{Example: Leverage and ROE}
\begin{itemize}
    \item Revenues = \$1,000; Expenses = \$800; Assets = \$1,000.
    \item Financing:
    \begin{enumerate}[label=(\alph*)]
        \item \textbf{100\% Equity:}
        \[
        \text{Net Income} = 200, \quad \text{Equity} = 1,000
        \]
        \[
        ROE = \frac{200}{1,000} = 20\%
        \]
        \item \textbf{50/50 Debt + Equity (Debt = 500, Equity = 500, Interest = 10\% of 500 = 50):}
        \[
        \text{Net Income} = 200 - 50 = 150, \quad \text{Equity} = 500
        \]
        \[
        ROE = \frac{150}{500} = 30\%
        \]
    \end{enumerate}
\end{itemize}

\textbf{Interpretation:}  
Leverage increases ROE as long as return on assets ($200/1000 = 20\%$) $>$ cost of debt (10\%).

\textbf{Example: Impact of 15\% Revenue Decrease}
\begin{itemize}
    \item New Revenues = \$850, Expenses = \$800.
    \item EBIT = \$50.
    \begin{enumerate}[label=(\alph*)]
        \item \textbf{100\% Equity:}
        \[
        \text{Net Income} = 50, \quad ROE = \frac{50}{1,000} = 5\%
        \]
        \item \textbf{50/50 Debt + Equity:}
        \[
        \text{Interest} = 50, \quad \text{Net Income} = 0, \quad ROE = 0\%
        \]
    \end{enumerate}
\end{itemize}

\textbf{Conflict of Interest:}
\begin{itemize}
    \item Shareholders may prefer risky projects (potential upside).
    \item Debtholders prefer safety (default risk reduces their value).
    \item Covenants (e.g., max leverage, min interest coverage) restrict management to protect debtholders.
\end{itemize}

---

\subsubsection*{LOS 23.b: Stakeholder Groups and Their Interests}

\begin{itemize}
    \item \textbf{Shareholder Theory:} Focus on maximizing equity value (owners vs. managers conflict).
    \item \textbf{Stakeholder Theory:} Broader view $\rightarrow$ conflicts among many groups.
\end{itemize}

\begin{table}[H]
\centering
\resizebox{\textwidth}{!}{%
\footnotesize
\begin{tabular}{|l|p{5.5cm}|p{6.5cm}|}
\hline
\textbf{Stakeholder} & \textbf{Role / Interests} & \textbf{Notes / Conflicts} \\
\hline
\textbf{Shareholders} & Residual claimants, voting rights, want profitability \& growth & Interested in maximizing share value, potential conflict with debt investors and managers \\
\hline
\textbf{Lenders} & Provide debt capital (public bonds, private loans) & Concerned with repayment, prefer lower risk; protected by covenants \\
\hline
\textbf{Board of Directors} & Protect shareholder interests, hire/fire managers, set strategy & One-tier (U.S., U.K.) vs. Two-tier (continental Europe) structures \\
\hline
\textbf{Managers (Executives)} & Run operations, compensated with salary + bonuses + perks & Incentives linked to firm performance; may pursue personal goals \\
\hline
\textbf{Employees} & Provide labor and skills, want pay, training, career growth & Can align via stock options or stock participation plans \\
\hline
\textbf{Suppliers} & Provide goods/services, want stability and solvency & Short-term creditors; sensitive to firm liquidity \\
\hline
\textbf{Customers} & Demand quality goods/services at fair price, after-sale support & Increasing interest in ESG (environment, ethics, sustainability) \\
\hline
\textbf{Governments/Regulators} & Collect taxes, ensure compliance, promote employment \& social welfare & Impose laws/regulations, may conflict with shareholders \\
\hline
\end{tabular}}
\caption{Corporate Stakeholder Groups and Interests}
\end{table}

---

\subsubsection*{LOS 23.c: ESG Factors Considered by Investors}

\textbf{Why ESG matters:}
\begin{enumerate}
    \item Governments prioritize climate/social issues via regulation.
    \item ESG factors can materially impact results (loss of goodwill, fines, governance failures).
    \item Younger investors increasingly demand ESG-aligned investments.
\end{enumerate}

\textbf{Key ESG Factors:}
\begin{itemize}
    \item \textbf{Environmental:}
    \begin{itemize}
        \item Climate change, carbon footprint, pollution, deforestation, water scarcity.
        \item \textbf{Risks:} physical (weather events), transition (regulations), stranded assets.
        \item Example: Oil spill $\rightarrow$ penalties, cleanup, litigation, reputation loss.
    \end{itemize}
    
    \item \textbf{Social:}
    \begin{itemize}
        \item Customer privacy, employee engagement, diversity \& inclusion, labor relations, community ties.
        \item Strong social practices $\rightarrow$ lower turnover, higher productivity, brand loyalty.
    \end{itemize}
    
    \item \textbf{Governance:}
    \begin{itemize}
        \item Board structure, independence, audit committee, executive compensation, anti-corruption, lobbying.
        \item Adequate governance ensures managers act ethically, lawfully, and in shareholder interests.
    \end{itemize}
\end{itemize}

\textbf{Evaluation of ESG Risks:}
\begin{itemize}
    \item Analysts must assess ESG impact on future cash flows.
    \item \textbf{Equity investors:} Most exposed (residual claimants).
    \item \textbf{Debt investors:} Less exposed unless ESG issues threaten solvency.
    \item Long-term debtholders $\Rightarrow$ more exposed than short-term debtholders (e.g., stranded coal plant).
\end{itemize}

---
\section*{24 CORPORATE GOVERNANCE: CONFLICTS, MECHANISMS, RISKS, AND BENEFITS}

\subsection*{Module 24.1: Corporate Governance}

\subsubsection*{LOS 24.a: Principal-Agent Relationship and Conflicts}

\begin{itemize}
    \item \textbf{Principal–Agent Relationship:}
    \begin{itemize}
        \item Principal hires agent to act in their interest.
        \item Conflict arises when agent’s incentives $\neq$ principal’s goals.
        \item \textbf{Agency Costs:}
        \begin{itemize}
            \item Direct costs: monitoring agents (auditors, compliance staff).
            \item Indirect costs: lost opportunities, inefficient decisions.
        \end{itemize}
    \end{itemize}

    \item \textbf{Example: Insurance Agent Conflict}
    \begin{itemize}
        \item Agent earns commission $\Rightarrow$ incentive to write risky policies.
        \item Principal (insurance company) wants only good risks.
        \item Mitigation: underwriting standards + termination of bad agents.
    \end{itemize}

    \item \textbf{Shareholders vs. Managers/Directors}
    \begin{itemize}
        \item Shareholders = principals (want growth, can diversify).
        \item Managers/Directors = agents (prefer lower risk, job security, perks).
        \item \textbf{Information asymmetry:} managers know more than shareholders, making monitoring difficult.
    \end{itemize}

    \item \textbf{Common Principal-Agent Conflicts:}
    \begin{itemize}
        \item Insufficient managerial effort $\rightarrow$ poor investments.
        \item Compensation misalignment:
        \begin{itemize}
            \item Options $\rightarrow$ managers may take excessive risk (no downside for them).
            \item Cash-based pay $\rightarrow$ managers too risk-averse.
        \end{itemize}
        \item Empire building (value-destroying acquisitions to grow size).
        \item Entrenchment (protecting own position, copying competitors).
        \item Self-dealing (using company resources for personal gain).
    \end{itemize}

    \item \textbf{Conflicts Between Shareholders:}
    \begin{itemize}
        \item Controlling shareholders vs. minority shareholders:
        \begin{itemize}
            \item Controlling block may push diversification to reduce personal wealth risk.
            \item Minority shareholders prefer efficient capital allocation.
        \end{itemize}
        \item Dual-class shares (different voting rights) can entrench founders.
        \item CFA Institute \textbf{opposes dual-class structures}.
    \end{itemize}

    \item \textbf{Conflicts Between Creditors and Shareholders:}
    \begin{itemize}
        \item Shareholders prefer higher risk (potential upside).
        \item Creditors have fixed upside (interest) but face higher default risk.
        \item Management actions against creditors:
        \begin{itemize}
            \item Issuing new debt (raising leverage).
            \item Paying out higher dividends (reducing collateral).
        \end{itemize}
        \item Conflict is worse for long-term debtholders.
    \end{itemize}
\end{itemize}

---

\subsubsection*{LOS 24.b: Corporate Governance and Mechanisms to Manage Stakeholder Relationships}

\begin{itemize}
    \item \textbf{Corporate Governance:}  
    System of internal controls and procedures to align interests and minimize conflicts.

    \item \textbf{Stakeholder Management:}  
    Effective communication, transparency, and fair treatment of all stakeholders.

    \item \textbf{Transparency in Reporting:}
    \begin{itemize}
        \item Public companies disclose in annual reports, proxy statements, notices.
        \item Information includes financial performance, related-party transactions, executive pay, governance.
        \item Improves monitoring by reducing information asymmetry.
    \end{itemize}
\end{itemize}

\textbf{Stakeholder Mechanisms:}

\begin{table}[H]
\centering
\resizebox{\textwidth}{!}{%
\footnotesize
\begin{tabular}{|l|p{6.5cm}|p{6.5cm}|}
\hline
\textbf{Mechanism} & \textbf{Description} & \textbf{Examples / Notes} \\
\hline
Annual General Meeting (AGM) & Management reports audited results, strategy, answers Qs. Shareholders vote. & Ordinary resolutions = simple majority (auditors, directors). Extraordinary = mergers, takeovers, liquidation. \\
\hline
Proxy Voting & Shareholders delegate voting rights. & Proxy can specify instructions or give discretion. \\
\hline
Activist Shareholders & Push for changes to improve shareholder value. & Lawsuits, board representation, proxy contests, tender offers. \\
\hline
Hostile Takeovers & Unfriendly acquisitions pressure boards to align with shareholder interests. & Defense: staggered boards, poison pills. \\
\hline
Creditors & Bond indentures define covenants; may form committees in distress. & Covenants: restrictions on leverage, collateral requirements. \\
\hline
Board Committees & Specialized groups oversee governance. & Audit, Compensation, Governance, Risk, Investment. \\
\hline
Employees/Customers/Suppliers & Labor laws, unions, ESOPs, contracts, social media influence. & Employees may have board representation in some countries. \\
\hline
Governments/Regulators & Enforce laws, listing requirements, governance codes. & Workplace safety, environmental regulation, stock exchange rules. \\
\hline
\end{tabular}}
\caption{Mechanisms of Corporate Governance and Stakeholder Management}
\end{table}

\textbf{Board Committees (examples):}
\begin{itemize}
    \item Audit Committee: financial reporting, internal controls, external auditors.
    \item Nominating/Governance Committee: board elections, code of ethics, compliance.
    \item Compensation Committee: exec/director pay (independent directors only).
    \item Risk Committee (banks): risk tolerance, enterprise-wide risk.
    \item Investment Committee (insurance): capital management, prudent investment policies.
\end{itemize}

---

\subsubsection*{LOS 24.c: Risks of Poor Governance and Benefits of Effective Governance}

\begin{itemize}
    \item \textbf{Risks of Poor Governance:}
    \begin{itemize}
        \item Weak audits/board oversight $\rightarrow$ fraud, poor recordkeeping.
        \item Managers pursue self-interest (risk avoidance, empire building, self-dealing).
        \item Misaligned executive compensation.
        \item Related-party transactions benefiting insiders.
        \item Poor compliance $\rightarrow$ legal and reputational risks.
        \item Violating stakeholder rights $\rightarrow$ lawsuits, defaults, bankruptcy.
    \end{itemize}

    \item \textbf{Benefits of Effective Governance:}
    \begin{itemize}
        \item Aligns management incentives with shareholder interests.
        \item Improves efficiency and monitoring.
        \item Reduces legal/regulatory risks.
        \item Formal conflict-of-interest policies improve performance.
        \item Better creditor protection $\rightarrow$ lower cost of debt financing.
        \item Strong governance $\Rightarrow$ higher company valuation.
    \end{itemize}
\end{itemize}

---
\section*{25 WORKING CAPITAL AND LIQUIDITY}

\subsection*{Module 25.1: Liquidity Measures and Management}

\subsubsection*{LOS 25.a: Cash Conversion Cycle (CCC)}

\begin{itemize}
    \item \textbf{Definition:} Measures efficiency of cash flow management:
    \begin{align*}
CCC &= \text{Days of Inventory on Hand (DOH)} \\
    &+ \text{Days Sales Outstanding (DSO)} \\
    &- \text{Days Payables Outstanding (DPO)}
\end{align*}
    \item Represents time taken to turn investments in inventory into cash inflows.
    \item Lower CCC $\Rightarrow$ faster cash generation, higher efficiency.
    \item High CCC $\Rightarrow$ slower conversion, potential liquidity issues.
\end{itemize}

\textbf{Management of CCC:}
\begin{itemize}
    \item Reduce DOH (inventory) → risk of shortages.
    \item Reduce DSO (receivables) → risk of lost sales.
    \item Increase DPO (payables) → implicit supplier financing, but may strain supplier relations.
\end{itemize}

\textbf{Supplier Financing Terms:}
\begin{itemize}
    \item Terms: $a/b \ \text{net} \ c$ = discount $a$ if paid within $b$ days; otherwise full payment due in $c$ days.
    \item Effective Annual Rate (EAR) of supplier financing:
    \[
    EAR = \left( 1 + \frac{a}{1-a} \right)^{\frac{365}{c-b}} - 1
    \]
\end{itemize}

\textbf{Example: EAR of Supplier Financing}
\begin{itemize}
    \item Terms: 2/10 net 30.
    \item Financing period = $30 - 10 = 20$ days.
    \[
    EAR = \left( 1 + \frac{0.02}{0.98} \right)^{\frac{365}{20}} - 1 = 44.6\%
    \]
    \item Compare with bank loan at 8\% → cheaper to borrow from bank to pay early.
\end{itemize}

\textbf{Industry Comparison:}
\begin{itemize}
    \item Pharma: long CCC (large, expensive inventories).
    \item Airlines: short CCC (prepaid sales, minimal inventory).
    \item Compare CCC \textbf{within industries}.
\end{itemize}

\textbf{Working Capital:}
\[
\text{Net Working Capital} = \text{Operating Current Assets} - \text{Operating Current Liabilities}
\]
\begin{itemize}
    \item Expressed relative to sales for comparability across companies.
    \item Closely linked with CCC.
\end{itemize}

---

\subsubsection*{LOS 25.b: Liquidity and Issuer Liquidity Levels}

\begin{itemize}
    \item \textbf{Liquidity of Assets:} Nearness to cash.
    \begin{itemize}
        \item Cash \& marketable securities = highly liquid.
        \item AR less liquid; Inventory least liquid.
    \end{itemize}
    \item \textbf{Liquidity of Issuer:} Ability to meet short-term obligations.
\end{itemize}

\textbf{Primary Liquidity Sources:}
\begin{itemize}
    \item Cash \& marketable securities.
    \item Bank borrowings.
    \item Operating cash flow.
\end{itemize}

\textbf{Secondary Liquidity Sources:}
\begin{itemize}
    \item Suspend dividends.
    \item Delay/reduce capex.
    \item Sell assets.
    \item Issue equity.
    \item Restructure debt.
    \item Bankruptcy protection (last resort).
\end{itemize}

\textbf{Example: Cost of Liquidity}
\begin{itemize}
    \item Drake Corp sells \$600k assets for \$480k immediate cash.
    \item Cost of liquidity = \$120k = 20\%.
    \item Interpretation: forced sale at discount = liquidity cost.
\end{itemize}

\textbf{Factors Affecting Liquidity:}
\begin{itemize}
    \item \textbf{Drags} = delayed inflows (slow AR collections, obsolete inventory).
    \item \textbf{Pulls} = accelerated outflows (suppliers reduce credit, demand faster payment).
\end{itemize}

\textbf{Liquidity Ratios:}
\[
\text{Current Ratio} = \frac{CA}{CL}, \quad 
\text{Quick Ratio} = \frac{CA - Inventory}{CL}, \quad
\text{Cash Ratio} = \frac{\text{Cash \& Equivalents}}{CL}
\]

\textbf{Example: Drake Corp Liquidity Ratios}
\begin{table}[H]
\centering
\footnotesize
\begin{tabular}{|l|c|c|}
\hline
\textbf{Ratio} & \textbf{20X1} & \textbf{20X2} \\
\hline
Current Ratio & 1.99 & 1.73 \\
\hline
Quick Ratio & 1.00 & 0.58 \\
\hline
Cash Ratio & 0.45 & 0.19 \\
\hline
\end{tabular}
\caption{Liquidity Ratios for Drake Corporation}
\end{table}

\textbf{Interpretation:} Declining across the board → worsening liquidity.

---

\subsubsection*{LOS 25.c: Working Capital and Liquidity Management}

\begin{itemize}
    \item \textbf{Objective:} Balance profitability with sufficient liquidity to operate and meet obligations.
    \item \textbf{Trade-off:} 
    \begin{itemize}
        \item More ST assets = safety but lower returns.
        \item ST financing = cheaper but riskier than LT debt.
    \end{itemize}
\end{itemize}

\textbf{Working Capital Approaches:}

\begin{table}[H]
\centering
\footnotesize
\begin{tabular}{|l|p{6cm}|p{6cm}|}
\hline
\textbf{Approach} & \textbf{Description} & \textbf{Benefits / Risks} \\
\hline
Conservative & Hold high ST assets, financed by LT debt/equity. & Safer (less rollover risk), flexibility, higher costs, lower profitability. \\
\hline
Aggressive & Low ST assets, financed by ST debt. & Lower cost, higher risk of liquidity shortfalls, vulnerable to disruptions. \\
\hline
Moderate & Match permanent assets with LT financing; seasonal with ST financing. & Balanced cost and risk. \\
\hline
\end{tabular}
\caption{Working Capital Management Approaches}
\end{table}

\textbf{Short-Term Liquidity Sources:}
\begin{itemize}
    \item Factors influencing access:
    \begin{itemize}
        \item Company size (small firms = limited options).
        \item Creditworthiness (affects rates and covenants).
        \item Legal system protections for lenders.
        \item Regulatory environment (restrictions in banking/utilities).
        \item Available collateral (underlying assets).
    \end{itemize}
\end{itemize}

---

\section*{26 CAPITAL INVESTMENTS AND CAPITAL ALLOCATION}

\subsection*{Module 26.1: Capital Investments and Project Measures}

\subsubsection*{LOS 26.a: Types of Capital Investments}

\begin{itemize}
    \item \textbf{Going Concern Projects:}
    \begin{itemize}
        \item Maintain business operations or reduce costs.
        \item Examples: replacing obsolete equipment, efficiency improvements.
        \item Financing approach: match-funding (finance with capital sources consistent with project life).
        \item Analysts often approximate by annual depreciation expense.
    \end{itemize}

    \item \textbf{Regulatory / Compliance Projects:}
    \begin{itemize}
        \item Required by government or insurers (e.g., safety, environmental).
        \item Generate little to no revenue; focus on compliance alternatives.
    \end{itemize}

    \item \textbf{Expansion Projects:}
    \begin{itemize}
        \item Grow the business: enter new markets, new products.
        \item Require detailed forecasting of revenues/expenses.
    \end{itemize}

    \item \textbf{Other Projects:}
    \begin{itemize}
        \item Outside current business lines (startups, new tech, acquisitions).
        \item High uncertainty and risk (e.g., risk of overpaying in acquisitions).
    \end{itemize}
\end{itemize}

---

\subsubsection*{LOS 26.b: Capital Allocation Process and Project Measures}

\textbf{Capital Allocation Steps:}
\begin{enumerate}
    \item Idea generation (sources: mgmt, employees, external).
    \item Analyze proposals (forecast expected cash flows).
    \item Create firm-wide capital budget (prioritize projects by strategy and resources).
    \item Monitor and post-audit (compare actual vs. forecasted results, improve forecasting).
\end{enumerate}

\textbf{Net Present Value (NPV):}
\[
NPV = \sum_{t=1}^{T} \frac{CF_t}{(1+r)^t} - C_0
\]
\begin{itemize}
    \item $CF_t$ = incremental after-tax cash flow.
    \item $r$ = discount rate (cost of capital).
    \item $C_0$ = initial investment.
    \item Decision rule: Accept if NPV $>$ 0.
\end{itemize}

\textbf{Example: NPV Calculation}
\begin{itemize}
    \item Assume $C_0 = 1,000$, cash inflows = 400 per year for 3 years, cost of capital = 9\%.
    \[
    NPV = \frac{400}{1.09} + \frac{400}{1.09^2} + \frac{400}{1.09^3} - 1,000
    = 34.6 \ (\text{positive, accept project})
    \]
\end{itemize}

---

\textbf{Internal Rate of Return (IRR):}
\[
NPV = 0 \quad \Rightarrow \quad \sum_{t=1}^{T} \frac{CF_t}{(1+IRR)^t} = C_0
\]
\begin{itemize}
    \item IRR = discount rate that makes NPV = 0.
    \item Decision rule: Accept if IRR $>$ required rate of return.
    \item Required rate of return = usually cost of capital (adjusted for project risk).
    \item Hurdle rate = minimum acceptable IRR.
\end{itemize}

\textbf{Example: IRR Calculation}
\begin{itemize}
    \item Same cash flows as NPV example.
    \item IRR = 19.4\% (calculated with calculator/Excel).
    \item Since IRR (19.4\%) $>$ required rate (9\%) $\Rightarrow$ accept project.
\end{itemize}

\textbf{Cash Flow Patterns:}
\begin{itemize}
    \item Conventional: one sign change (outflow → inflows).
    \item Unconventional: multiple sign changes (may produce multiple IRRs).
\end{itemize}

---

\textbf{NPV vs. IRR — Comparison}

\begin{table}[H]
\centering
\footnotesize
\begin{tabular}{|l|p{6cm}|p{6cm}|}
\hline
\textbf{Method} & \textbf{Advantages} & \textbf{Disadvantages} \\
\hline
NPV & Direct measure of value added to shareholders; theoretically best. & Requires cost of capital; less intuitive to non-financial managers. \\
\hline
IRR & Intuitive \% return measure; shows margin of safety. & Assumes reinvestment at IRR; multiple IRRs possible with unconventional flows; may conflict with NPV. \\
\hline
\end{tabular}
\caption{NPV vs. IRR Comparison}
\end{table}

---

\textbf{Return on Invested Capital (ROIC):}
\[
ROIC = \frac{NOPAT}{\text{Average Total Capital}} = \frac{\text{After-tax Operating Profit}}{\text{Debt + Equity}}
\]

\begin{itemize}
    \item $NOPAT =$ Net Operating Profit After Tax = Net Income + After-tax Interest.
    \item Measures return to all capital providers (debt + equity).
    \item Decomposition:
    \[
    ROIC = \text{Operating Margin After-tax} \times \text{Capital Turnover}
    \]
    \item Compare ROIC to cost of capital:
    \begin{itemize}
        \item ROIC $>$ cost of capital $\Rightarrow$ value creation.
        \item ROIC $<$ cost of capital $\Rightarrow$ value destruction.
    \end{itemize}
\end{itemize}

\textbf{Attractions of ROIC:}
\begin{itemize}
    \item Based on accounting data (available to outsiders).
    \item Firm-wide measure (not project-specific).
    \item Useful for investors who can’t access internal project details.
\end{itemize}

\textbf{Limitations of ROIC:}
\begin{itemize}
    \item Not comparable across firms due to accounting differences.
    \item Backward-looking, volatile year-to-year.
    \item Firm-wide measure may mask bad projects within good performance.
\end{itemize}

---

\subsubsection*{LOS 26.c (Implied): Contrast of Measures in Capital Allocation}

\begin{itemize}
    \item \textbf{NPV:} Best theoretical measure, absolute dollar impact on firm value.
    \item \textbf{IRR:} Useful as % return, intuitive, but may mislead in certain cases.
    \item \textbf{ROIC:} Firm-wide accounting-based measure, useful for external investors.
\end{itemize}

\begin{table}[H]
\centering
\footnotesize
\begin{tabular}{|l|p{4.5cm}|p{4.5cm}|p{4.5cm}|}
\hline
\textbf{Metric} & \textbf{What it Measures} & \textbf{Strengths} & \textbf{Weaknesses} \\
\hline
NPV & Dollar value added & Direct impact on shareholder wealth; best decision criterion & Needs cost of capital, less intuitive \\
\hline
IRR & Percentage return & Easy to communicate; shows margin of safety & Assumes reinvestment at IRR; multiple IRRs possible \\
\hline
ROIC & Firm-wide return on capital employed & Accounting data available to outsiders; compares firm return to cost of capital & Backward-looking; may mask poor projects; non-comparable across firms \\
\hline
\end{tabular}
\caption{Comparison of NPV, IRR, and ROIC}
\end{table}

\subsection*{Module 26.2: Capital Allocation Principles and Real Options}

\subsubsection*{LOS 26.c: Principles of Capital Allocation and Common Pitfalls}

\textbf{Principles of Capital Allocation:}
\begin{itemize}
    \item \textbf{Decisions based on after-tax cash flows, not accounting income.}
    \begin{itemize}
        \item Accounting income uses accruals $\Rightarrow$ ignores timing.
        \item Taxes reduce firm value $\Rightarrow$ include tax effects and tax shields.
        \item Non-cash deductions (depreciation, amortization) provide tax savings and must be included.
    \end{itemize}

    \item \textbf{Only incremental cash flows matter.}
    \begin{itemize}
        \item \textbf{Sunk costs:} irrecoverable costs (e.g., past consulting fees) should be ignored.
        \item \textbf{Cannibalization:} negative impact on existing product sales (e.g., diet soda reduces regular soda sales).
        \item \textbf{Positive externalities:} complementary effects (e.g., iPhone boosts MacBook sales).
    \end{itemize}

    \item \textbf{Timing of cash flows is important.}
    \begin{itemize}
        \item Time value of money (earlier cash flows = more valuable).
        \item Projects with identical totals but earlier inflows are better.
    \end{itemize}
\end{itemize}

\textbf{Common Mistakes in Capital Allocation:}
\begin{itemize}
    \item \textbf{Cognitive Errors (calculation errors):}
    \begin{itemize}
        \item Poor forecasting (misallocating overhead, ignoring competitor reactions).
        \item Ignoring cost of internal funds (retained earnings $\neq$ free).
        \item Inflation misaccounting (nominal vs. real mismatch).
    \end{itemize}

    \item \textbf{Behavioral Biases (judgment errors):}
    \begin{itemize}
        \item Pet projects of senior management (optimistic projections, low scrutiny).
        \item Inertia in capital budgets (anchoring to last year rather than new opportunities).
        \item EPS/ROE focus (rejecting good NPV projects if short-term EPS/ROE falls).
        \item Failure to generate alternatives (settling for the first “good” project).
    \end{itemize}
\end{itemize}

\begin{table}[H]
\centering
\footnotesize
\begin{tabular}{|l|p{6cm}|p{6cm}|}
\hline
\textbf{Category} & \textbf{Examples} & \textbf{Consequences} \\
\hline
Cognitive Errors & Ignoring competitor response; not adjusting for inflation; ignoring cost of equity & Misleading projections, poor project selection \\
\hline
Behavioral Biases & Pet projects; static budgets; EPS/ROE myopia; lack of alternatives & Biased decisions, inefficient allocation, shareholder value destruction \\
\hline
\end{tabular}
\caption{Common Pitfalls in Capital Allocation}
\end{table}

---

\subsubsection*{LOS 26.d: Real Options in Capital Investments}

\textbf{Definition:}
\begin{itemize}
    \item Real options = flexibility embedded in projects.
    \item Similar to financial options: right (not obligation) to take future action.
    \item Options never have negative value (if worthless, simply not exercised).
\end{itemize}

\textbf{Types of Real Options:}
\begin{itemize}
    \item \textbf{Timing Options:} Delay investment until better information is available.  
    \emph{Example:} A firm postpones building a factory until demand forecasts improve.

    \item \textbf{Abandonment Options:} Exit if losses exceed benefits.  
    \emph{Example:} A mining project abandoned if commodity prices fall below break-even.

    \item \textbf{Expansion (Growth) Options:} Invest further if project is successful.  
    \emph{Example:} Build additional data centers if cloud demand grows.

    \item \textbf{Flexibility Options:} Adjust operations dynamically.
    \begin{itemize}
        \item \textbf{Price-setting:} Raise prices when demand is strong.  
        \item \textbf{Production-flexibility:} Use different materials, overtime, or adjust product mix.
    \end{itemize}

    \item \textbf{Fundamental Options:} Project itself behaves like an option on underlying asset.  
    \emph{Example:} A copper mine is valuable only if copper prices are high (option to open/close).
\end{itemize}

\textbf{Incorporating Real Options into Valuation:}
\begin{itemize}
    \item Estimate value of options and add to base NPV (subtract cost of option).
    \item Tools: option pricing models, decision trees.
    \item NPV without options = minimum project value.
\end{itemize}

\begin{table}[H]
\centering
\footnotesize
\begin{tabular}{|l|p{6cm}|p{6cm}|}
\hline
\textbf{Option Type} & \textbf{Description} & \textbf{Example} \\
\hline
Timing & Delay investment until uncertainty resolves & Pharmaceutical firm delays R\&D until regulatory clarity \\
\hline
Abandonment & Exit project if future cash flows $<$ salvage value & Close factory if losses persist \\
\hline
Expansion & Increase scale if successful & Add new stores if pilot store is profitable \\
\hline
Flexibility & Adjust operations (pricing, production) & Raise price or switch input materials \\
\hline
Fundamental & Project linked to underlying asset price & Copper mine depends on copper spot price \\
\hline
\end{tabular}
\caption{Types of Real Options in Capital Investments}
\end{table}

\section*{27 CAPITAL STRUCTURE}

\subsection*{Module 27.1: Weighted-Average Cost of Capital (WACC)}

\subsubsection*{LOS 27.a: Calculate and Interpret WACC}

\textbf{Definition:}
\begin{itemize}
    \item WACC = blended cost of debt and equity financing. 
    \item Reflects a firm's opportunity cost of capital for new projects.
\end{itemize}

\[
\text{WACC} = \big[w_d \times r_d \times (1 - T)\big] + \big[w_e \times r_e\big]
\]
Where:
\begin{itemize}
    \item $w_d$ = weight of debt
    \item $r_d$ = pretax cost of debt
    \item $T$ = corporate tax rate
    \item $w_e$ = weight of equity
    \item $r_e$ = cost of equity
\end{itemize}

\textbf{Key Notes:}
\begin{itemize}
    \item Cost of debt $<$ cost of equity (due to priority of claims and tax deductibility).
    \item After-tax adjustment: interest expense is tax deductible.
    \item Weights may be based on:
    \begin{itemize}
        \item \textbf{Target weights} (management’s intended capital structure).
        \item \textbf{Market value weights} (reflect current opportunity cost of capital, preferred for valuation).
    \end{itemize}
\end{itemize}

\textbf{Example: ABC Inc.}
\begin{itemize}
    \item Capital structure: 50\% debt, 50\% equity
    \item Cost of debt = 8\%
    \item Cost of equity = 11\%
    \item Corporate tax rate = 30\%
\end{itemize}

\[
\text{WACC} = (0.50 \times 0.08 \times (1-0.30)) + (0.50 \times 0.11) = 0.083 = 8.3\%
\]

\begin{table}[H]
\centering
\footnotesize
\begin{tabular}{|l|c|c|}
\hline
\textbf{Input} & \textbf{Value} & \textbf{Contribution to WACC} \\
\hline
Debt weight ($w_d$) & 0.50 & $0.50 \times 0.08 \times (1-0.30) = 0.028$ \\
Equity weight ($w_e$) & 0.50 & $0.50 \times 0.11 = 0.055$ \\
\hline
\textbf{Total WACC} & -- & \textbf{0.083 = 8.3\%} \\
\hline
\end{tabular}
\caption{Example: WACC for ABC Inc.}
\end{table}

---

\subsubsection*{LOS 27.b: Factors Affecting Capital Structure and WACC}

\textbf{Objective:}
\begin{itemize}
    \item Firms target a capital structure that minimizes WACC.
    \item Consider the \textbf{capacity to service debt}.
\end{itemize}

\textbf{Internal Factors:}
\begin{itemize}
    \item \textbf{Revenue growth \& stability:} More stable $\Rightarrow$ higher debt capacity.
    \item \textbf{Cash flow predictability:} Stable/growing $\Rightarrow$ more debt support.
    \item \textbf{Business risk:} Higher risk $\Rightarrow$ lower debt capacity.
    \item \textbf{Asset base:} Tangible/liquid assets $\Rightarrow$ better collateral, cheaper debt.
\end{itemize}

\textbf{External Factors:}
\begin{itemize}
    \item Market conditions (credit spreads, investor appetite).
    \item Business cycle phase (recession $\Rightarrow$ higher spreads).
    \item Regulation \& industry norms.
\end{itemize}

\textbf{Firm-Specific Characteristics:}
\begin{itemize}
    \item High existing leverage $\Rightarrow$ lower additional debt capacity.
    \item Volatile revenues/earnings $\Rightarrow$ less ability to borrow.
    \item Leverage and coverage ratios (e.g., $\text{Interest Coverage} = \tfrac{\text{EBIT}}{\text{Interest Expense}}$) measure debt service capacity.
\end{itemize}

\textbf{Impact of Life Cycle Stages:}
\begin{enumerate}
    \item \textbf{Start-up:}
    \begin{itemize}
        \item Sales/cash flow low or negative.
        \item High business risk, limited collateral.
        \item Mostly equity financing; may use \emph{convertible debt} or \emph{leasing}.
    \end{itemize}
    \item \textbf{Growth:}
    \begin{itemize}
        \item Rising revenue/cash flow, reduced business risk.
        \item Conservative use of debt (secured by assets/receivables).
    \end{itemize}
    \item \textbf{Mature:}
    \begin{itemize}
        \item Slowing growth, stable cash flows.
        \item Greater access to low-cost debt, including unsecured.
    \end{itemize}
\end{enumerate}

\begin{table}[H]
\centering
\footnotesize
\begin{tabular}{|l|p{4cm}|p{4cm}|p{4cm}|}
\hline
\textbf{Stage} & \textbf{Characteristics} & \textbf{Debt Capacity} & \textbf{Financing Sources} \\
\hline
Start-up & Low/negative CF, high risk, little collateral & Very low & Equity, convertible debt, leasing \\
\hline
Growth & Rising CF, reduced risk & Moderate (secured debt possible) & Equity + secured loans \\
\hline
Mature & Stable CF, low risk & High (unsecured debt available) & Debt + retained earnings \\
\hline
\end{tabular}
\caption{Life Cycle Stages and Capital Structure}
\end{table}

\textbf{Top-Down (Macroeconomic) Factors:}
\begin{itemize}
    \item Benchmark interest rates (e.g., U.S. Treasury).
    \item Credit spreads (wider in downturns).
    \item Inflation, GDP growth, monetary policy, FX rates.
    \item Cyclical industries more affected than non-cyclical.
\end{itemize}

\textbf{Industry-Specific Examples:}
\begin{itemize}
    \item Oil sector: profitability linked to oil prices. Credit spreads narrow when oil prices rise.
    \item Subscription-based firms (e.g., SaaS): high debt capacity due to stable revenues.
    \item Cyclical manufacturers: lower debt tolerance due to volatile demand.
\end{itemize}

\subsection*{27.2 Capital Structure Theories}

\begin{itemize}
  \item \textbf{MM Proposition I (No Taxes): Capital Structure Irrelevance}
  \begin{itemize}
    \item Assumptions:
    \begin{itemize}
      \item Perfect capital markets: no taxes, transaction costs, or bankruptcy costs.
      \item Homogeneous expectations of investors.
      \item Risk-free borrowing/lending possible.
      \item No agency costs (no conflict between managers and shareholders).
      \item Investment decisions independent of financing decisions.
    \end{itemize}
    \item Key Idea: Firm value does not depend on how it is financed (``the size of the pie is constant regardless of how sliced'').
    \item Explanation:
      \begin{itemize}
        \item In an all-equity firm, value = PV of EBIT discounted at cost of equity.
        \item In a levered firm, EBIT is split between debt and equity holders, but total value = same as all-equity case.
      \end{itemize}
  \end{itemize}

  \item \textbf{MM Proposition II (No Taxes): Cost of Equity and Leverage}
  \begin{itemize}
    \item As leverage ($D/E$ ratio) increases:
    \begin{itemize}
      \item Cost of equity rises linearly due to higher risk of residual claims.
      \item Cost of debt remains lower than cost of equity (priority claims).
      \item WACC remains constant (benefit of cheaper debt is offset by higher cost of equity).
    \end{itemize}
    \item Formula:
    \[
      k_e = k_0 + (k_0 - k_d)\frac{D}{E}
    \]
    where:
    \begin{itemize}
      \item $k_e =$ cost of equity
      \item $k_0 =$ cost of capital of unlevered firm
      \item $k_d =$ cost of debt
    \end{itemize}
  \end{itemize}

  \item \textbf{MM With Taxes: Debt Creates Value}
  \begin{itemize}
    \item Interest expense is tax-deductible, creating a \textbf{tax shield}.
    \item Value of a levered firm:
    \[
      V_L = V_U + (t \times D)
    \]
    where $t$ is corporate tax rate.
    \item WACC declines with leverage since after-tax debt is cheaper:
    \[
      WACC = \frac{E}{V}k_e + \frac{D}{V}k_d(1 - t)
    \]
    \item Implication: Value maximized at 100\% debt (theoretically).
    \item Example: If tax rate = 30\%, Debt = 100m, tax shield = $0.30 \times 100 = 30$m increase in firm value.
  \end{itemize}

  \item \textbf{Costs of Financial Distress}
  \begin{itemize}
    \item \textbf{Direct costs}: Legal + administrative fees during bankruptcy.
    \item \textbf{Indirect costs}: Lost customers, suppliers, reputation, and investment opportunities.
    \item \textbf{Agency costs of debt}: Conflicts between equity holders and debtholders during distress.
    \item Higher leverage $\Rightarrow$ higher probability of distress.
  \end{itemize}

  \item \textbf{Static Tradeoff Theory}
  \begin{itemize}
    \item Balances:
    \[
      V_L = V_U + (t \times D) - PV(\text{costs of financial distress})
    \]
    \item Firm value initially rises with leverage (tax shield), but beyond a point expected costs of distress dominate.
    \item Optimal capital structure = point where WACC is minimized and firm value is maximized.
  \end{itemize}

  \item \textbf{Target Capital Structure}
  \begin{itemize}
    \item Firms choose a long-run average mix of debt and equity to maximize value.
    \item Approaches to estimate:
    \begin{itemize}
      \item Use firm’s current structure (market-value based).
      \item Use industry averages.
      \item Incorporate observed trends in leverage.
    \end{itemize}
  \end{itemize}

  \item \textbf{Agency Costs of Equity}
  \begin{itemize}
    \item Conflicts of interest between managers and shareholders.
    \item Components:
    \begin{enumerate}
      \item Monitoring costs: supervising management, board costs.
      \item Bonding costs: commitments, insurance, non-competes.
      \item Residual losses: costs that persist despite monitoring/bonding.
    \end{enumerate}
    \item \textbf{Free Cash Flow Hypothesis}: More debt reduces excess free cash, forcing managers to act more efficiently.
  \end{itemize}

  \item \textbf{Pecking Order Theory (Asymmetric Information)}
  \begin{itemize}
    \item Firms prefer financing options in this order:
    \begin{enumerate}
      \item Internal funds (no negative signals).
      \item Debt (moderate signals).
      \item Equity (often negative signal, managers may issue when overvalued).
    \end{enumerate}
    \item Result: Capital structure is an outcome of sequential financing choices, not a pre-set optimal ratio.
  \end{itemize}
\end{itemize}

\bigskip

\begin{table}[H]
\centering
\caption{Summary of Capital Structure Theories}
\begin{tabular}{|l|p{6cm}|p{6cm}|}
\hline
\textbf{Theory} & \textbf{Key Idea} & \textbf{Implications} \\
\hline
MM I (No Taxes) & Firm value independent of capital structure. & Financing choice irrelevant for value. \\
\hline
MM II (No Taxes) & Cost of equity rises with leverage; WACC constant. & No advantage to debt or equity. \\
\hline
MM with Taxes & Tax shield of debt increases firm value. & Value maximized at 100\% debt. \\
\hline
Static Tradeoff & Tradeoff between tax shield and distress costs. & Optimal leverage exists where WACC minimized. \\
\hline
Pecking Order & Financing choice follows hierarchy (internal → debt → equity). & Capital structure is path-dependent, not fixed. \\
\hline
Agency Costs & Conflicts between managers vs. shareholders and debt vs. equity holders. & More debt can discipline managers (reduce free cash flow misuse). \\
\hline
\end{tabular}
\end{table}

\section*{28 BUSINESS MODELS}

\subsection*{Module 28.1: Business Model Features and Types}

\subsubsection*{LOS 28.a: Key Features of Business Models}

\begin{itemize}
    \item \textbf{Definition:} 
    A business model explains how a firm creates, delivers, and captures value. 
    It answers \textbf{who, what, how, where, and how much}.
    
    \item \textbf{Who (Customers):}
    \begin{itemize}
        \item Identify potential customers and segment them (e.g., geographic, demographic, or niche groups).
        \item Analyze cost of customer acquisition and methods to monitor satisfaction.
        \item \textbf{Example:} A dog-food startup may define its market as ``urban dog owners aged 20--40.'' 
    \end{itemize}
    
    \item \textbf{How (Assets and Suppliers):}
    \begin{itemize}
        \item Key assets: patents, proprietary software, strong brand, or skilled employees.
        \item Key suppliers: battery suppliers for Tesla, lithium miners for battery makers.
    \end{itemize}
    
    \item \textbf{What (Products/Services):}
    \begin{itemize}
        \item Define differentiation: low price, premium quality, or unique features.
        \item \textbf{Example:} Apple differentiates via design + ecosystem integration.
    \end{itemize}
    
    \item \textbf{Where (Channels):}
    \begin{itemize}
        \item Direct vs. intermediaries (wholesalers, franchisees).
        \item Omnichannel strategies = digital + physical integration.
        \item B2B vs. B2C business distinctions.
    \end{itemize}
    
    \item \textbf{How Much (Pricing Strategy):}
    Pricing depends on industry competition and differentiation.
\end{itemize}

\subsubsection*{Pricing Strategies and Models}

\begin{table}[H]
\centering
\resizebox{\textwidth}{!}{%
\small
\begin{tabular}{|p{3cm}|p{6cm}|p{5cm}|}
\hline
\textbf{Model} & \textbf{Explanation} & \textbf{Example} \\
\hline
\textbf{Price Takers} & Firms in commodity industries must accept market price. & Oil producers, home loans. \\
\hline
\textbf{Pricing Power} & Firms with unique products/oligopolies can charge premiums. & Patented drugs, luxury goods. \\
\hline
\textbf{Price Discrimination} & Different prices for different segments. & Airlines: off-peak vs. peak tickets. \\
\hline
\textbf{Bundling} & Multiple complementary products sold together. & Telecom: internet + TV package. \\
\hline
\textbf{Razors-and-Blades} & Low-margin equipment, high-margin consumables. & Printers and ink cartridges. \\
\hline
\textbf{Add-on Pricing} & Extra options with high margins. & Car upgrades (GPS, leather seats). \\
\hline
\textbf{Penetration Pricing} & Low price initially to gain scale. & Netflix’s early subscription model. \\
\hline
\textbf{Freemium} & Free base, paid premium features. & Spotify free vs. premium. \\
\hline
\textbf{Hidden Revenue} & Free product, revenue via ads/data. & Google Search, Facebook. \\
\hline
\textbf{Subscription} & Pay recurring fee for access. & Microsoft Office 365. \\
\hline
\textbf{Licensing / Franchising} & Rights sold to third parties. & McDonald’s franchises, biotech licensing. \\
\hline
\end{tabular}}
\caption{Common Pricing Models and Examples}
\end{table}

\subsubsection*{Value Proposition and Value Chain}

\begin{itemize}
    \item \textbf{Value Proposition:} Why customers buy the product (quality, price, innovation).
    \item \textbf{Value Chain:} How the firm executes value creation (Michael Porter’s 5 activities):
    \begin{enumerate}
        \item Inbound logistics
        \item Operations
        \item Outbound logistics
        \item Marketing
        \item Sales and service
    \end{enumerate}
    \item \textbf{Example:} Amazon’s value chain integrates warehousing (logistics), Prime subscription (marketing), and fast delivery (operations).
\end{itemize}

\subsubsection*{LOS 28.b: Types of Business Models}

\begin{itemize}
    \item \textbf{Conventional Models:} 
    Industry-specific (manufacturers, retailers, banks, brokers, software firms).
    
    \item \textbf{Private Label / Contract Manufacturing:} 
    Firms produce goods for other brands.  
    \textbf{Example:} Costco’s Kirkland brand.
    
    \item \textbf{Licensing Agreements:} 
    Brand used by others for a fee.  
    \textbf{Example:} Marvel characters on toys.
    
    \item \textbf{Value-Added Resellers:} 
    Add services/customization to existing products.  
    \textbf{Example:} IT resellers adding installation + support.
    
    \item \textbf{Innovative Models:} 
    \begin{itemize}
        \item \textbf{SaaS:} Subscription-based software (e.g., Salesforce).
        \item \textbf{Ultra-Low-Cost Airlines:} Ryanair, AirAsia. 
        \item \textbf{Discount Brokers:} Robinhood, Trade Republic. 
    \end{itemize}
    
    \item \textbf{Network Effects:} Value rises with more users.  
    \textbf{Examples:} WhatsApp, Facebook, Airbnb. 
    
    \item \textbf{Crowdsourcing Models:} Rely on user contributions.  
    \textbf{Examples:} Wikipedia, Waze, Open-source software.
\end{itemize}

\end{document}