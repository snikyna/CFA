\documentclass[12pt]{article}
\usepackage{amsmath}
\usepackage{geometry}
\usepackage{graphicx} % for including images and figures
\usepackage{booktabs}
\usepackage{caption}
\usepackage{titlesec}
\usepackage{float}
\usepackage{makecell}
\usepackage{tabularx}
\usepackage{enumitem}
\usepackage[utf8]{inputenc}
\usepackage{textcomp}
\usepackage{adjustbox}  % put in preamble
\usepackage{array}      % for >{\raggedright\arraybackslash}X

\geometry{margin=1in}

% Define an X column that is ragged-right and allows line breaks
\newcolumntype{Y}{>{\raggedright\arraybackslash}X}

\title{Portfolio Managament}
\author{}
\date{}

\begin{document}
\maketitle

\section*{MODULE 20.1: Historical Risk and Return}

\subsection*{Learning Objective}
\textbf{LOS 20.a:} Describe characteristics of the major asset classes that investors consider in forming portfolios.

\subsection*{1. Overview: The Risk–Return Tradeoff}

\textbf{Concept:}  
Investors face a fundamental tradeoff between \textbf{expected return} and \textbf{risk (volatility)} across asset classes.

\begin{itemize}
    \item Higher expected returns are associated with higher risk.
    \item Lower-risk assets provide more stability but lower long-term returns.
\end{itemize}

\textbf{Empirical Observation (U.S. Data, 1926–2017):}
\begin{itemize}
    \item Small-capitalization stocks had the \textbf{highest average returns} and \textbf{highest standard deviation (risk)}.
    \item Large-cap stocks had lower average returns and volatility than small-cap stocks.
    \item Long-term government bonds had moderate risk and return.
    \item Treasury bills (T-bills) had the \textbf{lowest risk and lowest return}.
\end{itemize}

\subsection*{2. Comparative Summary of Major Asset Classes}

\begin{center}
\begin{tabular}{|p{4cm}|p{4cm}|p{4cm}|p{4cm}|}
\hline
\textbf{Asset Class} & \textbf{Avg. Nominal Return} & \textbf{Risk (Std. Dev.)} & \textbf{Key Characteristics} \\
\hline
\textbf{Small-Cap Stocks} & Highest ($\approx$ 12--13\%) & Highest ($\approx$ 30\%) & High growth potential, highly volatile. \\
\hline
\textbf{Large-Cap Stocks} & Moderate ($\approx$ 10\%) & Moderate ($\approx$ 20\%) & Broad diversification, lower risk than small-caps. \\
\hline
\textbf{Long-Term Gov. Bonds} & Lower ($\approx$ 6\%) & Low–Moderate ($\approx$ 10\%) & Fixed income, sensitive to interest rate changes. \\
\hline
\textbf{T-Bills (Short-Term Gov. Securities)} & Lowest ($\approx$ 3\%) & Lowest ($\approx$ 1\%) & Near risk-free, used as a proxy for the risk-free rate. \\
\hline
\end{tabular}
\end{center}

\textbf{Verbal explanation:}  
Across decades of data, small firms offered greater reward potential but at much higher variability. Safe assets like T-bills barely beat inflation but provide security and liquidity.

\subsection*{3. Real vs. Nominal Returns}

\textbf{Definitions:}
\[
\text{Real Return} = \text{Nominal Return} - \text{Inflation Rate}
\]

\textbf{Example:}  
If the nominal return on large-cap stocks is \(10\%\) and average inflation is \(2.7\%\),  
\[
\text{Real Return} = 10\% - 2.7\% = 7.3\%
\]

\textbf{Historical Evidence (U.S. 1926–2017):}
\begin{itemize}
    \item \textbf{T-Bills:} Real return ≈ 0.5\%
    \item \textbf{Large-cap stocks:} Real return ≈ 7.3\%
    \item Real returns are more stable than nominal returns because inflation varies significantly year to year.
\end{itemize}

\textbf{Interpretation:}
\begin{itemize}
    \item Inflation erodes purchasing power.
    \item Long-term investors must focus on \textbf{real returns} to maintain wealth.
    \item Nominal volatility partly reflects inflation fluctuations, not just asset performance.
\end{itemize}

\subsection*{4. Risk Characteristics: Variability and Distribution of Returns}

\textbf{Standard deviation ($\sigma$)} measures the dispersion of returns — the most common risk metric.  
However, historical data show returns are \textbf{not perfectly normal (Gaussian)}.

\textbf{Two important deviations:}
\begin{itemize}
    \item \textbf{Skewness} ($\neq 0$): Measures asymmetry of return distribution.
    \begin{itemize}
        \item \textbf{Negative skew:} frequent small gains, few large losses (common in equities).
        \item \textbf{Positive skew:} frequent small losses, few large gains.
    \end{itemize}
    \item \textbf{Kurtosis} ($\neq 3$): Measures “fat tails” or frequency of extreme outcomes.
    \begin{itemize}
        \item \textbf{Excess kurtosis $>$ 0:} More extreme outcomes than a normal distribution — greater downside and upside volatility.
        \item Real-world return distributions often have \textbf{fat tails and negative skew}, indicating higher crash risk.
    \end{itemize}
\end{itemize}

\textbf{Verbal Explanation:}  
While models often assume normally distributed returns, real data show more frequent extreme losses.  
Thus, relying solely on mean and variance underestimates downside risk.

\subsection*{5. Liquidity as an Additional Investment Characteristic}

\textbf{Definition:}  
Liquidity refers to how easily and quickly an asset can be sold at its fair market value without significant price discount.

\textbf{Key points:}
\begin{itemize}
    \item \textbf{High liquidity:} Assets like large-cap stocks and T-bills can be traded quickly with minimal price impact.
    \item \textbf{Low liquidity:} Assets like emerging market securities, thinly traded corporate bonds, or private equity can be difficult to sell.
    \item Liquidity risk is reflected in \textbf{higher required returns} for illiquid assets.
\end{itemize}

\textbf{Example (verbal):}  
An investor holding a rare municipal bond might have to sell it at a discount in a hurry, even if its intrinsic value hasn’t changed — this is a liquidity premium.

\textbf{Effect on Expected Return:}
\[
E(R_{\text{illiquid}}) = E(R_{\text{liquid}}) + \text{Liquidity Premium}
\]

\subsection*{6. Key Relationships Summary Table}

\begin{center}
\begin{tabular}{|l|l|}
\hline
\textbf{Concept} & \textbf{Key Relationship or Interpretation} \\
\hline
Risk–Return Tradeoff & Higher risk $\Rightarrow$ higher expected return \\
\hline
Real Return & Nominal Return $-$ Inflation \\
\hline
Negative Skew & Frequent small gains, rare large losses \\
\hline
Excess Kurtosis & Fat tails — more extreme events than normal distribution \\
\hline
Liquidity & Ease of converting to cash; illiquid assets demand higher returns \\
\hline
\end{tabular}
\end{center}

\subsection*{7. Key Takeaways}

\begin{itemize}
    \item \textbf{Major asset classes differ in risk and return:}  
    Small-cap stocks are the riskiest and most rewarding; T-bills are the safest with the lowest return.
    \item \textbf{Real returns matter more than nominal:} Inflation-adjusted returns better measure long-term purchasing power.
    \item \textbf{Return distributions are not normal:}  
    They are typically negatively skewed and have excess kurtosis — more frequent extreme losses.
    \item \textbf{Liquidity influences returns:}  
    Illiquid assets require a premium for their trading constraints.
    \item Overall, investors must consider both quantitative measures (mean, variance) and qualitative factors (liquidity, distribution shape) when constructing portfolios.
\end{itemize}

\section*{MODULE 20.2: Risk Aversion and Optimal Portfolio Selection}

\subsection*{Learning Objectives}
\begin{itemize}
    \item \textbf{LOS 20.b:} Explain risk aversion and its implications for portfolio selection.
    \item \textbf{LOS 20.c:} Explain the selection of an optimal portfolio, given an investor's utility (risk aversion) and the capital allocation line.
\end{itemize}

\subsection*{1. Risk Aversion: Definition and Investor Preferences}

\textbf{Concept:}  
A \textbf{risk-averse investor} prefers less risk to more risk when expected returns are equal.  
Financial models assume all rational investors are risk averse.

\textbf{Investor Classifications:}
\begin{center}
\begin{tabular}{|p{4cm}|p{4cm}|p{7cm}|}
\hline
\textbf{Investor Type} & \textbf{Attitude Toward Risk} & \textbf{Behavioral Preference} \\
\hline
\textbf{Risk-Averse} & Dislikes risk; prefers certainty. & Chooses lower-risk investment for same return. \\
\hline
\textbf{Risk-Neutral} & Indifferent to risk. & Cares only about expected return; no risk preference. \\
\hline
\textbf{Risk-Seeking (Risk-Loving)} & Prefers more risk. & Chooses higher-risk investment even if expected return is equal. \\
\hline
\end{tabular}
\end{center}

\textbf{Example (Gamble Illustration):}
\[
\text{Gamble: } 50\% \text{ chance of } \$100,\; 50\% \text{ chance of } \$0 \Rightarrow E(\text{Payoff}) = \$50
\]
\begin{itemize}
    \item \textbf{Risk-Averse:} Prefers a certain \$50 to the gamble.
    \item \textbf{Risk-Neutral:} Indifferent between gamble and certain \$50.
    \item \textbf{Risk-Seeking:} Prefers gamble to certain \$50.
\end{itemize}

\textbf{Implication:}  
For the same expected return, risk-averse investors always choose the portfolio with the lower standard deviation (\(\sigma\)).

\subsection*{2. Investor Utility Function and Indifference Curves}

\textbf{Utility Function:}
\[
U = E(R) - \frac{1}{2}A\sigma^2
\]
where:
\begin{itemize}
    \item \(U\) = utility (satisfaction level)
    \item \(E(R)\) = expected return of portfolio
    \item \(\sigma^2\) = variance of portfolio return
    \item \(A\) = coefficient of risk aversion (\(A > 0\) for risk-averse investors)
\end{itemize}

\textbf{Interpretation:}
\begin{itemize}
    \item \(E(R)\): provides positive utility (benefit).
    \item \(\frac{1}{2}A\sigma^2\): risk penalty term; higher for more risk-averse investors.
\end{itemize}

\textbf{Implications:}
\begin{itemize}
    \item Higher \(A\) → investor dislikes risk strongly → steeper indifference curve.
    \item Lower \(A\) → investor tolerates risk → flatter indifference curve.
\end{itemize}

\subsubsection*{Indifference Curves (Qualitative Description)}
\begin{itemize}
    \item Each curve shows combinations of \(E(R)\) and \(\sigma\) that provide the same utility.
    \item Curves slope \textbf{upward} for risk-averse investors — more risk requires more return.
    \item \textbf{Higher curves} represent higher utility levels.
\end{itemize}

\textbf{Example (Verbal Illustration):}
\begin{itemize}
    \item Investor A (highly risk-averse): requires 8\% return for 10\% risk.
    \item Investor B (less risk-averse): satisfied with 6\% return for same 10\% risk.
    \item Therefore, A’s indifference curve is steeper.
\end{itemize}

\subsection*{3. Combining Risk-Free and Risky Assets}

\textbf{Portfolio Composition:}
A portfolio may consist of:
\[
\text{Risk-Free Asset (return } R_f, \sigma = 0) \quad \text{and} \quad \text{Risky Portfolio (return } E(R_R), \sigma_R)
\]

\textbf{Portfolio Expected Return:}
\[
E(R_P) = w_R E(R_R) + (1 - w_R)R_f
\]
\textbf{Portfolio Standard Deviation:}
\[
\sigma_P = w_R \sigma_R
\]
where:
\begin{itemize}
    \item \(w_R\) = proportion invested in the risky portfolio
    \item \((1 - w_R)\) = proportion invested in the risk-free asset
\end{itemize}

\textbf{Interpretation:}
\begin{itemize}
    \item Increasing \(w_R\) increases both expected return and risk linearly.
    \item Relationship between \(E(R_P)\) and \(\sigma_P\) is a straight line.
\end{itemize}

\textbf{Example:}
If 40\% invested in risky portfolio (\(E(R_R)=10\%, \sigma_R=15\%\)) and 60\% in risk-free asset (\(R_f=3\%\)):
\[
E(R_P) = 0.4(10\%) + 0.6(3\%) = 5.8\%, \quad \sigma_P = 0.4(15\%) = 6\%
\]

\subsection*{4. The Capital Allocation Line (CAL)}

\textbf{Equation of CAL:}
\[
E(R_P) = R_f + \left( \frac{E(R_R) - R_f}{\sigma_R} \right)\sigma_P
\]

\textbf{Interpretation:}
\begin{itemize}
    \item The CAL represents all possible combinations of the risk-free asset and the risky portfolio.
    \item Its \textbf{slope} is the \textbf{Sharpe Ratio}:
    \[
    \text{Sharpe Ratio} = \frac{E(R_R) - R_f}{\sigma_R}
    \]
    \item A higher slope implies a better risk–return tradeoff.
\end{itemize}

\textbf{Economic Meaning:}
\begin{itemize}
    \item If investors can borrow/lend at \(R_f\), any desired risk-return point along the CAL can be reached.
    \item Lending (investing in \(R_f\)): moves left on CAL → lower return, lower risk.
    \item Borrowing (leveraging): moves right on CAL → higher return, higher risk.
\end{itemize}

\textbf{Numerical Illustration:}
Suppose:
\[
R_f = 2\%, \quad E(R_R) = 8\%, \quad \sigma_R = 10\%
\]
\[
\text{Slope of CAL} = \frac{8\% - 2\%}{10\%} = 0.6
\]
For a portfolio with \(\sigma_P = 5\%\):
\[
E(R_P) = 2\% + 0.6(5\%) = 5\%
\]

\subsection*{5. Two-Fund Separation Theorem}

\textbf{Statement:}  
All investors, regardless of risk aversion, will:
\begin{enumerate}
    \item Hold the same optimal risky portfolio (the “market portfolio”).
    \item Combine it with the risk-free asset in different proportions.
\end{enumerate}

\textbf{Implications:}
\begin{itemize}
    \item Portfolio selection can be divided into two independent decisions:
    \begin{enumerate}
        \item Identify the optimal risky portfolio.
        \item Allocate between that portfolio and the risk-free asset based on personal risk tolerance.
    \end{enumerate}
\end{itemize}

\subsection*{6. Selecting the Optimal Portfolio (Utility Maximization)}

\textbf{Goal:}  
Find the combination of risky and risk-free assets that maximizes the investor’s expected utility.

\textbf{Optimization Condition:}
Investor chooses the portfolio where:
\[
\text{Slope of Indifference Curve} = \text{Slope of CAL}
\]
i.e., the point of tangency between investor’s preferences and market opportunities.

\textbf{Mathematical Derivation:}
\[
\max_{w_R} \; U = R_f + w_R(E(R_R) - R_f) - \frac{1}{2}A(w_R\sigma_R)^2
\]
First-order condition:
\[
\frac{\partial U}{\partial w_R} = 0 \Rightarrow w_R^* = \frac{E(R_R) - R_f}{A\sigma_R^2}
\]

\textbf{Interpretation:}
\begin{itemize}
    \item \(w_R^*\): optimal proportion of wealth in the risky portfolio.
    \item As risk aversion (\(A\)) increases → smaller \(w_R^*\).
    \item As risk premium (\(E(R_R) - R_f\)) increases → larger \(w_R^*\).
    \item As portfolio risk (\(\sigma_R^2\)) increases → smaller \(w_R^*\).
\end{itemize}

\textbf{Example:}
\[
E(R_R) = 8\%, \quad R_f = 3\%, \quad \sigma_R = 10\%, \quad A = 4
\]
\[
w_R^* = \frac{8 - 3}{4(10^2)} = \frac{5}{400} = 0.0125 \Rightarrow 1.25\%
\]
If \(A = 1\):
\[
w_R^* = \frac{5}{100} = 0.05 \Rightarrow 5\%
\]
\textbf{Interpretation:} The less risk-averse investor (A = 1) takes a higher allocation to the risky asset.

\subsection*{7. Portfolio Choices for Investors with Different Risk Aversions}

\begin{center}
\begin{tabular}{|p{3cm}|p{3cm}|p{4cm}|p{5cm}|}
\hline
\textbf{Investor Type} & \textbf{Risk Aversion (A)} & \textbf{Indifference Curve Shape} & \textbf{Portfolio Composition} \\
\hline
\textbf{Highly Risk-Averse} & Large & Steep & Mostly risk-free assets, little risky portfolio. \\
\hline
\textbf{Moderately Risk-Averse} & Medium & Moderate slope & Balanced allocation between risky and risk-free. \\
\hline
\textbf{Less Risk-Averse} & Small & Flat & Mostly risky assets, little risk-free portion. \\
\hline
\textbf{Risk-Neutral} & 0 & Horizontal & 100\% risky portfolio (maximizes return). \\
\hline
\end{tabular}
\end{center}

\subsection*{8. Conceptual Summary (Verbal Explanation)}

\begin{itemize}
    \item Risk aversion determines how much extra return investors require to bear extra risk.
    \item Utility curves represent investors’ satisfaction level given risk and return.
    \item Combining a risky portfolio with a risk-free asset yields a straight line (CAL) of possible portfolio outcomes.
    \item The optimal portfolio is chosen at the tangency between the CAL and the investor’s highest attainable indifference curve.
    \item All investors hold the same optimal risky portfolio (market portfolio), differing only in how much they mix it with the risk-free asset.
    \item More risk-averse investors hold more in the risk-free asset; less risk-averse investors hold more of the risky portfolio.
\end{itemize}

\subsection*{9. Key Takeaways}

\begin{itemize}
    \item \textbf{Risk Aversion:} Investors prefer less risk for the same return; quantified by the coefficient \(A\).
    \item \textbf{Utility Maximization:} Optimal portfolio maximizes \(U = E(R) - \frac{1}{2}A\sigma^2\).
    \item \textbf{Capital Allocation Line (CAL):} Linear relationship between risk and return combining risk-free and risky assets.
    \item \textbf{Two-Fund Separation:} All investors hold the same risky portfolio and mix it with the risk-free asset differently.
    \item \textbf{Diversification Benefit:} Risk-free asset allows investors to reach their desired risk level without altering market efficiency.
\end{itemize}

\section*{MODULE 20.3: Portfolio Standard Deviation}

\subsection*{Learning Objectives}
\begin{itemize}
    \item \textbf{LOS 20.d:} Calculate and interpret the mean, variance, and covariance (or correlation) of asset returns based on historical data.
    \item \textbf{LOS 20.e:} Calculate and interpret portfolio standard deviation.
\end{itemize}

\subsection*{1. Measures of Risk and Return for Individual Assets}

In finance, the \textbf{variance} and \textbf{standard deviation} of returns are fundamental measures of risk.  
They quantify how much an investment’s returns deviate from its expected (mean) return.

\subsubsection*{1.1 Mean (Expected) Return}
\[
\bar{R} = \frac{1}{T}\sum_{t=1}^{T}R_t
\]
where:
\begin{itemize}
    \item \(R_t\): return in period \(t\)
    \item \(T\): number of observed periods
\end{itemize}

\textbf{Interpretation:}  
The mean represents the central tendency (average performance) of the investment.

\subsubsection*{1.2 Population and Sample Variance}

\textbf{Population variance:}
\[
\sigma^2 = \frac{\sum_{t=1}^{T}(R_t - \mu)^2}{T}
\]
\textbf{Sample variance:}
\[
s^2 = \frac{\sum_{t=1}^{T}(R_t - \bar{R})^2}{T - 1}
\]

\textbf{Notes:}
\begin{itemize}
    \item In finance, we usually use \textbf{sample variance} because we work with historical samples.
    \item Standard deviation (\(\sigma\)) is the square root of variance — it is expressed in the same units as returns.
\end{itemize}

\textbf{Interpretation:}
\begin{itemize}
    \item Higher variance or standard deviation = greater dispersion around the mean = higher risk.
\end{itemize}

\subsection*{2. Covariance and Correlation Between Two Assets}

\textbf{Covariance} and \textbf{correlation} measure how two assets move together.  
They are essential for understanding diversification effects in portfolios.

\subsubsection*{2.1 Covariance Formula}

\[
\text{Cov}_{1,2} = \frac{\sum_{t=1}^{T}(R_{t,1} - \bar{R}_1)(R_{t,2} - \bar{R}_2)}{T - 1}
\]
where:
\begin{itemize}
    \item \(R_{t,1}, R_{t,2}\): returns on assets 1 and 2 in period \(t\)
    \item \(\bar{R}_1, \bar{R}_2\): average returns of assets 1 and 2
    \item \(T\): number of observations
\end{itemize}

\textbf{Interpretation:}
\begin{itemize}
    \item \(\text{Cov}_{1,2} > 0\): assets move together (positive relationship).
    \item \(\text{Cov}_{1,2} < 0\): assets move in opposite directions (negative relationship).
    \item \(\text{Cov}_{1,2} = 0\): no linear relationship.
\end{itemize}

\textbf{Units:} Covariance is measured in \emph{returns squared}, making it difficult to compare directly.

\subsubsection*{2.2 Correlation Coefficient}

To standardize covariance, we divide by the product of standard deviations:
\[
\rho_{1,2} = \frac{\text{Cov}_{1,2}}{\sigma_1 \sigma_2}
\]

\textbf{Properties of Correlation:}
\begin{itemize}
    \item \(-1 \leq \rho_{1,2} \leq +1\)
    \item \(\rho = +1\): perfectly positively correlated — assets move exactly together.
    \item \(\rho = -1\): perfectly negatively correlated — assets move exactly opposite.
    \item \(\rho = 0\): no linear relationship.
\end{itemize}

\textbf{Interpretation:}  
Correlation captures both the direction and strength of co-movement between assets.

\subsubsection*{Example: Calculating Mean, Variance, Covariance, and Correlation}

\textbf{Given:} Returns (\%) over 3 years:

\begin{center}
\begin{tabular}{|c|c|c|}
\hline
\textbf{Year} & \textbf{Asset A (\%)} & \textbf{Asset B (\%)} \\
\hline
1 & 10 & 12 \\
2 & 8  & 9  \\
3 & 6  & 7  \\
\hline
\end{tabular}
\end{center}

\textbf{Step 1: Mean returns}
\[
\bar{R}_A = (10 + 8 + 6)/3 = 8\%, \quad \bar{R}_B = (12 + 9 + 7)/3 = 9.33\%
\]

\textbf{Step 2: Variance of Asset A}
\[
s_A^2 = \frac{(10-8)^2 + (8-8)^2 + (6-8)^2}{3 - 1} = \frac{8}{2} = 4 \quad \Rightarrow \sigma_A = 2\%
\]
\textbf{Variance of Asset B:}
\[
s_B^2 = \frac{(12-9.33)^2 + (9-9.33)^2 + (7-9.33)^2}{2} = 6.66 \quad \Rightarrow \sigma_B = 2.58\%
\]

\textbf{Step 3: Covariance}
\[
\text{Cov}_{A,B} = \frac{(10-8)(12-9.33) + (8-8)(9-9.33) + (6-8)(7-9.33)}{2} = 5.16
\]

\textbf{Step 4: Correlation}
\[
\rho_{A,B} = \frac{5.16}{(2)(2.58)} = 1.00
\]
\textbf{Interpretation:} Assets A and B are perfectly positively correlated.

\subsection*{3. Portfolio Variance and Standard Deviation (LOS 20.e)}

A portfolio’s total risk depends on:
\begin{enumerate}
    \item The risk of each individual asset (\(\sigma_1, \sigma_2\)),
    \item The proportion invested in each asset (\(w_1, w_2\)),
    \item The correlation between asset returns (\(\rho_{1,2}\)).
\end{enumerate}

\subsubsection*{3.1 Formula for Portfolio Variance (Two Assets)}
\[
\sigma_P^2 = w_1^2\sigma_1^2 + w_2^2\sigma_2^2 + 2w_1w_2\sigma_1\sigma_2\rho_{1,2}
\]
and the \textbf{portfolio standard deviation}:
\[
\sigma_P = \sqrt{w_1^2\sigma_1^2 + w_2^2\sigma_2^2 + 2w_1w_2\sigma_1\sigma_2\rho_{1,2}}
\]

\textbf{Interpretation:}
\begin{itemize}
    \item The third term (\(2w_1w_2\sigma_1\sigma_2\rho_{1,2}\)) represents the \textbf{co-movement effect}.
    \item Portfolio risk decreases as correlation \(\rho_{1,2}\) decreases.
    \item Diversification benefit: combining less-than-perfectly correlated assets reduces overall portfolio risk.
\end{itemize}

\subsubsection*{3.2 Special Correlation Cases}

\begin{center}
\begin{tabular}{|c|p{10cm}|}
\hline
\textbf{Correlation Value} & \textbf{Interpretation} \\
\hline
\(\rho = +1\) & No diversification benefit; portfolio risk = weighted average of individual risks. \\
\hline
\(\rho = 0\) & Partial diversification; risk is lower than weighted average. \\
\hline
\(\rho = -1\) & Perfect diversification possible; risk can be eliminated with proper weights. \\
\hline
\end{tabular}
\end{center}

\subsubsection*{3.3 Example: Calculating Portfolio Standard Deviation}

\textbf{Given:}
\begin{itemize}
    \item \(w_{\text{stocks}} = 0.3\), \(\sigma_{\text{stocks}} = 20\%\)
    \item \(w_{\text{bonds}} = 0.7\), \(\sigma_{\text{bonds}} = 12\%\)
    \item Correlation \(\rho = 0.60\)
\end{itemize}

\textbf{Step 1: Portfolio Variance}
\[
\sigma_P^2 = (0.3)^2(0.20)^2 + (0.7)^2(0.12)^2 + 2(0.3)(0.7)(0.20)(0.12)(0.60)
\]
\[
\sigma_P^2 = 0.0036 + 0.007056 + 0.006048 = 0.016704
\]

\textbf{Step 2: Portfolio Standard Deviation}
\[
\sigma_P = \sqrt{0.016704} = 0.1293 = 12.93\%
\]

\textbf{Step 3: Perfectly Positively Correlated Case (\(\rho = +1\))}
\[
\sigma_P = (0.3)(20\%) + (0.7)(12\%) = 14.4\%
\]

\textbf{Interpretation:}
\begin{itemize}
    \item When correlation = +1, no diversification benefit → linear combination of risks.
    \item When correlation < +1, portfolio risk (12.93\%) is less than the weighted average (14.4\%).
    \item Diversification reduces total risk without reducing expected return.
\end{itemize}

\subsection*{4. Conceptual Interpretation of Diversification}

\textbf{Why diversification works:}
\begin{itemize}
    \item The imperfect correlation between asset returns means their ups and downs offset each other partially.
    \item As long as \(\rho_{1,2} < +1\), total portfolio risk is reduced below the weighted average of individual risks.
    \item The lower the correlation, the greater the benefit.
\end{itemize}

\textbf{Extreme Case:}
\begin{itemize}
    \item \(\rho = -1\): portfolio can theoretically achieve zero standard deviation if asset weights are chosen perfectly.
\end{itemize}

\textbf{General Insight:}  
Diversification cannot eliminate all risk — only the portion specific to individual assets (unsystematic risk).  
Systematic market-wide risk remains even in well-diversified portfolios.

\subsection*{5. Summary Table of Key Formulas and Relationships}

\begin{center}
\begin{tabular}{|l|l|}
\hline
\textbf{Concept} & \textbf{Formula / Interpretation} \\
\hline
Mean Return & $\bar{R} = \frac{1}{T}\sum R_t$ \\
\hline
Sample Variance & $s^2 = \frac{\sum (R_t - \bar{R})^2}{T - 1}$ \\
\hline
Covariance & $\text{Cov}_{1,2} = \frac{\sum (R_{t,1} - \bar{R}_1)(R_{t,2} - \bar{R}_2)}{T - 1}$ \\
\hline
Correlation & $\rho_{1,2} = \frac{\text{Cov}_{1,2}}{\sigma_1 \sigma_2}$ \\
\hline
Portfolio Variance & $\sigma_P^2 = w_1^2\sigma_1^2 + w_2^2\sigma_2^2 + 2w_1w_2\sigma_1\sigma_2\rho_{1,2}$ \\
\hline
Diversification Condition & Lower $\rho_{1,2}$ $\Rightarrow$ Lower $\sigma_P$ \\
\hline
\end{tabular}
\end{center}

\subsection*{6. Key Takeaways}

\begin{itemize}
    \item \textbf{Variance and standard deviation} measure the volatility of individual asset returns.
    \item \textbf{Covariance and correlation} measure how assets move together; correlation is a standardized, bounded measure.
    \item \textbf{Portfolio risk} depends not only on individual risks but also on how asset returns co-move.
    \item \textbf{Diversification} reduces portfolio risk when assets are not perfectly correlated (\(\rho < +1\)).
    \item The benefit of diversification increases as correlation approaches zero or becomes negative.
\end{itemize}

\section*{MODULE 20.4: The Efficient Frontier}

\subsection*{Learning Objectives}
\begin{itemize}
    \item \textbf{LOS 20.f:} Describe the effect on a portfolio's risk of investing in assets that are less than perfectly correlated.
    \item \textbf{LOS 20.g:} Describe and interpret the minimum-variance and efficient frontiers of risky assets and the global minimum-variance portfolio.
\end{itemize}

\subsection*{1. Effect of Correlation on Portfolio Risk (LOS 20.f)}

\textbf{Concept:}  
The correlation (\(\rho_{12}\)) between asset returns determines the degree of diversification benefit in a portfolio.  
Portfolio risk depends not only on individual asset risks but also on how returns move together.

\subsubsection*{1.1 Portfolio Variance Formula (Two-Asset Case)}

\[
\sigma_P^2 = w_1^2\sigma_1^2 + w_2^2\sigma_2^2 + 2w_1w_2\sigma_1\sigma_2\rho_{12}
\]
where:
\begin{itemize}
    \item \(w_1, w_2\): portfolio weights (\(w_1 + w_2 = 1\))
    \item \(\sigma_1, \sigma_2\): individual standard deviations
    \item \(\rho_{12}\): correlation coefficient between the two asset returns
\end{itemize}

\textbf{Interpretation:}
\begin{itemize}
    \item The third term \(2w_1w_2\sigma_1\sigma_2\rho_{12}\) represents the \textbf{interaction effect} — the impact of correlation.
    \item Portfolio variance is reduced whenever \(\rho_{12} < +1\).
    \item The lower (or more negative) the correlation, the greater the diversification benefit.
\end{itemize}

\subsubsection*{1.2 Perfect Positive Correlation (\(\rho_{12} = +1\))}

If two assets move perfectly together:
\[
\sigma_P = w_1\sigma_1 + w_2\sigma_2
\]
This means portfolio risk is simply the \textbf{weighted average} of individual risks — no diversification benefit.

\textbf{Example:}  
25\% invested in Asset 1 (\(\sigma_1 = 20\%\)) and 75\% in Asset 2 (\(\sigma_2 = 10\%\)):
\[
\sigma_P = 0.25(20\%) + 0.75(10\%) = 12.5\%
\]
There is \textbf{no reduction in risk} because returns move perfectly together.

\subsubsection*{1.3 Less Than Perfect Correlation (\(-1 < \rho_{12} < +1\))}

As correlation decreases:
\[
\sigma_P < w_1\sigma_1 + w_2\sigma_2
\]
→ Portfolio risk falls below the weighted average.

\textbf{Interpretation:}
\begin{itemize}
    \item Correlation of \(+1\): maximum risk (no diversification).
    \item Correlation of \(0\): moderate diversification; covariance term = 0.
    \item Correlation of \(-1\): maximum diversification; covariance term becomes negative.
\end{itemize}

\subsubsection*{1.4 Perfect Negative Correlation (\(\rho_{12} = -1\))}

In this special case, the two assets move in exactly opposite directions.  
There exists a combination of weights (\(w_1, w_2\)) such that portfolio standard deviation can be reduced to zero:
\[
\sigma_P = 0
\]
→ A risk-free portfolio can theoretically be constructed, even using risky assets.

\subsubsection*{1.5 Example: Portfolio Risk as Correlation Changes}

\textbf{Given:}
\[
\sigma_1 = 25\%, \quad \sigma_2 = 18\%, \quad w_1 = w_2 = 0.5
\]
\[
\sigma_P^2 = 0.5^2(0.25^2) + 0.5^2(0.18^2) + 2(0.5)(0.5)(0.25)(0.18)\rho_{12}
\]

\textbf{Compute for various correlations:}
\begin{center}
\begin{tabular}{|c|c|c|}
\hline
\textbf{Correlation} & \textbf{Portfolio Variance (\(\sigma_P^2\))} & \textbf{Portfolio Std. Dev. (\(\sigma_P\))} \\
\hline
\(+1.0\) & \(0.0441\) & \(21.0\%\) \\
\hline
\(+0.5\) & \(0.0304\) & \(17.4\%\) \\
\hline
\(0.0\) & \(0.0235\) & \(15.3\%\) \\
\hline
\(-0.5\) & \(0.0166\) & \(12.9\%\) \\
\hline
\end{tabular}
\end{center}

\textbf{Interpretation:}
\begin{itemize}
    \item As correlation decreases, portfolio risk (\(\sigma_P\)) falls.
    \item When \(\rho_{12} = -0.5\), risk reduction is substantial.
    \item The lower the correlation, the greater the \textbf{diversification benefit}.
\end{itemize}

\textbf{Key Insight:}
\[
\text{Diversification benefit } \propto (1 - \rho_{12})
\]
→ Holding assets with low or negative correlations provides the greatest reduction in overall risk.

\subsection*{2. Minimum-Variance and Efficient Frontiers (LOS 20.g)}

\subsubsection*{2.1 Concept Overview}

When combining two or more risky assets, each possible combination (set of weights) gives a distinct portfolio with:
\begin{itemize}
    \item An expected return: \(E(R_P) = w_1E(R_1) + w_2E(R_2)\)
    \item A risk level (standard deviation): \(\sigma_P = \sqrt{w_1^2\sigma_1^2 + w_2^2\sigma_2^2 + 2w_1w_2\sigma_1\sigma_2\rho_{12}}\)
\end{itemize}

\textbf{Plotting all possible portfolios:}
\begin{itemize}
    \item Creates a curve showing the tradeoff between expected return and risk.
    \item This curve is known as the \textbf{minimum-variance frontier}.
\end{itemize}

\subsubsection*{2.2 Minimum-Variance Frontier (MVF)}

\textbf{Definition:}  
The set of all portfolios that provide the \textbf{lowest risk (σ)} for each possible expected return level.

\textbf{Interpretation:}
\begin{itemize}
    \item Every point on this curve represents an optimal portfolio for a given return target.
    \item Portfolios below the MVF are inefficient — they have unnecessary risk for the same return.
\end{itemize}

\textbf{Mathematically:}  
For each target return \(E(R_T)\), minimize portfolio variance:
\[
\min_{w_1, w_2} \; \sigma_P^2 \quad \text{subject to } E(R_P) = E(R_T)
\]

\subsubsection*{2.3 Efficient Frontier (EF)}

\textbf{Definition:}  
The upper half of the minimum-variance frontier that offers:
\[
\text{Highest expected return for a given level of risk.}
\]
\textbf{Properties:}
\begin{itemize}
    \item Portfolios below the efficient frontier are suboptimal.
    \item Portfolios above the frontier are unattainable (impossible combinations).
    \item Rational, risk-averse investors will always choose portfolios on the efficient frontier.
\end{itemize}

\textbf{Interpretation in Words:}
\begin{itemize}
    \item The efficient frontier forms a curve that begins with the least-risk (global minimum-variance) portfolio and bends upward as expected return increases with risk.
    \item Each investor selects a point on the efficient frontier based on personal risk tolerance.
\end{itemize}

\subsubsection*{2.4 Global Minimum-Variance Portfolio (GMVP)}

\textbf{Definition:}  
The single portfolio on the minimum-variance frontier that has the \textbf{lowest possible standard deviation} among all combinations of risky assets.

\textbf{Characteristics:}
\begin{itemize}
    \item It lies at the bottom (leftmost point) of the minimum-variance frontier.
    \item It serves as the foundation of the efficient frontier — all efficient portfolios lie above it.
\end{itemize}

\textbf{Mathematically:}
\[
\frac{d\sigma_P^2}{dw_1} = 0 \quad \Rightarrow \quad w_1^{GMVP} = \frac{\sigma_2^2 - \sigma_{12}}{\sigma_1^2 + \sigma_2^2 - 2\sigma_{12}}
\]
where \(\sigma_{12} = \rho_{12}\sigma_1\sigma_2\).

\subsubsection*{2.5 Example: Efficient Frontier Construction (Verbal Explanation)}

\textbf{Step 1:} Start with two risky assets with given expected returns, standard deviations, and correlations.  
\textbf{Step 2:} Calculate expected return and risk for multiple weight combinations (e.g., 0\%, 25\%, 50\%, 75\%, 100\% in Asset 1).  
\textbf{Step 3:} Plot each combination — the lower boundary of all points forms the \textbf{minimum-variance frontier}.  
\textbf{Step 4:} Retain only the upper half (where higher returns accompany higher risks) — that is the \textbf{efficient frontier}.  
\textbf{Step 5:} Identify the lowest point on this curve — the \textbf{global minimum-variance portfolio (GMVP)}.

\textbf{Intuitive Meaning:}
\begin{itemize}
    \item Portfolios on the efficient frontier dominate all portfolios below it.
    \item Portfolios below the frontier are inefficient because they offer lower returns for the same risk.
\end{itemize}

\subsection*{3. Diversification and the Shape of the Efficient Frontier}

\textbf{Effect of Correlation:}
\begin{itemize}
    \item \(\rho_{12} = +1\): Frontier is a straight line — no diversification.
    \item \(0 < \rho_{12} < +1\): Frontier becomes concave — some diversification benefit.
    \item \(\rho_{12} < 0\): Frontier becomes more bowed — strong diversification benefit.
    \item \(\rho_{12} = -1\): Possible to achieve zero-risk (risk-free) portfolio.
\end{itemize}

\textbf{Interpretation in Words:}
\begin{itemize}
    \item The curve’s “bowing” shape visually reflects the degree of diversification.
    \item The more negatively correlated the assets, the more the curve bends leftward (lower risk for same return).
\end{itemize}

\subsection*{4. Conceptual Summary}

\begin{itemize}
    \item Portfolio risk is not a simple average of individual asset risks — it depends on how returns co-move.
    \item Correlation less than +1 allows risk reduction through diversification.
    \item The \textbf{minimum-variance frontier} represents all lowest-risk portfolios for given returns.
    \item The \textbf{efficient frontier} (upper part) shows portfolios offering the highest return for each risk level.
    \item The \textbf{global minimum-variance portfolio (GMVP)} is the least-risk combination of risky assets.
    \item Investors will only select portfolios on the efficient frontier since they dominate all others in risk–return space.
\end{itemize}

\subsection*{5. Key Takeaways Table}

\begin{center}
\begin{tabular}{|l|p{10cm}|}
\hline
\textbf{Concept} & \textbf{Key Insight / Formula} \\
\hline
Portfolio Variance & $\sigma_P^2 = w_1^2\sigma_1^2 + w_2^2\sigma_2^2 + 2w_1w_2\sigma_1\sigma_2\rho_{12}$ \\
\hline
Perfect Positive Correlation ($\rho=+1$) & No diversification; $\sigma_P = w_1\sigma_1 + w_2\sigma_2$ \\
\hline
Zero Correlation ($\rho=0$) & Partial diversification; risk reduction. \\
\hline
Perfect Negative Correlation ($\rho=-1$) & Complete diversification; possible zero-risk combination. \\
\hline
Minimum-Variance Frontier & Set of portfolios with lowest $\sigma$ for each $E(R_P)$. \\
\hline
Efficient Frontier & Upper portion of MVF; highest $E(R_P)$ for each $\sigma_P$. \\
\hline
Global Minimum-Variance Portfolio & Lowest-risk portfolio among all risky portfolios. \\
\hline
Diversification Benefit & Increases as $\rho_{12}$ decreases. \\
\hline
\end{tabular}
\end{center}

\section*{MODULE 21.1: Systematic Risk and Beta}

\subsection*{Learning Objectives}
\begin{itemize}
    \item \textbf{LOS 21.a:} Describe the implications of combining a risk-free asset with a portfolio of risky assets.
    \item \textbf{LOS 21.b:} Explain the capital allocation line (CAL) and the capital market line (CML).
    \item \textbf{LOS 21.c:} Explain systematic and nonsystematic risk, including why an investor should not expect additional return for bearing nonsystematic risk.
    \item \textbf{LOS 21.d:} Explain return-generating models (including the market model) and their uses.
    \item \textbf{LOS 21.e:} Calculate and interpret beta.
\end{itemize}

\subsection*{1. Combining a Risk-Free Asset with a Risky Portfolio (LOS 21.a)}

\textbf{Concept:}  
Adding a risk-free asset (with zero variance and zero correlation with risky assets) to a risky portfolio creates a \textbf{linear relationship} between portfolio risk and return.

\subsubsection*{1.1 Portfolio Equations}

\[
E(R_P) = w_P E(R_R) + (1 - w_P) R_f
\]
\[
\sigma_P = w_P \sigma_R
\]
where:
\begin{itemize}
    \item \(E(R_P)\): expected return on the combined portfolio
    \item \(E(R_R)\): expected return on the risky portfolio
    \item \(R_f\): risk-free rate
    \item \(\sigma_P\): standard deviation of portfolio
    \item \(w_P\): weight invested in the risky portfolio
\end{itemize}

\textbf{Implications:}
\begin{itemize}
    \item As \(w_P\) increases, both risk and expected return rise proportionally.
    \item The relationship between risk and return is a straight line through \((R_f, 0)\) and \((E(R_R), \sigma_R)\).
    \item The line is called the \textbf{Capital Allocation Line (CAL)}.
\end{itemize}

\subsection*{2. Capital Allocation Line (CAL) and Capital Market Line (CML) (LOS 21.b)}

\subsubsection*{2.1 The Capital Allocation Line (CAL)}

\textbf{Equation:}
\[
E(R_P) = R_f + \left( \frac{E(R_R) - R_f}{\sigma_R} \right) \sigma_P
\]

\textbf{Interpretation:}
\begin{itemize}
    \item \(R_f\): y-intercept (return at zero risk).
    \item Slope = \(\frac{E(R_R) - R_f}{\sigma_R}\) — the \textbf{Sharpe Ratio} of the risky portfolio.
    \item The higher the slope, the better the risk–return tradeoff.
\end{itemize}

\textbf{Borrowing and Lending:}
\begin{itemize}
    \item If investors can borrow and lend at \(R_f\), they can achieve portfolios beyond the risky portfolio (rightward extension of CAL).
    \item Lend → invest partly in risk-free asset (low risk, low return).
    \item Borrow → leverage risky portfolio (high risk, high return).
\end{itemize}

\subsubsection*{2.2 The Capital Market Line (CML)}

\textbf{Concept:}  
If all investors have the same expectations (homogeneous expectations) about risk, return, and correlations, they all face the same efficient frontier.  
They will all hold the same optimal risky portfolio — the \textbf{market portfolio}.

\textbf{CML Equation:}
\[
E(R_P) = R_f + \frac{E(R_M) - R_f}{\sigma_M} \sigma_P
\]
where:
\begin{itemize}
    \item \(E(R_M)\): expected return on the market portfolio
    \item \(\sigma_M\): standard deviation of the market portfolio
    \item \(E(R_P), \sigma_P\): expected return and standard deviation of any efficient portfolio
\end{itemize}

\textbf{Slope of the CML:}
\[
\text{Slope} = \frac{E(R_M) - R_f}{\sigma_M} = \text{market Sharpe ratio}
\]

\textbf{Interpretation:}
\begin{itemize}
    \item All efficient portfolios lie on the CML.
    \item CML represents the best possible risk–return combinations.
    \item Individual investors differ only in their allocation between the market portfolio and the risk-free asset (risk preference).
\end{itemize}

\textbf{Passive vs. Active Strategies:}
\begin{itemize}
    \item \textbf{Passive strategy:} Hold the market portfolio and risk-free asset in some proportion (index fund + Treasury bills).
    \item \textbf{Active strategy:} Deviate from market weights based on perceived mispricing (overweight undervalued assets, underweight overvalued ones).
\end{itemize}

\subsection*{3. Systematic and Unsystematic Risk (LOS 21.c)}

\subsubsection*{3.1 Risk Components of a Security}

Total risk (\(\sigma^2_{Total}\)) can be decomposed into:
\[
\sigma^2_{Total} = \sigma^2_{Systematic} + \sigma^2_{Unsystematic}
\]
where:
\begin{itemize}
    \item \textbf{Systematic Risk:} Market-related risk that cannot be diversified away (e.g., interest rates, inflation, GDP growth).
    \item \textbf{Unsystematic Risk:} Firm-specific or idiosyncratic risk (e.g., product recall, management change, strike).
\end{itemize}

\subsubsection*{3.2 Diversification Effect}

\textbf{Observation:}
As the number of securities in a portfolio increases, total risk decreases — but only up to a point.  
Beyond approximately 30–40 securities, further diversification provides little additional risk reduction.

\begin{center}
\begin{tabular}{|l|p{10cm}|}
\hline
\textbf{Risk Type} & \textbf{Description and Compensation} \\
\hline
Systematic (Market) Risk & Common to all securities; cannot be eliminated. Investors are compensated for bearing it. \\
\hline
Unsystematic (Firm-Specific) Risk & Can be eliminated by diversification; no additional expected return. \\
\hline
\end{tabular}
\end{center}

\textbf{Example (Conceptual):}
\begin{itemize}
    \item A biotech firm with one drug in trials has high total risk — mostly unsystematic.
    \item A machine tool manufacturer, exposed to business cycles, has lower total risk but higher systematic risk.
    \item Capital market theory: only systematic risk is priced → the manufacturer may have higher expected return despite lower total risk.
\end{itemize}

\subsubsection*{3.3 Key Insight}

\[
\text{Investors are compensated only for systematic risk.}
\]
Unsystematic (diversifiable) risk offers no premium because it can be costlessly diversified away in a broad portfolio.

\subsection*{4. Return Generating Models (LOS 21.d)}

\textbf{Purpose:}  
To estimate the expected return of a security based on its sensitivity to one or more risk factors.

\subsubsection*{4.1 General Multifactor Model}

\[
E(R_i) - R_f = \beta_{i1}E(F_1) + \beta_{i2}E(F_2) + \cdots + \beta_{ik}E(F_k)
\]
where:
\begin{itemize}
    \item \(E(R_i) - R_f\): expected excess return of Asset \(i\)
    \item \(F_1, F_2, \ldots, F_k\): systematic risk factors (e.g., GDP growth, inflation, firm size)
    \item \(\beta_{ij}\): sensitivity (factor loading) of Asset \(i\) to factor \(j\)
\end{itemize}

\textbf{Types of Factors:}
\begin{itemize}
    \item \textbf{Macroeconomic:} GDP growth, interest rates, inflation.
    \item \textbf{Fundamental:} Firm earnings, book-to-market ratio, firm size.
    \item \textbf{Statistical:} Factors extracted by data analysis (e.g., principal components).
\end{itemize}

\textbf{Fama–French–Carhart Model:}
\[
E(R_i) - R_f = \beta_M E(R_M - R_f) + \beta_{SMB} E(\text{Size}) + \beta_{HML} E(\text{Value}) + \beta_{MOM} E(\text{Momentum})
\]
\begin{itemize}
    \item SMB = Small Minus Big (size effect)
    \item HML = High Minus Low (value effect)
    \item MOM = Momentum (Carhart’s addition)
\end{itemize}

\subsubsection*{4.2 Single-Factor (Market) Model}

\[
E(R_i) - R_f = \beta_i [E(R_M) - R_f]
\]
\textbf{Interpretation:}
\begin{itemize}
    \item Market portfolio return is the single systematic factor.
    \item \(\beta_i\) measures sensitivity to market movements.
\end{itemize}

\subsubsection*{4.3 Market Model (Regression Form)}

\[
R_i = \alpha_i + \beta_i R_M + e_i
\]
where:
\begin{itemize}
    \item \(R_i\): actual return on security \(i\)
    \item \(R_M\): market return
    \item \(\alpha_i\): intercept (expected return independent of market)
    \item \(\beta_i\): slope coefficient (systematic risk measure)
    \item \(e_i\): residual (firm-specific or unsystematic return)
\end{itemize}

\textbf{Expected Return:} \(E(R_i) = \alpha_i + \beta_i E(R_M)\)

\textbf{Abnormal Return:} \(e_i = R_i - (\alpha_i + \beta_i R_M)\)

\textbf{Use:}
\begin{itemize}
    \item Estimate \(\beta_i\) empirically via regression of past returns.
    \item Identify security performance relative to the market (alpha).
\end{itemize}

\subsection*{5. Beta: Definition, Calculation, and Interpretation (LOS 21.e)}

\subsubsection*{5.1 Definition of Beta (\(\beta_i\))}

\[
\beta_i = \frac{\text{Cov}(R_i, R_M)}{\text{Var}(R_M)}
\]
\textbf{Alternative form using correlation:}
\[
\beta_i = \rho_{iM} \left( \frac{\sigma_i}{\sigma_M} \right)
\]
where:
\begin{itemize}
    \item \(\rho_{iM}\): correlation between Asset \(i\) and the market
    \item \(\sigma_i\): standard deviation of Asset \(i\)
    \item \(\sigma_M\): standard deviation of market returns
\end{itemize}

\subsubsection*{5.2 Interpretation of Beta}

\begin{center}
\begin{tabular}{|c|l|}
\hline
\textbf{Beta Value} & \textbf{Interpretation} \\
\hline
\(\beta = 1.0\) & Asset has same systematic risk as the market. \\
\hline
\(\beta > 1.0\) & Asset is more volatile; amplifies market movements. \\
\hline
\(\beta < 1.0\) & Asset is less volatile; dampens market movements. \\
\hline
\(\beta = 0\) & No relation to market movements (e.g., risk-free asset). \\
\hline
\(\beta < 0\) & Moves opposite to the market (e.g., gold, hedges). \\
\hline
\end{tabular}
\end{center}

\subsubsection*{5.3 Example: Calculating Beta}

\textbf{Example 1 (using correlation):}
\[
\sigma_M = 20\%, \; \sigma_A = 30\%, \; \rho_{AM} = 0.8
\]
\[
\beta_A = 0.8 \times \frac{30\%}{20\%} = 1.2
\]
→ Asset A has 20\% more systematic risk than the market.

\textbf{Example 2 (using covariance):}
\[
\text{Cov}(R_A, R_M) = 0.048, \; \text{Var}(R_M) = (0.20)^2 = 0.04
\]
\[
\beta_A = \frac{0.048}{0.04} = 1.20
\]
→ Consistent with previous calculation.

\subsubsection*{5.4 Estimating Beta (Empirical Regression)}

Regress excess returns on the asset against excess market returns:
\[
(R_i - R_f) = \alpha_i + \beta_i (R_M - R_f) + e_i
\]
\textbf{Interpretation:}
\begin{itemize}
    \item \(\beta_i\): slope of regression line (systematic risk measure).
    \item \(\alpha_i\): intercept (abnormal performance, or Jensen’s alpha).
    \item \(e_i\): residual (unsystematic, random variation).
\end{itemize}

\textbf{Security Characteristic Line (SCL):}  
This regression line graphically shows how a security’s returns vary with market returns.  
Its slope = \(\beta_i\); its scatter (residuals) = unsystematic risk.

\subsection*{6. Conceptual Summary}

\begin{itemize}
    \item Adding a risk-free asset to risky portfolios forms a straight line of possible combinations (CAL).
    \item With homogeneous expectations, all investors hold the same market portfolio — leading to the Capital Market Line (CML).
    \item Portfolio risk divides into:
        \begin{itemize}
            \item \textbf{Systematic risk:} cannot be diversified away; priced.
            \item \textbf{Unsystematic risk:} can be diversified away; not priced.
        \end{itemize}
    \item \textbf{Return generating models} estimate expected returns from exposure to systematic factors.
    \item \textbf{Market model} isolates the portion of return explained by market movements.
    \item \textbf{Beta} measures an asset’s sensitivity to market risk — only this component is relevant to required returns.
\end{itemize}

\subsection*{7. Key Formula Summary}

\begin{center}
\begin{tabular}{|l|l|}
\hline
\textbf{Concept} & \textbf{Formula} \\
\hline
Expected Portfolio Return & $E(R_P) = w_P E(R_R) + (1 - w_P) R_f$ \\
\hline
Portfolio Std. Dev. (Risk-Free + Risky) & $\sigma_P = w_P \sigma_R$ \\
\hline
Capital Market Line & $E(R_P) = R_f + \frac{E(R_M) - R_f}{\sigma_M}\sigma_P$ \\
\hline
Total Risk Decomposition & $\sigma^2_{Total} = \sigma^2_{Systematic} + \sigma^2_{Unsystematic}$ \\
\hline
Market Model & $R_i = \alpha_i + \beta_i R_M + e_i$ \\
\hline
Beta (Covariance Definition) & $\beta_i = \frac{\text{Cov}(R_i, R_M)}{\text{Var}(R_M)}$ \\
\hline
Beta (Correlation Form) & $\beta_i = \rho_{iM}\left(\frac{\sigma_i}{\sigma_M}\right)$ \\
\hline
Expected Return (Single-Factor) & $E(R_i) - R_f = \beta_i[E(R_M) - R_f]$ \\
\hline
\end{tabular}
\end{center}

\subsection*{8. Final Insights}

\begin{itemize}
    \item The \textbf{CML} describes efficient portfolios (market + risk-free asset), while the \textbf{Security Market Line (SML)} later describes individual assets.
    \item Beta is the essential link between individual asset risk and expected return.
    \item Diversification eliminates unsystematic risk, leaving only systematic risk to be rewarded.
    \item In equilibrium, expected returns depend only on exposure to market (systematic) risk — not on total volatility.
\end{itemize}

\section*{MODULE 21.2: The CAPM and the SML}

\subsection*{Learning Objectives}
\begin{itemize}
    \item \textbf{LOS 21.f:} Explain the Capital Asset Pricing Model (CAPM), its assumptions, and the Security Market Line (SML).
    \item \textbf{LOS 21.g:} Calculate and interpret the expected return of an asset using the CAPM.
    \item \textbf{LOS 21.h:} Describe and demonstrate applications of the CAPM and the SML (valuation and mispricing analysis).
    \item \textbf{LOS 21.i:} Calculate and interpret the Sharpe ratio, Treynor ratio, M2 measure, and Jensen’s alpha.
\end{itemize}

\subsection*{1. The Capital Asset Pricing Model (CAPM)}

\textbf{Core Idea:}  
Only \textbf{systematic risk} (market-related) is priced in equilibrium.  
The CAPM links expected return to systematic risk as measured by \(\beta_i\).

\[
E(R_i) = R_f + \beta_i [E(R_M) - R_f]
\]
where:
\begin{itemize}
    \item \(E(R_i)\): expected (or required) return on asset \(i\)
    \item \(R_f\): risk-free rate of return
    \item \(E(R_M)\): expected return on the market portfolio
    \item \(\beta_i\): sensitivity of asset \(i\) to market risk
    \item \([E(R_M) - R_f]\): market risk premium
\end{itemize}

\subsubsection*{1.1 CAPM Intuition}
\begin{itemize}
    \item Investors are only rewarded for bearing risk that cannot be diversified away.
    \item The greater an asset’s sensitivity (\(\beta_i\)) to market movements, the higher its required return.
    \item Assets with \(\beta_i = 0\) should earn \(R_f\).
    \item Assets with \(\beta_i = 1\) should earn the market return \(E(R_M)\).
    \item Assets with \(\beta_i > 1\) should have higher expected returns than the market.
\end{itemize}

\subsubsection*{1.2 Example: Expected Return via CAPM}
\[
R_f = 2\%, \quad E(R_M) = 8\%, \quad \beta_A = 1.2
\]
\[
E(R_A) = 2\% + 1.2(8\% - 2\%) = 9.2\%
\]
→ Stock A’s expected (required) return = 9.2\%, which exceeds the market return because \(\beta_A > 1\).

\subsection*{2. Assumptions of the CAPM}

\begin{enumerate}
    \item \textbf{Risk Aversion:} Investors require higher expected returns for higher risk.
    \item \textbf{Utility Maximization:} Investors select portfolios that maximize expected utility based on risk and return.
    \item \textbf{Frictionless Markets:} No transaction costs, taxes, or restrictions.
    \item \textbf{Single Period Horizon:} All investors plan for the same single investment period.
    \item \textbf{Homogeneous Expectations:} All investors have identical expectations about returns, risk, and correlations.
    \item \textbf{Divisible Assets:} All assets can be divided infinitely.
    \item \textbf{Competitive Markets:} Investors are price takers; no one can influence market prices.
\end{enumerate}

\textbf{Result:}  
Under these assumptions, all investors hold the same optimal risky portfolio — the \textbf{market portfolio}.  
This leads to the \textbf{Capital Market Line (CML)} for portfolios and the \textbf{Security Market Line (SML)} for individual assets.

\subsection*{3. The Security Market Line (SML)}

\textbf{Equation:}
\[
E(R_i) = R_f + \beta_i [E(R_M) - R_f]
\]

\subsubsection*{3.1 Interpretation}
\begin{itemize}
    \item The SML depicts the linear relationship between \textbf{systematic risk} (\(\beta\)) and \textbf{expected return}.
    \item The slope of the SML = \([E(R_M) - R_f]\), the market risk premium.
    \item All assets and portfolios (efficient or not) should lie on the SML in equilibrium.
\end{itemize}

\textbf{Key Points:}
\begin{itemize}
    \item Risk-free asset (\(\beta = 0\)) → return = \(R_f\).
    \item Market portfolio (\(\beta = 1\)) → return = \(E(R_M)\).
    \item High-\(\beta\) assets (\(\beta > 1\)) → higher expected return.
    \item Low-\(\beta\) assets (\(\beta < 1\)) → lower expected return.
\end{itemize}

\subsection*{4. Comparing the CML and the SML}

\begin{center}
\begin{tabular}{|l|p{5cm}|p{5cm}|}
\hline
\textbf{Feature} & \textbf{Capital Market Line (CML)} & \textbf{Security Market Line (SML)} \\
\hline
\textbf{Applicable To} & Efficient portfolios only & All individual securities and portfolios \\
\hline
\textbf{X-Axis Measure} & Total risk (\(\sigma_P\)) & Systematic risk (\(\beta\)) \\
\hline
\textbf{Y-Axis Measure} & Expected return (\(E(R_P)\)) & Expected return (\(E(R_i)\)) \\
\hline
\textbf{Slope} & \(\frac{E(R_M) - R_f}{\sigma_M}\) & \(E(R_M) - R_f\) \\
\hline
\textbf{Risk Interpretation} & Combines risk-free and market portfolios & Captures beta risk of any asset \\
\hline
\textbf{Plot Points} & Efficient portfolios only & All properly priced assets \\
\hline
\end{tabular}
\end{center}

\textbf{Summary Interpretation (Verbal):}
\begin{itemize}
    \item The CML shows the best achievable combinations of risk and return using the market portfolio and risk-free asset.
    \item The SML shows the equilibrium expected return for any asset based on its \(\beta\).
    \item Individual securities (even inefficient ones) lie on the SML if correctly priced.
\end{itemize}

\subsection*{5. CAPM Applications: Mispricing and Valuation (LOS 21.h)}

\textbf{Concept:}  
If a stock’s expected return differs from its CAPM-predicted (required) return, the stock is \textbf{mispriced}.

\[
\text{Required Return} = R_f + \beta_i [E(R_M) - R_f]
\]
\[
\text{Alpha} = E(R_i) - [R_f + \beta_i(E(R_M) - R_f)]
\]
\begin{itemize}
    \item \(\alpha > 0\): stock is undervalued (plots above SML)
    \item \(\alpha < 0\): stock is overvalued (plots below SML)
\end{itemize}

\subsubsection*{Example: Identifying Mispriced Securities}

\textbf{Given:}
\[
R_f = 7\%, \quad E(R_M) = 15\%
\]
\begin{center}
\begin{tabular}{|c|c|c|}
\hline
\textbf{Stock} & \textbf{Beta} & \textbf{Expected Return (Forecast)} \\
\hline
A & 1.0 & 12.0\% \\
\hline
B & 0.8 & 17.5\% \\
\hline
C & 1.2 & 16.6\% \\
\hline
\end{tabular}
\end{center}

\textbf{Step 1: Compute Required Return using CAPM}
\[
E(R_i) = 7\% + \beta_i(15\% - 7\%) = 7\% + 8\%\beta_i
\]
\begin{center}
\begin{tabular}{|c|c|c|}
\hline
\textbf{Stock} & \textbf{Required Return (CAPM)} & \textbf{Comparison} \\
\hline
A & 15.0\% & 12.0\% forecast → \textbf{Overvalued} \\
\hline
B & 13.4\% & 17.5\% forecast → \textbf{Undervalued} \\
\hline
C & 16.6\% & 16.6\% forecast → \textbf{Fairly Valued} \\
\hline
\end{tabular}
\end{center}

\textbf{Step 2: Trading Strategy}
\begin{itemize}
    \item \textbf{Stock A (Overvalued):} Expected return < required → Sell or short.
    \item \textbf{Stock B (Undervalued):} Expected return > required → Buy.
    \item \textbf{Stock C (Fairly Valued):} Hold or ignore.
\end{itemize}

\textbf{Graphical Explanation (Verbal):}
\begin{itemize}
    \item Stocks plotting \textbf{above} the SML have positive alpha (undervalued).
    \item Stocks plotting \textbf{below} the SML have negative alpha (overvalued).
    \item Stocks \textbf{on} the SML are fairly priced.
\end{itemize}

\subsection*{6. Risk-Adjusted Performance Measures (LOS 21.i)}

Portfolio performance must be adjusted for risk to fairly evaluate managers’ skill.

\subsubsection*{6.1 Sharpe Ratio (Total Risk-Based Measure)}

\[
\text{Sharpe Ratio} = \frac{R_P - R_f}{\sigma_P}
\]

\textbf{Interpretation:}
\begin{itemize}
    \item Measures excess return per unit of total risk.
    \item Higher Sharpe ratio → better risk-adjusted performance.
    \item Can be used for diversified or concentrated portfolios.
\end{itemize}

\textbf{Comparison:} The Sharpe ratio equals the \textbf{slope of the Capital Allocation Line (CAL)} for the portfolio.

\subsubsection*{6.2 M-Squared (M\textsuperscript{2}) Measure}

\[
M^2 = R_f + \left(\frac{R_P - R_f}{\sigma_P}\right)\sigma_M
\]
where:
\begin{itemize}
    \item \(R_P, \sigma_P\): portfolio return and standard deviation
    \item \(R_f\): risk-free rate
    \item \(\sigma_M\): market portfolio risk
\end{itemize}

\textbf{Interpretation:}
\begin{itemize}
    \item Converts the Sharpe ratio into a percentage measure.
    \item \(M^2\) represents the return the portfolio would have achieved if it had the same total risk as the market.
    \item \(M^2_\alpha = M^2 - R_M\) indicates outperformance relative to the market.
\end{itemize}

\textbf{Example:}
\[
R_P = 10\%, \quad \sigma_P = 20\%, \quad R_f = 5\%, \quad R_M = 11\%, \quad \sigma_M = 30\%
\]
\[
\text{Sharpe Ratio} = \frac{10 - 5}{20} = 0.25
\]
\[
M^2 = 0.05 + 0.25(0.30) = 0.125 = 12.5\%
\]
→ \(M^2 - R_M = 1.5\%\): Portfolio outperforms the market on a risk-adjusted basis.

\subsubsection*{6.3 Treynor Ratio (Systematic Risk-Based Measure)}

\[
\text{Treynor Ratio} = \frac{R_P - R_f}{\beta_P}
\]

\textbf{Interpretation:}
\begin{itemize}
    \item Measures excess return per unit of systematic risk.
    \item Appropriate for well-diversified portfolios (where unsystematic risk ≈ 0).
    \item Slope of the SML for a portfolio.
\end{itemize}

\subsubsection*{6.4 Jensen’s Alpha (\(\alpha_J\))}

\[
\alpha_J = R_P - [R_f + \beta_P(E(R_M) - R_f)]
\]

\textbf{Interpretation:}
\begin{itemize}
    \item Measures actual return above that predicted by CAPM.
    \item \(\alpha_J > 0\): portfolio outperformed relative to its risk.
    \item \(\alpha_J < 0\): portfolio underperformed.
\end{itemize}

\subsection*{7. Choosing the Right Performance Measure}

\begin{center}
\begin{tabular}{|l|l|l|}
\hline
\textbf{Measure} & \textbf{Uses} & \textbf{Risk Basis} \\
\hline
Sharpe Ratio & Total performance across all risk sources & Total risk (\(\sigma_P\)) \\
\hline
M\textsuperscript{2} & Percentage version of Sharpe ratio & Total risk (\(\sigma_P\)) \\
\hline
Treynor Ratio & Diversified portfolio evaluation & Systematic risk (\(\beta_P\)) \\
\hline
Jensen's Alpha & CAPM-based active return measure & Systematic risk (\(\beta_P\)) \\
\hline
\end{tabular}
\end{center}

\textbf{Interpretation Guidelines:}
\begin{itemize}
    \item Use Sharpe or M\textsuperscript{2} if the portfolio is not well diversified.
    \item Use Treynor or Jensen’s alpha if the portfolio is well diversified.
\end{itemize}

\subsection*{8. Conceptual Summary}

\begin{itemize}
    \item \textbf{CAPM:} Links expected return to systematic (beta) risk only.
    \item \textbf{SML:} Linear relationship between \(\beta\) and expected return — all properly priced assets lie on it.
    \item \textbf{Mispricing:} Assets above the SML are undervalued; below it are overvalued.
    \item \textbf{CML vs. SML:} CML applies to efficient portfolios; SML applies to all securities.
    \item \textbf{Performance Measures:}
        \begin{itemize}
            \item Sharpe / M\textsuperscript{2} → total risk-based (useful for single managers).
            \item Treynor / Jensen’s Alpha → systematic risk-based (useful for multi-manager diversified funds).
        \end{itemize}
\end{itemize}

\section*{MODULE 85.1: Portfolio Management Process}

\subsection*{Learning Objectives}
\begin{itemize}
    \item \textbf{LOS 85.a:} Describe the portfolio approach to investing.
    \item \textbf{LOS 85.b:} Describe the steps in the portfolio management process.
    \item \textbf{LOS 85.c:} Describe types of investors and their distinctive characteristics and needs.
    \item \textbf{LOS 85.d:} Describe defined contribution and defined benefit pension plans.
\end{itemize}

\subsection*{1. The Portfolio Approach to Investing (LOS 85.a)}

\subsubsection*{1.1 Concept of Portfolio Perspective}

\textbf{Definition:}  
The \textbf{portfolio perspective} evaluates each investment based on its \textit{contribution} to the portfolio’s total risk and return, not in isolation.

\begin{itemize}
    \item A rational investor focuses on how a new investment affects the overall portfolio’s expected return and risk (standard deviation).
    \item Holding a single asset (e.g., one stock) is inefficient — it exposes the investor to unnecessary unsystematic (idiosyncratic) risk.
    \item Diversification reduces total portfolio risk without necessarily lowering expected return.
\end{itemize}

\textbf{Contrast:}  
Evaluating assets in isolation ignores diversification benefits — a key mistake avoided by the portfolio approach.

\subsubsection*{1.2 Modern Portfolio Theory (MPT)}

\begin{itemize}
    \item Originated from \textbf{Harry Markowitz (1950s)} — introduced mathematical modeling of diversification.
    \item Risk is measured by the \textbf{standard deviation of returns}.
    \item Portfolio risk depends on:
    \[
    \sigma_P = \sqrt{w_1^2\sigma_1^2 + w_2^2\sigma_2^2 + 2w_1w_2\sigma_1\sigma_2\rho_{12}}
    \]
    \item Unless asset returns are perfectly positively correlated (\(\rho = +1\)), combining them reduces total risk.
    \item \textbf{Treynor, Sharpe, Lintner, and Mossin (1960s)} extended Markowitz’s model into equilibrium pricing theory — leading to the \textbf{Capital Asset Pricing Model (CAPM)}.
\end{itemize}

\subsubsection*{1.3 Diversification Ratio}

\[
\text{Diversification Ratio} = \frac{\text{Risk of equally weighted portfolio}}{\text{Average risk of individual securities}}
\]

\textbf{Example:}
\begin{itemize}
    \item Average individual stock risk = 25\%.
    \item Equally weighted portfolio risk = 18\%.
    \item Diversification Ratio \(= 18 / 25 = 0.72\).
\end{itemize}

\textbf{Interpretation:}
\begin{itemize}
    \item If ratio = 1 → no diversification benefit.
    \item If ratio < 1 → portfolio risk reduced through diversification.
    \item Lower ratio → higher diversification benefit.
\end{itemize}

\textbf{Limitation:}
\begin{itemize}
    \item An equal-weighted portfolio does not necessarily yield the minimum possible portfolio variance.
    \item Computerized \textbf{optimization models} determine optimal weights that minimize portfolio risk for a given expected return.
\end{itemize}

\textbf{Important Caveat:}
\begin{itemize}
    \item During financial crises, asset correlations tend to increase (\(\rho \uparrow\)), reducing diversification benefits.
    \item Example: During 2008 global credit contagion, correlations between equities, bonds, and real estate spiked, limiting risk reduction.
\end{itemize}

\subsection*{2. The Portfolio Management Process (LOS 85.b)}

\textbf{Definition:}  
A structured process designed to align investment decisions with investor objectives, constraints, and market conditions.

\subsubsection*{The Three Major Steps}

\begin{enumerate}
    \item \textbf{Planning Step: Establishing the Investor Profile}
        \begin{itemize}
            \item Analyze the investor’s:
            \begin{itemize}
                \item Risk tolerance (ability and willingness to bear risk)
                \item Return objectives (required or desired returns)
                \item Time horizon (short, medium, or long term)
                \item Liquidity needs (cash flow or spending requirements)
                \item Tax considerations (jurisdictional effects)
                \item Legal or regulatory constraints
                \item Unique circumstances or ethical preferences
            \end{itemize}
            \item Result: Draft an \textbf{Investment Policy Statement (IPS)} — a document outlining investment goals, constraints, and benchmarks.
            \item IPS must specify:
            \begin{itemize}
                \item \textbf{Objectives:} Risk tolerance and return requirement.
                \item \textbf{Constraints:} Time horizon, taxes, liquidity, laws, unique needs.
                \item \textbf{Benchmark:} Objective reference for performance evaluation (e.g., S\&P 500).
            \end{itemize}
            \item Update IPS periodically (at least every few years) or when investor’s situation changes significantly.
        \end{itemize}

    \item \textbf{Execution Step: Asset Allocation and Security Selection}
        \begin{itemize}
            \item Analyze asset classes for risk–return tradeoffs.
            \item Allocate funds among major asset types (e.g., cash, fixed income, equities, real estate, private equity, hedge funds, commodities).
            \item Use \textbf{top-down analysis:}
            \begin{itemize}
                \item Begin with macroeconomic outlook (GDP, inflation, interest rates).
                \item Select the most attractive asset classes and regions.
            \end{itemize}
            \item Then apply \textbf{bottom-up analysis:}
            \begin{itemize}
                \item Identify undervalued securities within chosen asset classes using valuation models.
            \end{itemize}
        \end{itemize}

    \item \textbf{Feedback Step: Monitoring and Rebalancing}
        \begin{itemize}
            \item Continuously monitor portfolio and market changes.
            \item Adjust asset allocations to restore target weights as market values fluctuate.
            \item Rebalance periodically (e.g., quarterly, annually, or threshold-based).
            \item Evaluate performance relative to the benchmark specified in the IPS.
            \item Performance attribution may identify sources of under/overperformance (e.g., asset allocation, security selection, timing).
        \end{itemize}
\end{enumerate}

\textbf{Summary Table: Steps in the Portfolio Management Process}

\begin{center}
\begin{tabular}{|l|p{7cm}|p{7cm}|}
\hline
\textbf{Step} & \textbf{Key Activities} & \textbf{Outputs} \\
\hline
Planning & Assess investor objectives and constraints & Investment Policy Statement (IPS) \\
\hline
Execution & Asset allocation and security selection & Portfolio construction \\
\hline
Feedback & Monitor, rebalance, and evaluate performance & Rebalanced portfolio; performance report \\
\hline
\end{tabular}
\end{center}

\subsection*{3. Types of Investors and Their Characteristics (LOS 85.c)}

\textbf{Investor categories differ in objectives, risk tolerance, liquidity, and horizon.}

\subsubsection*{3.1 Individual Investors}
\begin{itemize}
    \item Objectives: Wealth accumulation, retirement savings, education funding, home purchase.
    \item Constraints: Moderate risk tolerance, liquidity for personal goals, taxable income.
    \item May use tax-advantaged retirement accounts (e.g., 401(k), IRA).
    \item Investment horizon typically long term.
\end{itemize}

\subsubsection*{3.2 Institutional Investors}

\begin{center}
\begin{tabular}{|p{3cm}|p{4cm}|p{3cm}|p{2cm}|p{3cm}|}
\hline
\textbf{Investor Type} & \textbf{Objective} & \textbf{Risk Tolerance} & \textbf{Horizon} & \textbf{Liquidity Needs} \\
\hline
\textbf{Endowment} & Support specific program (e.g., university) & High & Long & Low (except spending needs) \\
\hline
\textbf{Foundation} & Fund research or charitable activities & High & Long & Low \\
\hline
\textbf{Bank} & Earn more on loans/investments than deposits & Low & Short to Medium & High (deposit withdrawals) \\
\hline
\textbf{Insurance – Life} & Fund long-term life insurance payouts & Moderate to High & Long & Low \\
\hline
\textbf{Insurance – P\&C} & Fund short-term claim obligations & Low to Moderate & Short & High \\
\hline
\textbf{Investment Company (Mutual Fund)} & Manage pooled investor funds within strategy & Depends on fund type & Variable & High (daily redemptions) \\
\hline
\textbf{Sovereign Wealth Fund (SWF)} & Manage national surplus or stabilization fund & Variable (depends on mandate) & Long & Low to Moderate \\
\hline
\end{tabular}
\end{center}

\textbf{Examples:}
\begin{itemize}
    \item \textbf{University Endowment:} Harvard or Yale endowment — aims for real growth while funding annual operations.
    \item \textbf{Foundation:} Gates Foundation — invests to support ongoing grants and research.
    \item \textbf{Life Insurer:} Prudential — invests in bonds to match long-term liabilities.
    \item \textbf{Property \& Casualty Insurer:} Allianz — invests conservatively due to frequent claim payouts.
    \item \textbf{Sovereign Wealth Fund:} Abu Dhabi Investment Authority (aprox \$700B) — invests globally for intergenerational equity.
\end{itemize}

\subsection*{4. Pension Plan Types (LOS 85.d)}

\subsubsection*{4.1 Defined Contribution (DC) Pension Plan}

\textbf{Definition:}  
Employer contributes a fixed amount each period into an employee’s retirement account.  
Employee decides how to invest and bears all investment risk.

\begin{itemize}
    \item Employer’s obligation ends with contributions.
    \item Future benefit depends on investment performance.
    \item Investment decisions made by employee.
    \item Common example: 401(k) plan.
\end{itemize}

\textbf{Formula:}
\[
\text{Future Benefit} = \text{Contributions} \times (1 + r)^n
\]
where \(r\) = portfolio return, \(n\) = number of compounding periods.

\textbf{Implications:}
\begin{itemize}
    \item Investment risk → employee.
    \item Flexibility and portability across employers.
    \item Employer’s cost predictable and limited.
\end{itemize}

\subsubsection*{4.2 Defined Benefit (DB) Pension Plan}

\textbf{Definition:}  
Employer promises to pay a specified benefit amount to retirees, usually based on salary and years of service.

\textbf{Formula Example:}
\[
\text{Annual Pension Payment} = (\text{Final Salary}) \times (\text{Benefit \% per Year}) \times (\text{Years of Service})
\]
\textbf{Example:}
\[
\$100,000 \times 2\% \times 20 = \$40,000 \text{ per year after retirement.}
\]

\textbf{Implications:}
\begin{itemize}
    \item Investment risk → employer (must ensure assets cover future liabilities).
    \item Contributions vary depending on fund performance.
    \item Employer manages assets through a pension fund.
\end{itemize}

\textbf{Comparison of DC vs. DB Plans}

\begin{center}
\begin{tabular}{|p{3cm}|p{6cm}|p{6cm}|}
\hline
\textbf{Feature} & \textbf{Defined Contribution (DC)} & \textbf{Defined Benefit (DB)} \\
\hline
Promised Benefit & None (depends on investment returns) & Predetermined formula (salary, service years) \\
\hline
Investment Decision & Employee & Employer / Plan Manager \\
\hline
Investment Risk & Employee bears risk & Employer bears risk \\
\hline
Cost Predictability & Predictable for employer & Variable (depends on performance) \\
\hline
Portability & High (account can transfer) & Low (usually employer-tied) \\
\hline
Example & 401(k) Plan & Traditional Company Pension \\
\hline
\end{tabular}
\end{center}

\textbf{Summary:}
\begin{itemize}
    \item DC plans shift risk from employer to employee.
    \item DB plans require careful actuarial and investment management by the employer.
\end{itemize}

\subsection*{5. Conceptual Summary}

\begin{itemize}
    \item The \textbf{portfolio approach} emphasizes total portfolio risk and return, not individual asset risk.
    \item \textbf{Diversification} reduces risk — effectiveness depends on correlations between asset returns.
    \item The \textbf{portfolio management process} involves:
    \begin{enumerate}
        \item Planning (create IPS)
        \item Execution (allocate and select)
        \item Feedback (monitor and rebalance)
    \end{enumerate}
    \item \textbf{Investor types} differ in their objectives, horizons, and liquidity needs.
    \item \textbf{Pension plan distinction:}
        \begin{itemize}
            \item DC → employee risk, flexible.
            \item DB → employer risk, guaranteed benefits.
        \end{itemize}
\end{itemize}

\section*{MODULE 85.2: Asset Management and Pooled Investments}

\subsection*{Learning Objectives}
\begin{itemize}
    \item \textbf{LOS 85.e:} Describe aspects of the asset management industry.
    \item \textbf{LOS 85.f:} Describe mutual funds and compare them with other pooled investment products.
\end{itemize}

\subsection*{1. Overview of the Asset Management Industry (LOS 85.e)}

\subsubsection*{1.1 Definition and Structure}

\textbf{Asset Management Industry:}  
Comprises firms that manage investments professionally on behalf of clients — both individuals and institutions. These firms are collectively known as the \textbf{buy-side} of the financial market, in contrast to the \textbf{sell-side} (broker-dealers, investment banks).

\textbf{Organizational Forms:}
\begin{itemize}
    \item \textbf{Independent firms:} Stand-alone investment managers (e.g., BlackRock, Fidelity).
    \item \textbf{Divisions of financial conglomerates:} Asset management arms of banks or insurers (e.g., JPMorgan Asset Management, Allianz Global Investors).
    \item \textbf{Multi-boutique firms:} Holding companies with several specialist managers under one structure, each focusing on specific styles or asset classes.
\end{itemize}

\subsubsection*{1.2 Management Styles}

\begin{center}
\begin{tabular}{|p{3cm}|p{6cm}|p{6cm}|}
\hline
\textbf{Style} & \textbf{Objective} & \textbf{Features} \\
\hline
\textbf{Active Management} & Outperform benchmark via manager skill & Uses fundamental, technical, or quantitative analysis. Higher fees. \\
\hline
\textbf{Passive Management} & Replicate benchmark performance & Index tracking or smart-beta strategies. Lower fees. \\
\hline
\end{tabular}
\end{center}

\textbf{Notes:}
\begin{itemize}
    \item Passive management ≈ 20\% of total assets under management (AUM), but a smaller share of total industry revenue due to lower fees.
    \item \textbf{Smart Beta:} A hybrid between active and passive — tracks an index but emphasizes specific factors (e.g., value, size, momentum).
\end{itemize}

\subsubsection*{1.3 Traditional vs. Alternative Asset Managers}

\begin{center}
\begin{tabular}{|p{4cm}|p{5cm}|p{5cm}|}
\hline
\textbf{Type} & \textbf{Asset Focus} & \textbf{Typical Characteristics} \\
\hline
\textbf{Traditional Managers} & Equities, fixed income, balanced funds & Lower fees, higher liquidity, widely accessible. \\
\hline
\textbf{Alternative Managers} & Private equity, hedge funds, real estate, commodities & Higher fees, lower liquidity, higher return potential. \\
\hline
\end{tabular}
\end{center}

\textbf{Trend:}  
Boundaries are blurring — many traditional firms are expanding into alternatives to capture higher margins.

\subsubsection*{1.4 Industry Trends}

\begin{enumerate}
    \item \textbf{Rise of Passive Investing:}
        \begin{itemize}
            \item Driven by low fees and skepticism about active managers’ ability to consistently add alpha in efficient markets.
            \item Passive index funds and ETFs gaining market share globally.
        \end{itemize}

    \item \textbf{Data and Technology Expansion:}
        \begin{itemize}
            \item Explosion in big data and analytics tools.
            \item Increased use of AI, machine learning, and third-party data vendors.
            \item Quicker decision-making and algorithmic strategies.
        \end{itemize}

    \item \textbf{Emergence of Robo-Advisors:}
        \begin{itemize}
            \item Automated portfolio allocation using algorithms.
            \item Targeted at younger or lower-asset investors.
            \item Lower cost, standardized asset allocation advice.
            \item Examples: Betterment, Wealthfront.
        \end{itemize}
\end{enumerate}

\textbf{Summary Insight:}
\begin{itemize}
    \item Passive strategies are growing rapidly due to low costs and transparency.
    \item Technology and data are transforming how asset managers analyze markets and interact with clients.
    \item Industry convergence is reducing the distinction between traditional and alternative managers.
\end{itemize}

\subsection*{2. Mutual Funds and Pooled Investment Products (LOS 85.f)}

\subsubsection*{2.1 Concept of Pooled Investments}

\textbf{Definition:}  
A \textbf{pooled investment vehicle} combines capital from multiple investors into one fund, managed by professionals, where each investor owns shares representing proportional ownership of the total pool.

\[
\text{Net Asset Value (NAV)} = \frac{\text{Total Value of Fund Assets} - \text{Liabilities}}{\text{Number of Shares Outstanding}}
\]

\textbf{NAV Interpretation:}  
Represents the per-share market value of the fund; determines buy/sell price for investors.

\subsubsection*{2.2 Open-End vs. Closed-End Mutual Funds}

\begin{center}
\begin{tabular}{|l|p{6cm}|p{6cm}|}
\hline
\textbf{Feature} & \textbf{Open-End Fund} & \textbf{Closed-End Fund} \\
\hline
\textbf{Share Issuance} & New shares created/redeemed at NAV & Fixed number of shares issued; traded on exchanges \\
\hline
\textbf{Pricing} & Always transacts at NAV (end of day) & Market price may differ from NAV (premium/discount) \\
\hline
\textbf{Liquidity Source} & Fund company (buys/sells shares) & Secondary market trading \\
\hline
\textbf{Management Fees} & Ongoing (usually annual expense ratio) & Similar, but trading involves brokerage costs \\
\hline
\textbf{Load Fees} & May have front-end or back-end loads & Typically none; trading fees apply \\
\hline
\end{tabular}
\end{center}

\textbf{Terminology:}
\begin{itemize}
    \item \textbf{No-load fund:} No entry or exit fees; only annual management fee.
    \item \textbf{Load fund:} May charge:
        \begin{itemize}
            \item \textbf{Front-end load:} Fee when buying shares.
            \item \textbf{Back-end load (redemption fee):} Fee when selling shares.
        \end{itemize}
\end{itemize}

\subsubsection*{2.3 Types of Mutual Funds}

\begin{enumerate}
    \item \textbf{Money Market Funds:}
        \begin{itemize}
            \item Invest in short-term, high-quality debt (T-bills, commercial paper).
            \item Aim for stability (NAV ≈ 1.00).
            \item Provide interest income with minimal price volatility.
        \end{itemize}

    \item \textbf{Bond Funds:}
        \begin{itemize}
            \item Invest in fixed-income securities.
            \item Differentiated by maturity, issuer type, credit quality, and region:
            \item Examples: government bond funds, municipal (tax-exempt) funds, high-yield funds, global bond funds.
        \end{itemize}

    \item \textbf{Stock Funds:}
        \begin{itemize}
            \item Invest primarily in equities.
            \item \textbf{Index Funds:} Passively managed to replicate benchmark performance (e.g., S\&P 500 Index Fund).
            \item \textbf{Actively Managed Funds:} Select securities to outperform benchmarks; higher fees and turnover.
            \item Active funds generally incur more taxes due to higher realized capital gains.
        \end{itemize}

    \item \textbf{Balanced (Hybrid) Funds:}
        \begin{itemize}
            \item Combine stocks, bonds, and sometimes cash equivalents.
            \item Aim for both income and capital appreciation.
        \end{itemize}
\end{enumerate}

\subsubsection*{2.4 Comparison: Mutual Funds vs. ETFs vs. Other Pooled Products}

\begin{center}
\begin{tabular}{|p{3cm}|p{4cm}|p{4cm}|p{4cm}|}
\hline
\textbf{Feature} & \textbf{Open-End Mutual Fund} & \textbf{Closed-End Fund} & \textbf{ETF (Exchange-Traded Fund)} \\
\hline
Trading Venue & Fund company (end of day) & Exchange & Exchange (intraday) \\
\hline
Transaction Price & NAV only & Market price (premium/discount) & Close to NAV (via arbitrage) \\
\hline
Management Style & Active or passive & Active or passive & Mostly passive (index-based) \\
\hline
Liquidity & High (daily redemptions) & High (market trading) & High (intraday trading) \\
\hline
Capital Gains Liability & Higher (fund must sell assets for redemptions) & Moderate & Lower (in-kind redemption structure) \\
\hline
Dividends & Often reinvested automatically & Paid as declared & Usually paid in cash \\
\hline
Short Selling / Margin & Not allowed & Allowed & Allowed \\
\hline
Investor Costs & Management + load fees & Brokerage costs & Brokerage commissions + bid-ask spread \\
\hline
\end{tabular}
\end{center}

\textbf{Interpretation (Verbal):}
\begin{itemize}
    \item ETFs trade intraday like stocks, while open-end funds trade only once daily.
    \item Closed-end funds can trade at prices different from NAV depending on investor sentiment and liquidity.
    \item ETFs generally have lower tax liabilities due to in-kind redemptions.
\end{itemize}

\subsubsection*{2.5 Separately Managed Accounts (SMAs)}

\textbf{Definition:}  
A professionally managed investment portfolio owned by a single investor, tailored to that investor’s objectives and constraints.

\textbf{Characteristics:}
\begin{itemize}
    \item Direct ownership of underlying securities (no fund shares issued).
    \item Customized strategies and tax optimization.
    \item Typically for high-net-worth individuals or institutions.
    \item Fees based on assets under management (AUM).
\end{itemize}

\subsubsection*{2.6 Hedge Funds}

\textbf{Definition:}  
Privately organized investment pools with limited investor access and less regulatory oversight than mutual funds.

\textbf{Features:}
\begin{itemize}
    \item Restricted to qualified or accredited investors (e.g., net worth or income thresholds).
    \item Use leverage, derivatives, and short positions.
    \item Wide range of strategies (long/short equity, global macro, event-driven, arbitrage, etc.).
    \item Typical fee structure: “2 and 20” — 2\% of AUM + 20\% of profits (incentive fee).
\end{itemize}

\textbf{Example:}  
A long/short equity hedge fund may buy undervalued stocks and short overvalued ones to generate absolute returns regardless of market direction.

\subsubsection*{2.7 Private Equity and Venture Capital Funds}

\textbf{Private Equity:}
\begin{itemize}
    \item Invests in established firms, often taking them private via leveraged buyouts (LBOs).
    \item Goal: Restructure company to improve efficiency and cash flows, then exit via resale or IPO within 3–5 years.
\end{itemize}

\textbf{Venture Capital:}
\begin{itemize}
    \item Focuses on financing start-ups and early-stage firms.
    \item Expects that only a small fraction of investments will succeed, but with very high potential returns.
    \item Fund managers often take active roles in advising and mentoring management.
\end{itemize}

\begin{center}
\begin{tabular}{|l|p{6cm}|p{6cm}|}
\hline
\textbf{Feature} & \textbf{Private Equity} & \textbf{Venture Capital} \\
\hline
Target Firms & Mature, undervalued, cash-flow positive & Start-ups, early-stage, high-growth potential \\
\hline
Typical Holding Period & 3–7 years & 5–10 years \\
\hline
Primary Goal & Operational improvement and exit via sale or IPO & Support innovation and achieve exponential growth \\
\hline
Manager Involvement & High (governance and restructuring) & Very high (advisory and mentoring role) \\
\hline
Expected Success Rate & Majority of deals successful & Minority succeed, but winners produce high returns \\
\hline
\end{tabular}
\end{center}

\subsection*{3. Key Insights and Comparative Overview}

\subsubsection*{3.1 Asset Management Spectrum}

\begin{center}
\begin{tabular}{|p{4cm}|p{6cm}|p{6cm}|}
\hline
\textbf{Category} & \textbf{Examples} & \textbf{Risk/Return Profile} \\
\hline
Traditional (Public Markets) & Mutual funds, ETFs, SMAs & Moderate risk, high liquidity \\
\hline
Alternative (Private Markets) & Hedge funds, Private equity, Venture capital & Higher risk, lower liquidity, higher return potential \\
\hline
\end{tabular}
\end{center}

\subsubsection*{3.2 Summary of Pooled Investment Characteristics}

\begin{center}
\begin{tabular}{|p{3cm}|p{3cm}|p{3cm}|p{3cm}|p{3cm}|}
\hline
\textbf{Investment Type} & \textbf{Liquidity} & \textbf{Regulation} & \textbf{Minimum Investment} & \textbf{Investor Base} \\
\hline
Open-End Mutual Fund & High (daily) & High & Low & General public \\
\hline
Closed-End Fund & High (market trading) & High & Low to moderate & Public investors \\
\hline
ETF & Very High (intraday) & High & Low & Public investors \\
\hline
SMA & Moderate & Moderate & High & Wealthy individuals/institutions \\
\hline
Hedge Fund & Low to Moderate & Low & High (≥ \$250k–\$1M) & Accredited investors \\
\hline
Private Equity/Venture Capital & Very Low (illiquid) & Low & Very High & Institutional / accredited investors \\
\hline
\end{tabular}
\end{center}

\subsection*{4. Conceptual Summary}

\begin{itemize}
    \item The \textbf{asset management industry} consists of firms that manage investments for clients, either actively (seeking alpha) or passively (tracking indices).
    \item \textbf{Active vs. Passive:} Active seeks outperformance via research; passive replicates benchmark returns with lower fees.
    \item \textbf{Traditional vs. Alternative:} Traditional managers focus on public equities/bonds; alternatives include private equity, hedge funds, and real assets.
    \item \textbf{Key Trends:}
        \begin{itemize}
            \item Growth of passive management.
            \item Expansion into alternatives.
            \item Use of technology and robo-advisors.
        \end{itemize}
    \item \textbf{Pooled Investments:}
        \begin{itemize}
            \item Mutual funds (open/closed) → accessible, diversified.
            \item ETFs → tradable, tax-efficient, low-cost.
            \item SMAs → customized, single-investor portfolios.
            \item Hedge funds, private equity, venture capital → high-risk, high-return, limited access.
        \end{itemize}
\end{itemize}

\section*{MODULE 86.1: Portfolio Planning and Construction}

\subsection*{Learning Objectives}
\begin{itemize}
    \item \textbf{LOS 86.a:} Explain the reasons for a written Investment Policy Statement (IPS).
    \item \textbf{LOS 86.b:} Describe the major components of an IPS.
    \item \textbf{LOS 86.c:} Describe risk and return objectives and how they are developed.
    \item \textbf{LOS 86.d:} Explain willingness vs. ability to take risk.
    \item \textbf{LOS 86.e:} Describe investment constraints (liquidity, time horizon, taxes, legal, unique circumstances).
    \item \textbf{LOS 86.f:} Explain specification of asset classes for asset allocation.
    \item \textbf{LOS 86.g:} Describe principles of portfolio construction and role of asset allocation.
    \item \textbf{LOS 86.h:} Describe integration of ESG factors into portfolio planning and construction.
\end{itemize}

\subsection*{1. Purpose of a Written Investment Policy Statement (IPS) (LOS 86.a)}

\textbf{Definition:}  
The \textbf{Investment Policy Statement (IPS)} is a formal, written document that outlines an investor’s objectives, constraints, and the policies to guide portfolio management decisions.

\textbf{Key Reasons for a Written IPS:}
\begin{itemize}
    \item Serves as a \textbf{contract} between client and manager — clarifying expectations.
    \item Ensures that portfolio decisions are consistent with client’s \textbf{risk tolerance} and \textbf{return goals}.
    \item Promotes discipline and reduces emotional decision-making.
    \item Provides a \textbf{benchmark} for evaluating performance.
    \item Helps manage long-term relationships through transparency and accountability.
\end{itemize}

\textbf{Important:}  
High return goals must align with acceptable risk. Unrealistically low risk with high return is incompatible — the IPS reconciles these expectations.

\subsection*{2. Major Components of the IPS (LOS 86.b)}

A comprehensive IPS should address the following elements:

\begin{enumerate}
    \item \textbf{Client Description:}  
          Financial situation, investment objectives, constraints, risk tolerance, time horizon, and other personal factors.

    \item \textbf{Purpose of IPS:}  
          Defines the scope and intent of the document (e.g., managing family trust or pension fund).

    \item \textbf{Duties and Responsibilities:}  
          Specifies roles of client, investment manager, custodian, and advisors.

    \item \textbf{Procedures for Review:}  
          Details frequency of updates, triggers for IPS revision, and communication procedures.

    \item \textbf{Investment Objectives:}  
          States the \textbf{risk and return objectives} derived from client discussions.

    \item \textbf{Investment Constraints:}  
          Identifies liquidity needs, time horizon, taxes, legal/regulatory factors, and unique considerations (see Section 5).

    \item \textbf{Investment Guidelines:}  
          Specifies asset classes permitted, diversification standards, use of derivatives, and leverage policies.

    \item \textbf{Performance Evaluation:}  
          Defines benchmark(s) and methods for measuring and comparing results.

    \item \textbf{Appendices:}  
          Include target (strategic) asset allocation, allowable deviations, and rebalancing rules.
\end{enumerate}

\textbf{Minimum IPS Requirements:}
\begin{itemize}
    \item Clear statement of client’s objectives and constraints.
    \item Investment strategy based on these factors.
    \item Benchmark for performance evaluation.
\end{itemize}

\subsection*{3. Risk and Return Objectives (LOS 86.c)}

\subsubsection*{3.1 Risk Objectives}

\textbf{Forms of Risk Objectives:}
\begin{itemize}
    \item \textbf{Absolute Risk Objective:} Expressed in absolute terms.
        \begin{itemize}
            \item Example: “No decrease in portfolio value during any 12-month period.”
            \item Example: “Not decrease in value by more than 2\% over any 12-month period.”
            \item Very conservative → implies near risk-free assets (e.g., T-bills).
        \end{itemize}

    \item \textbf{Probability-Based Absolute Risk:}
        \begin{itemize}
            \item “No greater than 5\% probability of return below –5\% in a year.”
            \item “No greater than 4\% probability of a \$20,000 loss over a year.”
        \end{itemize}

    \item \textbf{Relative Risk Objective:}  
          Related to performance of a benchmark.
        \begin{itemize}
            \item Example (strict): “Returns will not be less than 12-month LIBOR in any 12-month period.”
            \item Example (probabilistic): “No greater than 5\% probability of underperforming MSCI World Index by more than 4\%.”
        \end{itemize}
\end{itemize}

\subsubsection*{3.2 Return Objectives}

\textbf{Absolute Return Objectives:}
\begin{itemize}
    \item “Achieve 6\% per annum nominal return.”
    \item “Achieve 3\% real return above inflation.”
\end{itemize}

\textbf{Relative Return Objectives:}
\begin{itemize}
    \item “Exceed S\&P 500 return by 2\% per annum.”
    \item “Exceed bank deposit rate (cost of funds) by 1.5\%.”
\end{itemize}

\textbf{Compatibility Rule:}  
Risk and return objectives must be internally consistent — higher expected return requires higher accepted risk.

\textbf{Non-Investable Benchmarks Warning:}  
Peer performance objectives (e.g., “top quartile endowment performance”) are not investable — less suitable for IPS benchmarks.

\subsection*{4. Ability vs. Willingness to Take Risk (LOS 86.d)}

\begin{center}
\begin{tabular}{|l|l|}
\hline
\textbf{Dimension} & \textbf{Key Factors} \\
\hline
\textbf{Ability (Capacity)} & Depends on objective financial circumstances. \\
\hline
 & • Longer time horizon → higher ability. \\
 & • Greater wealth vs. liabilities → higher ability. \\
 & • Stable income or insured risks → higher ability. \\
 & • Short horizon or high liquidity needs → lower ability. \\
\hline
\textbf{Willingness (Psychological)} & Depends on personal attitude and comfort with volatility. \\
\hline
 & • Determined via discussion or questionnaires. \\
 & • Influenced by past experience, beliefs, emotions. \\
\hline
\end{tabular}
\end{center}

\textbf{Conflict Resolution:}
\begin{itemize}
    \item If \textbf{willingness > ability:} Ability dominates → portfolio risk lowered to protect capital.
    \item If \textbf{ability > willingness:} Educate investor, but final risk level should respect lower tolerance.
\end{itemize}

\textbf{Principle:}  
Construct portfolios using the \textbf{lower} of ability or willingness to bear risk to ensure client comfort and adherence.

\subsection*{5. Investment Constraints (LOS 86.e)}

\textbf{Mnemonic:} RRTTLLU — Risk, Return, Time horizon, Tax, Liquidity, Legal, Unique.  

\subsubsection*{5.1 Liquidity}
\begin{itemize}
    \item Ability to convert assets into cash without significant loss.
    \item Required for expenses such as tuition, healthcare, or living costs.
    \item Institutional example: P\&C insurers need high liquidity for unpredictable claims.
    \item Illiquid assets (e.g., hedge funds, private equity) unsuitable for short-term needs.
\end{itemize}

\subsubsection*{5.2 Time Horizon}
\begin{itemize}
    \item Longer horizons allow more risk-taking and less liquidity.
    \item Short horizons require conservative, liquid investments.
    \item Example:  
        - 20-year horizon → equities acceptable.  
        - 1-year horizon → cash equivalents or short-term bonds.
\end{itemize}

\subsubsection*{5.3 Tax Considerations}
\begin{itemize}
    \item Evaluate after-tax returns; tax rates vary by income and account type.
    \item High-tax investors may prefer tax-exempt bonds or capital-gains-focused equities.
    \item Tax-deferred accounts (e.g., retirement plans) should hold high-tax-income assets like bonds.
    \item Taxable accounts should hold assets generating tax-preferred income (e.g., dividends, long-term capital gains).
\end{itemize}

\subsubsection*{5.4 Legal and Regulatory Factors}
\begin{itemize}
    \item Certain accounts (trusts, pensions, corporations) face investment restrictions.
    \item Compliance with fiduciary standards is essential.
    \item Corporate insiders face trading limits on their own firm’s stock.
\end{itemize}

\subsubsection*{5.5 Unique Circumstances}
\begin{itemize}
    \item Ethical or religious screens (e.g., excluding tobacco, weapons, or interest-based securities).
    \item Employment-related diversification: avoid investments tied to investor’s primary income source.
    \item Socially responsible or ESG-based exclusions.
\end{itemize}

\subsection*{6. Specification of Asset Classes (LOS 86.f)}

\textbf{Purpose:}  
Define the broad categories of assets to be included in the strategic asset allocation.

\textbf{Guiding Principles:}
\begin{itemize}
    \item Assets within a class should have \textbf{high internal correlation} (similar performance characteristics).
    \item Asset classes should have \textbf{low correlation with each other} to provide diversification benefits.
\end{itemize}

\textbf{Traditional Asset Classes:}
\begin{itemize}
    \item Equities, Bonds, Cash, Real Estate.
\end{itemize}

\textbf{Alternative Asset Classes:}
\begin{itemize}
    \item Hedge funds, Private equity, Commodities, Infrastructure, Art, Intellectual property.
\end{itemize}

\textbf{Subdivisions:}
\begin{itemize}
    \item \textbf{Equities:} Domestic vs. Foreign, Large-cap vs. Small-cap, Developed vs. Emerging markets.
    \item \textbf{Bonds:} Domestic vs. International, Government vs. Corporate, Investment grade vs. High-yield, Short vs. Long maturity.
\end{itemize}

\textbf{Data Required:}
\begin{itemize}
    \item Expected returns, standard deviations, and correlations between asset classes.
\end{itemize}

\subsection*{7. Portfolio Construction and Asset Allocation (LOS 86.g)}

\textbf{Step 1: Construct Efficient Frontier}
\begin{itemize}
    \item Use expected return, risk, and correlation data of asset classes.
    \item Identify portfolios that minimize risk for each expected return level.
\end{itemize}

\textbf{Step 2: Select Optimal Portfolio}
\begin{itemize}
    \item Combine investor’s risk–return objectives (from IPS) with efficient frontier data.
    \item Choose the portfolio that maximizes expected utility — this becomes the \textbf{strategic asset allocation}.
\end{itemize}

\textbf{Active vs. Passive Portfolio Management:}

\begin{center}
\begin{tabular}{|l|l|}
\hline
\textbf{Strategy Type} & \textbf{Description} \\
\hline
\textbf{Passive} & Maintain strategic asset allocation using index-based investments. \\
\hline
\textbf{Active} & Deviate from strategic allocation to exploit short-term opportunities. \\
\hline
\end{tabular}
\end{center}

\textbf{Active Strategies:}
\begin{itemize}
    \item \textbf{Tactical Asset Allocation (TAA):} Temporarily adjust weights across asset classes based on market outlook.
    \item \textbf{Security Selection:} Over/underweight securities within asset classes.
\end{itemize}

\textbf{Risk Budgeting:}
\begin{itemize}
    \item Assign total portfolio risk across:
        \begin{enumerate}
            \item Systematic risk (strategic allocation),
            \item Tactical risk,
            \item Security-selection risk.
        \end{enumerate}
\end{itemize}

\textbf{Core–Satellite Approach:}
\begin{itemize}
    \item \textbf{Core:} Passive index holdings representing most of the portfolio.
    \item \textbf{Satellite:} Smaller allocations to active strategies.
    \item Benefits: Minimizes offsetting active bets, reduces excessive trading, improves tax efficiency.
\end{itemize}

\subsection*{8. ESG Integration into Portfolio Planning (LOS 86.h)}

\textbf{ESG = Environmental, Social, and Governance factors.}  
ESG integration aligns investment decisions with ethical, sustainability, or governance considerations.

\textbf{ESG Approaches:}
\begin{enumerate}
    \item \textbf{Negative Screening:} Exclude industries (e.g., tobacco, fossil fuels, weapons).
    \item \textbf{Positive Screening:} Include firms with strong ESG practices.
    \item \textbf{Thematic Investing:} Target ESG themes (e.g., clean energy, gender equality).
    \item \textbf{Impact Investing:} Seek measurable social/environmental impact alongside returns.
    \item \textbf{Engagement / Active Ownership:} Use shareholder rights to influence ESG improvements.
    \item \textbf{ESG Integration:} Systematically include ESG factors throughout security selection and asset allocation.
\end{enumerate}

\textbf{Key Implementation Considerations:}
\begin{itemize}
    \item Constrained investment universe → benchmark must align with ESG exclusions.
    \item Positive or thematic strategies require customized performance benchmarks.
    \item ESG may reduce diversification or incur costs but can improve long-term risk management.
\end{itemize}

\textbf{Example:}
\begin{itemize}
    \item Negative screen: Exclude fossil fuel producers.
    \item Thematic: Invest in renewable energy firms.
    \item Active ownership: Vote proxies for better corporate governance.
\end{itemize}

\subsection*{9. Conceptual Summary}

\begin{itemize}
    \item The \textbf{IPS} anchors portfolio design — defining objectives, constraints, and benchmarks.
    \item \textbf{Risk and return} objectives must be realistic and aligned with investor tolerance.
    \item \textbf{Constraints (RRTTLLU)} ensure the portfolio fits practical realities (liquidity, taxes, etc.).
    \item \textbf{Asset allocation} determines most of the portfolio’s risk and return.
    \item \textbf{Risk budgeting} allocates total risk among strategic, tactical, and security-selection sources.
    \item \textbf{ESG integration} customizes portfolios for social and environmental alignment while managing long-term sustainability risks.
\end{itemize}

\section*{MODULE 87.1: Cognitive Errors vs. Emotional Biases}

\subsection*{Overview}
Traditional finance assumes that investors are rational, risk-averse, and process all available information efficiently.  
Behavioral finance, pioneered by \textbf{Kahneman and Tversky}, challenges this assumption by showing that humans display \textbf{systematic biases} in decision-making due to both \textbf{cognitive limitations} and \textbf{emotional influences}.  
These biases cause deviations from rational behavior and can result in suboptimal financial decisions.



\subsection*{LOS 87.a: Cognitive Errors vs. Emotional Biases}

\begin{itemize}
    \item \textbf{Cognitive Errors:}  
    Result from faulty reasoning, improper information processing, or lack of understanding of probability/statistics.  
    They are often due to poor logic, misinterpretation, or memory errors.  
    \textbf{→ Can often be mitigated through education, awareness, or better information.}
    \item \textbf{Emotional Biases:}  
    Arise from feelings, impulses, or intuition rather than conscious reasoning.  
    \textbf{→ Difficult to correct; must often be accommodated rather than eliminated.}
\end{itemize}

\textbf{Key Difference:}
\begin{center}
\begin{tabular}{|p{8cm}|p{8cm}|}
\hline
\textbf{Cognitive Errors} & \textbf{Emotional Biases} \\
\hline
Errors in information processing or reasoning & Errors due to feelings and emotions \\
\hline
Can be reduced by training and awareness & Hard to control; often require behavioral adjustment \\
\hline
Example: Anchoring on old data & Example: Loss aversion or overconfidence \\
\hline
\end{tabular}
\end{center}

\textbf{Note:}  
Many biases contain both cognitive and emotional elements. When both exist, focus on mitigating the cognitive component first — it is more manageable.



\subsection*{LOS 87.b: Commonly Recognized Behavioral Biases and Their Implications}

Behavioral biases can be grouped into two categories of \textbf{cognitive errors}:

\begin{enumerate}
    \item \textbf{Belief Perseverance Biases} – irrationally holding on to prior beliefs.  
    \item \textbf{Information-Processing Biases} – faulty processing or evaluation of new data.
\end{enumerate}



\subsection*{1. Belief Perseverance Biases}

\textbf{Definition:}  
Biases that cause individuals to hold on to prior beliefs or conclusions, even when presented with new evidence.

\textbf{Main Types:}

\begin{center}
\begin{tabular}{|l|p{10cm}|}
\hline
\textbf{Bias} & \textbf{Description and Implications} \\
\hline
\textbf{Conservatism Bias} & Failure to update beliefs or forecasts when new information arises.  
Investors overweight prior probabilities and underweight new data.  
\textbf{Example:} An investor keeps a recession probability at 20\% despite central bank tightening — ignoring updated signals.  
\textbf{Consequence:} Holding losing positions too long or reacting too slowly to new information. \\
\hline
\textbf{Confirmation Bias} & Seeking or interpreting information that confirms existing beliefs while ignoring contradictory data.  
\textbf{Example:} Reading only positive reviews after buying a car (or stock).  
\textbf{Consequence:} Overconfidence, poor decision validation, ignoring downside risk.  
\textbf{Mitigation:} Actively seek disconfirming evidence and opposing viewpoints. \\
\hline
\textbf{Representativeness Bias} & Classifying new information based on superficial similarity to existing stereotypes or small samples.  
Two forms:  
(i) \textbf{Base-rate neglect} – ignoring actual statistical frequency.  
(ii) \textbf{Sample-size neglect} – overgeneralizing from small samples.  
\textbf{Example:} Assuming a fund manager is “skilled” after three strong years without considering randomness.  
\textbf{Consequence:} Misclassification, trend-chasing, ignoring mean reversion. \\
\hline
\textbf{Illusion of Control Bias} & Belief that one can control or influence outcomes that are actually random.  
Linked to \textbf{overconfidence}, \textbf{illusion of knowledge}, and \textbf{self-attribution}.  
\textbf{Example:} Overweighting stock of one’s employer because one feels “in control.”  
\textbf{Consequence:} Underdiversification, excessive trading. \\
\hline
\textbf{Hindsight Bias} & Believing after an event that it was predictable all along.  
Selective memory of past predictions (“I knew it!”).  
\textbf{Example:} Claiming foresight after a stock rally.  
\textbf{Consequence:} Overconfidence, abandonment of sound analysis in favor of luck-based reasoning. \\
\hline
\end{tabular}
\end{center}

\textbf{Summary Insight:}  
Belief perseverance biases distort how investors update beliefs — causing underreaction, overconfidence, or inflexible decision-making.



\subsection*{2. Information-Processing Biases}

\textbf{Definition:}  
Biases that result from incorrect information processing, weighting, or mental accounting rather than from the persistence of prior beliefs.

\textbf{Main Types:}

\begin{center}
\begin{tabular}{|l|p{10cm}|}
\hline
\textbf{Bias} & \textbf{Description and Implications} \\
\hline
\textbf{Anchoring and Adjustment Bias} & Basing decisions on an initial value (anchor) and failing to adjust sufficiently when new data emerges.  
\textbf{Example:} Estimating fair value of a stock relative to its current price instead of its intrinsic value.  
\textbf{Consequence:} Underreaction to new information; inertia in forecast updates.  
\textbf{Mitigation:} Re-evaluate data independently of prior anchor. \\
\hline
\textbf{Mental Accounting Bias} & Treating money differently depending on its source or purpose.  
\textbf{Example:} Investing a work bonus in risky stocks (“found money”) while keeping savings ultra-safe.  
\textbf{Consequence:} Suboptimal total portfolio, inefficient diversification, conflicting sub-portfolios.  
\textbf{Note:} Investors should adopt a \textbf{total-portfolio perspective}. \\
\hline
\textbf{Framing Bias} & Decision outcomes are influenced by the way choices or data are presented (gain vs. loss framing).  
\textbf{Example:} Prefer “saving 200 lives” (gain frame) vs. avoiding “400 deaths” (loss frame) — though outcomes identical.  
\textbf{In investments:} Overemphasizing short-term losses over long-term potential.  
\textbf{Consequence:} Risk aversion in gains, risk-seeking in losses, mis-assessed risk tolerance. \\
\hline
\textbf{Availability Bias} & Overweighting information that is recent, easily recalled, or vivid.  
\textbf{Example:} After a market crash, investors overestimate probability of another crash.  
\textbf{Consequence:} Poor diversification, overreaction to news, familiarity bias (investing only in known firms).  
\textbf{Mitigation:} Broaden data sources, rely on statistical evidence. \\
\hline
\end{tabular}
\end{center}



\subsection*{3. Cognitive Bias Examples Explained}

\subsubsection*{Conservatism Bias Example}
\textbf{Scenario:}  
An analyst keeps forecasting stable growth despite central bank tightening.  
\textbf{Error:} Fails to adjust recession probability given new conditions → underreaction.

\textbf{Implication:}  
Investor may hold outdated positions too long, ignoring relevant information.



\subsubsection*{Confirmation Bias Example}
\textbf{Scenario:}  
An investor reads only bullish articles on Tesla after buying it.  
\textbf{Error:} Selectively processing information to confirm preexisting belief.  
\textbf{Implication:}  
Overconfidence, reduced willingness to reconsider investment thesis.



\subsubsection*{Representativeness Bias Example}
\textbf{Scenario:}  
XYZ Corp has long been a “growth” stock. Despite recent stagnation, an analyst still recommends buying because of its “growth” label.  
\textbf{Error:} Stereotyping based on past category — ignoring changed fundamentals.  
\textbf{Implication:}  
Misclassification and failure to recognize structural shift.



\subsubsection*{Anchoring and Adjustment Example}
\textbf{Scenario:}  
A stock once traded at \$100 but is now \$60. Analyst sets price target of \$80 simply “because it used to be 100.”  
\textbf{Error:} Using past price as anchor rather than intrinsic value.  
\textbf{Implication:}  
Slow reaction to new valuation realities; possible overvaluation bias.



\subsubsection*{Framing Bias Example}
\textbf{Scenario:}  
Client A told “Your portfolio gained 10\% this year.” Client B told “You missed 5\% vs. market.” Both have same return, but reactions differ.  
\textbf{Error:} Perception shifts due to framing of gain/loss.  
\textbf{Implication:}  
Emotionally inconsistent risk preferences.



\subsection*{4. Summary: Classification of Key Cognitive Biases}

\begin{center}
\begin{tabular}{|l|p{10cm}|}
\hline
\textbf{Category} & \textbf{Biases Included} \\
\hline
\textbf{Belief Perseverance} & Conservatism, Confirmation, Representativeness, Illusion of Control, Hindsight \\
\hline
\textbf{Information Processing} & Anchoring and Adjustment, Mental Accounting, Framing, Availability \\
\hline
\end{tabular}
\end{center}



\subsection*{5. Implications for Financial Decision-Making}

\textbf{Key Effects of Biases:}
\begin{itemize}
    \item Holding losing investments too long (conservatism, confirmation).
    \item Underdiversification and concentrated portfolios (illusion of control, availability).
    \item Overtrading and overconfidence (hindsight, illusion of control).
    \item Misestimation of probabilities or expected returns (representativeness, anchoring).
    \item Poor portfolio optimization due to segmentation of wealth (mental accounting).
    \item Inconsistent risk preferences depending on framing (framing bias).
\end{itemize}

\textbf{Practical Applications:}
\begin{itemize}
    \item Advisors must identify client biases and adapt strategy — some can be corrected, others accommodated.
    \item Educate investors on the effects of anchoring, overconfidence, and framing in decision-making.
    \item Incorporate behavioral insights into portfolio rebalancing and risk assessment.
\end{itemize}



\subsection*{6. CFA-Style Exam Pointers}

\begin{itemize}
    \item \textbf{Cognitive errors} → can be \textit{mitigated} (education, analysis).  
    \textbf{Emotional biases} → must often be \textit{accommodated} (behavioral adjustment).
    \item \textbf{Belief perseverance} = failure to revise view; \textbf{Information-processing} = incorrect data use.
    \item \textbf{Framing and availability} often appear in client risk-tolerance questions.
    \item \textbf{Anchoring vs. conservatism:}  
          – Anchoring = fixating on an initial number.  
          – Conservatism = under-adjusting to new info.
    \item \textbf{Representativeness vs. confirmation:}  
          – Representativeness = stereotyping new data.  
          – Confirmation = seeking info supporting old belief.
\end{itemize}



\subsection*{7. Conceptual Summary Table}

\begin{center}
\begin{tabular}{|l|p{10cm}|}
\hline
\textbf{Bias Type} & \textbf{Practical Investment Implications} \\
\hline
Conservatism & Fails to adjust portfolio to new data; may miss turning points. \\
\hline
Confirmation & Ignores disconfirming info; reinforces wrong forecasts. \\
\hline
Representativeness & Misclassifies investments; buys “hot” stocks or managers. \\
\hline
Illusion of Control & Overtrading, underdiversification, excessive risk. \\
\hline
Hindsight & Overconfidence, poor learning from mistakes. \\
\hline
Anchoring & Biased price targets, slow reaction to new events. \\
\hline
Mental Accounting & Inefficient asset allocation; suboptimal total return. \\
\hline
Framing & Inconsistent risk attitude; misjudged risk tolerance. \\
\hline
Availability & Overreaction to recent or vivid events; familiarity bias. \\
\hline
\end{tabular}
\end{center}



\subsection*{8. Key Takeaways}

\begin{itemize}
    \item Behavioral biases distort rational investment decisions.
    \item Cognitive errors stem from reasoning flaws — \textbf{can be educated away.}
    \item Emotional biases stem from feelings — \textbf{must be managed or accepted.}
    \item Advisors must diagnose which bias dominates client behavior.
    \item Overcoming biases leads to more consistent, long-term investment discipline.
\end{itemize}

\section*{MODULE 87.2: Emotional Biases and Market Implications}

\subsection*{Overview}
Traditional finance assumes that investors are rational and markets are efficient.  
Behavioral finance, however, recognizes that investors exhibit \textbf{emotional biases} — decisions driven by feelings, impulses, and psychological comfort rather than logic or probability.  
These biases often persist even when investors are aware of them and cannot be corrected through training alone.



\subsection*{LOS 87.b (continued): Emotional Biases in Decision-Making}

\textbf{Key Idea:}  
Emotional biases originate from feelings and affective reactions, not cognitive miscalculations.  
They are difficult to mitigate — often requiring behavioral adaptation or portfolio design rather than education.

\subsubsection*{Summary of Major Emotional Biases}

% Bullet-list version of the behavioral-bias content (compact)
\begin{itemize}[leftmargin=*,noitemsep,topsep=0pt]
  \item \textbf{Loss‑Aversion Bias} — Investors feel greater pain from losses than pleasure from equal gains (``losses loom larger than gains'') and evaluate outcomes relative to a reference point (current wealth or purchase price).
    \begin{itemize}[noitemsep,topsep=0pt,leftmargin=*]
      \item \textbf{Example:} Prefer a certain gain (\$5) over a 50–50 chance to gain \$10, but prefer a risky loss over a certain loss.
      \item \textbf{Consequences:} Selling winners too soon; holding losers too long; taking excessive risk to recover losses; trading too frequently.
      \item \textbf{Distinction:} Risk aversion (preference for lower variance given equal return) vs.\ loss aversion (asymmetric emotional reaction to losses vs.\ gains).
    \end{itemize}

  \item \textbf{Overconfidence Bias} — Excessive faith in one’s own judgment or skill.
    \begin{itemize}[noitemsep,topsep=0pt,leftmargin=*]
      \item \textbf{Forms:} Illusion of knowledge; self‑attribution (crediting success to skill, blaming failure on external factors).
      \item \textbf{Consequences:} Underestimating risk, underdiversified portfolios, excessive trading, ignoring contrary evidence.
      \item \textbf{Mitigation:} Maintain performance records; compare forecasts to outcomes and track forecast accuracy.
    \end{itemize}

  \item \textbf{Self‑Control Bias} — Preference for immediate gratification over long‑term benefit (hyperbolic discounting).
    \begin{itemize}[noitemsep,topsep=0pt,leftmargin=*]
      \item \textbf{Example:} Spending now instead of saving for retirement.
      \item \textbf{Consequences:} Insufficient retirement savings; overemphasis on short‑term income; risky catch‑up behavior later.
      \item \textbf{Mitigation:} Automatic savings plans, automatic payroll deductions, pre‑commitment mechanisms.
    \end{itemize}

  \item \textbf{Status Quo Bias} — Emotional comfort with the current situation leads to resistance to change.
    \begin{itemize}[noitemsep,topsep=0pt,leftmargin=*]
      \item \textbf{Example:} Not updating default 401(k) allocations or failing to opt into a better fund.
      \item \textbf{Consequences:} Portfolios become misaligned with changing goals or risk tolerance; missed opportunities.
      \item \textbf{Behavioral application:} Automatic enrollment (with opt‑out) increases participation (Thaler \& Sunstein).
    \end{itemize}

  \item \textbf{Endowment Bias} — Tendency to overvalue an asset simply because one owns it (ownership effect).
    \begin{itemize}[noitemsep,topsep=0pt,leftmargin=*]
      \item \textbf{Example:} Holding inherited stock for sentimental reasons.
      \item \textbf{Consequences:} Emotional attachment leads to underdiversification and holding inappropriate or outdated assets.
      \item \textbf{Test question:} ``Would I buy this asset today with new money?'' — helps detect the bias.
    \end{itemize}

  \item \textbf{Regret‑Aversion Bias} — Avoiding decisions to reduce potential future regret; stronger regret for actions (commission) than for inaction (omission).
    \begin{itemize}[noitemsep,topsep=0pt,leftmargin=*]
      \item \textbf{Example:} Not buying a promising stock to avoid feeling foolish if it declines.
      \item \textbf{Consequences:} Excessive conservatism, missed opportunities, herding behavior (to share blame).
      \item \textbf{Mitigation:} Pre‑commitment, rules‑based investing, or documented decision rules to reduce emotional paralysis.
    \end{itemize}
\end{itemize}



\subsection*{Behavioral Bias Comparison Table}

\begin{center}
\begin{tabular}{|l|l|l|}
\hline
\textbf{Bias} & \textbf{Core Emotion} & \textbf{Typical Investor Behavior} \\
\hline
Loss Aversion & Fear of loss & Hold losers; sell winners too early \\
\hline
Overconfidence & Pride, ego & Overtrade; underdiversify; ignore risk \\
\hline
Self-Control & Impatience & Under-save; short-term focus \\
\hline
Status Quo & Comfort & Resist portfolio changes \\
\hline
Endowment & Attachment & Overvalue owned assets \\
\hline
Regret Aversion & Fear of mistake & Follow herd; avoid action \\
\hline
\end{tabular}
\end{center}

\textbf{Exam Tip:}  
If bias stems from emotional attachment, fear, or self-image → classify as \textit{emotional}.  
If it stems from faulty reasoning or data interpretation → classify as \textit{cognitive}.



\subsection*{LOS 87.c: Behavioral Biases and Market Characteristics}

\textbf{Key Idea:}  
Investor biases can aggregate into market-wide phenomena — explaining anomalies and inefficiencies that traditional finance struggles to justify.

\subsubsection*{1. Market Bubbles and Crashes}
\begin{itemize}
    \item \textbf{Overconfidence:} Investors overestimate skill and underweight risk, leading to overtrading and inflated valuations.  
          – During bull markets, good results reinforce confidence (self-attribution).  
    \item \textbf{Confirmation Bias:} Investors ignore signals that prices are too high; dismiss warnings as noise.  
    \item \textbf{Anchoring:} Investors treat recent high prices as “normal,” refusing to revise valuations downward.  
    \item \textbf{Regret Aversion:} Fear of missing out keeps investors in markets even when they believe prices are excessive.  
    \item \textbf{Outcome:} Excess demand inflates bubbles; when sentiment reverses, sharp corrections follow.  
\end{itemize}

\textbf{Example:}  
Dot-com bubble (late 1990s) — investors extrapolated short-term growth, ignored valuations, and herded into tech stocks.

\textbf{Behavioral Explanation:}  
Market participants as a group exhibit \textbf{collective cognitive dissonance} — refusing to accept contradictory evidence until forced by large losses.



\subsubsection*{2. The Value vs. Growth Anomaly}
\begin{itemize}
    \item Traditional models found that \textbf{value stocks} (low P/E, high book-to-market) outperform \textbf{growth stocks}.  
    \item \textbf{Fama-French (1992):} Explained excess returns by additional risk factors (size, value).  
    \item \textbf{Behavioral View:}  
        – Investors overpay for ``glamour’’ growth stocks due to \textbf{representativeness} and the \textbf{halo effect}.  
        – They underreact to positive information about unpopular (value) stocks.  
    \item \textbf{Halo Effect:} Extending one positive attribute (e.g., rapid growth) into unjustified general optimism about quality and future performance.  
\end{itemize}

\textbf{Implication:}  
Behavioral mispricing persists until fundamentals reassert, generating excess value-stock returns.



\subsubsection*{3. Home Bias and Familiarity Effects}
\begin{itemize}
    \item \textbf{Home Bias:} Investors overweight domestic or regional securities despite the diversification benefits of global investing.  
    \item \textbf{Reasons:}
        \begin{itemize}
            \item Emotional comfort and familiarity.  
            \item Illusion of information advantage (“I know these firms better”).  
            \item Preference for proximity and cultural similarity.  
        \end{itemize}
    \item \textbf{Consequences:}
        \begin{itemize}
            \item Reduced diversification.  
            \item Underexposure to international opportunities.  
        \end{itemize}
\end{itemize}

\textbf{Behavioral Drivers:} Familiarity bias (availability) and overconfidence in local knowledge.



\subsection*{4. Behavioral Finance and Market Inefficiency}

\textbf{Traditional View:}  
Anomalies are due to model misspecification or random chance.

\textbf{Behavioral View:}  
Anomalies reflect predictable patterns of human bias and emotion, especially:
\begin{itemize}
    \item Overreaction (overconfidence, representativeness).  
    \item Underreaction (conservatism, anchoring).  
    \item Excessive volatility (herding, loss aversion).  
    \item Momentum and reversal patterns (due to delayed or emotional adjustments).  
\end{itemize}

\textbf{Limitations:}  
Behavioral finance explains tendencies, not precise market equilibria.  
Biases help interpret deviations but not always forecast them.



\subsection*{5. Summary of Behavioral Market Effects}

\begin{center}
\begin{tabular}{|l|p{9cm}|}
\hline
\textbf{Bias / Emotion} & \textbf{Market-Level Effect} \\
\hline
Overconfidence & Overtrading, bubbles, excess volatility. \\
\hline
Confirmation & Ignoring negative news; persistent overvaluation. \\
\hline
Anchoring & Slow adjustment to fundamentals after shocks. \\
\hline
Regret Aversion & Herding; delayed exits from bubbles. \\
\hline
Representativeness / Halo & Overpricing of growth stocks; value premium. \\
\hline
Home / Familiarity & Underdiversification; regional concentration. \\
\hline
Loss Aversion & Selling winners too early; holding losers too long. \\
\hline
\end{tabular}
\end{center}



\subsection*{6. Conceptual Summary}

\begin{itemize}
    \item \textbf{Emotional biases} are rooted in feelings — difficult to eliminate.  
    \item They lead to predictable investor errors (holding losers, underdiversifying, herding).  
    \item When biases aggregate across many participants, markets display anomalies such as bubbles, crashes, and the value premium.  
    \item Recognizing these behavioral patterns improves both portfolio management and client advisory quality.  
\end{itemize}

\section*{MODULE 88.1: Introduction to Risk Management}

\subsection*{LOS 88.a: Defining Risk Management}

\begin{itemize}
    \item \textbf{Risk management} is the process of:
    \begin{enumerate}
        \item Identifying the organization's \textbf{risk tolerance}.
        \item Identifying and measuring the \textbf{risks faced}.
        \item Modifying and monitoring these risks over time.
    \end{enumerate}

    \item The goal is not to eliminate risk but to achieve the \textbf{optimal bundle of risks} consistent with the organization’s objectives and capacity.
    \item \textbf{Rationale:} 
    \begin{itemize}
        \item Risk cannot and should not be completely avoided — returns above the risk-free rate require accepting risk.
        \item Management controls the amount and types of risk, not actual realized returns.
    \end{itemize}

    \item \textbf{In essence:}
    \[
    \text{Risk Management} = \text{Identify} + \text{Measure} + \text{Modify} + \text{Monitor}
    \]
    while aligning all exposures with the firm’s risk tolerance.
\end{itemize}



\subsection*{LOS 88.b: Features of a Risk Management Framework}

\begin{itemize}
    \item A comprehensive framework must include:
    \begin{enumerate}
        \item \textbf{Risk Governance:} Senior management oversight and policy setting.
        \item \textbf{Risk Tolerance Determination:} Define acceptable total exposure.
        \item \textbf{Risk Identification \& Measurement:} Map all existing risks.
        \item \textbf{Risk Management \& Mitigation:} Adjust exposures toward the optimal mix.
        \item \textbf{Risk Monitoring:} Continuous tracking and reporting.
        \item \textbf{Communication:} Ensure cross-departmental awareness.
        \item \textbf{Strategic Risk Analysis:} Link risk outcomes to business objectives.
    \end{enumerate}

    \item \textbf{Key Objective:} Align organizational risk exposures with its strategic goals and stakeholder interests.
\end{itemize}



\subsection*{LOS 88.c: Risk Governance and Effective Oversight}

\begin{itemize}
    \item \textbf{Risk Governance:} Senior management’s responsibility for:
    \begin{itemize}
        \item Determining firmwide \textbf{risk tolerance}.
        \item Designing the \textbf{risk management structure}.
        \item Overseeing execution and compliance.
    \end{itemize}

    \item \textbf{Objective:} Manage risk to support the firm’s strategic objectives — pursue efficient risks, limit excessive ones, avoid those outside tolerance.

    \item \textbf{Implementation:}
    \begin{itemize}
        \item Create a \textbf{Risk Management Committee} to integrate firmwide perspectives.
        \item Provide clear reporting lines and independent risk oversight.
    \end{itemize}
\end{itemize}



\subsection*{LOS 88.d: Risk Tolerance}

\begin{itemize}
    \item \textbf{Definition:} The total amount and types of risk an organization is willing and able to accept.
    \item \textbf{Factors Influencing Risk Tolerance:}
    \begin{enumerate}
        \item Firm’s expertise in its business lines.
        \item Financial strength and ability to absorb losses.
        \item Regulatory environment and external constraints.
        \item Management’s ability to respond to negative events.
    \end{enumerate}
    \item The objective is to \textbf{match risk exposure with capacity and goals}.
\end{itemize}



\subsection*{LOS 88.e: Risk Budgeting}

\begin{itemize}
    \item \textbf{Definition:} Allocation of total acceptable risk (the ``risk budget’’) across business units, asset classes, or risk factors.
    \item \textbf{Purpose:} Achieve maximum expected return for a given total risk level.

    \item \textbf{Approaches:}
    \begin{itemize}
        \item \textbf{Single-measure basis:} e.g., total portfolio variance, beta, duration, or VaR.
        \item \textbf{Asset-class basis:} e.g., domestic equity, bonds, international equity, etc.
        \item \textbf{Risk-factor basis:} e.g., interest rate, FX, equity market, and credit risks.
    \end{itemize}

    \item \textbf{Goal:} Combine these exposures to fit within the organization’s risk tolerance.
\end{itemize}



\subsection*{LOS 88.f: Financial vs. Non-Financial Risks}

\begin{center}
\begin{tabular}{|l|p{11cm}|}
\hline
\textbf{Risk Type} & \textbf{Definition and Examples} \\
\hline
\textbf{Financial Risks} & Arise from exposure to financial markets. \\
\hline
Credit Risk & Counterparty fails to fulfill contractual obligations. \\
\hline
Liquidity Risk & Loss from inability to sell assets quickly at fair value. \\
\hline
Market Risk & Loss due to changes in market prices or rates (stocks, commodities, FX, interest rates). \\
\hline
\textbf{Non-Financial Risks} & Arise from operations or external environment. \\
\hline
Operational Risk & Loss from human error, system failure, or fraud (includes \textit{cyber risk}). \\
\hline
Solvency Risk & Organization cannot continue operations due to lack of cash. \\
\hline
Regulatory Risk & Unfavorable change in regulation increases costs or limits activities. \\
\hline
Government/Political Risk & Policy or tax changes impose costs outside the regulatory framework. \\
\hline
Legal Risk & Exposure to litigation or contractual uncertainty. \\
\hline
Model Risk & Analytical or valuation models are incorrect. \\
\hline
Tail Risk & Extreme outcomes more probable than predicted by normal models. \\
\hline
Accounting Risk & Accounting choices or estimates are judged incorrect. \\
\hline
\end{tabular}
\end{center}

\textbf{Special cases for individuals:}
\begin{itemize}
    \item \textbf{Mortality risk:} Death before providing for dependents (mitigated by life insurance).  
    \item \textbf{Longevity risk:} Living longer than assets last (mitigated by annuities).  
    \item \textbf{Health-care risk:} Managed with health insurance.
\end{itemize}

\textbf{Interaction of Risks:}
\begin{itemize}
    \item Risks are interrelated, not additive.  
    \item Example: Market decline $\Rightarrow$ counterparty defaults $\Rightarrow$ legal disputes $\Rightarrow$ liquidity strain $\Rightarrow$ solvency risk.
    \item During crises, correlations among risks increase — amplifying total exposure.
\end{itemize}



\subsection*{LOS 88.g: Measuring and Modifying Risk Exposures}

\subsubsection*{(1) Common Measures of Risk}
\begin{center}
\begin{tabular}{|l|p{11cm}|}
\hline
\textbf{Measure} & \textbf{Description / Use} \\
\hline
Standard Deviation & Measures total volatility (stand-alone risk). Inadequate for non-normal distributions (fat tails). \\
\hline
Beta & Measures systematic (market) risk for equities within diversified portfolios. \\
\hline
Duration & Measures price sensitivity of bonds to interest-rate changes. \\
\hline
\textbf{Option Greeks} & Sensitivities for derivative positions: \\
\hline
Delta & Change in derivative value per unit change in underlying price. \\
\hline
Gamma & Change in delta per unit change in underlying price. \\
\hline
Vega & Change in derivative value per unit change in volatility. \\
\hline
Rho & Change in derivative value per unit change in risk-free rate. \\
\hline
\end{tabular}
\end{center}

\subsubsection*{(2) Tail Risk Metrics}
\begin{itemize}
    \item \textbf{Value at Risk (VaR):} Minimum expected loss over a time horizon at a given probability level.  
    Example: One-month 5\% VaR of \$1 million → 5\% chance of losing at least \$1 million in one month.
    \item \textbf{Conditional VaR (CVaR):} Expected loss \emph{given} the loss exceeds VaR (average of the tail losses).  
    Useful complement to VaR, similar to “loss given default’’.
\end{itemize}

\textbf{Limitations:}  
VaR is sensitive to model assumptions and distributions — it does not give the maximum possible loss.



\subsubsection*{(3) Subjective and Market-Based Risk Assessment}
\begin{itemize}
    \item \textbf{Stress Testing:} Examine effect of extreme but plausible shocks in one variable (e.g., interest-rate spike).  
    \item \textbf{Scenario Analysis:} Multi-variable ``what-if’’ testing — combined macro and market changes (e.g., rate hike + oil shock + FX move).  
    \item \textbf{Market-Implied Risk:} Use option prices, insurance premiums, or credit spreads to infer the market’s aggregate perception of risk.
    \item \textbf{Operational / Political Risks:} Often require subjective estimates when quantitative data are unavailable.
\end{itemize}



\subsubsection*{(4) Modifying Risk Exposures}

\begin{center}
\begin{tabular}{|l|p{10cm}|}
\hline
\textbf{Method} & \textbf{Description and Example} \\
\hline
\textbf{Avoidance} & Do not engage in the activity.  
Example: Do not invest in high-political-risk country. \\
\hline
\textbf{Prevention / Reduction} & Implement controls to reduce probability or impact.  
Example: Cybersecurity measures to prevent data breaches. \\
\hline
\textbf{Retention (Self-Insurance)} & Accept the risk internally; may create reserves for potential losses.  
Example: Firm keeps exposure to routine operational losses. \\
\hline
\textbf{Transfer} & Shift risk to another party through contracts.  
Example: Buy insurance, surety bonds, or fidelity bonds.  
– \textit{Insurance:} Pay premium to transfer risk of fire, theft, etc.  
– \textit{Surety bond:} Pays if third party fails contractual obligation.  
– \textit{Fidelity bond:} Covers employee theft or fraud. \\
\hline
\textbf{Shifting (via Derivatives)} & Alter distribution of outcomes through derivative instruments.  
Examples:  
– Buy put options (establish floor value).  
– Sell calls (reduce upside, collect premium).  
– Use swaps or forwards to hedge FX or interest-rate risk. \\
\hline
\end{tabular}
\end{center}

\textbf{Choosing Among Methods:}
\begin{itemize}
    \item Always weigh \textbf{costs vs. benefits}.  
    \item Combine tools — e.g., diversify, insure, and hedge simultaneously.  
    \item Aim for a final risk profile that matches the organization’s risk tolerance and strategic objectives.
\end{itemize}



\subsection*{Key Takeaways}

\begin{itemize}
    \item \textbf{Goal:} Not to eliminate risk, but to manage it efficiently.
    \item \textbf{Framework:} Governance → Measurement → Modification → Monitoring.
    \item \textbf{Effective Risk Governance:} Aligns all risk decisions with corporate strategy.
    \item \textbf{Risk Budgeting:} Distributes risk to maximize expected return per unit of accepted risk.
    \item \textbf{Risk Interactions:} Must be analyzed jointly — crises often reveal hidden correlations.
    \item \textbf{Modification Tools:} Avoid, prevent, retain, transfer, or shift — depending on cost-benefit trade-off.
\end{itemize}


\end{document}
