% Financial Analysis Study Notes for CFA Level I
% This document contains comprehensive study notes covering the Financial Statement Analysis framework
% and key concepts for the CFA Level I examination
% Author: CFA Level I Candidate
% Created: 2024 Study Session
% Last Updated: October 2025

\documentclass[12pt]{article}
\usepackage{amsmath}
\usepackage{geometry}
\usepackage{graphicx}
\usepackage{booktabs}
\usepackage{caption}
\usepackage{titlesec}
\usepackage{graphicx} % make sure to include in preamble
\usepackage{float}
\usepackage{makecell}
\usepackage{tabularx}
\usepackage{enumitem}


\geometry{margin=1in}

\title{Financial Analysis}
\date{}

\begin{document}
\maketitle
\subsection*{Module 29.1: Financial Statement Roles}

\subsubsection*{LOS 29.a: Steps in the Financial Statement Analysis Framework}
\begin{itemize}
  \item \textbf{Step 1: State the objective and context}
    \begin{itemize}
      \item Define key questions: e.g., “Should we invest in this company’s bonds?”
      \item Decide reporting format (memo, detailed report, presentation).
      \item Consider time and resources available.
    \end{itemize}
  \item \textbf{Step 2: Gather data}
    \begin{itemize}
      \item Collect company’s financial statements (10-K, annual reports).
      \item Industry reports, macroeconomic data.
      \item Field research: interviews with management, suppliers, site visits.
    \end{itemize}
  \item \textbf{Step 3: Process the data}
    \begin{itemize}
      \item Adjust statements (e.g., leases capitalized).
      \item Compute ratios: liquidity, profitability, leverage.
      \item Prepare exhibits: graphs, common-size balance sheets.
    \end{itemize}
  \item \textbf{Step 4: Analyze and interpret the data}
    \begin{itemize}
      \item Compare with peers and historical data.
      \item Identify risk factors and growth opportunities.
    \end{itemize}
  \item \textbf{Step 5: Report conclusions or recommendations}
    \begin{itemize}
      \item Ensure compliance with CFA Code and Standards.
      \item Adapt report to audience (investors, management, regulators).
    \end{itemize}
  \item \textbf{Step 6: Update the analysis}
    \begin{itemize}
      \item Continuous monitoring of new data.
      \item Adjust recommendations as conditions change.
    \end{itemize}
\end{itemize}

\subsubsection*{LOS 29.b: Roles of Financial Statement Analysis}
\begin{itemize}
  \item Uses accounting information to support \textbf{economic decisions}.
  \item Examples of decisions:
    \begin{itemize}
      \item Buy/sell recommendations for equity or debt securities.
      \item Assigning credit ratings.
      \item Extending trade or bank credit.
    \end{itemize}
  \item Analysts evaluate:
    \begin{itemize}
      \item Past performance and financial position.
      \item Future ability to generate profits and cash flows.
      \item Risk factors impacting profitability and stability.
    \end{itemize}
\end{itemize}

\subsubsection*{LOS 29.c: Importance of Regulatory Filings and Disclosures}
\begin{itemize}
  \item \textbf{Standard-setters:}
    \begin{itemize}
      \item \textbf{FASB (U.S.)}: U.S. GAAP.
      \item \textbf{IASB (International)}: IFRS.
    \end{itemize}
  \item \textbf{Regulators:}
    \begin{itemize}
      \item SEC (U.S.), FCA (UK), ESMA (EU).
      \item Members of \textbf{IOSCO} regulate $95\%$ of global markets.
    \end{itemize}
  \item \textbf{IOSCO Objectives:}
    \begin{enumerate}
      \item Protect investors.
      \item Ensure fair, efficient, transparent markets.
      \item Reduce systemic risk.
    \end{enumerate}
  \item \textbf{SEC Example Requirements:}
    \begin{itemize}
      \item Compliance with Sarbanes–Oxley Act (SOX 2002).
      \item CEO/CFO certification of financial statements.
      \item Auditor independence (cannot provide consulting services).
      \item Internal controls effectiveness statement.
    \end{itemize}
\end{itemize}

\paragraph{Financial Statement Notes (Footnotes):}
\begin{itemize}
  \item Provide basis of presentation (IFRS vs U.S. GAAP, fiscal year end).
  \item Disclose accounting methods, assumptions, estimates.
  \item Contain details on acquisitions, legal contingencies, pensions, related parties.
  \item Segment disclosures:
    \begin{itemize}
      \item Revenue (external + inter-segment).
      \item Assets, liabilities, profit/loss.
      \item CapEx, D\&A, income taxes.
    \end{itemize}
  \item Segments must account for $\geq 75\%$ of external sales.
\end{itemize}

\paragraph{Management Commentary (MD\&A):}
\begin{itemize}
  \item Nature of business, strategy, past performance.
  \item Key risks, relationships, forward-looking statements.
  \item U.S. SEC requires MD\&A to cover:
    \begin{itemize}
      \item Liquidity and capital resources.
      \item Effects of inflation.
      \item Off-balance sheet obligations.
      \item Critical accounting policies.
    \end{itemize}
\end{itemize}

\paragraph{Audit Reports:}
\begin{itemize}
  \item \textbf{Unqualified opinion (clean)}: No material errors.
  \item \textbf{Qualified opinion}: Exceptions exist.
  \item \textbf{Adverse opinion}: Misstated or misleading.
  \item \textbf{Disclaimer}: No opinion possible (scope limitation).
  \item Key Audit Matters (KAMs) / Critical Audit Matters (CAMs) disclose:
    \begin{itemize}
      \item Most significant accounting judgments.
      \item Challenging/subjective areas of audit.
    \end{itemize}
\end{itemize}

\subsubsection*{LOS 29.d: Alternative Reporting Systems and Monitoring}
\begin{itemize}
  \item \textbf{Key issue:} IFRS vs. U.S. GAAP differences can distort cross-border comparisons.
  \item Example differences:
    \begin{itemize}
      \item \textbf{IFRS:} Principle-based, allows revaluation of PP\&E.
      \item \textbf{U.S. GAAP:} Rule-based, historical cost model.
    \end{itemize}
  \item Analysts must track:
    \begin{itemize}
      \item New products and financial innovations.
      \item Emerging accounting standards.
      \item Significant changes in company disclosures.
    \end{itemize}
  \item Sources: IASB, FASB websites, CFA Institute position papers.
\end{itemize}

\subsubsection*{LOS 29.e: Additional Information Sources}
\begin{itemize}
  \item \textbf{Issuer sources:}
    \begin{itemize}
      \item Earnings calls (Q\&A with management).
      \item Ad hoc presentations, press releases.
      \item Direct communications with management / IR.
    \end{itemize}
  \item \textbf{Public third-party sources:}
    \begin{itemize}
      \item Industry reports, whitepapers, trade journals.
      \item Government statistics.
      \item Media and social media.
    \end{itemize}
  \item \textbf{Proprietary third-party sources:}
    \begin{itemize}
      \item Bloomberg, FactSet, Wind.
      \item Analyst/consultancy reports.
    \end{itemize}
  \item \textbf{Proprietary primary research:}
    \begin{itemize}
      \item Commissioned studies.
      \item First-hand product usage.
      \item Technical expert consultations.
    \end{itemize}
\end{itemize}

\subsubsection*{Exhibit: Comparison Table}
\begin{table}[H]
\centering
\resizebox{\textwidth}{!}{%
\footnotesize
\begin{tabular}{|l|p{5cm}|p{6cm}|}
\hline
\textbf{Source} & \textbf{Strengths} & \textbf{Limitations} \\
\hline
Financial Statements & Audited, standardized (IFRS/GAAP) & Backward-looking, limited qualitative info \\
\hline
Management Commentary & Forward-looking, strategic insights & Partially unaudited, potential bias \\
\hline
Footnotes & Detail on assumptions, methods, risks & Complex, requires expertise to interpret \\
\hline
Audit Report & Provides assurance, highlights key issues & Only “reasonable assurance,” not absolute \\
\hline
Earnings Calls / Press Releases & Timely updates, direct access to management & Not audited, selective disclosure risk \\
\hline
Third-party Reports (Bloomberg, FactSet) & Independent analysis, benchmarks & Expensive, potential conflicts of interest \\
\hline
Proprietary Research & Tailored, unique insights & Costly, time-intensive \\
\hline
\end{tabular}}
\caption{Comparison of Information Sources in Financial Analysis}
\end{table}

\subsection*{Module 30.1: Revenue Recognition}

\subsubsection*{LOS 30.a: General Principles of Revenue Recognition}
\begin{itemize}
  \item \textbf{Core principle:} Revenue is recognized when control of goods/services transfers to the customer, not necessarily when cash is received.
  \item \textbf{Accrual basis:} 
    \begin{itemize}
      \item Credit sales → Revenue recognized at sale; Accounts Receivable created.
      \item Cash received in advance → Recorded as \textit{Unearned Revenue (liability)} until goods/services delivered.
      \item Example: Magazine subscription → Cash received upfront, liability recognized, revenue recognized as issues delivered.
    \end{itemize}
  \item \textbf{Revenue is reported net of:}
    \begin{itemize}
      \item Returns
      \item Allowances
      \item Discounts
      \item Warranty provisions
    \end{itemize}
\end{itemize}

\subsubsection*{Five-Step Model under Converged IFRS/US GAAP (IFRS 15 / ASC 606)}
\begin{enumerate}
  \item Identify the \textbf{contract(s)} with a customer.
  \item Identify distinct \textbf{performance obligations}.
  \item Determine the \textbf{transaction price}.
  \item Allocate the transaction price to the performance obligations.
  \item Recognize revenue when/as performance obligations are satisfied.
\end{enumerate}

\paragraph{Definitions:}
\begin{itemize}
  \item \textbf{Contract:} Agreement with enforceable rights/obligations; collectability must be probable (definition of “probable” differs under IFRS vs US GAAP).
  \item \textbf{Performance obligation:} Promise to deliver a distinct good/service.
    \begin{itemize}
      \item Distinct if:
        \begin{enumerate}
          \item Customer can benefit independently or with other resources.
          \item Transfer promise is identifiable separately.
        \end{enumerate}
    \end{itemize}
  \item \textbf{Transaction price:} Expected amount of consideration (fixed or variable).
  \item \textbf{Revenue recognition:} Only when highly probable it won’t be reversed.
  \item \textbf{Indicators of control transfer:} Physical possession, acceptance, transfer of risks/benefits, legal title, right to payment.
\end{itemize}

\subsubsection*{Revenue Recognition in Long-Term Contracts}
\begin{itemize}
  \item Revenue recognized \textbf{over time} if:
    \begin{enumerate}
      \item Customer benefits continuously as supplier performs (e.g., maintenance contracts).
      \item Customer controls asset being created/enhanced (e.g., construction projects).
      \item Asset has no alternative use + supplier has right to payment for completed work.
    \end{enumerate}
  \item Measurement:
    \begin{itemize}
      \item \textbf{Input method:} % of completion costs incurred.
      \item \textbf{Output method:} Engineering milestones, % delivered.
    \end{itemize}
  \item Costs to secure contracts (e.g., sales commissions) must be \textbf{capitalized}.
\end{itemize}

\subsubsection*{Examples (IFRS 15 Applications)}
\paragraph{1. Long-term contract (Warehouse construction)}
\begin{itemize}
  \item Contract price = \$10m; total costs estimated = \$8m.
  \item Year 1: Costs incurred = \$4m (50\% completion) → Revenue recognized = 0.5 × \$10m = \$5m.
  \item Year 2: Costs incurred additional \$2m → Cumulative costs = \$6m (75\% completion).  
  Revenue to date = 0.75 × \$10m = \$7.5m.  
  Revenue recognized in Year 2 = \$7.5m – \$5m = \$2.5m.
  \item Equivalent to \textbf{Percentage-of-Completion Method}.
\end{itemize}

\paragraph{2. Acting as an Agent (Travel Agent)}
\begin{itemize}
  \item Ticket price = \$10,000.  
  \item Commission = \$1,000 (no credit or inventory risk).
  \item Revenue recognized = \$1,000 (net).  
  \item If treated as principal → Revenue = \$10,000, Expense = \$9,000, GP = \$1,000.  
  \item \textbf{Gross profit margin differences:}
    \begin{itemize}
      \item As principal: \( \frac{1,000}{10,000} = 10\% \).
      \item As agent: \( \frac{1,000}{1,000} = 100\% \).
    \end{itemize}
\end{itemize}

\paragraph{3. Franchising and Licensing (Fast Food Chain)}
\begin{itemize}
  \item Revenue categories:
    \begin{enumerate}
      \item Company-owned restaurants.
      \item Franchise royalties \& fees (deferred then amortized over franchise term).
      \item Supplies to franchisees (equipment, food).
    \end{enumerate}
  \item Royalties (e.g., 2\% turnover) recognized when payable.
\end{itemize}

\paragraph{4. Service vs License (Software Supplier)}
\begin{itemize}
  \item \textbf{Case A: License with continuous updates.}  
  Revenue recognized over contract life.
  \item \textbf{Case B: License “as is”.}  
  Revenue recognized at outset; updates covered in separate contract.
  \item \textbf{Cloud service (SaaS).}  
  Revenue recognized over subscription life (service).
\end{itemize}

\paragraph{5. Bill-and-Hold Agreements}
\begin{itemize}
  \item Customer pays ahead of shipping; normally → deferred revenue.
  \item Revenue may be recognized before delivery if:
    \begin{itemize}
      \item Customer requests arrangement.
      \item Goods are identified as belonging to customer.
      \item Goods are complete and ready to ship.
      \item Supplier cannot redirect goods.
    \end{itemize}
\end{itemize}

\subsubsection*{Required Disclosures (IFRS 15 / ASC 606)}
\begin{itemize}
  \item Disaggregation of revenue (by product/service category).
  \item Assets \& liabilities from contracts (balances, changes).
  \item Outstanding performance obligations + allocated transaction prices.
  \item Management judgments on timing/amount of revenue.
\end{itemize}

\subsubsection*{Exhibit: Examples of Revenue Recognition}
\begin{table}[H]
\centering
\resizebox{\textwidth}{!}{%
\footnotesize
\begin{tabular}{|l|p{5cm}|p{5cm}|}
\hline
\textbf{Scenario} & \textbf{Revenue Recognition} & \textbf{Implications for Analysis} \\
\hline
Credit Sale & Recognized at sale (A/R created) & Cash flow timing differs from revenue; analysts adjust for working capital. \\
\hline
Advance Payment (Magazine subscription) & Initially liability (unearned revenue); recognized as delivered & Liability inflates until service performed. \\
\hline
Long-term Contract & Over time using input/output methods & Smooths revenue; requires estimate reliability. \\
\hline
Agent vs Principal & Agent → Net revenue (commission only). Principal → Gross revenue & Gross margin ratios differ; important for comparability. \\
\hline
Franchise Fees \& Royalties & Fees deferred, amortized; royalties when payable & Analysts separate recurring vs one-time revenue streams. \\
\hline
Software License vs SaaS & License revenue upfront vs over contract term & Recognition timing significantly affects earnings profile. \\
\hline
Bill-and-Hold & Recognize if customer controls goods & May accelerate revenue; analysts should check substance. \\
\hline
\end{tabular}}
\caption{Revenue Recognition Scenarios and Implications}
\end{table}

\subsection*{Module 30.2: Expense Recognition}

\subsubsection*{LOS 30.b: General Principles of Expense Recognition}
\begin{itemize}
  \item \textbf{Definition (IASB):} Expenses = decreases in economic benefits during an accounting period in the form of:
    \begin{itemize}
      \item Outflows or depletions of assets
      \item Increases in liabilities
      \item Resulting in decreases in equity (other than distributions to owners)
    \end{itemize}
  \item \textbf{Accrual vs Cash Basis:}
    \begin{itemize}
      \item \textbf{Cash basis:} Expense when paid.
      \item \textbf{Accrual basis:} Expense when economic benefit is consumed.
    \end{itemize}
  \item \textbf{Three recognition methods:}
    \begin{enumerate}
      \item \textbf{Matching principle:} Match expense with revenue generated (e.g., COGS, warranty provisions).
      \item \textbf{Capitalization:} Record as asset → amortized/depreciated as benefits consumed.
      \item \textbf{Expensing as incurred:} Period costs (admin, rent, utilities).
    \end{enumerate}
  \item \textbf{Conservatism vs Aggressiveness:}
    \begin{itemize}
      \item Expensing earlier = conservative.
      \item Deferring via capitalization = aggressive.
    \end{itemize}
\end{itemize}

\subsubsection*{Example: Matching Principle with Inventory}
\begin{itemize}
  \item Firm sells 100 units during the year.
  \item Beginning inventory = 20 units @ \$400 total.
  \item Purchases = 90 units (various costs). Units available = 110.
  \item Ending inventory = 10 units (8 from most recent purchase, 2 from prior).
  \item \textbf{Matching:} Remove 10 units from COGS → report them as inventory (asset).
  \item Ensures COGS = cost of 100 units sold.
\end{itemize}

\paragraph{Note:} If exact identification is not possible → use cost flow methods:
\begin{enumerate}
  \item FIFO (First-in, First-out)
  \item LIFO (Last-in, First-out)
  \item Weighted Average Cost
\end{enumerate}

\subsubsection*{Capitalization vs Expensing}
\begin{itemize}
  \item \textbf{Capitalization:} 
    \begin{itemize}
      \item Expected future economic benefit → recorded as asset.
      \item Cost spread via depreciation, amortization, or depletion.
      \item Land and indefinite-life intangibles (goodwill) not amortized.
    \end{itemize}
  \item \textbf{Expensing:}
    \begin{itemize}
      \item No future benefit or highly uncertain → expense immediately.
      \item Reduces current pretax income fully in period incurred.
    \end{itemize}
  \item \textbf{Subsequent expenditures:}
    \begin{itemize}
      \item \textbf{Extend life/increase benefits} → capitalize.
      \item \textbf{Maintenance/repairs} → expense.
    \end{itemize}
\end{itemize}

\subsubsection*{Example: Northwood Equipment}
\begin{itemize}
  \item Equipment cost = \$250,000 (incl. freight + taxes).
  \item Installation = \$10,000 → capitalize.
  \item Training = \$7,500 → expense (benefits employees, not asset).
  \item Repairs \& maintenance = \$35,000 → expense.
  \item Motor rebuild = \$85,000 → capitalize (extends life).
\end{itemize}

\subsubsection*{Example: Chair Ltd. (Impact of Capitalization vs Expensing)}
\begin{itemize}
  \item Equipment cost = £12,000, useful life = 4 years, straight-line depreciation.
  \item Annual revenue = £30,000, operating margin = 40\%, tax = 30\%.
\end{itemize}

\paragraph{Impacts:}
\begin{itemize}
  \item \textbf{Income Statement:}
    \begin{itemize}
      \item Capitalization: Expense spread (£3,000/year depreciation).
      \item Expensing: Entire £12,000 in Year 1.
      \item Result: Expensing = lower NI in Year 1, higher NI in later years. Capitalization smooths earnings.
    \end{itemize}
  \item \textbf{Balance Sheet:}
    \begin{itemize}
      \item Capitalization: Higher assets (equipment net of depreciation), higher retained earnings.
      \item Expensing: No asset recorded → lower equity in early years.
    \end{itemize}
  \item \textbf{Cash Flow Statement:}
    \begin{itemize}
      \item Capitalization: Cash outflow → investing activities.
      \item Expensing: Cash outflow → operating activities.
      \item Expensing gives full tax benefit upfront, capitalization spreads it.
    \end{itemize}
  \item \textbf{Ratios:}
    \begin{itemize}
      \item Asset turnover = lower if capitalized (assets higher).
      \item Net profit margin = higher in Year 1 if capitalized.
      \item ROE = higher in Year 1 if capitalized, lower in later years.
    \end{itemize}
\end{itemize}

\subsubsection*{Capitalized Interest}
\begin{itemize}
  \item When firm builds asset for own use or resale → interest during construction is capitalized.
  \item Treatment:
    \begin{itemize}
      \item Included in asset cost.
      \item Expensed later via depreciation (if held for use) or COGS (if held for sale).
    \end{itemize}
  \item Cash flow effect:
    \begin{itemize}
      \item Capitalized interest → investing outflow.
      \item Expensed interest → operating outflow (GAAP) or operating/financing (IFRS).
    \end{itemize}
  \item \textbf{Analyst adjustment:} Add capitalized interest back to interest expense for solvency ratios.
\end{itemize}

\paragraph{Example: Willock AG}
\begin{itemize}
  \item EBIT = €160m, reported interest expense = €80m.
  \item €20m capitalized, €10m depreciation from prior capitalized interest.
  \item Adjusted EBIT = €160m + €10m = €170m.  
  \item Adjusted interest = €80m + €20m = €100m.  
  \item Interest coverage = \( \frac{170}{100} = 1.7 \) (vs reported \( \frac{160}{80} = 2.0 \)).
\end{itemize}

\subsubsection*{R\&D Costs}
\begin{itemize}
  \item \textbf{IFRS:}
    \begin{itemize}
      \item Research costs → expensed.
      \item Development costs → capitalized if criteria met (e.g., feasibility, intent to use/sell).
    \end{itemize}
  \item \textbf{U.S. GAAP:}
    \begin{itemize}
      \item R\&D → expensed.
      \item Software development: expensed until technological feasibility, then capitalized.
    \end{itemize}
  \item \textbf{Analyst adjustment:}  
    \begin{itemize}
      \item Expense capitalized development costs for comparability.
      \item Remove amortization of past capitalized costs.
      \item Adjust CFO downward (include costs in operations).
    \end{itemize}
\end{itemize}

\subsubsection*{Bad Debt \& Warranty Expense Recognition}
\begin{itemize}
  \item Matching principle requires recognition \textbf{at time of sale}.
  \item Estimates involved → risk of earnings management.
  \item Analyst checks:
    \begin{itemize}
      \item Compare to peers (e.g., unusually low warranty expense).
      \item Assess whether estimate changes reflect real improvements or manipulation.
    \end{itemize}
\end{itemize}

\subsubsection*{Exhibit: Capitalization vs Expensing – Financial Statement Effects}
\begin{table}[H]
\centering
\footnotesize
\begin{tabular}{|l|p{5cm}|p{5cm}|}
\hline
\textbf{Aspect} & \textbf{Capitalization} & \textbf{Expensing} \\
\hline
Income Statement & Spreads cost over asset life (depreciation) & Entire cost in Year 1 \\
\hline
Balance Sheet & Higher assets (PP\&E), higher equity (retained earnings) & No asset, lower equity early \\
\hline
Cash Flow Statement & Outflow under investing activities & Outflow under operating activities \\
\hline
Tax Effect & Tax benefit spread over years & Immediate tax benefit in Year 1 \\
\hline
Ratios & Lower asset turnover, smoother NI, higher margins in Year 1 & Higher turnover, volatile NI, margins lower in Year 1 \\
\hline
Earnings Profile & Smooth, less volatile & Front-loaded cost, volatile earnings \\
\hline
\end{tabular}
\caption{Comparison of Capitalization vs Expensing}
\end{table}

\subsection*{Module 30.3: Nonrecurring Items}

\subsubsection*{LOS 30.c: Financial Reporting Treatment and Analysis of Nonrecurring Items}

\subsubsection*{1. Unusual or Infrequent Items}
\begin{itemize}
  \item \textbf{Definition:} Events that are unusual in nature or infrequent in occurrence, and \textbf{material} enough to affect decisions.
  \item \textbf{Examples:}
    \begin{itemize}
      \item Gains/losses from sale of assets or business units (not part of ordinary operations).
      \item Impairments, write-offs, write-downs.
      \item Restructuring costs.
    \end{itemize}
  \item \textbf{Reporting:}
    \begin{itemize}
      \item Included in \textit{income from continuing operations}.
      \item Reported \textbf{before tax}.
    \end{itemize}
  \item \textbf{Analyst consideration:}
    \begin{itemize}
      \item Should assess whether such items are truly nonrecurring.
      \item Some firms report “one-off” charges frequently → signals recurring issues.
      \item Adjust forecasts by excluding these from “core” earnings if justified.
    \end{itemize}
\end{itemize}

\subsubsection*{2. Discontinued Operations}
\begin{itemize}
  \item \textbf{Definition:} Component of business that is physically and operationally distinct, and management has decided to dispose of.
  \item \textbf{Phases:}
    \begin{itemize}
      \item \textbf{Measurement date:} When formal plan to dispose is announced.
      \item \textbf{Phaseout period:} Between measurement date and disposal.
    \end{itemize}
  \item \textbf{Accounting treatment:}
    \begin{itemize}
      \item Reported separately in income statement, \textbf{net of tax}, after continuing operations.
      \item Prior-period statements restated for comparability.
      \item Losses during phaseout and estimated loss on sale recognized at measurement date.
      \item Gains only recognized when disposal completed.
    \end{itemize}
  \item \textbf{Analyst treatment:}
    \begin{itemize}
      \item Exclude discontinued operations from future earnings forecasts.
      \item Consider disposal impact on firm’s future cash flows and structure.
    \end{itemize}
\end{itemize}

\subsubsection*{3. Changes in Accounting Policies, Estimates, and Errors}
\begin{itemize}
  \item \textbf{Types of accounting changes:}
    \begin{enumerate}
      \item \textbf{Accounting policy changes:} (e.g., inventory method, capitalization vs expensing).  
      \begin{itemize}
        \item Require \textbf{retrospective application} unless impractical.  
        \item Enhances comparability across periods.  
        \item Example: IFRS 15 revenue recognition → allowed modified retrospective application (adjust cumulative balances, no restatement of prior periods).  
      \end{itemize}
      \item \textbf{Accounting estimate changes:} (e.g., useful life of asset, bad debt allowance).  
      \begin{itemize}
        \item Require \textbf{prospective application}.  
        \item Do not affect prior periods; only future results.  
        \item Do not directly affect cash flows.  
      \end{itemize}
      \item \textbf{Corrections of errors / prior-period adjustments:} (e.g., correcting from non-GAAP to GAAP method).  
      \begin{itemize}
        \item Require \textbf{retrospective restatement}.  
        \item Disclosure required (nature of error and impact).  
        \item May indicate weak internal controls.  
      \end{itemize}
    \end{enumerate}
  \item \textbf{Analyst adjustments:}  
    \begin{itemize}
      \item Scrutinize policy changes for earnings management.  
      \item Adjust comparability when firms adopt different policies.  
      \item For estimates, determine whether changes reflect genuine new information or manipulation.  
    \end{itemize}
\end{itemize}

\subsubsection*{4. Changes in Scope and Exchange Rates}
\begin{itemize}
  \item \textbf{Changes in scope:} Acquisitions, mergers, or disposals → affect comparability of financial statements before vs after.
  \item \textbf{Exchange rates:} Affect overseas subsidiaries’ revenues, expenses, and assets when translated to reporting currency.
  \item \textbf{Disclosure:} Not explicitly required, but analysts should monitor.
\end{itemize}

\subsubsection*{Exhibit: Summary of Nonrecurring Items Treatment}
\begin{table}[H]
\centering
\resizebox{\textwidth}{!}{%
\footnotesize
\begin{tabular}{|l|p{5cm}|p{5cm}|}
\hline
\textbf{Item} & \textbf{Reporting Treatment} & \textbf{Analyst Implications} \\
\hline
Unusual / Infrequent Items & Included in continuing operations (before tax) & Adjust if not truly one-off; recurring charges reduce quality of earnings \\
\hline
Discontinued Operations & Separate line, net of tax, after continuing operations; prior periods restated & Exclude from future earnings forecasts; assess disposal impact on cash flows \\
\hline
Change in Accounting Policy & Retrospective application (unless impractical) & Improves comparability, but check for management bias \\
\hline
Change in Accounting Estimate & Prospective application & No restatement; assess impact on future earnings \\
\hline
Correction of Errors (Prior-period Adjustment) & Retrospective restatement; disclosure required & May signal weak internal controls; usually no cash flow effect \\
\hline
Change in Scope (M\&A) & Not separately disclosed & Reduces comparability; analyst should adjust historical trends \\
\hline
Exchange Rate Effects & Not separately disclosed & Affects revenues/assets of foreign subsidiaries; adjust for FX volatility \\
\hline
\end{tabular}}
\caption{Nonrecurring Items – Reporting Treatment and Analyst Considerations}
\end{table}

\subsubsection*{Key Analytical Insights}
\begin{itemize}
  \item Nonrecurring items distort earnings comparability.  
  \item Analysts should focus on \textbf{income from continuing operations} as basis for forecasting.  
  \item Frequent “one-off” losses may reveal poor operations or aggressive accounting.  
  \item Restatements (policy or error corrections) improve comparability but highlight potential internal control issues.  
  \item Scope and FX changes → require careful normalization in trend analysis.  
\end{itemize}

\subsection*{Module 30.4: Earnings Per Share (EPS)}

\subsubsection*{LOS 30.d: Basic and Diluted EPS – Principles and Calculations}

\subsubsection*{1. Overview}
\begin{itemize}
  \item EPS = most widely used measure of corporate profitability for publicly traded firms.
  \item EPS is reported only for \textbf{common stock}.
  \item \textbf{Capital structure types:}
    \begin{itemize}
      \item \textbf{Simple:} Only common stock, nonconvertible debt, nonconvertible preferred.  
      $\Rightarrow$ Report only \textbf{basic EPS}.
      \item \textbf{Complex:} Contains potentially dilutive securities (options, warrants, convertible bonds, convertible preferred).  
      $\Rightarrow$ Report both \textbf{basic and diluted EPS}.
    \end{itemize}
\end{itemize}

\subsubsection*{2. Basic EPS}
\paragraph{Formula:}
\[
\text{Basic EPS} = \frac{\text{Net income} - \text{Preferred dividends}}{\text{Weighted average number of common shares outstanding}}
\]

\begin{itemize}
  \item Preferred dividends are subtracted (common shareholders’ claim).  
  \item Common dividends are \textbf{not} subtracted.  
  \item Weighted average shares = shares outstanding adjusted for:
    \begin{itemize}
      \item Issue or repurchase (time-weighted by days/months).
      \item Stock splits/dividends → applied retroactively to beginning of year and prior periods.
    \end{itemize}
\end{itemize}

\paragraph{Example – Weighted Average Shares (Johnson Co.):}
\begin{itemize}
  \item 10,000 shares at start.
  \item April 1: issue 4,000 shares.
  \item July 1: 10\% stock dividend (retroactive adjustment).
  \item Sept 1: repurchase 3,000 shares.
\end{itemize}
\[
\text{Weighted Average Shares} = \text{time-adjusted and dividend-adjusted count (new shares)}
\]

\paragraph{Example – Basic EPS (Johnson Co.):}
\begin{itemize}
  \item Net income = \$10,000.
  \item Preferred dividends = \$1,000.  
  \item Weighted average shares (from above) = used in denominator.  
  \item Cash dividends to common (\$1,750) ignored in EPS.  
  \item \[
  \text{Basic EPS} = \frac{10,000 - 1,000}{\text{Weighted Avg. Shares}}
  \]
\end{itemize}

\subsubsection*{3. Diluted EPS}
\paragraph{Definition:}
\begin{itemize}
  \item Diluted EPS considers effects of all \textbf{potentially dilutive securities}.
  \item \textbf{Dilutive security:} reduces EPS if converted (included).  
  \item \textbf{Antidilutive security:} increases EPS if converted (excluded).
\end{itemize}

\paragraph{Formula:}
\[
\text{Diluted EPS} = \frac{\text{Net income available to common (adjusted)}}{\text{Weighted average shares outstanding + shares from conversion (if dilutive)}}
\]

\paragraph{Adjustments:}
\begin{itemize}
  \item \textbf{Convertible Preferred Stock:} Add back preferred dividends to numerator if dilutive.
  \item \textbf{Convertible Debt:} Add back after-tax interest expense:
  \[
  \text{Adj. Net Income} = \text{Net Income} + \text{Interest} \times (1 - t)
  \]
  \item \textbf{Options/Warrants:} Use Treasury Stock Method:
    \begin{itemize}
      \item Assumes exercise proceeds used to buy back shares at average market price.
      \item Net increase = new shares issued – shares repurchased.
    \end{itemize}
\end{itemize}

\subsubsection*{4. Worked Examples}

\paragraph{Example 1 – Convertible Preferred Stock (ZZZ Corp.)}
\begin{itemize}
  \item Net income = \$4.35m.
  \item Shares outstanding = 2m.
  \item Preferred stock = \$5m par, 7\% dividend, convertible 1.1 shares per \$10 par.
  \item Step 1 – Basic EPS:
  \[
  \text{Basic EPS} = \frac{4.35m - 0.35m}{2m} = 2.00
  \]
  \item Step 2 – Diluted EPS:
    \begin{itemize}
      \item New shares = (5m / 10) × 1.1 = 550,000 shares.
      \item Add back preferred dividends (\$0.35m).  
      \item Diluted EPS:
      \[
      \frac{4.35m}{2.55m} = 1.71
      \]
    \end{itemize}
  \item Since diluted EPS (1.71) < basic (2.00) → dilutive.
\end{itemize}

\paragraph{Example 2 – Convertible Debt (YYY Corp.)}
\begin{itemize}
  \item Net income available = \$2.5m.
  \item Shares outstanding = 1m.
  \item Basic EPS = 2.50.
  \item Convertible bonds = 2,000 bonds × \$1,000 × 5\% = \$100,000 interest.
  \item Tax rate = 30\%.
  \item Step 1 – Extra shares if converted:
  \[
  2,000 \times 120 = 240,000
  \]
  \item Step 2 – Add back after-tax interest:
  \[
  100,000 \times (1 - 0.30) = 70,000
  \]
  \item Step 3 – Diluted EPS:
  \[
  \frac{2.5m + 70,000}{1m + 240,000} = 2.07
  \]
  \item Since 2.07 < 2.50 → dilutive.
\end{itemize}

\paragraph{Example 3 – Stock Options/Warrants (XXX Corp.)}
\begin{itemize}
  \item Net income = \$1.2m.
  \item Shares = 500,000.
  \item Basic EPS = 2.40.
  \item Options outstanding = 100,000 @ \$15 exercise price.
  \item Average market price = \$20.
  \item Step 1 – Shares issued if exercised = 100,000.
  \item Step 2 – Proceeds = 100,000 × 15 = 1.5m.
  \item Step 3 – Shares repurchased = 1.5m / 20 = 75,000.
  \item Step 4 – Net new shares = 25,000.
  \item Step 5 – Diluted EPS:
  \[
  \frac{1.2m}{500,000 + 25,000} = 2.29
  \]
  \item Options are dilutive since 2.29 < 2.40.
\end{itemize}

\subsubsection*{5. Summary Table – Basic vs Diluted EPS}
\begin{table}[H]
\centering
\footnotesize
\begin{tabular}{|l|p{5cm}|p{6cm}|}
\hline
\textbf{Case} & \textbf{Numerator Adjustment} & \textbf{Denominator Adjustment} \\
\hline
Basic EPS & Net income – preferred dividends & Weighted average shares outstanding \\
\hline
Convertible Preferred & Add back preferred dividends if dilutive & Add new shares if converted \\
\hline
Convertible Debt & Add back interest × (1 – tax) if dilutive & Add new shares if converted \\
\hline
Options/Warrants & No adjustment & Treasury stock method: new shares – repurchased shares \\
\hline
Antidilutive Securities & Excluded & Excluded (ignored in denominator) \\
\hline
\end{tabular}
\caption{Basic vs Diluted EPS Adjustments}
\end{table}

\subsubsection*{6. Key Analyst Considerations}
\begin{itemize}
  \item Always test each potential security separately for dilution.  
  \item Exclude antidilutive securities even if they are convertible.  
  \item Stock splits/dividends → retroactively adjust prior years to ensure comparability.  
  \item Diluted EPS provides more conservative measure of per-share profitability.  
  \item Frequent issuance of dilutive securities = red flag for shareholders (dilution of ownership).  
\end{itemize}

\subsection*{Module 30.5: Ratios and Common-Size Income Statements}

\subsubsection*{LOS 30.e: Evaluate Performance Using Common-Size Income Statements and Ratios}

\subsubsection*{1. Common-Size Income Statements}
\begin{itemize}
  \item \textbf{Definition:} Expresses each line item as a percentage of revenue.  
  \item \textbf{Purpose:}
    \begin{itemize}
      \item Eliminates firm size effect $\Rightarrow$ allows \textbf{time-series} and \textbf{cross-sectional} analysis.
      \item Facilitates comparison across peers and over time.
    \end{itemize}
  \item \textbf{Key points:}
    \begin{itemize}
      \item Reveals structural differences in costs and profitability.
      \item Highlights strategic focus (e.g., high R\&D vs low R\&D firms).
      \item Exception: Income tax is more meaningful as \textbf{percentage of pretax income} = effective tax rate.
    \end{itemize}
\end{itemize}

\subsubsection*{2. Example: North vs South Company}
\paragraph{Absolute Results (in \$):}
\begin{itemize}
  \item North: Revenue = 75,000,000; Gross Profit = 22,500,000; Operating Profit = 7,500,000.
  \item South: Revenue = 3,500,000; Gross Profit = 2,800,000; Operating Profit = 1,575,000.
  \item North larger and higher absolute profit.
\end{itemize}

\paragraph{Common-Size Results (relative \% of revenue):}
\begin{table}[H]
\centering
\footnotesize
\begin{tabular}{|l|c|c|}
\hline
\textbf{Metric} & \textbf{North (\% of Revenue)} & \textbf{South (\% of Revenue)} \\
\hline
Gross Profit Margin & 30\% & 80\% \\
Operating Profit Margin & 10\% & 45\% \\
R\&D Expense & Lower proportion & Higher proportion \\
\hline
\end{tabular}
\caption{Common-Size Income Statement Comparison: North vs South}
\end{table}

\paragraph{Insights:}
\begin{itemize}
  \item South is \textbf{more profitable relatively}, despite smaller size.
  \item High gross margin suggests \textbf{technological differentiation or pricing power}.
  \item Higher R\&D share indicates innovation-driven strategy.
\end{itemize}

\subsubsection*{3. Margin Ratios (Profitability Metrics)}
\paragraph{Formulas:}
\begin{itemize}
  \item \textbf{Gross Profit Margin:}
  \[
  \text{GPM} = \frac{\text{Gross Profit}}{\text{Revenue}} = \frac{\text{Revenue – COGS}}{\text{Revenue}}
  \]
  \item \textbf{Operating Profit Margin:}
  \[
  \text{OPM} = \frac{\text{Operating Profit}}{\text{Revenue}}
  \]
  \item \textbf{Pretax Margin:}
  \[
  \text{Pretax Margin} = \frac{\text{Pretax Accounting Profit}}{\text{Revenue}}
  \]
  \item \textbf{Net Profit Margin:}
  \[
  \text{NPM} = \frac{\text{Net Income}}{\text{Revenue}}
  \]
  \item \textbf{Effective Tax Rate:}
  \[
  \text{ETR} = \frac{\text{Income Tax Expense}}{\text{Pretax Income}}
  \]
\end{itemize}

\paragraph{Interpretation:}
\begin{itemize}
  \item \textbf{Gross Profit Margin (GPM):}
    \begin{itemize}
      \item Indicates ability to cover production costs.
      \item Improved via: raising prices, reducing production costs.
      \item Higher GPM often reflects product differentiation (brand, technology, patents).
    \end{itemize}
  \item \textbf{Operating Profit Margin (OPM):}
    \begin{itemize}
      \item Accounts for operating expenses (R\&D, SG\&A).
      \item Measures efficiency of operations and cost control.
    \end{itemize}
  \item \textbf{Net Profit Margin (NPM):}
    \begin{itemize}
      \item Includes all expenses (interest, tax).
      \item Best measure of bottom-line profitability.
    \end{itemize}
  \item \textbf{Pretax Margin:}
    \begin{itemize}
      \item Useful for comparing firms across different tax jurisdictions.
    \end{itemize}
\end{itemize}

\subsubsection*{4. Example: Ratio Analysis of North vs South}
\begin{table}[H]
\centering
\footnotesize
\begin{tabular}{|l|c|c|c|}
\hline
\textbf{Ratio} & \textbf{Formula} & \textbf{North} & \textbf{South} \\
\hline
Gross Profit Margin & GP / Revenue & 30\% & 80\% \\
\hline
Operating Profit Margin & OP / Revenue & 10\% & 45\% \\
\hline
Net Profit Margin & NI / Revenue & Lower & Higher \\
\hline
R\&D as \% of Revenue & R\&D / Revenue & Low & High \\
\hline
Effective Tax Rate & Tax Expense / Pretax Income & Apply separately & Apply separately \\
\hline
\end{tabular}
\caption{Comparison of Profitability Ratios: North vs South}
\end{table}

\paragraph{Insights:}
\begin{itemize}
  \item North: Economies of scale, but low margins. Strategy: volume-driven.
  \item South: Differentiated products, high pricing power, innovation-focused.
  \item Indicates South may sustain higher profitability despite smaller size.
\end{itemize}

\subsubsection*{5. Key Analyst Considerations}
\begin{itemize}
  \item Common-size analysis reveals \textbf{underlying strategy and structure}, not visible in absolute figures.
  \item Margin ratios should be tracked \textbf{over time} (trend analysis) and \textbf{against peers} (cross-sectional).
  \item Tax effects should be separated using effective tax rate.
  \item High margins may indicate differentiation but may also suggest risk if not sustainable.
  \item Low margins may suggest commoditization, reliance on cost leadership.
\end{itemize}

\subsection*{Module 31.1: Intangible Assets and Marketable Securities}

\subsubsection*{LOS 31.a: Intangible Assets}
\begin{itemize}
  \item \textbf{Definition:} Non-monetary assets lacking physical substance.
  \item \textbf{Types:}
    \begin{itemize}
      \item \textbf{Identifiable:} Can be acquired separately (patents, trademarks, copyrights).
      \item \textbf{Unidentifiable:} Cannot be separated, often indefinite life (e.g., goodwill).
    \end{itemize}
  \item \textbf{IFRS Treatment:}
    \begin{itemize}
      \item Purchased intangibles: cost model or revaluation model (if active market exists).
      \item Internally created intangibles:
        \begin{itemize}
          \item Research costs $\Rightarrow$ expensed.
          \item Development costs $\Rightarrow$ capitalized if criteria met (technical feasibility, resources, market, intention).
        \end{itemize}
    \end{itemize}
  \item \textbf{U.S. GAAP Treatment:}
    \begin{itemize}
      \item Only cost model allowed.
      \item Internally created intangibles (R\&D) generally expensed (except certain legal costs).
    \end{itemize}
  \item \textbf{Subsequent Treatment:}
    \begin{itemize}
      \item Finite-lived $\Rightarrow$ amortized + impairment testing.
      \item Indefinite-lived $\Rightarrow$ no amortization, annual impairment test.
    \end{itemize}
  \item \textbf{Costs Always Expensed (IFRS \& GAAP):} start-up, training, admin, advertising, relocation, termination.
\end{itemize}

\paragraph{Example: Lowe S.A. R\&D Projects (IFRS)}
\begin{itemize}
  \item Project 1: Hydrogen fuel cells (research stage) $\Rightarrow$ costs expensed.
  \item Project 2: Catalytic converter (development stage, prototype exists, resources and market available) $\Rightarrow$ costs capitalized.
  \[
  \text{Capitalized Costs} = 120 + 60 + 30 = €210 \text{ million}
  \]
  \item Admin costs $\Rightarrow$ expensed.
\end{itemize}

\subsubsection*{LOS 31.b: Goodwill}
\begin{itemize}
  \item \textbf{Definition:} Excess purchase price over fair value of net assets in an acquisition.
  \[
  \text{Goodwill} = \text{Purchase Price} - \text{Fair Value of Net Assets}
  \]
  \item \textbf{Key Points:}
    \begin{itemize}
      \item Created only in acquisitions (not internally generated).
      \item Indefinite life $\Rightarrow$ not amortized, but tested annually for impairment.
      \item Impairment recognized as loss (no cash flow impact).
    \end{itemize}
  \item \textbf{Special Case:} If purchase price $<$ fair value $\Rightarrow$ gain recognized in income statement.
  \item \textbf{Analyst Considerations:}
    \begin{itemize}
      \item Some analysts exclude goodwill from balance sheets (improves comparability).
      \item Goodwill impairments can signal poor acquisitions.
      \item Firms may allocate more cost to goodwill (not amortized) vs assets (which depreciate), inflating net income.
    \end{itemize}
\end{itemize}

\paragraph{Types of Goodwill:}
\begin{itemize}
  \item \textbf{Accounting Goodwill:} Arises from past acquisitions.
  \item \textbf{Economic Goodwill:} PV of expected future excess returns.
\end{itemize}

\subsubsection*{LOS 31.c: Financial Instruments}
\begin{itemize}
  \item \textbf{Definition:} Contracts that create both a financial asset (for one party) and a liability/equity instrument (for the other).
  \item \textbf{Examples (Assets):} investment securities, derivatives, loans, receivables.
  \item \textbf{Measurement Bases:}
    \begin{itemize}
      \item \textbf{Historical Cost:} e.g., unquoted equity investments, loans.
      \item \textbf{Amortized Cost:} held-to-maturity (GAAP), debt securities with intent to hold to maturity.
      \item \textbf{Fair Value:} trading securities, available-for-sale (GAAP), derivatives.
    \end{itemize}
\end{itemize}

\paragraph{U.S. GAAP Classification:}
\begin{table}[H]
\centering
\footnotesize
\begin{tabular}{|l|l|l|}
\hline
\textbf{Category} & \textbf{Measurement} & \textbf{Income Statement Impact} \\
\hline
Held-to-Maturity & Amortized cost & Interest income only \\
Trading Securities & Fair value & Unrealized gains/losses + income \\
Available-for-Sale & Fair value & Realized gains/losses + income; \\
 & & Unrealized gains/losses $\to$ OCI \\
\hline
\end{tabular}
\caption{Financial Assets under U.S. GAAP}
\end{table}

\paragraph{Example: Triple D Bond (\$1M, 6\%, decline by \$20k)}
\begin{itemize}
  \item \textbf{Held-to-Maturity:} Report \$1,000,000, interest income \$60,000.
  \item \textbf{Trading:} Report \$980,000, interest \$60,000 + unrealized loss \$20,000.
  \item \textbf{Available-for-Sale:} Report \$980,000, interest \$60,000 in IS, \$20,000 unrealized loss in OCI.
\end{itemize}

\paragraph{IFRS Classification:}
\begin{itemize}
  \item Amortized Cost (hold to collect).
  \item Fair Value through OCI (collect + sell).
  \item Fair Value through P\&L (trading/default).
  \item Key Differences: Equity can be FVOCI under IFRS (choice at purchase), not under U.S. GAAP.
\end{itemize}

\subsubsection*{LOS 31.d: Non-Current Liabilities}
\begin{itemize}
  \item \textbf{Examples:} Bank loans, notes payable, bonds payable, some derivatives.
  \item \textbf{Measurement:}
    \begin{itemize}
      \item Usually reported at amortized cost:
     \[
\begin{aligned}
\text{Amortized Cost} &= \text{Issue Price} \\
&- \text{Principal Payments} \\
&+ \text{Amortized Discount} \\
&- \text{Amortized Premium}
\end{aligned}
\]
      \item Premium/discount amortized into interest expense.
      \item Liability approaches face value at maturity.
      \item Some liabilities (e.g., trading, derivatives, hedged) measured at fair value.
    \end{itemize}
  \item \textbf{Deferred Tax Liabilities (DTL):}
    \begin{itemize}
      \item Taxes payable in future due to timing differences between financial vs tax reporting.
      \item Created when:
        \begin{itemize}
          \item Tax deductions occur before expense recognition (e.g., accelerated tax depreciation).
          \item Revenues recognized before taxable (e.g., subsidiary earnings).
        \end{itemize}
      \item Eventually reverse when taxes are paid.
    \end{itemize}
\end{itemize}

\subsection*{Module 32.1: Cash Flow Introduction and Direct Method CFO}

\textbf{LOS 32.a: How the Cash Flow Statement Links to Income Statement and Balance Sheet}

\begin{itemize}
    \item The \textbf{Cash Flow Statement (CFS)} provides insights not visible in the Income Statement (IS) and Balance Sheet (BS):
    \begin{itemize}
        \item Cash receipts and cash payments during the period.
        \item Classification into: Operating (CFO), Investing (CFI), Financing (CFF).
        \item Quality of earnings: accrual vs. cash-backed profits.
    \end{itemize}
    
    \item \textbf{Uses of CFS by analysts:}
    \begin{itemize}
        \item Liquidity $\rightarrow$ ability to sustain business with operating cash.
        \item Solvency $\rightarrow$ ability to meet long-term obligations.
        \item Financial flexibility $\rightarrow$ ability to fund growth or meet surprises.
    \end{itemize}

    \item \textbf{Link to Financial Statements:}
    \begin{itemize}
        \item IS = performance between two BS dates (flow statement).
        \item CFS reconciles change in cash between beginning and end of BS period.
        \item Operating Activities $\leftrightarrow$ Current Assets \& Liabilities. \\
              Investing Activities $\leftrightarrow$ Noncurrent Assets. \\
              Financing Activities $\leftrightarrow$ Noncurrent Liabilities \& Equity.
    \end{itemize}
\end{itemize}

\bigskip
\textbf{Example: Accounts Receivable (AR)}

\[
\text{Ending AR} = \text{Beginning AR} + \text{Sales} - \text{Cash Collections}
\]

\[
\text{Cash Collections} = \text{Sales} - (\text{Ending AR} - \text{Beginning AR})
\]

\begin{exampleblock}{Numerical Example}
Beginning AR = €10,000, Ending AR = €15,000, Sales = €68,000  
\[
\text{Cash Collections} = 68{,}000 - (15{,}000 - 10{,}000) = 63{,}000
\]
\end{exampleblock}

- An \textbf{increase in AR} $\rightarrow$ use of cash.  
- An \textbf{increase in Unearned Revenue} $\rightarrow$ source of cash.

\bigskip
\textbf{General Rules (Sources vs Uses of Cash):}
\begin{itemize}
    \item Increase in an Asset $\rightarrow$ Use of Cash (−).
    \item Decrease in an Asset $\rightarrow$ Source of Cash (+).
    \item Increase in a Liability $\rightarrow$ Source of Cash (+).
    \item Decrease in a Liability $\rightarrow$ Use of Cash (−).
\end{itemize}

\bigskip
\textbf{LOS 32.b: Direct vs. Indirect CFO Presentation}

\begin{itemize}
    \item CFO can be presented using:
    \begin{enumerate}
        \item \textbf{Direct Method:} Lists actual cash inflows/outflows.  
        \item \textbf{Indirect Method:} Adjusts net income for non-cash items and accruals.  
    \end{enumerate}
    \item CFI and CFF are presented the same under both methods.  
\end{itemize}

\bigskip
\textbf{Direct Method for CFO: Step-by-Step}

\begin{enumerate}
    \item Start with Revenue (top of IS).
    \item Adjust for changes in related BS accounts:
    \begin{itemize}
        \item Subtract increase in asset (use of cash).
        \item Add decrease in asset (source of cash).
        \item Add increase in liability (source of cash).
        \item Subtract decrease in liability (use of cash).
    \end{itemize}
    \item Treat expenses as negative values before adjustments.
    \item Ignore non-cash items (e.g., depreciation, unrealized gains/losses).
    \item Sum adjusted inflows/outflows = CFO.
\end{enumerate}

\bigskip
\textbf{Components of Direct Method CFO:}
\begin{itemize}
    \item Cash collected from customers.
    \item Cash paid to suppliers (COGS adjusted for Inventory \& AP).
    \item Cash operating expenses (e.g., wages, rent).
    \item Cash interest paid.
    \item Cash taxes paid.
\end{itemize}

\bigskip
\textbf{Example: Direct Method CFO Calculation}

\begin{table}[H]
\centering
\caption*{Income Statement (20X7)}
\begin{tabular}{|l|r|}
\hline
Sales Revenue & 500,000 \\
COGS & (300,000) \\
Depreciation & (20,000) \\
Operating Expenses & (100,000) \\
Interest Expense & (10,000) \\
Tax Expense & (20,000) \\
\hline
Net Income & 50,000 \\
\hline
\end{tabular}
\end{table}

\begin{table}[H]
\centering
\caption*{Balance Sheet Changes (20X6 $\rightarrow$ 20X7)}
\begin{tabular}{|l|r|}
\hline
Accounts Receivable & +15,000 \\
Inventory & +5,000 \\
Accounts Payable & +8,000 \\
Taxes Payable & +3,000 \\
\hline
\end{tabular}
\end{table}

\bigskip
\textbf{Step-by-Step CFO:}

\begin{itemize}
    \item Cash Collected from Customers:
    \[
    500{,}000 - 15{,}000 = 485{,}000
    \]
    \item Cash Paid to Suppliers (COGS adj.):
    \[
    300{,}000 + 5{,}000 - 8{,}000 = 297{,}000
    \]
    \item Cash Operating Expenses:
    \[
    100{,}000 \quad (\text{no adjustment assumed})
    \]
    \item Cash Interest Paid:
    \[
    10{,}000 \quad (\text{no adjustment assumed})
    \]
    \item Cash Taxes Paid:
    \[
    20{,}000 - 3{,}000 = 17{,}000
    \]
\end{itemize}

\[
\text{CFO} = 485{,}000 - 297{,}000 - 100{,}000 - 10{,}000 - 17{,}000 = 61{,}000
\]

\bigskip
\textbf{Interpretation:}
\begin{itemize}
    \item CFO = 61,000 (positive cash flow from operations).
    \item Quality of earnings is high if CFO $\geq$ Net Income (50,000).
    \item Indicates earnings are backed by real cash collections.
\end{itemize}

\subsection*{Module 32.2: Indirect Method CFO}

\subsubsection*{LOS 32.b: Prepare and interpret CFO using the indirect method}

\paragraph{Core idea}
\begin{itemize}
  \item Start from \textbf{Net Income (NI)} and reconcile to \textbf{Cash Flow from Operations (CFO)} by:
  \begin{enumerate}[label=\arabic*.]
    \item \textbf{Adding back} noncash charges and \textbf{removing} non-operating gains/losses that flowed through NI.
    \item \textbf{Adjusting for working capital} changes (operating current assets and operating current liabilities).
  \end{enumerate}
\end{itemize}

\paragraph{Bridge formula}
\[
\boxed{\;\; \text{CFO} \;=\; \text{NI} \;+\; \text{NCC} \;-\; \text{WCINV} \;\;}
\]
\begin{itemize}
  \item \textbf{NCC (Noncash Charges):} Items in NI with no current-period cash effect (e.g., depreciation). Gains reduce NI without cash classification in CFO; losses increase NI similarly. Under the indirect method: \textit{add back charges}, \textit{subtract gains}, \textit{add losses}.
  \item \textbf{WCINV (Working Capital Investment):} Net increase in \emph{noncash} operating current assets minus the net increase in \emph{operating} current liabilities.
\end{itemize}

\paragraph{What counts as operating for WCINV}
\begin{itemize}
  \item \textbf{Include (typical):} Accounts receivable, inventory, prepaid expenses, other operating CAs; Accounts payable, accrued expenses, taxes payable, unearned (deferred) revenue.
  \item \textbf{Exclude:} Cash and cash equivalents; short-term \emph{interest-bearing} debt and dividends payable (CFF); short-term investments (except trading securities which are CFO under U.S. GAAP).
\end{itemize}

\paragraph{Sign rules for working capital adjustments}
\begin{table}[H]
\centering
\resizebox{\textwidth}{!}{%
\footnotesize
\begin{tabular}{|l|c|p{7cm}|}
\hline
\textbf{Account change} & \textbf{CFO effect} & \textbf{Logic} \\
\hline
$\uparrow$ Operating current asset (e.g., AR, Inventory) & $-$ & Cash not yet received or tied up in inventory \\
$\downarrow$ Operating current asset & $+$ & Release of cash (collection or inventory run-down) \\
$\uparrow$ Operating current liability (e.g., AP, Accrued, Taxes Payable) & $+$ & Paying later preserves cash \\
$\downarrow$ Operating current liability & $-$ & Paying earlier uses cash \\
\hline
\end{tabular}}
\caption{Working capital adjustments under the indirect method}
\end{table}

\paragraph{Typical noncash charges, gains, and losses (NCC bucket)}
\begin{table}[H]
\centering
\footnotesize
\begin{tabular}{|l|c|}
\hline
\textbf{Income statement item} & \textbf{Indirect CFO adjustment} \\
\hline
Depreciation and amortization & Add back \\
Impairment losses, write-downs & Add back \\
Stock-based compensation expense & Add back \\
Bad-debt expense (allowance build) & Add back \\
Deferred tax expense (net) & Add back (if noncash) \\
Unrealized FX losses (noncash) & Add back \\
Gains on sale of PPE/investments (CFI item) & Subtract \\
Losses on sale of PPE/investments (CFI item) & Add back \\
\hline
\end{tabular}
\caption{Common noncash and non-operating items in the NI-to-CFO bridge}
\end{table}

\paragraph{Step-by-step algorithm (indirect method)}
\begin{enumerate}[label=\arabic*.]
  \item Begin with \textbf{Net Income}.
  \item \textbf{Add back} noncash charges and \textbf{remove} non-operating gains/losses: $+\,$Depreciation/Amortization, $+\,$Impairments, $-\,$Gains on asset sales, $+\,$Losses on asset sales, $+\,$Deferred tax expense, etc.
  \item Adjust for \textbf{working capital}:
  \begin{itemize}
    \item Subtract increases / add decreases in operating current \emph{assets}.
    \item Add increases / subtract decreases in operating current \emph{liabilities}.
  \end{itemize}
  \item Resulting total is \textbf{CFO}.
\end{enumerate}

\subsubsection*{Worked example (reconciling to the 32.1 direct-method numbers)}

\textbf{Given (from 32.1):}
\begin{itemize}
  \item Income Statement (20X7): NI = \$50{,}000; Depreciation = \$20{,}000; no gains/losses assumed.
  \item Balance sheet changes (20X6 $\rightarrow$ 20X7): AR = $+15{,}000$; Inventory = $+5{,}000$; AP = $+8{,}000$; Taxes Payable = $+3{,}000$.
\end{itemize}

\textbf{Indirect method CFO:}
\[
\begin{aligned}
\text{CFO} &= \text{NI} \;+\; \text{NCC} \;+\;(\Delta \text{AP}) \;+\;(\Delta \text{Taxes Payable}) \;-\;(\Delta \text{AR}) \;-\;(\Delta \text{Inventory}) \\
           &= 50{,}000 \;+\; 20{,}000 \;+\; 8{,}000 \;+\; 3{,}000 \;-\; 15{,}000 \;-\; 5{,}000 \\
           &= \boxed{61{,}000}
\end{aligned}
\]
This matches the \textbf{direct method CFO} computed in Module 32.1, as required.

\subsubsection*{Aggregate working capital formulation}
\[
\text{WCINV} \;=\; \Delta(\text{AR} + \text{Inventory} + \text{Prepaids} + \text{Other Op. CAs}) \;-\; \Delta(\text{AP} + \text{Accrued} + \text{Taxes Pay.} + \text{Unearned})
\]
\[
\text{CFO} \;=\; \text{NI} \;+\; \text{NCC} \;-\; \text{WCINV}
\]

\subsubsection*{Standards notes and exam tips}
\begin{itemize}
  \item Both IFRS and U.S. GAAP \emph{encourage} the direct format, but most issuers present \textbf{indirect} CFO. Under U.S. GAAP, if direct is shown, an indirect reconciliation is required in the notes.
  \item Remember classification options: under IFRS, interest and dividends received/paid may be classified as CFO or CFI/CFF (policy choice), while U.S. GAAP generally classifies interest paid/received and dividends received as \textbf{CFO}, and dividends paid as \textbf{CFF}.
  \item Quality of earnings: persistent gap where NI $>$ CFO can indicate aggressive accruals or working-capital build.
\end{itemize}

\subsection*{Module 32.3: Investing and Financing Cash Flows and IFRS/U.S. GAAP Differences}

\subsubsection*{Cash Flow From Investing (CFI) and From Financing (CFF)}

\paragraph{Definitions}
\begin{itemize}
  \item \textbf{CFI:} Cash inflows/outflows from acquiring or disposing of \emph{long-term assets} and certain investments.
  \item \textbf{CFF:} Cash inflows/outflows from transactions affecting \emph{capital structure} (debt and equity).
\end{itemize}

\paragraph{U.S. GAAP classification (typical)}
\begin{table}[H]
\centering
\footnotesize
\begin{tabular}{|l|p{5.3cm}|p{5.3cm}|}
\hline
\textbf{Bucket} & \textbf{Inflows} & \textbf{Outflows} \\
\hline
\textbf{CFI} &
Proceeds from sale of PP\&E, intangibles, long-term investments &
Purchase of PP\&E and intangibles; purchase of debt/equity investments (other than trading); loans made to others \\
\hline
\textbf{CFF} &
Proceeds from issuing debt (bonds/notes) or equity (shares); proceeds from new borrowings &
Repayment of principal on debt; share repurchases (treasury stock); cash dividends paid \\
\hline
\textbf{CFO (relevant contrasts)} &
Interest received; dividends received &
Interest paid; income taxes paid (all taxes under U.S. GAAP) \\
\hline
\end{tabular}
\caption{Cash flow classifications under U.S. GAAP (selected items)}
\end{table}

\paragraph{Professor's note}
\begin{itemize}
  \item Do not confuse \emph{dividends received} (CFO under U.S. GAAP) with \emph{dividends paid} (CFF).
\end{itemize}

\subsubsection*{Worked example: Computing CFI}
\textbf{Given:}
\begin{itemize}
  \item Footnote: PP\&E purchases during the year = \$25{,}000.
  \item Gross PP\&E: beginning = \$60{,}000; ending = \$69{,}000.
  \item Depreciation expense = \$7{,}000.
  \item Accumulated depreciation increased only \$3{,}000 (implies disposal removed accumulated depreciation).
  \item Loss on sale of PP\&E in the income statement (therefore proceeds $<$ carrying value).
  \item Land: beginning carrying value = \$40{,}000; ending = \$35{,}000; no land purchases disclosed.
\end{itemize}

\paragraph{1) Identify PP\&E disposals (gross and accumulated depreciation)}
\[
\begin{aligned}
\text{Disposed gross cost} &= \text{Beg. gross PP\&E} + \text{Additions} - \text{End. gross PP\&E} \\
&= 60{,}000 + 25{,}000 - 69{,}000 = 16{,}000
\end{aligned}
\]
Accumulated depreciation removed with the disposal:
\[
\text{AD removed} = \text{Beg. AD} + \text{Depreciation} - \text{End. AD} = 7{,}000 - 3{,}000 = 4{,}000
\]
\textit{(We only need the net change; exact beg/end AD levels are not required.)}

\paragraph{2) Carrying value and proceeds on PP\&E sale}
\[
\text{Carrying value disposed} = \text{Disposed gross cost} - \text{AD removed} = 16{,}000 - 4{,}000 = 12{,}000
\]
Given a \emph{loss} on disposal and that only \emph{proceeds} are cash:
\[
\text{Proceeds on PP\&E sale} = 10{,}000 \quad (\text{implied by loss of \$2{,}000})
\]

\paragraph{3) Land disposal}
\[
\text{Carrying value disposed (land)} = 40{,}000 - 35{,}000 = 5{,}000 \quad (\text{no depreciation on land})
\]
\[
\text{Proceeds on land sale} = 15{,}000
\]

\paragraph{4) Compute CFI}
\[
\begin{aligned}
\text{CFI} &= -\text{Cash paid for PP\&E additions} + \text{Proceeds on PP\&E sale} + \text{Proceeds on land sale} \\
&= -25{,}000 + 10{,}000 + 15{,}000 \\
&= \boxed{0}
\end{aligned}
\]
\textit{In this case, new asset purchases were exactly offset by disposal proceeds.}

\paragraph{One-line shortcut for carrying value of PP\&E disposed}
\[
\text{Carrying value disposed} = \text{Beg. net PP\&E} - \text{Depreciation} + \text{Additions} - \text{End. net PP\&E}
\]
\textit{Then use: Proceeds = Carrying value $+$ Gain (or $-$ Loss).}

\subsubsection*{Worked example: Computing CFF}
\textbf{Given:}
\begin{itemize}
  \item Bonds outstanding issued at par. Bonds payable: beginning = \$10{,}000; ending = \$15{,}000.
  \item Contributed capital (Common stock + APIC): beginning = \$40{,}000; ending = \$50{,}000.
  \item Retained earnings: beginning = \$30{,}500; ending = \$61{,}000. Net income = \$39{,}000.
  \item Dividends payable change as applicable (not shown below; include if provided).
\end{itemize}

\paragraph{1) Net principal cash flow from debt (issued at par)}
\[
\Delta \text{Bonds payable} = 15{,}000 - 10{,}000 = \boxed{+5{,}000 \text{ (CFF inflow)}}
\]

\paragraph{2) Net equity cash flow}
\[
\Delta \text{Contributed capital} = 50{,}000 - 40{,}000 = \boxed{+10{,}000 \text{ (CFF inflow)}}
\]
\textit{If contributed capital decreased, it would be an outflow (share repurchase).}

\paragraph{3) Cash dividends paid}
\[
\text{Dividends declared} = \text{Beg. RE} + \text{Net income} - \text{End. RE} = 30{,}500 + 39{,}000 - 61{,}000 = 8{,}500
\]
Adjust for change in dividends payable (DP):
\[
\text{Dividends paid} = \text{Dividends declared} + \text{Beg. DP} - \text{End. DP}
\]
\textit{Use provided DP figures; if none, assume declared = paid.} \\
Thus, \(\boxed{\text{Dividends paid} = 8{,}500 \text{ (CFF outflow)}}\).

\paragraph{4) Total CFF (sign convention: inflows positive)}
\[
\text{CFF} = (+5{,}000) + (+10{,}000) - (8{,}500) = \boxed{+6{,}500}
\]

\subsubsection*{Completing the cash flow statement}
\begin{itemize}
  \item Compute \textbf{CFO} (from 32.1/32.2), \textbf{CFI}, and \textbf{CFF}.
  \item \(\text{Net change in cash} = \text{CFO} + \text{CFI} + \text{CFF}\).
  \item Check: \(\text{End cash} - \text{Beg cash} = \text{Net change in cash}\).
\end{itemize}

\subsubsection*{Converting Indirect CFO to Direct CFO (LOS 32.c)}
\paragraph{Three-step method}
\begin{enumerate}[label=\arabic*.]
  \item \textbf{Aggregate} all revenues \& gains and all expenses \& losses.
  \item \textbf{Remove} all noncash items and \textbf{disaggregate} the remainder into natural cash categories.
  \item \textbf{Convert accruals to cash} by adjusting each category for related working-capital changes.
\end{enumerate}

\paragraph{Useful direct-method building blocks}
\[
\text{Cash collected from customers} = \text{Sales} - \Delta \text{Accounts receivable}
\]
\[
\text{Cash paid to suppliers} = \text{COGS} + \Delta \text{Inventory} - \Delta \text{Accounts payable}
\]
\[
\text{Cash operating expenses} \approx \text{SG\&A} - \Delta \text{Accrued expenses} - \Delta \text{Prepaids}
\]
\[
\text{Cash interest paid} = \text{Interest expense} - \Delta \text{Interest payable}
\]
\[
\text{Cash taxes paid} = \text{Tax expense} - \Delta \text{Taxes payable} - \Delta \text{Deferred taxes}
\]
Sum the adjusted cash inflows/outflows to get \(\text{CFO}_{\text{direct}}\) (it must equal indirect CFO).

\subsubsection*{IFRS vs U.S. GAAP classification differences (LOS 32.d)}
\begin{table}[H]
\centering
\footnotesize
\begin{tabular}{|l|c|c|}
\hline
\textbf{Item} & \textbf{U.S. GAAP} & \textbf{IFRS} \\
\hline
Interest received & CFO & CFO or CFI \\
Interest paid & CFO & CFO or CFF \\
Dividends received & CFO & CFO or CFI \\
Dividends paid & CFF & CFO or CFF \\
Income taxes paid & CFO (all) & CFO unless specifically attributable to CFI or CFF \\
\hline
\end{tabular}
\caption{Key classification differences: IFRS vs U.S. GAAP}
\end{table}

\paragraph{Illustration: tax on investing transaction}
Sell land for \$1{,}000{,}000; income tax on sale = \$160{,}000.
\begin{itemize}
  \item \textbf{U.S. GAAP:} CFI inflow \$1{,}000{,}000; CFO outflow \$160{,}000.
  \item \textbf{IFRS:} May present net CFI inflow \$840{,}000 if taxes are directly attributable to the investing transaction.
\end{itemize}

\subsubsection*{Exam tips and analyst notes}
\begin{itemize}
  \item When bonds are issued at par (Level I simplification), \(\Delta\) Bonds payable equals \emph{cash} from debt issuance/repayment.
  \item Premium/discount amortization affects interest expense and carrying value but not cash; focus on principal cash flows in CFF.
  \item For equity, \(\Delta\) (Common stock + APIC) approximates net share issuance (inflow) or repurchase (outflow); differences from issue price affect retained earnings in practice.
  \item Always reconcile totals to the change in cash as a validation step.
\end{itemize}


\end{document}