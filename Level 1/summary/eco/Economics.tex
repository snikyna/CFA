% Economics Study Notes for CFA Level I
% Comprehensive notes covering microeconomics and macroeconomics principles
% Topics include market structures, firms behavior, supply and demand analysis
% Essential for understanding economic foundations in investment analysis
% Study Material: 2024 CFA Level I Curriculum

\documentclass[12pt]{article}
\usepackage{amsmath}
\usepackage{geometry}
\usepackage{graphicx}
\usepackage{booktabs}
\usepackage{caption}
\usepackage{titlesec}
\usepackage{graphicx} % make sure to include in preamble
\usepackage{float}
\usepackage{makecell}
\usepackage{tabularx}


\geometry{margin=1in}

\title{Economics}
\date{}

\begin{document}
\maketitle
\section*{12 FIRMS AND MARKET STRUCTURES}

\subsection*{12.1 Breakeven, Shutdown, and Scale}

\begin{itemize}
    \item \textbf{Short run vs. long run:}
    \begin{itemize}
        \item \textbf{Short run:} Some inputs (e.g., capital) are fixed, so plant size/scale cannot change. Fixed costs must still be paid.
        \item \textbf{Long run:} All costs are variable, and the firm can change scale (e.g., let leases expire, sell equipment).
    \end{itemize}

    \item \textbf{Breakeven point (BEP):}
    \begin{itemize}
        \item Occurs where \(\text{Price} = AR = ATC\).
        \item \(\pi_{econ} = 0\), firm earns normal profit.
    \end{itemize}

    \item \textbf{Shutdown decision under perfect competition:}
    \begin{itemize}
        \item If \(AR < AVC\): shut down immediately in short run.
        \item If \(AVC \leq AR < ATC\): continue in short run (covering variable costs, but not all fixed costs).
        \item If \(AR \geq ATC\): continue in both short and long run.
    \end{itemize}

    \item \textbf{General decision rules:}
    \begin{table}[H]
    \centering
    \begin{tabular}{|c|c|c|}
    \hline
    Condition & Short Run & Long Run \\
    \hline
    \(AR \geq ATC\) & Operate & Stay in market \\
    \(AVC \leq AR < ATC\) & Operate (minimize losses) & Exit in long run \\
    \(AR < AVC\) & Shut down & Exit in long run \\
    \hline
    \end{tabular}
    \caption{Shutdown and Breakeven Rules (Perfect Competition)}
    \end{table}

    \item \textbf{Shutdown decision under imperfect competition:}
    \begin{itemize}
        \item Marginal revenue \(\neq\) price (since demand is downward sloping).
        \item Better to compare total revenue (TR), total cost (TC), and total variable cost (TVC).
    \end{itemize}
    \begin{table}[H]
    \centering
    \begin{tabular}{|c|c|}
    \hline
    Condition & Decision \\
    \hline
    \(TR = TC\) & Break even \\
    \(TC > TR > TVC\) & Operate in short run, exit in long run \\
    \(TR < TVC\) & Shut down in both short and long run \\
    \hline
    \end{tabular}
    \caption{Shutdown and Breakeven Rules (Imperfect Competition)}
    \end{table}

    \item \textbf{Examples:}
    \begin{itemize}
        \item \textbf{Short-run shutdown:} \\
        Legion Gaming: 
        \[
        TR = 700{,}000,\; TVC = 800{,}000,\; TFC = 400{,}000
        \] 
        \(\Rightarrow TR < TVC\). \\
        If operate: Loss \(= TC - TR = 1{,}200{,}000 - 700{,}000 = 500{,}000\). \\
        If shut down: Loss \(= TFC = 400{,}000\). \\
        $\Rightarrow$ Better to shut down.
        
        \item \textbf{Long-run shutdown:} \\
        If \(TR = 850{,}000\), then:
        \[
        TVC = 800{,}000,\; TFC = 400{,}000
        \]
        Short run: \(TR > TVC\) → continue operating. \\
        Long run: \(TR < TC = 1{,}200{,}000\) → exit eventually.
    \end{itemize}

    \item \textbf{Economies and diseconomies of scale:}
    \begin{itemize}
        \item Long-run average total cost (LRATC) is U-shaped.
        \item \textbf{Economies of scale (downward-sloping LRATC):}
        \begin{itemize}
            \item Factors: labor specialization, mass production, more efficient technology, bulk input purchases.
            \item Result: average costs fall as output increases.
        \end{itemize}
        \item \textbf{Minimum Efficient Scale (MES):} Point where LRATC is minimized.
        \item \textbf{Constant returns to scale:} LRATC flat over a range of output.
        \item \textbf{Diseconomies of scale (upward-sloping LRATC):}
        \begin{itemize}
            \item Causes: bureaucracy, motivation/communication issues in large firms, barriers to innovation.
            \item Result: average costs rise as firm grows too large.
        \end{itemize}
    \end{itemize}

    \item \textbf{Graphical intuition:}
    \begin{itemize}
        \item \textbf{Short-run:} ATC, AVC, and MC curves determine shutdown/breakeven points.
        \item \textbf{Long-run:} LRATC envelops all SRATC curves, showing optimal scale decisions.
    \end{itemize}
\end{itemize}

\subsection*{12.2 Characteristics of Market Structures}

\textbf{LOS 12.b: Characteristics of Market Structures}

\begin{itemize}
    \item \textbf{Criteria to compare market structures:}
    \begin{enumerate}
        \item Number of firms and relative sizes
        \item Degree of product differentiation
        \item Bargaining power in pricing
        \item Barriers to entry/exit
        \item Non-price competition (advertising, branding, features)
    \end{enumerate}

    \item \textbf{Market structures overview:}
    \begin{table}[H]
\centering
\resizebox{\textwidth}{!}{%
\begin{tabular}{|l|c|c|c|c|}
\hline
Characteristic & Perfect Comp. & Monopolistic Comp. & Oligopoly & Monopoly \\
\hline
Firms & Many small & Many small/medium & Few large & Single \\
Products & Identical & Differentiated & Similar or diff. & Unique \\
Demand curve & Perfectly elastic & Downward-sloping & Downward-sloping & Market demand \\
Price power & None & Limited (brand-based) & Significant, interdependent & Full price-setter \\
Barriers & Very low & Low & High & Very high \\
Example & Wheat & Toothpaste & Autos, oil & Local utility, patents \\
\hline
\end{tabular}}
\caption{Comparison of Market Structures}
\end{table}
\end{itemize}

---

\textbf{LOS 12.c: Monopolistic Competition – Price/Output and Strategy}

\begin{itemize}
    \item \textbf{Key features:}
    \begin{itemize}
        \item Many independent sellers with small market share.
        \item Products differentiated (quality, features, branding).
        \item Demand curve is downward sloping.
        \item Low entry/exit barriers.
        \item High advertising/marketing costs.
    \end{itemize}

    \item \textbf{Short run:}
    \begin{itemize}
        \item Profit maximization where \(MR = MC\).
        \item Firm charges price from demand curve: \(P^* > ATC^*\).
        \item Positive economic profit attracts new entrants.
    \end{itemize}

    \item \textbf{Long run:}
    \begin{itemize}
        \item Entry of new firms shifts demand downwards.
        \item Long-run equilibrium when \(P = ATC\) and \(\pi_{econ} = 0\).
        \item Firms still produce where \(MR = MC\).
    \end{itemize}

    \item \textbf{Comparison with perfect competition:}
    \begin{table}[H]
    \centering
    \begin{tabular}{|l|c|c|}
    \hline
    Feature & Monopolistic Comp. & Perfect Comp. \\
    \hline
    Price & \(P > MC\) & \(P = MC\) \\
    ATC & Not at minimum (excess capacity) & Minimum ATC \\
    Product differentiation & Present & None \\
    Long-run profit & Zero & Zero \\
    Efficiency & Lower & Higher \\
    \hline
    \end{tabular}
    \caption{Monopolistic vs. Perfect Competition}
    \end{table}

    \item \textbf{Example: Toothpaste market}
    \begin{itemize}
        \item Firms differentiate with whitening, freshness, decay-prevention claims.
        \item Advertising is key.
        \item Consumers switch brands, but not perfectly elastic demand.
    \end{itemize}
\end{itemize}

---

\textbf{LOS 12.d: Oligopoly – Price/Output and Strategy}

\begin{itemize}
    \item \textbf{Key features:}
    \begin{itemize}
        \item Few large firms, interdependent decisions.
        \item High barriers to entry (scale, branding, capital intensity).
        \item Products may be similar (oil) or differentiated (autos).
    \end{itemize}

    \item \textbf{Models of Oligopoly:}
    \begin{enumerate}
        \item \textbf{Kinked Demand Curve:}
        \begin{itemize}
            \item Competitors unlikely to match price increase, but likely to match decrease.
            \item Demand is elastic above \(P_K\), inelastic below.
            \item Leads to kink at \(P_K, Q_K\).
            \item MR has a discontinuity → price rigidity.
        \end{itemize}

        \item \textbf{Cournot Duopoly Model:}
        \begin{itemize}
            \item Two firms choose output simultaneously.
            \item Each assumes other’s output fixed.
            \item Equilibrium: firms split market, price between monopoly and competition.
            \item Generalizes to more firms → price approaches competitive level.
        \end{itemize}

        \item \textbf{Stackelberg Model:}
        \begin{itemize}
            \item Sequential decision-making.
            \item Leader sets price/output first, follower reacts.
            \item Leader captures larger share of profits.
        \end{itemize}

        \item \textbf{Nash Equilibrium (strategic games):}
        \begin{itemize}
            \item Equilibrium when no firm can improve profits by changing strategy unilaterally.
            \item Example payoff matrix:
            \[
            \begin{array}{|c|c|c|}
            \hline
             & \text{Firm B: High P} & \text{Firm B: Low P} \\
            \hline
            \text{Firm A: High P} & (1000, 600) & (600, 700) \\
            \hline
            \text{Firm A: Low P} & (500, 400) & (300, 300) \\
            \hline
            \end{array}
            \]
            Nash equilibrium: Firm A = High P, Firm B = Low P.
        \end{itemize}
    \end{enumerate}

    \item \textbf{Collusion (cartels):}
    \begin{itemize}
        \item Firms agree to fix prices/output (illegal in many countries).
        \item Maximizes joint profits (like monopoly).
        \item Example: OPEC.
        \item Success factors: fewer firms, homogeneous products, similar costs, frequent small purchases, credible retaliation, weak outside competition.
    \end{itemize}

    \item \textbf{Dominant Firm Model:}
    \begin{itemize}
        \item One large firm (DF) sets price, smaller competitive firms (CFs) take it as given.
        \item DF determines price where \(MR_{DF} = MC_{DF}\).
        \item CFs produce where \(MC_{CF} = P^*\).
        \item If CFs cut prices, DF cuts price further → CFs lose share in long run.
    \end{itemize}

    \item \textbf{Range of outcomes:}
    \begin{itemize}
        \item Price lies between:
        \[
        P_{competition} < P_{oligopoly} < P_{monopoly}
        \]
        \item \(\pi_{econ}\) between zero (competition) and maximum (monopoly).
    \end{itemize}

    \item \textbf{Examples:}
    \begin{itemize}
        \item \textbf{Autos:} Toyota vs. Ford, interdependent strategies.
        \item \textbf{Oil:} OPEC cartel, collusion with occasional cheating.
    \end{itemize}
\end{itemize}

\subsection*{12.3 Identifying Market Structures}

\textbf{LOS 12.e: Identify market structure and describe use/limitations of concentration measures}

\begin{itemize}
    \item \textbf{Identifying market structure:}
    \begin{itemize}
        \item Based on characteristics: 
        number of firms, entry barriers, product differentiation, substitutes, 
        and pricing interdependence.  
        \item \textbf{Perfect Competition:} many firms, homogeneous products, no pricing power.  
        \item \textbf{Monopolistic Competition:} many firms, differentiated products, limited pricing power.  
        \item \textbf{Oligopoly:} few large firms, high barriers, interdependence, significant pricing power.  
        \item \textbf{Monopoly:} single firm, unique product, very high barriers, full price control.
    \end{itemize}

    \item \textbf{Market structure overview (recap):}
    \begin{table}[H]
    \centering
\resizebox{\textwidth}{!}{%
    \begin{tabular}{|l|c|c|c|c|}
    \hline
    Characteristic & Perfect Comp. & Monopolistic Comp. & Oligopoly & Monopoly \\
    \hline
    Firms & Many small & Many small/medium & Few large & Single \\
    Products & Identical & Differentiated & Similar or diff. & Unique \\
    Demand curve & Perfectly elastic & Downward-sloping & Downward-sloping & Market demand \\
    Price power & None & Limited (branding) & Significant, interdependent & Full control \\
    Barriers & Very low & Low & High & Very high \\
    Example & Wheat & Toothpaste & Autos, oil & Local utility, patents \\
    \hline
    \end{tabular}}
    \caption{Comparison of Market Structures (recap)}
    \end{table}

    \item \textbf{Measuring market concentration:}
    \begin{itemize}
        \item Regulators assess market power via concentration ratios.
        \item \textbf{N-firm concentration ratio:}
        \[
        CR_N = \sum_{i=1}^N s_i
        \]
        where \(s_i\) = market share (\%).
        \begin{itemize}
            \item Simple to calculate, but insensitive to distribution among top firms.
            \item E.g., 2 firms each with 30\% vs. one firm 60\% → same \(CR_2 = 60\%\).
        \end{itemize}
        \item \textbf{Herfindahl-Hirschman Index (HHI):}
        \[
        HHI = \sum_{i=1}^N s_i^2
        \]
        (shares expressed as decimals).
        \begin{itemize}
            \item Gives more weight to larger firms.
            \item Sensitive to mergers → better measure of market concentration.
        \end{itemize}
    \end{itemize}

    \item \textbf{Example – Concentration ratio vs. HHI:}
    \begin{table}[H]
    \centering
    \begin{tabular}{|c|c|}
    \hline
    Firm & Market Share (\%) \\
    \hline
    Acme & 25 \\
    Blake & 15 \\
    Carter & 15 \\
    Delta & 10 \\
    Others & 35 \\
    \hline
    \end{tabular}
    \caption{Market Shares Before Merger}
    \end{table}

    \begin{itemize}
        \item \textbf{Before merger:}
        \[
        CR_4 = 25 + 15 + 15 + 10 = 65\%
        \]
        \[
        HHI = 0.25^2 + 0.15^2 + 0.15^2 + 0.10^2 = 0.1175
        \]
        
        \item \textbf{After merger (Acme + Blake = 40\%):}
        \[
        CR_4 = 40 + 15 + 10 + 5 = 70\%
        \]
        \[
        HHI = 0.40^2 + 0.15^2 + 0.10^2 + 0.05^2 = 0.1950
        \]

        \item \textbf{Interpretation:}
        \begin{itemize}
            \item CR increased slightly (65\% → 70\%), suggesting modest concentration rise.
            \item HHI rose significantly (0.1175 → 0.1950), reflecting higher dominance of merged firm.
            \item Shows why HHI is more informative for mergers.
        \end{itemize}
    \end{itemize}

    \item \textbf{Limitations of concentration measures:}
    \begin{itemize}
        \item Do not consider entry barriers: high share firms may still lack pricing power if entry is easy.
        \item Do not directly measure demand elasticity.
        \item Potential competition may discipline pricing even in concentrated markets.
        \item Sensitive to market definition (geographic, product scope).
    \end{itemize}
\end{itemize}

\section*{13 UNDERSTANDING BUSINESS CYCLES}

\subsection*{13.1 Business Cycles}

\textbf{LOS 13.a: Describe the business cycle and its phases}

\begin{itemize}
    \item \textbf{Definition:} Business cycle = fluctuations in economic activity, measured mainly by real GDP.  
    \item \textbf{Phases:}
    \begin{enumerate}
        \item \textbf{Expansion:} 
        \begin{itemize}
            \item Real GDP rising, employment and investment increasing.  
            \item Inflation may accelerate near the peak.  
        \end{itemize}
        \item \textbf{Peak:} 
        \begin{itemize}
            \item Real GDP growth slows and turns negative.  
            \item Inflation typically highest here.  
        \end{itemize}
        \item \textbf{Contraction (Recession):}
        \begin{itemize}
            \item Real GDP declining, unemployment rising, investment and spending falling.  
            \item Inflation decreases.  
        \end{itemize}
        \item \textbf{Trough:} 
        \begin{itemize}
            \item GDP growth turns positive again.  
            \item Inflation moderate/low, unemployment high but falling.  
        \end{itemize}
    \end{enumerate}

    \item \textbf{Alternative measures of cycles:}
    \begin{itemize}
        \item \textbf{Classical cycle:} Real GDP relative to a base level.  
        \item \textbf{Growth cycle:} GDP relative to trend value.  
        \item \textbf{Growth rate cycle:} Changes in annualized growth rate (peaks/troughs detected earlier).  
        \item Preferred by economists = growth rate cycle.  
    \end{itemize}

    \item \textbf{Dating rules:}
    \begin{itemize}
        \item Expansion: usually 2 consecutive quarters of positive GDP growth.  
        \item Recession: 2 consecutive quarters of negative GDP growth.  
        \item Official dating (e.g., NBER in the U.S.) uses broader data: unemployment, industrial production, inflation.  
    \end{itemize}

    \item \textbf{Important:} Cycles recur, but not at regular intervals (duration varies: 1 year to 10+ years).
\end{itemize}

---

\textbf{LOS 13.b: Credit Cycles}

\begin{itemize}
    \item \textbf{Definition:} Cyclical fluctuations in interest rates and credit availability.  
    \item \textbf{Expansion:} Loose credit, easier lending, lower interest rates → may fuel bubbles.  
    \item \textbf{Contraction:} Tight credit, higher rates, less lending.  
    \item \textbf{Amplification:} Expansions stronger and recessions deeper when coinciding with credit cycles.  
    \item \textbf{Example:} Subprime mortgage bubble (2007–2009). Easy credit fueled housing bubble; crash amplified recession.  
    \item \textbf{Duration:} Credit cycles often longer than business cycles.  
\end{itemize}

---

\textbf{LOS 13.c: Resource Use, Consumer/Business, Housing, External Trade}

\begin{itemize}
    \item \textbf{Inventories (resource use):}
    \begin{itemize}
        \item Firms balance sales vs. inventory.  
        \item Near peak: sales slow, inventories rise → firms cut production.  
        \item Near trough: sales pick up, inventories depleted → firms raise production.  
    \end{itemize}

    \item \textbf{Labor and capital use:}
    \begin{itemize}
        \item Short-term: adjust overtime, utilization of current workers and machines.  
        \item Long-term: hiring/layoffs and capital investment changes.  
    \end{itemize}

    \item \textbf{Consumer sector:}
    \begin{itemize}
        \item Spending ↑ in expansions, ↓ in recessions.  
        \item \textbf{Durable goods:} highly cyclical (cars, furniture, appliances).  
        \item \textbf{Nondurables:} relatively stable (food, household supplies).  
        \item \textbf{Services:} moderately cyclical (travel, restaurants) but partly stable (healthcare, insurance).  
    \end{itemize}

    \item \textbf{Housing sector:}
    \begin{enumerate}
        \item \textbf{Mortgage rates:} low → more buying, high → less.  
        \item \textbf{Income vs. housing costs:} high affordability → ↑ demand.  
        \item \textbf{Speculative activity:} expectations of rising prices → bubble risk (e.g., 2007–08).  
        \item \textbf{Demographics:} young population (25–40 yrs) → ↑ demand; urbanization drives housing booms.  
    \end{enumerate}

    \item \textbf{External trade sector:}
    \begin{itemize}
        \item Domestic GDP ↑ → imports ↑.  
        \item Trading partners’ GDP ↑ → exports ↑.  
        \item Exchange rate ↑ (domestic currency stronger) → exports ↓, imports ↑.  
        \item Exchange rate ↓ → exports ↑, imports ↓.  
    \end{itemize}
\end{itemize}

---

\textbf{Summary of Phases and Characteristics}

\begin{table}[H]
\centering
\footnotesize
\begin{tabular}{|l|l|l|l|l|}
\hline
 & \textbf{Trough} & \textbf{Expansion} & \textbf{Peak} & \textbf{Contraction} \\
\hline
GDP Growth & Neg. → Pos. & Increasing & Decreasing & Negative \\
Unemployment & High, ↓ overtime/temp & Decreasing & Low but ↓ slower & Increasing \\
Consumer Spending & Durables ↑ & Strong ↑ & Slower ↑ & ↓ (esp. durables) \\
Investment & Low → rising & Rising & Slowing & Falling \\
Housing & May ↑ early & Rising & Slowing & Falling \\
Inflation & Moderate/↓ & Rising & Highest & ↓ (lagged) \\
Imports & Low & Increasing & High & Decreasing \\
\hline
\end{tabular}
\caption{Characteristics of Business Cycle Phases}
\end{table}

---

\textbf{Economic Indicators Classification}

\begin{itemize}
    \item \textbf{Leading Indicators:} Change before peaks/troughs.  
    Examples: Average weekly hours, new orders, stock prices, consumer expectations, housing permits, initial jobless claims.  
    \item \textbf{Coincident Indicators:} Change at same time as economy.  
    Examples: GDP, industrial production, personal income, non-farm employment.  
    \item \textbf{Lagging Indicators:} Change after economy turns.  
    Examples: Unemployment duration, inflation, labor cost per unit, commercial loans outstanding.  
\end{itemize}

\section*{14 FISCAL POLICY}
\subsection*{14.1 Fiscal Policy Objectives}

\textbf{LOS 14.a: Compare monetary and fiscal policy}

\begin{itemize}
    \item \textbf{Fiscal Policy:} Government use of taxation and spending to influence economic activity.  
    \begin{itemize}
        \item Balanced budget: Tax revenues = expenditures.  
        \item Surplus: Revenues $>$ expenditures.  
        \item Deficit: Expenditures $>$ revenues.  
        \item Expansionary: Increase deficit (or reduce surplus) → ↑ GDP.  
        \item Contractionary: Reduce deficit (or increase surplus) → ↓ GDP.  
    \end{itemize}

    \item \textbf{Monetary Policy:} Central bank actions influencing money and credit supply.  
    \begin{itemize}
        \item Expansionary (easy): Increase quantity of money/credit.  
        \item Contractionary (tight): Decrease quantity of money/credit.  
    \end{itemize}

    \item \textbf{Both policies aim to:}
    \begin{itemize}
        \item Maintain price stability.  
        \item Foster sustainable economic growth.  
        \item Fiscal policy additionally redistributes wealth and allocates resources.  
    \end{itemize}
\end{itemize}

\begin{table}[H]
\centering
\small
\resizebox{\textwidth}{!}{%
\begin{tabular}{|l|c|c|}
\hline
 & \textbf{Fiscal Policy} & \textbf{Monetary Policy} \\
\hline
Authority & Government (legislature, treasury) & Central Bank \\
Tools & Taxes, government spending & Money supply, interest rates, credit \\
Objective & Demand mgmt., redistribution, resource allocation & Inflation control, stability, growth \\
Expansionary & $\downarrow$ taxes, $\uparrow$ spending & $\downarrow$ interest rates, $\uparrow$ money supply \\
Contractionary & $\uparrow$ taxes, $\downarrow$ spending & $\uparrow$ interest rates, $\downarrow$ money supply \\
Time Lag & Longer (political process) & Shorter (policy committees) \\
\hline
\end{tabular}}
\caption{Comparison: Fiscal vs. Monetary Policy}
\end{table}

---

\textbf{LOS 14.b: Roles, Objectives, and National Debt Concerns}

\begin{itemize}
    \item \textbf{Roles and objectives of fiscal policy:}
    \begin{enumerate}
        \item Influence aggregate demand and economic activity.  
        \item Redistribute income/wealth.  
        \item Allocate resources among sectors (e.g., infrastructure, education).  
    \end{enumerate}

    \item \textbf{Mechanics:}
    \begin{itemize}
        \item Expansionary: $\downarrow$ taxes or $\uparrow$ government spending → ↑ AD, ↑ GDP, ↑ employment.  
        \item Contractionary: $\uparrow$ taxes or $\downarrow$ spending → ↓ AD, ↓ GDP, ↓ inflation.  
        \item Automatic stabilizers: Built-in mechanisms (e.g., tax receipts fall + welfare spending ↑ in recession).  
        \item Discretionary policy: Active changes in taxation/spending to stabilize economy.  
    \end{itemize}

    \item \textbf{Schools of thought:}
    \begin{itemize}
        \item \textbf{Keynesians:} Fiscal policy strongly effective at less than full employment.  
        \item \textbf{Monetarists:} Fiscal stimulus only temporary; monetary policy better for long-term inflation control.  
    \end{itemize}

    \item \textbf{Debt-to-GDP dynamics:}
    \[
    \text{Debt Ratio} = \frac{\text{Government Debt}}{\text{GDP}}
    \]
    \begin{itemize}
        \item If $r > g$ (real interest rate $>$ real GDP growth) → debt ratio worsens over time.  
        \item If $r < g$ → debt ratio improves over time.  
    \end{itemize}

    \item \textbf{Arguments \underline{for concern} with high deficits:}
    \begin{itemize}
        \item Higher future taxes $\Rightarrow$ ↓ incentives to work/invest $\Rightarrow$ ↓ growth.  
        \item Default/inflation risk if markets lose confidence (esp. if foreign currency debt).  
        \item Crowding-out: Government borrowing ↑ interest rates → ↓ private investment.  
    \end{itemize}

    \item \textbf{Arguments \underline{against concern}:}
    \begin{itemize}
        \item If debt mostly held domestically, risk overstated.  
        \item If debt finances productive investment, future gains repay debt.  
        \item Deficits may encourage tax reform.  
        \item Ricardian equivalence: Households save more today in anticipation of future taxes → offsetting effect.  
        \item At less than full capacity, deficits can boost GDP and employment without displacing private capital.  
    \end{itemize}
\end{itemize}

\begin{table}[H]
\centering
\footnotesize
\begin{tabular}{|l|p{6cm}|p{6cm}|}
\hline
 & \textbf{Arguments For Concern} & \textbf{Arguments Against Concern} \\
\hline
Future Taxes & Higher deficits → higher taxes, lower growth & Ricardian equivalence may offset via higher savings \\
Default Risk & Risk of default or inflation if markets lose confidence & Domestic-held debt less risky; gov. can service in local currency \\
Crowding-Out & Gov. borrowing ↑ interest rates, ↓ private investment & At less than full capacity, deficits ↑ GDP without displacing investment \\
Productivity & Debt not always productive & Productive debt → future gains cover repayment \\
Reform Impact & Higher taxes discourage work & Deficits may prompt tax reform \\
\hline
\end{tabular}
\caption{Arguments For vs. Against Concern Over Fiscal Deficits}
\end{table}

\subsection*{14.2 Fiscal Policy Tools and Implementation}

\textbf{LOS 14.c: Tools of Fiscal Policy (Advantages \& Disadvantages)}

\begin{itemize}
    \item \textbf{Spending Tools:}
    \begin{itemize}
        \item \textbf{Transfer payments:} Redistribution (e.g., Social Security, unemployment insurance). Not included in GDP.  
        \item \textbf{Current spending:} Ongoing purchases (e.g., defense, education, administration).  
        \item \textbf{Capital spending:} Infrastructure investment (roads, bridges, schools, hospitals). Boosts long-term productivity.  
    \end{itemize}

    \item \textbf{Revenue Tools:}
    \begin{itemize}
        \item \textbf{Direct taxes:} Income, wealth, estate, corporate, capital gains, Social Security. Often progressive → redistribution.  
        \item \textbf{Indirect taxes:} VAT, sales taxes, excise taxes. Can discourage undesirable consumption (alcohol, tobacco).  
    \end{itemize}

    \item \textbf{Desirable tax attributes:}
    \begin{itemize}
        \item Simplicity (easy to use and enforce).  
        \item Efficiency (minimal distortion to markets).  
        \item Fairness (horizontal = equal treatment, vertical = higher income $\rightarrow$ higher tax).  
        \item Sufficiency (enough revenue to meet spending needs).  
    \end{itemize}
\end{itemize}

\begin{table}[H]
\centering
\footnotesize
\begin{tabular}{|l|p{6cm}|p{6cm}|}
\hline
 & \textbf{Advantages} & \textbf{Disadvantages} \\
\hline
Indirect taxes & Quick to implement, raise revenue cheaply, influence social behavior (e.g., tobacco tax) & Can be regressive (hit low-income harder), distort consumption \\
Direct taxes & Redistributive, predictable revenue & Take time to implement, may reduce work incentives \\
Spending tools & Direct boost to AD, capital spending raises long-term productivity & Slow to implement (esp. infrastructure), effects lag actual need \\
Announcements & Expectations adjust immediately (can curb AD fast) & May dampen confidence, reduce investment \\
\hline
\end{tabular}
\caption{Advantages and Disadvantages of Fiscal Policy Tools}
\end{table}

---

\textbf{Fiscal Multiplier}

\begin{itemize}
    \item Spending multiplier formula:
    \[
    \text{Multiplier} = \frac{1}{1 - MPC \times (1 - t)}
    \]
    where $MPC =$ marginal propensity to consume, $t =$ tax rate.
    \item Example: $MPC = 0.8$, $t = 0.25$:  
    \[
    \text{Multiplier} = \frac{1}{1 - 0.8(1-0.25)} = 2.5
    \]  
    $\Rightarrow$ A \$100 spending increase $\rightarrow$ \$250 increase in AD.
\end{itemize}

\textbf{Balanced Budget Multiplier}
\begin{itemize}
    \item Example: Increase spending by \$100, financed by \$100 tax increase.  
    \item Tax effect: Initial $\Delta C = 100 \times 0.8 = 80$; total effect = $100(0.8)(2.5) = 200$ decrease.  
    \item Net effect: $+250 - 200 = +50$.  
    \item $\Rightarrow$ Balanced budget multiplier is \textbf{positive}.  
\end{itemize}

\textbf{Ricardian Equivalence}
\begin{itemize}
    \item If taxpayers fully anticipate future taxes to service debt, they increase savings today → offsetting government deficit.  
    \item Then fiscal stimulus has no effect on AD.  
    \item In reality, may not hold (taxpayers underestimate liability).  
\end{itemize}

---

\textbf{LOS 14.d: Implementation of Fiscal Policy}

\begin{itemize}
    \item \textbf{Expansionary:} $\uparrow$ spending, $\downarrow$ taxes → increase AD, GDP, employment.  
    \item \textbf{Contractionary:} $\downarrow$ spending, $\uparrow$ taxes → reduce AD, inflation.  
    \item Fiscal stance often measured by change in budget balance (surplus or deficit).  
\end{itemize}

\begin{table}[H]
\centering
\small
\begin{tabular}{|l|c|c|}
\hline
 & \textbf{Expansionary Policy} & \textbf{Contractionary Policy} \\
\hline
Taxes & Decrease & Increase \\
Spending & Increase & Decrease \\
Budget & Larger deficit / smaller surplus & Larger surplus / smaller deficit \\
Impact & Boost AD, GDP, jobs & Slow AD, inflation control \\
\hline
\end{tabular}
\caption{Expansionary vs. Contractionary Fiscal Policy}
\end{table}

---

\textbf{Difficulties in Implementation}

\begin{enumerate}
    \item \textbf{Recognition lag:} Time to identify recession/inflation.  
    \item \textbf{Action lag:} Time for political debate and legislation.  
    \item \textbf{Impact lag:} Time for spending/tax changes to affect economy.  
\end{enumerate}

\begin{table}[H]
\centering
\small
\begin{tabular}{|l|p{10cm}|}
\hline
Lag Type & Description \\
\hline
Recognition & Delay in recognizing downturn/overheating \\
Action & Time for government to approve policy changes \\
Impact & Delay between enactment and economic effect \\
\hline
\end{tabular}
\caption{Fiscal Policy Lags}
\end{table}

---

\textbf{Complications and Limitations}

\begin{itemize}
    \item \textbf{Misreading economy:} If economy is already at full capacity, expansionary policy only fuels inflation.  
    \item \textbf{Crowding-out effect:} Government borrowing raises interest rates $\rightarrow$ ↓ private investment.  
    \item \textbf{Supply constraints:} If slowdown is due to supply shortages, AD stimulus won’t help.  
    \item \textbf{Deficit limits:} Too much borrowing → loss of investor confidence, ↑ interest rates.  
    \item \textbf{Multiple targets:} High unemployment + high inflation (stagflation) cannot be solved with one fiscal tool.  
\end{itemize}

\textbf{Structural Budget Deficit}
\begin{itemize}
    \item Adjusts deficit for cyclical effects (i.e., what deficit would be at full employment).  
    \item Used to distinguish automatic stabilizers vs. discretionary policy.  
\end{itemize}

\section*{15 MONETARY POLICY}

\subsection*{15.1 Central Bank Objectives and Tools}

\textbf{LOS 15.a: Roles and Objectives of Central Banks}

\begin{itemize}
    \item \textbf{Roles of central banks:}
    \begin{enumerate}
        \item Sole supplier of currency (legal tender, fiat money).
        \item Banker to government and other banks.
        \item Regulator/supervisor of payments system.
        \item Lender of last resort (prevents bank runs).
        \item Holder of FX and gold reserves.
        \item Conductor of monetary policy.
    \end{enumerate}

    \item \textbf{Primary objective:} Price stability (control inflation).  
    \begin{itemize}
        \item High inflation $\rightarrow$ uncertainty, menu costs, shoe leather costs.
        \item Target inflation = 2–3\% (avoid 0\%, which risks deflation).
    \end{itemize}

    \item \textbf{Other possible objectives:}
    \begin{itemize}
        \item Stable exchange rates.
        \item Full employment.
        \item Sustainable positive economic growth.
        \item Moderate long-term interest rates.
    \end{itemize}

    \item \textbf{Country differences:}
    \begin{itemize}
        \item U.S. Fed: dual mandate — maximum employment + moderate long-term rates (no explicit inflation target).
        \item Bank of Japan: focus on avoiding deflation (not inflation).
        \item Some developing countries: peg exchange rates to USD $\rightarrow$ manage money supply/interest rates to match U.S. inflation.
    \end{itemize}
\end{itemize}

\begin{table}[H]
\centering
\small
\begin{tabular}{|l|c|c|}
\hline
\textbf{Country/Region} & \textbf{Primary Goal} & \textbf{Other Goals} \\
\hline
U.S. Fed & Employment + moderate LT rates & Implicit inflation stability \\
ECB & Explicit 2\% inflation target & Growth, stability \\
BoJ (Japan) & Avoid deflation & Growth support \\
Pegging countries & Exchange rate stability (vs. USD) & Implied inflation convergence \\
\hline
\end{tabular}
\caption{Examples of Central Bank Objectives}
\end{table}

---

\textbf{LOS 15.b: Tools of Monetary Policy \& Transmission Mechanism}

\begin{itemize}
    \item \textbf{Main tools:}
    \begin{enumerate}
        \item \textbf{Policy rate (short-term interest rate):}
        \begin{itemize}
            \item U.S.: Discount rate (borrowing from Fed).  
            \item ECB: Refinancing rate.  
            \item BoE: Two-week repo rate.  
            \item Fed also targets \textbf{federal funds rate} (interbank overnight rate).  
            \item Lower policy rate $\rightarrow$ cheaper borrowing, ↑ lending, ↓ interest rates.  
        \end{itemize}

        \item \textbf{Reserve requirements:}
        \begin{itemize}
            \item $\uparrow$ reserve ratio $\rightarrow$ ↓ lending capacity, ↓ money supply, ↑ rates.  
            \item $\downarrow$ reserve ratio $\rightarrow$ ↑ lending capacity, ↑ money supply, ↓ rates.  
        \end{itemize}

        \item \textbf{Open market operations (OMO):}
        \begin{itemize}
            \item Central bank buys securities $\rightarrow$ ↑ reserves, ↑ lending, ↑ MS, ↓ rates.  
            \item Central bank sells securities $\rightarrow$ ↓ reserves, ↓ lending, ↓ MS, ↑ rates.  
            \item Fed’s most common tool to hit Fed funds target.  
        \end{itemize}
    \end{enumerate}

    \item \textbf{Monetary transmission mechanism:}
    \begin{itemize}
        \item Step 1: Policy rate ↑ $\rightarrow$ short-term bank lending rates ↑.  
        \item Step 2: ↑ discount rate applied to bonds, equities → asset values ↓ (wealth effect).  
        \item Step 3: Expectations of growth ↓ $\rightarrow$ less consumption + investment.  
        \item Step 4: Higher rates attract foreign capital → currency appreciates → exports ↓, imports ↑.  
        \item Net effect: ↓ aggregate demand → ↓ inflation, ↓ growth (short run).  
    \end{itemize}
\end{itemize}

\begin{table}[H]
\centering
\footnotesize
\begin{tabular}{|l|p{6cm}|p{6cm}|}
\hline
\textbf{Policy Action} & \textbf{Expansionary Effects} & \textbf{Contractionary Effects} \\
\hline
Policy Rate & ↓ borrowing costs, ↑ lending & ↑ borrowing costs, ↓ lending \\
Reserve Requirements & ↓ reserves held $\rightarrow$ ↑ money supply & ↑ reserves held $\rightarrow$ ↓ money supply \\
Open Market Ops & Buy securities $\rightarrow$ ↑ reserves, ↓ rates & Sell securities $\rightarrow$ ↓ reserves, ↑ rates \\
Currency Impact & Depreciation $\rightarrow$ ↑ exports & Appreciation $\rightarrow$ ↓ exports \\
\hline
\end{tabular}
\caption{Expansionary vs. Contractionary Monetary Tools}
\end{table}

---

\textbf{Relation to Growth, Inflation, Interest, FX}

\begin{itemize}
    \item \textbf{Expansionary Policy:}
    \begin{itemize}
        \item $\uparrow$ reserves $\rightarrow$ ↓ rates, ↑ lending.  
        \item ↓ real interest rates $\rightarrow$ domestic currency depreciation.  
        \item ↑ investment (business), ↑ durable goods demand (consumers).  
        \item Net exports ↑ due to weaker currency.  
        \item Overall effect: ↑ AD, ↑ inflation, ↑ GDP, ↑ employment.  
    \end{itemize}

    \item \textbf{Contractionary Policy:}
    \begin{itemize}
        \item Opposite effects: ↑ rates, ↓ asset values, ↓ investment, stronger currency.  
        \item ↓ AD, ↓ inflation.  
    \end{itemize}

    \item \textbf{Money neutrality:}
    \begin{itemize}
        \item Long run: Monetary policy only affects price level (not real growth).  
        \item Short run: Monetary policy affects real GDP, inflation, interest rates, and FX.  
    \end{itemize}
\end{itemize}

\begin{table}[H]
\centering
\small
\begin{tabular}{|l|c|c|}
\hline
\textbf{Variable} & \textbf{Expansionary Policy} & \textbf{Contractionary Policy} \\
\hline
GDP Growth & ↑ (short run) & ↓ \\
Inflation & ↑ & ↓ \\
Interest Rates & ↓ & ↑ \\
Currency (FX) & Depreciates & Appreciates \\
Unemployment & ↓ & ↑ \\
\hline
\end{tabular}
\caption{Effects of Monetary Policy on Key Variables}
\end{table}

\subsection*{15.2 Monetary Policy Effects and Limitations}

\subsubsection*{LOS 15.c: Qualities of Effective Central Banks}

\begin{itemize}
    \item \textbf{Independence}  
    \begin{itemize}
        \item \textit{Operational independence}: central bank sets policy rate independently.  
        \item \textit{Target independence}: central bank defines inflation computation, sets target, and time horizon.  
        \item Example: The ECB has both; most others only operational independence.
        \item Importance: prevents political interference, e.g., politicians may want expansion before elections.
    \end{itemize}

    \item \textbf{Credibility}  
    \begin{itemize}
        \item Central banks must follow through on commitments.  
        \item If credible, expectations align: e.g., if target inflation = 3\%, wage contracts are based on 3\%, making it self-fulfilling.  
        \item Governments with high debt lack credibility (incentive to allow higher inflation).
    \end{itemize}

    \item \textbf{Transparency}  
    \begin{itemize}
        \item Regular reports (inflation outlook, policy rationale).  
        \item Improves credibility and makes policy changes easier to anticipate.  
    \end{itemize}
\end{itemize}

\subsection*{Inflation, Interest Rate, and Exchange Rate Targeting}

\begin{itemize}
    \item \textbf{Interest Rate Targeting} (used in the past)  
    \begin{itemize}
        \item Increase money supply if interest rates rise above target band.  
        \item Decrease money supply if interest rates fall below target.  
    \end{itemize}

    \item \textbf{Inflation Targeting} (most common today)  
    \begin{itemize}
        \item Used by UK, Brazil, Canada, Australia, Mexico, ECB.  
        \item Typical target: 2\% with band 1--3\%.  
        \item Example: If current inflation = 4\%, central bank tightens policy; if 0\%, expansionary stance.  
    \end{itemize}

    \item \textbf{Exchange Rate Targeting} (mainly developing countries)  
    \begin{itemize}
        \item Peg currency to another (often USD).  
        \item If domestic currency falls: buy domestic currency using reserves $\Rightarrow$ reduces money supply, raises rates.  
        \item If currency rises: sell domestic currency $\Rightarrow$ increases money supply, lowers rates.  
        \item Long-run effect: inflation rate converges to pegged country’s inflation.  
        \item Limitation: foreign reserves may run out.  
    \end{itemize}
\end{itemize}

\subsubsection*{Limitations of Monetary Policy}

\begin{itemize}
    \item \textbf{Transmission mechanism limitations}:  
    \begin{itemize}
        \item Long-term rates may not move with short-term rates if expectations of inflation shift.  
        \item Example: If policy raises short-term rates but markets expect lower inflation, long-term rates may fall.  
    \end{itemize}

    \item \textbf{Liquidity Trap}:  
    \begin{itemize}
        \item Demand for money becomes elastic.  
        \item Individuals hold cash even with zero interest rates.  
        \item Policy becomes ineffective (e.g., Japan in 1990s, US/UK post-2008).  
    \end{itemize}

    \item \textbf{Deflation}:  
    \begin{itemize}
        \item Expansionary policy limited by zero lower bound (nominal rates cannot go below 0).  
        \item Harder to reverse than inflation.  
    \end{itemize}

    \item \textbf{Bank Lending Constraint}:  
    \begin{itemize}
        \item Even with higher reserves, banks may not lend (post-2008 credit crisis).  
        \item Led to \textbf{Quantitative Easing (QE)}:
        \begin{itemize}
            \item UK: bought government bonds (3--5 year maturities).  
            \item US: QE1 = mortgage-backed securities, QE2 = long-term Treasuries.  
            \item Goal: lower long-term rates, improve balance sheets, stimulate lending.  
        \end{itemize}
    \end{itemize}
\end{itemize}

\subsubsection*{Monetary Policy in Developing Economies}

\begin{itemize}
    \item Lack of liquid government debt market $\Rightarrow$ weak interest rate signals.  
    \item Rapid growth $\Rightarrow$ difficult to estimate neutral rate.  
    \item Financial innovation changes demand for money.  
    \item Weak credibility if past inflation targets were missed.  
    \item Limited independence from politics.  
\end{itemize}

\subsubsection*{LOS 15.d: Interaction of Monetary and Fiscal Policy}

\begin{table}[H]
\centering
\begin{tabular}{|p{4cm}|p{5cm}|p{5cm}|}
\hline
\textbf{Policy Mix} & \textbf{Effects on Economy} & \textbf{Example Scenario} \\
\hline
Expansionary Fiscal + Expansionary Monetary & Highly expansionary; lower rates, strong GDP growth. & Post-2008 stimulus + QE in US. \\
\hline
Contractionary Fiscal + Contractionary Monetary & Strongly contractionary; lower GDP, higher rates. & Tight austerity with rate hikes. \\
\hline
Expansionary Fiscal + Contractionary Monetary & GDP up (from fiscal); higher interest rates (crowding-out). Govt spending/GDP rises. & US 1980s: Reagan fiscal expansion + Volcker tight money. \\
\hline
Contractionary Fiscal + Expansionary Monetary & Lower interest rates; private sector grows; govt spending/GDP falls. & EU post-crisis austerity + ECB loose policy. \\
\hline
\end{tabular}
\caption{Interaction of Fiscal and Monetary Policy}
\end{table}

\begin{itemize}
    \item Fiscal multipliers vary:  
    \begin{itemize}
        \item \textbf{Highest}: direct government spending.  
        \item \textbf{Moderate}: transfers to poor.  
        \item \textbf{Lower}: tax cuts.  
    \end{itemize}
    \item \textbf{Key Insight:} Expansionary fiscal policy is most effective when combined with expansionary monetary policy.  
\end{itemize}

\section*{16 INTRODUCTION TO GEOPOLITICS}

\subsection*{16.1 Geopolitics}

\subsubsection*{LOS 16.a: Geopolitics – Cooperation vs Competition}
\begin{itemize}
  \item \textbf{Definition:} Geopolitics = interactions among nations (state and nonstate actors) shaped by geography, economics, culture, and politics.  
  \item \textbf{Cooperation:}
    \begin{itemize}
      \item Areas: diplomacy, military alliances, trade, tariffs, technology exchange, cultural flows.  
      \item Example: \textit{NATO as military cooperation; EU Single Market as economic cooperation}.  
    \end{itemize}
  \item \textbf{Competition / Noncooperation:}
    \begin{itemize}
      \item Countries prioritize national interest at expense of others.  
      \item Example: \textit{Trade wars, protectionist tariffs}.  
    \end{itemize}
  \item \textbf{Determinants of cooperation:} domestic politics, political cycles, resource endowment (e.g., mineral-rich but food-deficient nations must trade).  
  \item \textbf{Nonstate actors:} Firms and NGOs push for regulatory harmonization (e.g., IFRS adoption).  
  \item \textbf{Soft power:} Influence through culture, language, and IP (e.g., Hollywood exports).
\end{itemize}

\subsubsection*{LOS 16.b: Geopolitics and Globalization}
\begin{itemize}
  \item \textbf{Globalization:} Long-term integration of economies and cultures.  
    \begin{itemize}
      \item Trade openness grew from \textbf{25\% (1970s)} to \textbf{60\% (2008)} (World Bank).  
      \item Driven by cross-border flows of goods, capital, services, information.  
    \end{itemize}
  \item \textbf{Nationalism:} Prioritizing self-interest, protectionism, reduced openness.  
  \item Spectrum: Countries fall between extremes of globalization vs nationalism.  
\end{itemize}

\subsubsection*{Framework: Cooperation vs Globalization}
\begin{center}
\begin{tabular}{|l|l|l|}
\hline
 & \textbf{Cooperation} & \textbf{Noncooperation} \\
\hline
\textbf{Globalization} & Multilateralism & Hegemony \\
\hline
\textbf{Nationalism} & Bilateralism & Autarky \\
\hline
\end{tabular}
\end{center}

\begin{itemize}
  \item \textbf{Autarky:} Self-reliance, state dominance (e.g., North Korea).  
  \item \textbf{Hegemony:} Global influence without cooperation (e.g., U.S. in some policies).  
  \item \textbf{Bilateralism:} Two-country agreements (e.g., U.S.–Mexico trade).  
  \item \textbf{Multilateralism:} Broad cooperation (e.g., WTO, EU, ASEAN).  
\end{itemize}

\subsubsection*{LOS 16.c: International Trade Organizations}
\begin{itemize}
  \item \textbf{IMF (International Monetary Fund):}
    \begin{itemize}
      \item Goals: Monetary cooperation, exchange stability, balanced growth, BoP support.  
      \item Tools: Lending, surveillance, policy advice.  
    \end{itemize}
  \item \textbf{World Bank:}
    \begin{itemize}
      \item Mission: Poverty reduction, development support.  
      \item Institutions: IBRD (middle-income), IDA (poorest nations).  
      \item Provides: Low-interest loans, grants, capacity building.  
    \end{itemize}
  \item \textbf{WTO (World Trade Organization):}
    \begin{itemize}
      \item Role: Enforce trade rules, reduce barriers, settle disputes.  
      \item Agreements: Legally binding trade frameworks.  
    \end{itemize}
\end{itemize}

\subsubsection*{LOS 16.d: Geopolitical Risk}
\begin{itemize}
  \item \textbf{Types of Risk:}
    \begin{itemize}
      \item Event risk: Known timing, uncertain outcome (e.g., elections).  
      \item Exogenous risk: Unexpected events (e.g., war, terrorism, pandemics).  
      \item Thematic risk: Long-term patterns (e.g., cyber risk, migration).  
    \end{itemize}
  \item \textbf{Key dimensions:} Probability, Impact, Velocity.  
  \item \textbf{Black swan risk:} Low probability, high impact exogenous risk (e.g., 9/11, COVID-19).  
\end{itemize}

\subsubsection*{LOS 16.e: Tools of Geopolitics}
\begin{itemize}
  \item \textbf{National Security Tools:}  
    \begin{itemize}
      \item Armed conflict, espionage, treaties.  
      \item Active (in use) vs. Threatened (deterrent).  
    \end{itemize}
  \item \textbf{Economic Tools:}  
    \begin{itemize}
      \item Cooperative: Free trade areas, common markets, monetary unions.  
      \item Noncooperative: Tariffs, export restraints, nationalization.  
    \end{itemize}
  \item \textbf{Financial Tools:}  
    \begin{itemize}
      \item Cooperative: Capital mobility, FDI, currency convertibility.  
      \item Noncooperative: Capital controls, sanctions.  
    \end{itemize}
\end{itemize}

\subsection*{LOS 16.f: Investment Impact of Geopolitical Risk}
\begin{itemize}
  \item \textbf{Investment effects:}
    \begin{itemize}
      \item Raises/lower risk premiums on assets.  
      \item Discrete (industry-specific) vs Broad (country/region-wide).  
    \end{itemize}
  \item \textbf{Business cycle sensitivity:} Risks have greater effects during recessions.  
  \item \textbf{Analysis methods:}  
    \begin{itemize}
      \item Scenario analysis (qualitative/quantitative).  
      \item Signposts: Indicators signaling rising risk (e.g., CDS spreads, volatility indices).  
    \end{itemize}
\end{itemize}

\subsubsection*{Example Table: Risk Types vs Investment Impact}
\begin{center}
\begin{tabular}{|l|l|l|l|}
\hline
\textbf{Risk Type} & \textbf{Example} & \textbf{Velocity} & \textbf{Investment Impact} \\
\hline
Event & Election outcome & Medium & Policy shifts, sector winners/losers \\
Exogenous & War outbreak & High & Market crash, currency fall \\
Thematic & Cyber risk & Low–Medium & Tech sector costs, ESG implications \\
\hline
\end{tabular}
\end{center}

\section*{17 INTERNATIONAL TRADE}

\subsection*{17.1 International Trade}

\subsubsection*{LOS 17.a: Benefits and Costs of International Trade}

\begin{itemize}
    \item \textbf{Comparative Advantage:}
    \begin{itemize}
        \item Countries specialize in goods with lower \emph{opportunity cost}.
        \item Example: 
        \begin{itemize}
            \item Country A: lower cost of steel $\rightarrow$ exports steel.
            \item Country B: lower cost of textiles $\rightarrow$ exports textiles.
            \item Both gain: total world output increases.
        \end{itemize}
    \end{itemize}

    \item \textbf{New Trade Theory (beyond Ricardo):}
    \begin{itemize}
        \item Gains from \textbf{economies of scale}.
        \item Variety of products increases.
        \item Domestic monopolies lose pricing power $\rightarrow$ consumer welfare increases.
        \item Example: automobile industry: Germany exports BMW, Japan exports Toyota, USA exports Ford — consumers get variety and lower prices.
    \end{itemize}

    \item \textbf{Costs of Free Trade:}
    \begin{itemize}
        \item \textbf{Job losses} in import-competing industries.
        \item \textbf{Income inequality} may rise.
        \item Example: 
        \begin{itemize}
            \item U.S. textile industry loses jobs when cheaper imports from Bangladesh enter the market.
            \item Domestic steel consumers in importing country benefit from cheaper steel, but local steel workers lose.
        \end{itemize}
    \end{itemize}

    \item \textbf{Dynamic Adjustment:}
    \begin{itemize}
        \item In the long run, workers retrain and reallocate to other industries.
        \item Economic theory: \textbf{net welfare gains exceed losses}.
    \end{itemize}
\end{itemize}

\subsubsection*{LOS 17.b: Trade Restrictions and Their Economic Implications}

\begin{itemize}
    \item \textbf{Motivations for Restrictions:}
    \begin{itemize}
        \item Infant industry protection (valid argument).
        \item National security goods (valid argument).
        \item Protecting domestic jobs (weak argument).
        \item Retaliation, tariff revenue, political lobbying.
    \end{itemize}

    \item \textbf{Types of Restrictions:}
    \begin{enumerate}
        \item Tariffs (taxes on imports).
        \item Quotas (quantity limits on imports).
        \item Export subsidies (government payments to exporters).
        \item Minimum domestic content rules.
        \item Voluntary export restraints (VERs).
    \end{enumerate}

    \item \textbf{Economic Implications:}
    \begin{itemize}
        \item Tariffs $\rightarrow$ higher domestic price, lower imports, government gains revenue.
        \item Quotas $\rightarrow$ same effect as tariffs but revenue depends on whether government sells licenses.
        \item Export subsidies $\rightarrow$ help exporters but hurt domestic consumers.
        \item VERs $\rightarrow$ benefit foreign exporters with licenses (quota rents).
    \end{itemize}
\end{itemize}

\begin{table}[H]
\centering
\caption{Comparison of Trade Restrictions}
\resizebox{\textwidth}{!}{%
\begin{tabular}{|l|c|c|c|}
\hline
\textbf{Policy} & \textbf{Winners} & \textbf{Losers} & \textbf{Govt. Revenue?} \\
\hline
Tariff & Domestic producers, Govt. & Consumers, Foreign exporters & Yes \\
\hline
Quota (licenses sold) & Domestic producers, Govt. & Consumers, Foreign exporters & Yes \\
\hline
Quota (licenses free) & Domestic producers, Foreign exporters (quota rents) & Consumers & No \\
\hline
VER & Domestic producers, Foreign exporters with quota rents & Consumers & No \\
\hline
Export subsidy & Domestic exporters & Consumers (higher price), Govt. (costs subsidy) & Negative (budget loss) \\
\hline
\end{tabular}}
\end{table}

\textbf{Welfare Analysis:}
\begin{itemize}
    \item Consumer Surplus (CS) decreases.
    \item Producer Surplus (PS) increases.
    \item Govt. revenue depends on policy.
    \item Deadweight loss = welfare loss from lost efficiency.
\end{itemize}

\subsubsection*{LOS 17.c: Trading Blocs, Common Markets, and Economic Unions}

\begin{itemize}
    \item \textbf{Integration Types (in increasing order):}
    \begin{enumerate}
        \item Free Trade Area (FTA): no tariffs among members. (e.g., NAFTA/USMCA)
        \item Customs Union: FTA + common external tariffs.
        \item Common Market: Customs Union + free movement of labor/capital.
        \item Economic Union: Common Market + shared institutions/policies. (e.g., EU)
        \item Monetary Union: Economic Union + single currency. (e.g., Eurozone)
    \end{enumerate}

    \item \textbf{Benefits:}
    \begin{itemize}
        \item Increased efficiency and competition.
        \item Larger markets, economies of scale.
        \item Comparative advantage exploited more fully.
    \end{itemize}

    \item \textbf{Costs:}
    \begin{itemize}
        \item Job and wage losses in less competitive industries.
        \item Possible trade diversion (switching imports from cheaper non-members to higher-cost members).
    \end{itemize}
\end{itemize}

\begin{table}[H]
\centering
\caption{Levels of Economic Integration}
\resizebox{\textwidth}{!}{%
\begin{tabular}{|l|c|c|c|c|c|}
\hline
\textbf{Feature} & FTA & Customs Union & Common Market & Economic Union & Monetary Union \\
\hline
Free movement of goods/services & Yes & Yes & Yes & Yes & Yes \\
\hline
Common external tariffs & No & Yes & Yes & Yes & Yes \\
\hline
Free movement of labor/capital & No & No & Yes & Yes & Yes \\
\hline
Common policies/institutions & No & No & No & Yes & Yes \\
\hline
Single currency & No & No & No & No & Yes \\
\hline
Example & NAFTA/USMCA & MERCOSUR & --- & EU & Eurozone \\
\hline
\end{tabular}}
\end{table}

\section*{18 CAPITAL FLOWS AND THE FX MARKET}

\subsection*{18.1: The Foreign Exchange Market}

\begin{itemize}
  \item \textbf{Functions of the FX Market:}
    \begin{itemize}
      \item Facilitates cross-border trade in goods and services (denominated in foreign currencies).
      \item Enables capital flows: purchase of foreign financial securities and physical assets.
      \item Provides hedging opportunities against exchange rate risk (e.g., forward contracts).
      \item Allows speculation on future movements of exchange rates.
    \end{itemize}

  \item \textbf{Participants in the FX Market:}
    \begin{itemize}
      \item \textbf{Sell side:} Large multinational banks (primary dealers in FX and originators of forwards).
      \item \textbf{Buy side:}
        \begin{itemize}
          \item \textit{Corporations:} Conduct cross-border transactions, hedge FX risk.
          \item \textit{Investment accounts:}
            \begin{itemize}
              \item Real money accounts: pension funds, insurance firms, mutual funds.
              \item Leveraged accounts: hedge funds, proprietary trading firms.
            \end{itemize}
          \item \textit{Governments and central banks:} Hold reserves, intervene for policy.
          \item \textit{Retail investors:} Households and small institutions (tourism, investments).
        \end{itemize}
    \end{itemize}

  \item \textbf{Types of Exchange Rates:}
    \begin{itemize}
      \item \textbf{Nominal exchange rate:} Price of one currency in terms of another at current period. \\
        Example: \(1.25 \, \text{USD/EUR}\) means 1 EUR costs 1.25 USD.
      \item \textbf{Real exchange rate:} Adjusted for relative price levels between two countries.  
        \[
        R_{P/B} = \text{Nominal}_{P/B} \times \frac{CPI_{B}}{CPI_{P}}
        \]
        where $P$ = price currency country, $B$ = base currency country.
    \end{itemize}

  \item \textbf{Direct vs Indirect Quotes:}
    \begin{itemize}
      \item Direct quote: Price of one unit of foreign (base) currency in terms of domestic (price) currency.
      \item Indirect quote: Price of one unit of domestic currency in terms of foreign currency.
      \item Example:  
        \begin{tabular}{|c|c|c|}
        \hline
        Quote & USD-based investor & EUR-based investor \\
        \hline
        1.17 USD/EUR & Direct & Indirect \\
        0.855 EUR/USD & Indirect & Direct \\
        \hline
        \end{tabular}
    \end{itemize}

  \item \textbf{Effects of Exchange Rate Changes:}
    \begin{itemize}
      \item If USD/EUR rises from 1.10 to 1.15:
        \begin{itemize}
          \item EUR becomes more expensive for U.S. consumers.
          \item Purchasing power of USD decreases.
        \end{itemize}
      \item If Eurozone prices rise relative to U.S. prices:
        \begin{itemize}
          \item Real USD/EUR increases, reducing USD purchasing power.
        \end{itemize}
    \end{itemize}

  \item \textbf{Example – Real Exchange Rate:}  
  Base period:  
  \[
  \text{Exchange rate} = 1.70 \, \text{USD/£}, \quad CPI_{US}=100, \quad CPI_{UK}=100
  \]  
  After 3 years:  
  \[
  \text{Exchange rate} = 1.60 \, \text{USD/£}, \quad CPI_{US}=110, \quad CPI_{UK}=112
  \]  
  Calculation:  
  \[
  R_{USD/£} = 1.60 \times \frac{112}{110} = 1.629 \, \text{USD/£}
  \]  
  \textbf{Interpretation:} The real exchange rate fell from 1.70 to 1.629. The USD gained purchasing power, but less than if relative prices were unchanged.

  \item \textbf{Spot vs Forward Exchange Rates:}
    \begin{itemize}
      \item Spot: Immediate delivery (usually T+2).
      \item Forward: Agreed exchange at a future date (e.g., 30, 60, 90 days).
      \item Example: A French firm receives £10m in 6 months. With forward rate \(1.192 \, \text{EUR/GBP}\), it locks in:  
        \[
        10m \times 1.192 = 11.92m \, \text{EUR}
        \]
        => Eliminates FX risk.
    \end{itemize}

  \item \textbf{Percentage Change in Currency Value:}
    \begin{itemize}
      \item Formula:  
      \[
      \%\Delta = \frac{S_{t} - S_{0}}{S_{0}}
      \]
      where $S$ = exchange rate.
      \item Example: USD/EUR falls from 1.42 to 1.39.  
        \[
        \%\Delta_{\text{Euro}} = \frac{1.39}{1.42} - 1 = -2.11\% \quad \Rightarrow \text{Euro depreciated}
        \]
        Convert to EUR/USD:  
        \[
        1/1.42 = 0.7042, \quad 1/1.39 = 0.7194
        \]
        \[
        \%\Delta_{\text{USD}} = \frac{0.7194}{0.7042} - 1 = +2.16\% \quad \Rightarrow \text{USD appreciated}
        \]
    \end{itemize}

\end{itemize}

\subsection*{18.2: Managing Exchange Rates}

\begin{itemize}
  \item \textbf{Exchange Rate Regimes (IMF Classification):}
    \begin{itemize}
      \item \textbf{Countries without their own currency:}
        \begin{itemize}
          \item \textit{Formal Dollarization:} Use another country’s currency (e.g., Panama uses USD). 
            \begin{itemize}
              \item No independent monetary policy.
              \item Inflation rate = inflation of adopted currency.
            \end{itemize}
          \item \textit{Monetary Union:} Countries adopt a common currency (e.g., Eurozone).
            \begin{itemize}
              \item Shared monetary policy (ECB in Eurozone).
              \item Sacrifice national monetary sovereignty.
            \end{itemize}
        \end{itemize}

      \item \textbf{Countries with their own currency:}
        \begin{itemize}
          \item \textit{Currency Board Arrangement:}
            \begin{itemize}
              \item Explicitly backs domestic money with foreign reserves at fixed rate.
              \item Example: Hong Kong Monetary Authority backs HKD with USD.
              \item Pros: Stability, credibility. Cons: No independent monetary policy.
            \end{itemize}
          \item \textit{Conventional Fixed Peg:}
            \begin{itemize}
              \item Peg within ±1\% band to another currency or basket.
              \item Maintained by \textbf{direct intervention} (FX buying/selling) or \textbf{indirect intervention} (interest rates, regulation).
            \end{itemize}
          \item \textit{Pegged within Horizontal Bands (Target Zone):}
            \begin{itemize}
              \item Wider bands (e.g., ±2\%).
              \item Provides more monetary flexibility than conventional peg.
            \end{itemize}
          \item \textit{Crawling Peg:}
            \begin{itemize}
              \item Adjusted periodically to offset inflation differential.
              \item \textbf{Passive crawling peg:} Adjustments follow inflation.  
              \item \textbf{Active crawling peg:} Pre-announced adjustments to influence expectations.
            \end{itemize}
          \item \textit{Crawling Bands:}
            \begin{itemize}
              \item Bands around peg widen over time.
              \item Used as transition from fixed to floating system.
            \end{itemize}
          \item \textit{Managed Float:}
            \begin{itemize}
              \item Exchange rate market-determined but monetary authority intervenes if needed.
              \item No specific long-run target.
            \end{itemize}
          \item \textit{Independent Float:}
            \begin{itemize}
              \item Fully market-determined exchange rate.
              \item Intervention only to reduce volatility (not target levels).
            \end{itemize}
        \end{itemize}
    \end{itemize}

  \item \textbf{Summary Table of Regimes:}

  \begin{tabular}{|l|l|l|l|}
  \hline
  \textbf{Regime} & \textbf{Flexibility} & \textbf{Example} & \textbf{Policy Autonomy} \\
  \hline
  Dollarization & None & Panama (USD) & None \\
  Monetary Union & Shared & Eurozone & Limited (ECB level) \\
  Currency Board & Very low & Hong Kong (USD peg) & None \\
  Fixed Peg ±1\% & Low & Gulf states (USD peg) & Limited \\
  Target Zone ±2\% & Medium & ERM II (EUR) & Moderate \\
  Crawling Peg & Low & Some LatAm countries & Limited \\
  Crawling Bands & Increasing & Transition cases & Rising \\
  Managed Float & High & China (historically) & High \\
  Independent Float & Full & USD, EUR, JPY & Full \\
  \hline
  \end{tabular}

  \item \textbf{Effects of Exchange Rate Changes on Trade \& Capital Flows:}
    \begin{itemize}
      \item If \textbf{USD/EUR decreases}, then USD appreciates:
        \begin{itemize}
          \item Eurozone goods cheaper in USD → U.S. imports rise.
          \item U.S. goods more expensive in EUR → Eurozone imports from U.S. fall.
        \end{itemize}
      \item Goods trade (exports/imports) adjusts \textbf{slowly}.
      \item Capital flows (investment in assets, debt) adjust \textbf{quickly}.
    \end{itemize}

  \item \textbf{Balance of Payments (BoP) Identity:}
    \[
    (X - M) + (CI - CO) + (FI - FO) = 0
    \]
    where:
    \begin{itemize}
      \item $X-M$ = Net exports (trade balance).
      \item $CI-CO$ = Net capital inflows/outflows.
      \item $FI-FO$ = Net financial inflows/outflows.
    \end{itemize}

    Alternative representation (from macro identity):  
    \[
    (X - M) = (S - I) + (T - G)
    \]
    \begin{itemize}
      \item Trade deficit ($X-M < 0$) means savings $<$ investment → foreign capital needed.
      \item Example: U.S. trade deficit with China is offset by China’s purchase of U.S. Treasuries (capital inflow).
    \end{itemize}

  \item \textbf{Capital Flows:}
    \begin{itemize}
      \item \textbf{Short-run:} Determined mainly by capital flows (fast adjustment).
      \item \textbf{Long-run:} Determined mainly by trade flows (goods/services adjust slowly).
    \end{itemize}

  \item \textbf{Government Capital Restrictions (LOS 18.c):}
    \begin{itemize}
      \item Objectives:
        \begin{itemize}
          \item Reduce volatility of domestic asset prices.
          \item Maintain fixed exchange rate targets.
          \item Keep domestic interest rates low (restrict outflows).
          \item Protect strategic industries (e.g., defense, telecom).
        \end{itemize}
      \item Example: China restricts capital outflows to maintain RMB peg and keep domestic rates low.
    \end{itemize}

\end{itemize}

\section*{19 EXCHANGE RATE CALCULATIONS}

\subsection*{19.1: Foreign Exchange Rates}

\begin{itemize}
  \item \textbf{LOS 19.a: Currency Cross-Rates}
    \begin{itemize}
      \item Definition: A \textbf{cross rate} is the exchange rate between two currencies implied by their exchange rates with a common third currency (usually USD or EUR).
      \item Formula: 
      \[
        \text{Cross Rate (A/B)} = \frac{\text{A/USD}}{\text{B/USD}}
      \]
      if both rates are quoted against USD.
      \item Key Rule: Ensure the bases of quotation match. If needed, invert the rate.
      
      \item \textbf{Example 1: MXN/AUD}
        \begin{itemize}
          \item MXN/USD = 10.70, USD/AUD = 0.60
          \item Convert: MXN/AUD = (MXN/USD) $\times$ (USD/AUD) = 10.70 $\times$ 0.60 = 6.42
          \item Interpretation: 1 AUD = 6.42 MXN.
        \end{itemize}

      \item \textbf{Example 2: CHF/NZD}
        \begin{itemize}
          \item CHF/USD = 1.7799, NZD/USD = 2.2529
          \item CHF/NZD = (CHF/USD) / (NZD/USD) = 1.7799 / 2.2529 = 0.7900
          \item Interpretation: 1 NZD = 0.79 CHF.
        \end{itemize}
    \end{itemize}

  \item \textbf{LOS 19.b: Forward Exchange Rates and Arbitrage (Covered Interest Parity)}
    \begin{itemize}
      \item \textbf{No-Arbitrage Principle:}  
      An investor cannot earn more than the domestic risk-free rate by borrowing domestically, converting to foreign currency, investing abroad, and locking in a forward contract.
      \[
        F_{d/f} = S_{d/f} \times \frac{(1 + R_d)}{(1 + R_f)}
      \]
      where:
      \begin{itemize}
        \item $F_{d/f}$ = forward exchange rate (domestic per foreign)
        \item $S_{d/f}$ = spot exchange rate
        \item $R_d$ = domestic risk-free rate
        \item $R_f$ = foreign risk-free rate
      \end{itemize}

      \item \textbf{Example 1: 1-Year Forward ABE/DUB}
        \begin{itemize}
          \item Spot: ABE/DUB = 4.5671
          \item $R_{ABE}$ = 5\%, $R_{DUB}$ = 3\%
          \item $F = 4.5671 \times \frac{1.05}{1.03} = 4.6558$
          \item Forward premium = $\frac{4.6558}{4.5671} - 1 = 1.94\%$
          \item Interpretation: ABE has higher interest rate → expected depreciation.
        \end{itemize}

      \item \textbf{Example 2: Arbitrage Profit if Forward Mispriced}
        \begin{itemize}
          \item Given forward = 4.6000 (below no-arbitrage rate 4.6558).
          \item Strategy:
            \begin{enumerate}
              \item Borrow 1,000 DUB at 3\% -> repay 1,030 after 1 year.
              \item Convert to ABE: 1,000 $\times$ 4.5671 = 4,567.1 ABE.
              \item Invest at 5\% -> 4,795.45 ABE after 1 year.
              \item Sell forward at 4.6000 -> 4,795.45 / 4.6000 = 1,042.49 DUB.
              \item Net = 1,042.49 - 1,030 = 12.49 DUB arbitrage profit.
            \end{enumerate}
          \item Arbitrage pushes spot and forward rates back into parity.
        \end{itemize}

      \item \textbf{Short-Term Forwards:}  
        For 30, 90, 180-day contracts, use money market rates:
        \[
          F = S \times \frac{(1 + R_d \times \tfrac{t}{360})}{(1 + R_f \times \tfrac{t}{360})}
        \]

      \item \textbf{Forward Quotes in Points:}
        \begin{itemize}
          \item Each point = last decimal place in spot.
          \item Example: AUD/EUR = 0.7313, 1Y forward = +3.5 points.
          \item Forward = 0.7313 + 0.00035 = 0.73165.
        \end{itemize}

      \item \textbf{Forward Quotes in Percentage:}
        \begin{itemize}
          \item Example: AUD/EUR = 0.7313, forward = -0.062\% (120-day).
          \item Forward = 0.7313 (1 - 0.00062) = 0.7308.
        \end{itemize}

      \item \textbf{Forward Premium / Discount:}
        \[
          \text{Premium or Discount} = \frac{F}{S} - 1
        \]
        \begin{itemize}
          \item Example: USD/EUR spot = 1.312, forward = 1.320
          \item Premium = 1.320 / 1.312 - 1 = 0.61\% (Euro forward premium).
          \item Annualized Premium = 0.61\% $\times \frac{12}{3} = 2.44\%$
          \item Interpretation: More USD needed to buy 1 EUR -> EUR expected to appreciate.
        \end{itemize}
    \end{itemize}

  \item \textbf{Summary Table: Cross-Rates and Forward Rates}

\begin{center}
\resizebox{\textwidth}{!}{%
  \begin{tabular}{|l|l|l|l|}
  \hline
  \textbf{Concept} & \textbf{Formula} & \textbf{Example} & \textbf{Result} \\
  \hline
  Cross Rate & A/USD $\div$ B/USD & CHF/USD = 1.7799, NZD/USD = 2.2529 & 0.7900 CHF/NZD \\
  Forward Rate (CIP) & $S \times \frac{1+R_d}{1+R_f}$ & Spot 4.5671, $R_d=5\%$, $R_f=3\%$ & 4.6558 \\
  Points & $F = S + \tfrac{\text{points}}{10,000}$ & 0.7313, +35 points & 0.73165 \\
  Percent & $F = S(1 + q)$ & 0.7313, $q=-0.062\%$ & 0.7308 \\
  Premium/Discount & $\tfrac{F}{S} - 1$ & 1.312 vs 1.320 & +0.61\% premium \\
  \hline
  \end{tabular}}
\end{center}

\end{itemize}

\end{document}